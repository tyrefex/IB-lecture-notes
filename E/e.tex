%%% TO DO %%%
% implement siunitx package and change units

\documentclass[12pt]{article}

\usepackage{ishn}
\usepackage{siunitx}
\usepackage{etoolbox}

\makeindex[intoc]

\begin{document}
\robustify\dots
\sisetup{input-digits = 0123456789\dots}

\hypersetup{pageanchor=false}
\begin{titlepage}
	\begin{center}
		\vspace*{1em}
		\Huge
		\textbf{IB Electromagnetism}

		\vspace{1em}
		\large
		Ishan Nath, Lent 2023

		\vspace{1.5em}

		\Large

		Based on Lectures by Prof. Gordon Ogilvie

		\vspace{1em}

		\large
		\today
	\end{center}
	
\end{titlepage}
\hypersetup{pageanchor=true}

\tableofcontents

\newpage

\section{Introduction}
\label{sec:introduction}

\subsection{Charges and Currents}
\label{sub:charges_and_currents}

\emph{Electric charge}\index{electric charge} is a physical property of elementary particles. It is:
\begin{itemize}
	\item Positive, negative or zero.
	\item Quantized (an integer multiple of the \emph{elementary charge} $e$).
	\item Conserved (even if particles are created or destroyed).
\end{itemize}

By convention, the electron has charge $-e$, the proton has charge $+e$, and the neutron has charge $0$.

On macroscopic scales, the number of particles is so large that charge can be considered to have continuous \emph{electric charge density}\index{electric charge density} $\rho(\mathbf{x},t)$. The total charge in a volume $V$ is then
\[
Q = \int_{V} \rho \diff V
.\]

The \emph{electric current density}\index{electric charge density} $\mathbf{J}(\mathbf{x}, t)$ is the flux of electric charge per unit area. The current flowing through a surface $S$ is
\[
I = \int_{S} \mathbf{J} \cdot \diff \mathbf{S}
.\]

Consider a time-independent volume $V$ with boundary $S$. Since charge is conserved, we have
\begin{align*}
	\frac{\diff Q}{\diff t} &= -I, \\
	\frac{\diff}{\diff t} \int_{V} \rho \diff V + \int_{S} \mathbf{J} \cdot d \mathbf{S} &= 0, \\
	\int_{V} \biggl( \frac{\partial \rho}{\partial t} + \nabla \cdot \mathbf{J} \biggr) \diff V &= 0.
\end{align*}
Since this is true for any $V$, we must have
\[
\frac{\partial \rho}{\partial t} + \nabla \cdot \mathbf{J} = 0
.\]

This \emph{equation of charge conservation}\index{equation of charge conservation} has the typical form of a conservation law.

The discrete charge distribution of a single particle of charge $q_i$, and position vector $\mathbf{x}_i(t)$ is
\begin{align*}
	\rho &= q_i \delta( \mathbf{x} - \mathbf{x}_i(t)), \\
	\mathbf{J} &= q_i \mathbf{\dot x}_i \delta(\mathbf{x} - \mathbf{x}_i(t)).
\end{align*}
For $N$ particles, it is
\begin{align*}
	\rho &= \sum_{i = 1}^{N} q_i \delta(\mathbf{x} - \mathbf{x}_i(t)), \\
	\mathbf{J} &= \sum_{i = 1}^{N} q_i \mathbf{\dot x}_i \delta(\mathbf{x} - \mathbf{x}_i(t)).
\end{align*}
We can verify that these distributions satisfy the charge conservation equation.

\subsection{Fields and Forces}
\label{sub:fields_and_forces}

Electromagnetism is a \emph{field theory}\index{field theory}. Charged particles interact not directly, but by generating fields around them that are experienced by other charged particles.

In general, we have two time-dependent vector fields: the \emph{electric field}\index{electric field} $\mathbf{E}(\mathbf{x}, t)$, and the \emph{magnetic field}\index{magnetic field} $\mathbf{B}(\mathbf{x}, t)$.

The \emph{Lorentz force}\index{Lorentz force} on a particle of charge $q$ and velocity $\mathbf{v}$ is
\[
\mathbf{F} = q(\mathbf{E} + \mathbf{v} \times \mathbf{B})
.\]

\subsection{Maxwell's equations}
\label{sub:maxwells_equations}

In this course we will explore some consequences of \emph{Maxwell's equations}\index{Maxwell's equations}
\begin{align*}
	\nabla \cdot \mathbf{E} &= \frac{\rho}{\epsilon_0}, & \nabla \cdot \mathbf{B} &= 0, \\
	\nabla \times \mathbf{E} &= - \frac{\partial \mathbf{B}}{\partial t}, & \nabla \times \mathbf{B} &= \mu_0 \biggl( \mathbf{J} + \epsilon_0 \frac{\partial \mathbf{E}}{\partial t} \biggr).
\end{align*}
Some properties of Maxwell's equations are:
\begin{itemize}
	\item They are coupled linear PDE's in space and time.
	\item They involve two positive constants: $\epsilon_0$ (vacuum permittivity)\index{vacuum permittivity}, and $\mu_0$ (vacuum permeability)\index{vacuum permeability}.
	\item Charges $(\rho)$ and currents $(\mathbf{J})$ are the sources of the electromagnetic fields.
	\item Each equation has an equivalent integral form, related via the divergence theorem of Stokes' theorem.
	\item These are the vacuum equations that apply on microscopic scales (or in a vacuum). A related macroscopic version applies in media (for examples air).
	\item The equations are consistent with each other and with charge conservation. For example, $\nabla \cdot (M3) = \frac{\partial}{\partial t}(M2)$, and
		\[
		\frac{\partial \rho}{\partial t} + \nabla \cdot \mathbf{J} = \frac{\partial}{\partial t} (\epsilon_0 \nabla \cdot \mathbf{E}) + \nabla \cdot \biggl(- \epsilon_0 \frac{\partial \mathbf{E}}{\partial t} + \frac{1}{\mu_0} \nabla \times \mathbf{B} \biggr) = 0
		.\]
\end{itemize}

\subsection{Units}
\label{sub:units}

The SI unit of electric charge is the coulomb (\unit{\coulomb}). The elementary charge is (exactly)
\[
	e = \qty{1.602176634e-19}{\coulomb}
.\]
The SI unit of electric current is the ampere, or amp (\unit{\ampere}), equal to \qty{1}{\coulomb\per\second}.

The SI base units needed in electromagnetism are:
\begin{itemize}
	\item[] second (\unit{\second})
	\item[] metre (\unit{\metre})
	\item[] kilogram (\unit{\kilogram})
	\item[] ampere (\unit{\ampere})
\end{itemize}

From the Lorentz force law, we can see that the units of $\mathbf{E}$ and $\mathbf{B}$ must be
\begin{center}
	\unit{\kilogram\metre\per\second\cubed\per\ampere} and \unit{\kilogram\per\second\squared\per\ampere}.
\end{center}
The latter is also called the tesla (\unit{\tesla}). From Maxwell's equations, we can work out the units of $\epsilon_0$ and $\mu_0$. The experimentally determined values are
\begin{align*}
	\epsilon_0 &= \qty{8.854\dots e-12}{\per\kilogram\per\metre\cubed\second\tothe{4}\ampere\squared} \\
	\mu_0 &= \qty{1.256\dots e-6}{\kilogram\metre\per\second\squared\per\ampere\squared}\\
	      &\approx 4 \pi \times 10^{-7}\, \unit{\kilogram\metre\per\second\squared\per\ampere\squared}.
\end{align*}

The speed of light is (exactly)
\[
	c = \frac{1}{\sqrt{\mu_0 \eps_0}} = \qty{299792458}{\metre\per\second} \approx \qty{3e8}{\metre\per\second}
.\]

\newpage

\section{Electrostatics}
\label{sec:electrostatics}

In a time-independent situation, Maxwell's equations reduce to
\begin{align*}
	\nabla \cdot \mathbf{E} &= \frac{\rho}{\epsilon_0}, & \nabla \times \mathbf{E} &= \mathbf{0}, \\
	\nabla \cdot \mathbf{B} &= 0, & \nabla \times \mathbf{B} &= \mu_0 \mathbf{J}.
\end{align*}
Since $\mathbf{E}$ and $\mathbf{B}$ are decoupled, we can study them separately.

\emph{Electrostatics}\index{electrostatics} is the study of the electric field generated by a stationary charge distribution
\begin{align*}
	\nabla \cdot \mathbf{E} = \frac{\rho}{\epsilon_0}, & \nabla \times \mathbf{E} &= \mathbf{0}.
\end{align*}

\subsection{Gauss' Law}
\label{sub:gauss_law}

Consider a closed surface $S$ enclosing a volume $V$. Integrating (M1) over $V$ and using the divergence theorem, we obtain \emph{Gauss' law}\index{Gauss' law}
\[
\int_{S} \mathbf{E} \cdot \diff \mathbf{S} = \frac{Q}{\epsilon_0}
,\]
where $Q = \int_{V} \rho \diff V$ is the total charge in $V$.

Gauss' law is the integral version of (M1) and is valid generally. This says that the electric flux of a closed surface is proportional to the total charge enclosed.

In special situations, we can use Gauss' law together with symmetry to deduce $\mathbf{E}$ from $\rho$. By choosing the \emph{Gaussian surface}\index{Gaussian surface} $S$ appropriately.

\subsubsection{Spherical Symmetry}
\label{subsub:spherical_symmetry}

Consider a spherically symmetric charge distribution, $\rho(r)$ in spherical polar coordinates, with total charge $Q$ contained within an outer radius $R$.

To have spherical symmetry, the electric field should have the form
\[
	\mathbf{E} = E(r) \mathbf{e}_r
.\]
This will satisfy (M3'), as required. To find $E(r)$, we apply Gauss' law to a sphere of radius $r$. If $r > R$, then
\[
\int_{S}\mathbf{E}\cdot \diff \mathbf{S} = E(r) \int_{S} \mathbf{e}_r \cdot \diff \mathbf{S} = E(r) \int_{S} \diff S = E(r) \, 4 \pi r^2 = \frac{Q}{\epsilon_0}
.\]
Thus, outside of the sphere of radius $R$,
\[
\mathbf{E} = \frac{Q}{4 \pi \epsilon_0 r^2}\mathbf{e}_r
.\]
So the external electric field of a spherically symmetric body depends only on the total charge.

The Lorentz force on a particle of charge $q$ in $r > R$ is
\[
\mathbf{F} = q \mathbf{E} = \frac{Qq}{4 \pi \epsilon_0 r^2} \mathbf{e}_r
.\]
This is the \emph{Coulomb force}\index{Coulomb force} between charged particles. The force is repulsive if the charges have the same sign $(Qq > 0)$ and attractive if they have opposite signs $(Qq < 0)$.

If we take the limit $R \to 0$, we obtain the electric field of a \emph{point charge}\index{point charge} $Q$, corresponding to
\[
\rho = Q \delta(\mathbf{x})
.\]

There is a close analogy between the Coulomb force and the gravitational force between massive particles,
\[
	\mathbf{F} = - \frac{GMm}{r^2} \mathbf{e}_r
.\]
Both involve an inverse-square law, and the product of the charges/masses. However,
\begin{itemize}
	\item While gravity is always attractive, electric forces can be repulsive or attractive.
	\item Gravity is very much weaker than the Coulomb force, e.g. for two protons the ratio of the electric to gravitational forces is
		\[
		\frac{e^2}{4 \pi \epsilon_0 G m_p^2} \approx 10^{36}
		.\]
		On the atomic scale, gravity is irrelevant. But positive and negative charges balance so accurately that on the planetary scale, gravity is dominant.
\end{itemize}

\subsubsection{Cylindrical Symmetry}
\label{subsub:cylindrical_symmetry}

Consider a cylindrically symmetric charge distribution $\rho(r)$ in cylindrical polar coordinates, with total charge $\lambda$ per unit length, contained within an outer radius $R$.

To have cylindrical symmetry,
\[
\mathbf{E} = E(r) \mathbf{e}_r
.\]
To find  $E(r)$ we apply Gauss' law to a cylinder of radius $r$ and arbitrary length $L$. Again, we consider $r > R$. Then, since only the curved part of the cylinder contributes to the flux,
\[
\int_{S}\mathbf{E} \cdot \diff \mathbf{S} = E(r) \int_{S}\mathbf{e}_r \cdot \diff \mathbf{S} = E(r) \int_{S} \diff S = E(r) 2 \pi r L = \frac{\lambda L}{\epsilon_0}
.\]
Thus, we get
\[
\mathbf{E} = \frac{\lambda}{2 \pi \epsilon_0 r} \mathbf{e_r}
.\]
In the limit $R \to 0$, we obtain the electric field of a \emph{line charge}\index{line charge} $\lambda$ per unit length, corresponding to
\[
\rho = \lambda \delta(x) \delta(y)
.\]

\subsubsection{Planar Symmetry}
\label{subsub:planar_symmetry}

We consider a planar charge distribution $\rho(z)$ in Cartesian coordinates, with total charge $\sigma$ per unit area, contained within a region $-d < z < d$ of thickness $2d$. We assume reflectional symmetry, so $\rho(z)$ is even.

To have planar symmetry, we need
\[
\mathbf{E} = E(z) \mathbf{e}_z
,\]
which will satisfy (M3'). Reflectional symmetry implies $E(-z) = -E(z)$. To find $E(z)$ for $z > 0$, apply Gauss' law to a ``Gaussian pillbox'' of height $2z$ and arbitrary area $A$. If $z > d$, then
\[
\int_{S}\mathbf{E} \cdot \diff \mathbf{S} = E(z)A - E(-z)A = 2E(z)A = \frac{\sigma A}{\epsilon_0}
.\]
Thus,
\[
\mathbf{E} =
\begin{cases}
	\frac{\sigma}{2 \epsilon_0} \mathbf{e}_z & z > d, \\
	-\frac{\sigma}{2 \epsilon_0} \mathbf{e}_z & $z < -d$.
\end{cases}
\]
In the limit $d \to 0$, we obtain the electric field of a \emph{surface charge}\index{surface charge} $\sigma$ per unit area, corresponding to
\[
\rho = \sigma \delta(z)
.\]

\subsubsection{Surface Charge and Discontinuity}
\label{subsub:surface_charge_and_discontinuity}

Let $\mathbf{n}$ be a unit vector normal to the charged surface, pointing from region 1 to region 2. In our example, $\mathbf{n} = \mathbf{e}_z$.

The discontinuity in $\mathbf{E}$ is given by
\[
	[\mathbf{n} \cdot \mathbf{E}] = \frac{\sigma}{\epsilon_0}
,\]
where $\sigma$ is the surface charge density, and
\[
	[X] = X_2 - X_1
\]
denotes a discontinuity. The tangential components are continuous (they are both 0), so
\[
	[\mathbf{n} \times \mathbf{E}] = \mathbf{0}
.\]

\newpage

\printindex

\end{document}
