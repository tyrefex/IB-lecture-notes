%%% TO DO %%%

\documentclass[12pt]{article}

\usepackage{ishn}

\makeindex[intoc]

\begin{document}

\hypersetup{pageanchor=false}
\begin{titlepage}
	\begin{center}
		\vspace*{1em}
		\Huge
		\textbf{IB Complex Analysis}

		\vspace{1em}
		\large
		Ishan Nath, Lent 2023

		\vspace{1.5em}

		\Large

		Based on Lectures by Prof. Holly Krieger

		\vspace{1em}

		\large
		\today
	\end{center}
	
\end{titlepage}
\hypersetup{pageanchor=true}

\tableofcontents

\newpage

\section{Complex Differentiation}
\label{sec:complex_differentiation}

Our goal in this course is to study the theory of complex-valued differentiable functions in one complex variable. Example include:
\begin{itemize}
	\item Polynomials $p(z) = a_dz^{d} + \cdots + a_1 z + a_0$, with coefficients in $\mathbb{R}, \mathbb{Q}, \mathbb{Z}$ or $\mathbb{C}$.
	\item The infinite series
		\[
		\sum_{n = 1}^{\infty} \frac{1}{n^{z}}
		,\]
		which we showed convergence for $z$ having real part greater than $1$.
	\item Harmonic functions\index{harmonic function} $u(x, y) : \mathbb{R}^2 \to \mathbb{R}$, $u_{xx} + u_{yy} = 0$.
\end{itemize}

In this course, we make the convention that $\theta = \arg(z) \in [0, 2\pi)$.

\subsection{Basic Notions}
\label{sub:basic_notions}

\begin{itemize}
	\item $U \subset \mathbb{C}$ is \emph{open}\index{open} if for all $u \in U$, there exists $\eps > 0$ such that
		\[
			D(x, \eps) = \{z \in \mathbb{C} \mid |z - u| < \eps\} \subset U
		.\]
	\item A \emph{path}\index{path} in $U \subset \mathbb{C}$ is a continuous map $\gamma : [a, b] \to U$. We say the path is $C^{1}$ if $\gamma'$ exists and is continuous (we take one-sided derivatives at the endpoints).

		$\gamma$ is \emph{simple}\index{simple} if it is injective.
	\item $U \subset \mathbb{C}$ is \emph{path-connected}\index{path-connected} if for all $z, w \in U$, there exists a path in $U$ with endpoints at $z, w$.
\end{itemize}
\begin{remark}
	If $U$ is open, and $z, w \in U$ are connected by a path $\gamma$ in $U$, then there exists a path $\gamma$ in $U$ connected $z, w$ consisting of finitely many horizontal and vertical segments.
\end{remark}

\begin{definition}
	A \emph{domain}\index{domain} is a non-empty, open, path-connected subset of $\mathbb{C}$.
\end{definition}

\begin{definition}
	\begin{enumerate}[(i)]
		\item[]
		\item $f : U \to \mathbb{C}$ is \emph{differentiable}\index{complex differentiable} at $u \in U$ if
			\[
			f'(u) = \lim_{z \to u} \frac{f(z) - f(u)}{z - u}
			\]
			exists.
		\item $f : U \to \mathbb{C}$ is \emph{holomorphic}\index{holomorphic} at $u \in U$ if there exists $\eps > 0$ such that $f$ is differentiable at $z$, for all $z \in D(u, \eps)$. We may also call such a function \emph{analytic}\index{analytic}.
		\item $f : \mathbb{C} \to \mathbb{C}$ is \emph{entire} if it is holomorphic everywhere.
	\end{enumerate}
\end{definition}

\begin{remark}
	All differentiation rules (sum, products, ...) in $\mathbb{R}$ hold, by the same proofs.
\end{remark}

Identifying $\mathbb{C}$ with $\mathbb{R}^2$, we may write $f : U \to \mathbb{C}$ as $f(x+iy) = u(x,y) + iv(x, y)$, where $u, v$ are the real and imaginary parts of $f$.

From analysis and topology, recall that $u : U \to \mathbb{R}$ as a function of two real variables if $(\mathbb{R}^2)$ differentiable at $(c, d) \in \mathbb{R}^2$with $Du|_{(c,d)} = (\lambda, \mu)$ if
\[
	\frac{u(x,y) - u(c,d) - [\lambda(x-c) + \mu(y-d)]}{\sqrt{(x-c)^2 + (y-d)^2}} \to 0
,\]
as $(x, y) \to (c, d)$. However, \textbf{this is a weaker condition} than differentiability over $\mathbb{C}$.

\begin{proposition}[Cauchy-Riemann equations]\index{Cauchy-Riemann equations}
	Let $f : U \to \mathbb{C}$ on an open set $U \subset \mathbb{C}$. Then $f$ is differentiable at $w = c+id \in U$ if and only if, writing $f = u + iv$, we have $u, v$ are $\mathbb{R}^2$-differentiable at $(c, d)$, and
	\begin{align*}
		u_{x} &= v_y, & u_y &= -v_x.
\end{align*}
\end{proposition}

\begin{proofbox}
	$f$ is differentiable at $w$ if and only if $f'(w) = p + iq$ exists, so
	\[
	\lim_{z \to w}\frac{f(z) - f(w) - (z - w)(p + iq)}{|z - w| = 0}
	.\]
	Writing $f = u + iv$ and considering the real and imaginary parts in the quotient above, this holds if and only if
	\[
		\lim_{(x, y) \to (c, d)} \frac{u(x, y) - u(c, d) - [p(x-c) - q(y-d)]}{\sqrt{(x - c)^2 + (y - d)^2}} = 0
	,\]
	and
	\[
		\lim_{(x, y) \to (c, d)} \frac{v(x, y) - v(c, d) - [q(x-c) + p(y-d)]}{\sqrt{(x-c)^2+(y-d)^2}} = 0
	.\]
	This holds if and only if $u, v$ are $\mathbb{R}^2$-differentiable at $(c, d)$, and $u_x = v_y$, $u_y = -v_x$.
\end{proofbox}

\begin{remark}
	If the partial $u_x, u_y, v_x, v_y$ exist and are continuous on $U$, then $u, v$ are differentiable on $U$. So it suffices to check the partials exist and are continuous, and the Cauchy-Riemann equations hold to deduce complex differentiability.
\end{remark}

\begin{exbox}
	\begin{enumerate}
		\item Take $f(z) = \overline{z}$. Then $f$ has $u(x, y) = x$ and $v(x, y) = -y$, so $u_x = 1$, $v_y = -1$. So $f(z) = \overline{z}$ is not holomorphic or differentiable anywhere.
		\item Any polynomial $p(z) = a_d z^{d} + \cdots + a_1z + a_0$, with $a_i \in \mathbb{C}$ is entire.
		\item Rational function\index{rational functions}, which are quotients of polynomials $\frac{p(z)}{q(z)}$ are holomorphic on the open set $\mathbb{C}\setminus\{\text{zeroes of } q\}$.
	\end{enumerate}
\end{exbox}

Note that $f = u + iv$ satisfying the Cauchy-Riemann equations at a point does not mean it is differentiable at that point.

Some proofs in regular analysis have natural extensions to complex analysis. For example, if $f : U \to \mathbb{C}$ on a domain $U$ with $f'(z) = 0$ on $U$, then $f$ is constant on $U$.

Now we ask: why are we interested in complex analysis?
\begin{itemize}
	\item Unlike $\mathbb{R}^2$ differentiable functions, holomorphics functions are very constrained. For example, if $f$ is entire and bounded (so $|f(z)| < M$ for all $z \in \mathbb{C}$), then $f$ is constant. Contrast with $\sin$, for example.
	\item We will see that $f$ holomorphic on a domain $U$ has holomorphic derivative on $U$. This implies that $f$ is infinitely differentiable, as are $u$ and $v$.
\end{itemize}
In particular, we can differentiate the Cauchy-Riemann equations to get
\[
u_{xx} = v_{yx} = v_{xy} = -u_{yy}
,\]
so $u_{xx} + u_{yy} = 0$, and similarly $v_{xx} + v_{yy} = 0$. Hence the real and imaginary parts of a holomorphic function are harmonic.

Let $f : U \to \mathbb{C}$ be a holomorphic function on an open set $U_1$ and $w \in U$ with $f'f(w) \neq 0$. We want to look at the geometric behaviour of $f$ at $w$.

In fact, we claim $f$ is \emph{conformal}\index{conformal} at $w$. Let $\gamma_1, \gamma_2$ be $C^{1}$-paths through $w$, say $\gamma_1, \gamma_2 : [-1,1] \to U_1$, such that $\gamma_1(0) = \gamma_2(0) = w$, and $\gamma_i'(0) \neq 0$. If we write $\gamma_j(t) = w + r_j(t) = e^{i \theta_j(t)}$, then we have
\[
\arg(\gamma_j'(z)) = \theta_j(0)
,\]
and the argument of the image line is
\[
\arg((f \circ \gamma_j)'(0)) = \arg(\gamma_j'(0) f'(\gamma_j(0))) = \arg(\gamma_j'(0)) + \arg(f'(w)) + 2 \pi n
,\]
where crucially we use $\gamma_j'(0) f'(\gamma_j(0)) \neq 0$, so the direction of $\gamma_j$ at $w$ under the application of $f$ is rotated by $\arg(f'(w))$. This is independent of $\gamma_j$. Since the angle between $\gamma_1$ and $\gamma_2$ is the difference of the arguments $f$ preserves the angle. This is what it means to be conformal.

\begin{definition}
	Let $U, V$ be domains in $\mathbb{C}$. A map $f : U \to V$ is a \emph{conformal equivalence}\index{conformal equivalence} of $U$ and $V$ if $f$ is a bijective holomorphic map with $f'(z) \neq 0$, for all $z \in U$.
\end{definition}

\begin{remark}
	\begin{enumerate}[1.]
		\item[]
		\item Using the real inverse function theorem, one can show if $f : U \to V$ is a holomorphic bijection of open sets with $f'(z) \neq 0$ for all $z \in U$, then the inverse of $f$ is also holomorphic, so also conformal by the chain rule. So conformally equivalent domains are equal from the perspective of the functions $f$.
		\item We will later see than being injective and holomorphic on a domain implies $f'(z) \neq 0$ for all $z \in U$, so this requirement is redundant.
	\end{enumerate}	
\end{remark}

\begin{exbox}
	\begin{enumerate}[1.]
		\item Any change of coordinates: on $\mathbb{C}$, take $f(z) = az + b$, for $a \neq 0$ and $b$, which is a conformal equivalence $\mathbb{C} \to \mathbb{C}$. More generally, a M\"{o}bius map
			\[
			f(z) = \frac{az + b}{cz + d},
			\]
			for $ad - bc \neq 0$, is a conformal equivalence from the Riemann sphere to itself. This can eb seen as adding a point at infinity to make a sphere $\mathbb{C}_{\infty}$ (or gluing two copies of the unit disc with coordinates $z$ and $\frac{1}{z}$).

			If $f : \mathbb{C}_{\infty} \to \mathbb{C}_{\infty}$ is continuous, then
			\begin{itemize}
				\item if $f(\infty) = \infty$, then $f$ is holomorphic at $\infty$ if and only if $g(z) = \frac{1}{f(\frac{1}{z})}$ is holomorphic at $0$.
				\item If $f(\infty) \neq \infty$, then $f$ is homolorphic at $\infty$ if and only if $f(\frac{1}{z})$ is holomorphic at $0$.
				\item If $f(a) = \infty$ for $a \in \mathbb{C}$, then $f$ is holomorphic at $a$ if and only if $\frac{1}{f(z)}$ is holomorphic at $a$.
			\end{itemize}
			We can then think of M\"{o}bius maps as change of coordinates for the sphere.

			Choosing $z_1 \to 0$, $z_2 \to \infty$, $z_3 \to 1$ defined a M\"{o}bius map
			\[
			f(z) = \frac{z - z_1}{z - z_2} \cdot \frac{z_3 - z_2}{z_3 - z_1}
			,\]
			for distinct $z_1, z_2, z_3 \in \mathbb{C}$.
		\item For $n \in \mathbb{N}$, $f(z) = z^{n}$ is a conformal equivalence from the sector $\{z \in \mathbb{C}^{\times} \mid 0 < \arg z < \frac{\pi}{n}\}$ to the upper half plane $\mathbb{H} = \{z \in \mathbb{C} \mid \Im z > 0\}$.
		\item The M\"{o}bius map $f(z) = \frac{z-i}{z+i}$ is a conformal equivalence between $\mathbb{H}$ and $D(0,1)$. We can compute $f'(z) \neq 0$ on $\mathbb{H}$, and
			\begin{align*}
				z \in \mathbb{H} & \iff |z-i| < |z+i| \iff |f(z)) < 1.
			\end{align*}
			Note that $f^{-1}(w) = -i \frac{w+1}{w-1}$.
		\item We can use these examples to write down conformal equivalences. Let $U_1$ be the upper half semicircle, and $U_2$ the lower half plane. Considering $g(z) = \frac{z+1}{z-1}$, we know that sends $D(0,1)$ to the left half-plane, so it sends $U_1$ to the upper left quadrant.

			Then, the upper left quadrant if mapped by the squaring map to $U_2$. So $f(z) = (\frac{z+1}{z-1})^2$ is a conformal equivalence from $U_1 \to U_2$.
	\end{enumerate}
\end{exbox}

These are all examples of the deep \emph{Riemann mapping theorem}\index{Riemann mapping theorem}:

\begin{theorem}[Riemann mapping theorem]
	Let $U \subset \mathbb{C}$ be a proper domain which is simply connected. Then there exists a conformal equivalence between $U$ and $D(0,1)$.
\end{theorem}

Here, \emph{simply connected}\index{simply connected} means a subset $U \subset \mathbb{C}$ which is path-connected, and contractible: any loop in $U$ can be contracted to a point. So any continuous path $\gamma : S^{1} \to U$ extends to a continuous map $\hat \gamma : D(0,1) \to U_1$ with $\hat \gamma|_{S_1} = \gamma$.

In fact any domain bounded by a simple closed curve is simply connected, so all of these are conformally equivalent to $D(0,1)$.

\begin{exbox}
	We look at a domains in the Riemann sphere, with bounded and connected complement. This is simply connected as a subset of $\mathbb{C}_{\infty}$.

	Now, the Mandelbrot set is bounded and connected, so the complement of the Mandelbrot set is simply connected in $\mathbb{C}_{\infty}$.
\end{exbox}

Recall the following facts about functions defined by power series, or sequences of functions:

\begin{enumerate}[1.]
	\item A sequence $(f_n)$ of functions \emph{converges uniformly}\index{uniform convergence} to a function $f$ on some set $S$ if for all $\eps > 0$, there exists $N \in \mathbb{N}$ such that for all $n \geq N$ and for all $x \in S$,
		\[
		|f_n(x) - f(x)| < \eps
		.\]
	\item The uniform limit of continuous functions is continuous.
	\item The \emph{Weierstrass M-test}\index{Weierstrass M-test}: if there exists $M_n \in \mathbb{R}$ for all $n$ such that $0 \leq |f_n(x)| \leq M_n$ for all $x \in S$, then
		\[
			\sum_{n = 1}^{\infty}M_n < \infty \implies \sum_{n = 1}^{\infty}f_n(x) \text{ converges uniformly on $S$ as $N \to \infty$}
		.\]
	\item Let $(c_n)$ be complex numbers, and fix $a \in \mathbb{C}$. Then there exists unique $R \in [0, \infty]$ such that the function
		\[
		z \mapsto \sum_{n = 1}^{\infty}c_n (z - a)^{n}
		\]
		converges absolutely if $|z - a| < R$, and diverges if $|z-a| > R$. If $0 < r < R$, then the series converges uniformly in $D(a, r)$. $R$ is the \emph{radius of convergence}\index{radius of convergence} of the series. We can compute
		\[
			R = \sup\{r \geq 0 \mid |c_n|r^{n} \to 0\}
		,\]
		or
		\[
		R = \frac{1}{\lambda}, \qquad \lambda = \limsup_{n \to \infty} |c_n|^{1/n}
		.\]
\end{enumerate}

\begin{theorem} 
	\[
	f(z) = \sum_{n = 0}^{\infty} c_n(z - a)^{n}
	\]
	is a complex power series with radius of convergence $R$. Then,
	\begin{enumerate}[\normalfont(i)]
		\item $f$ is holomorphic on $D(a, R)$.
		\item $f$ has derivative
			\[
			f'(z) = \sum_{n = 1}^{\infty} n c_n (z - a)^{n-1}
			,\]
			with radius of convergence $R$ about $a$.
		\item $f$ has derivatives of all orders on $D(a, R)$, and $f^{(n)}(a) = n! c_n$.
	\end{enumerate}
\end{theorem}

\begin{proofbox}
	We can let $a = 0$ by change of variables $z \to z - a$. Consider the series
	\[
	\sum_{n = 1}^{\infty}n c_n z^{n-1}
	.\]
	Since $|n c_n| \geq |c_n|$, the radius of convergence of this series is no larger than $R$. If $0 < R_1 < R$, then for $|z| < R_1$, we have
	\[
	|n c_n z^{n-1}| = n |c_n| R_1^{n-1} \frac{|z|^{n-1}}{R_1^{n-1}}
	,\]
	and
	\[
	n \biggl( \frac{|z|}{R_1} \biggr)^{n-1} \to 0
	.\]
	Applying the M-test with $M_n = c_n R_1^{n-1}$, we have the convergence of the series. So the series has radius of convergence $R$.

	Now for $|z|, |w| < R$, we need to consider
	\[
	\frac{f(z) - f(w)}{z - w}
	.\]
	Taking the partial sums,
	\[
		\sum_{n = 0}^{N} c_n \frac{z^{n} - w^{n}}{z - w} = \sum_{n = 0}^{N} c_n \Biggl( \sum_{j = 0}^{n-1} z^{j}w^{n-1-j}\Biggr)
	.\]
	For $|z|, |w| < \rho < R$, we have
	\[
	\biggl| c_n \Biggl( \sum_{j = 0}^{n-1} z^{j} w^{n-1-j} \Biggr) \biggr| \leq |c_n| n \rho^{n-1}
	.\]
	Hence the partial sums converge uniformly on $\{(z, w) \mid |z|, |w| < \rho\}$. So the series converges to a continuous limit on $\{|z|, |w| < R\}$, say $g(z, w)$. When $z \neq w$, we know
	\[
	g(z, w) = \frac{f(z) - f(w)}{z - w}
	.\]
	When $z = w$, we have
	\[
	g(w, w) = \sum_{n = 0}^{\infty} n c_n w^{n-1}
	.\]
	Hence by the continuity of $g$, this proves (i) and (ii). Then (iii) follows from a simple induction.
\end{proofbox}

\begin{corollary}
	Suppose $0 < \rho < R$, where $R$ is the radius of convergence of the complex power series
	\[
	f(z) = \sum_{n = 0}^{\infty}c_n (z - a)^{n}
	,\]
	and $f(z) = 0$ for all $z \in D(a, \rho)$. Then $f \equiv 0$ on $D(a, R)$.
\end{corollary}

\begin{proofbox}
	Since $f \equiv 0$ on $D(a, \rho)$, we have $f^{(n)}(a) = 0$ for all $n$. Hence $c_n = 0$ for all $n$, so $f \equiv 0$ on $D(a, R)$.
\end{proofbox}

\subsection{Exponential and Logarithm}
\label{sub:exponential_and_logarithm}

We define the complex exponential
\[
e^{z} = \exp(z) = \sum_{n = 0}^{\infty} \frac{z^{n}}{n!}
.\]

The complex exponential has the following properties:
\begin{enumerate}[1.]
	\item It has radius of convergence $\infty$, so the function is entire, and we have $\frac{\diff}{\diff z} e^{z} = e^{z}$.
	\item For all $z, w \in \mathbb{C}$, $e^{z+w} = e^{z} e^{w}$, and $e^{z} \neq 0$.

		This follows from setting $F(z) = e^{z+w}e^{-z}$, then taking the derivative,
		\[
		F'(z) = e^{z+w}e^{-z} - e^{z+w}e^{-z} = 0
		,\]
		so $F$ is constant. Since $e^{0} = 1$, $F(z) = e^{w}$, and $e^{z+w} = e^{z}e^{w}$. Since $e^{z}e^{-z} = e^{0} = 1$, $e^{z} \neq 0$.
	\item Let $z = x + iy$. Then $e^{z} = e^{x + iy} = e^{x}e^{iy}$. But $e^{iy} = \cos y + i \sin y$, and note that $|e^{iy}| = 1$, so
		\[
		e^{z} = e^{x} (\cos y + i \sin y)
		,\]
		and $|e^{z}| = e^{x}$, so $e^{z} = 1$ if and only if $x = 0$ and $y = 2 \pi k$ for $k \in \mathbb{Z}$. In fact, for all $w \in \mathbb{C}^{\times}$, there exist infinitely many $z \in \mathbb{C}$ such that $e^{z} = w$, differing by integer multiples of $2 \pi i$.
\end{enumerate}

\begin{definition}
	Let $U \subset \mathbb{C}^{\times}$ be an open set. We say a continuous function $\lambda : U \to \mathbb{C}$ is a \emph{branch of the logarithm}\index{branch of the logarithm} if for all $z \in U$, $\exp(\lambda(z)) = z$.
\end{definition}

\begin{exbox}
	Let $U = \mathbb{C} \setminus \mathbb{R}_{\leq 0}$. Define $\log : U \to \mathbb{C}$ by
	\[
	\log(z) = \ln|z| + i \theta
	,\]
	where $\theta = \arg (z)$, and $\theta \in (-\pi, \pi)$. This is the \emph{principal branch of the logarithm}\index{principal branch of the logarithm}.
\end{exbox}

\begin{proposition}
	$\log(z)$ is holomorphic on $\mathbb{C}\setminus \mathbb{R}_{\leq 0}$ with derivative $\frac{1}{z}$. Moreover, if $|z| < 1$, then
	\[
	\log(1+z) = \sum_{n = 1}^{\infty} \frac{(-1)^{n-1} z^{n}}{n}
	.\]
\end{proposition}

\begin{proofbox}
	As an inverse to $e^{z}$ and by the chain rule, we have $\log z$ is holomorphic with $\frac{\diff}{\diff z}\log z = \frac{1}{z}$, We have
	\[
	\frac{\diff}{\diff z} \log (1 + z) = \frac{1}{z+1} = 1 - z + z^2 - z^3 + z^{4} - \cdots
	,\]
	which is the derivative of
	\[
	\sum_{n = 1}^{\infty} \frac{(-1)^{n-1} z^{n}}{n}
	.\]
	So $\log(1+z)$ agrees with this series up to a constant. Since $\log(1) = 0$, the equality holds.
\end{proofbox}

If $\alpha \in \mathbb{C}$, we can define $z^{\alpha} = \exp(\alpha \log z)$. This gives a definition of $z^{\alpha}$ on $\mathbb{C}\setminus \mathbb{R}_{\leq 0}$. We can compute that $\frac{\diff}{\diff z} z^{\alpha} = \alpha z^{\alpha - 1}$.

It is not necessarily true that $z^{\alpha}w^{\alpha} = (zw)^{\alpha}$. Take $\alpha = \frac{1}{2}$, then
\[
z^{1/2} = \exp \biggl( \frac{1}{2} \log z \biggr) = \exp \biggl( \frac{1}{2} \ln|z| + \frac{1}{2} i \theta \biggr)
,\]
for $\theta \in (-\pi, \pi)$. Hence the argument of $z^{1/2}$ is in $(-\frac{\pi}{2}, \frac{\pi}{2})$.

\subsection{Contour Integration}
\label{sub:contour_integration}

If $f : [a, b] \to \mathbb{C}$ is continuous, we define
\[
\int_{a}^{b} f(t) \diff t = \int_{a}^{b} \Re(f(t)) \diff t + i \int_{a}^{b} \Im(f(t)) \diff t
.\]

\begin{proposition}
	Let $f : [a, b] \to \mathbb{C}$ be continuous. Then,
	\[
	\biggl| \int_{a}^{b} f(t) \diff t \biggr| \leq (b - a)\sup_{a \leq t \leq b} |f(t)|
	,\]
	with equality if and only if $f$ is constant.
\end{proposition}

\begin{proofbox}
	Write $M = \sup_{a \leq t \leq b}|f(t)|$, and $\theta = \arg(\int_{a}^{b} f(t) \diff t)$. Then
	\begin{align*}
		\biggl| \int_{a}^{b}f(t) \diff t \biggr| &= e^{-i \theta} \int_{a}^{b} f(t) \diff t = \int_{a}^{b} e^{-i \theta} f(t) \diff t \\
							 &= \int_{a}^{b} \Re(e^{-i \theta}f(t)) \diff t \\
							 &\leq \int_{a}^{b} |f(t)| \diff t \leq M(b - a).
	\end{align*}
	If we have equality, then $|f(t)| = M$, and $\arg f(t) = \theta$, so $f$ is constant.
\end{proofbox}

\begin{definition}
	Let $\gamma : [a, b] \to \mathbb{C}$ be a $C^{1}$-smooth curve. Then we define the arc-length of $\gamma$ to be
	\[
		\rm{length}(\gamma) = \int_{a}^{b} |\gamma'(t)| \diff t
	.\]
	We say $\gamma$ is \emph{simple}\index{simple} if $\gamma(t_1) = \gamma(t_2) \iff t_1 = t_2$ or $\{t_1, t_2\} = \{a, b\}$. If $\gamma$ is simple, then $\rm{length}(\gamma)$ is the length of the image of $\gamma$.
\end{definition}

\begin{definition}
	Let $f : U \to \mathbb{C}$ be continuous, with $U$ open, and $\gamma : [a, b] \to U$ be a $C^{1}$-smooth curve. Then the integral of $f$ along $\gamma$ is
	\[
	\int_{\gamma} f(z) \diff z = \int_{a}^{b} f(\gamma(t)) \gamma'(t) \diff t
	.\]
\end{definition}

This integral satisfies the following properties:
\begin{enumerate}[1.]
	\item Linearity:
		\[
		\int_{\gamma} c_1 f_1 + c_2 f_2 \diff z = c_1 \int_{\gamma}f_1 \diff z + c_2 \int_{\gamma} f_2 \diff z
		.\]
	\item Additivity: if $a < a' < b$, then
		\[
			\int_{\gamma|_{[a,a']}}f(z) \diff z + \int_{\gamma|_{[a',b]}}f(z) \diff z = \int_{\gamma} f(z) \diff z
		.\]
	\item Inverse path: if $(- \gamma)(t) = \gamma(-t)$ on $[-b, -a]$, then
		\[
		\int_{-\gamma}f(z) \diff z = - \int_{\gamma} f(z) \diff z
		.\]
	\item Independence of parametrization: if $\phi : [a', b'] \to [a, b]$ is $C^{1}$-smooth with $\phi(a') = a$, $\phi(b') = b$ and $\delta = \gamma \circ \phi$, then
		\[
		\int_{\delta}f(z) \diff z = \int_{\gamma} f(z) \diff z
		.\]
		This lets us assume that $\gamma : [0, 1] \to U$.
\end{enumerate}

We can loosen the restriction that $\gamma$ is $C^{1}$-smooth and allow it to be piecewise $C^{1}$-smooth, i.e. there exist $a = a_0 < a_1 < \cdots < a_n = b$ such that $\gamma_i = \gamma|_{[a_{i-1},a_i]}$ is $C^{1}$-smooth. Define then
\[
\int_{\gamma}f(z) \diff z = \sum_{i = 1}^{n} \int_{\gamma_i}f(z) \diff z
.\]

\begin{remark}
	Any piecewise $C^{1}$-smooth curve can be reparametrized to be $C^{1}$: for such a $\gamma$ as above, replace $\gamma_i$ by $\gamma_i \circ h_i$ where $h_i$ is monotonic $C^{1}$-smooth bijection with endpoint derivative $0$.

	So $C^{1}$-smooth paths can have corners, for example
	\[
	\gamma(t) =
	\begin{cases}
		1 + i \sin (\pi t) & t \in [0, \frac{1}{2}], \\
		\sin(\pi t) + i & t \in [\frac{1}{2}, 1].
	\end{cases}
	\]
\end{remark}

We say a ``curve'' is a piecewise $C^{1}$-smooth path, and a ``contour'' is a simple \emph{closed} piecewise $C^{1}$-smooth path, where closed means the endpoints are equal.

\begin{proposition}
	For any continuous $f : U \to \mathbb{C}$ with $U$ open, and any curve $\gamma : [a, b] \to U$,
	\[
		\biggl| \int_{\gamma}f(z) \diff z \biggr| \leq \rm{length}(\gamma) \sup_{z \in \gamma}|f(z)|
	.\]
\end{proposition}

\begin{proofbox}
	
	\begin{align*}
		\biggl| \int_{\gamma}f(z) \diff z \biggr| &= \biggl| \int_{a}^{b} f(\gamma(t)) \gamma'(t) \diff t \biggr| \\
							  &\leq \int_{a}^{b} |f(\gamma(t)) \gamma'(t)| \diff t \\
							  &\leq \sup_{z \in \gamma}|f(z)| \rm{length}(\gamma).
	\end{align*}
\end{proofbox}

\begin{proposition}
	If $f_n : U \to \mathbb{C}$ for $n \in \mathbb{N}$ and $f : U \to \mathbb{C}$ are continuous, and $\gamma : [a, b] \to U$ is a curve in $U$ with $f_n \to f$ uniformly on $\gamma$, then
	\[
	\int_{\gamma}f_n(z) \diff z \to \int_{\gamma}f(z) \diff z
	,\]
	as $n \to \infty$.
\end{proposition}

\begin{proofbox}
	By uniform convergence, $\sup_{z \in \gamma}|f(z) - f_n(z)| \to 0$ as $n \to \infty$. So by the previous proposition,
	\begin{align*}
		\biggl| \int_{\gamma}f(z) \diff z - \int_{\gamma}f_n(z) \diff z \biggr| &\leq \rm{length}(\gamma) \sup_{\gamma}|f - f_n| \\
											&\to 0
	\end{align*}
	as $n \to \infty$.
\end{proofbox}

\begin{exbox}
	Let $f_n(z) = z^{n}$ for $n \in \mathbb{Z}$ on $C^{\times} = U$, and $\gamma : [0, 2\pi] \to U$ with $\gamma(t) = e^{it}$. Then,
	\[
	\int_{\gamma}f_n(z) \diff z = \int_{0}^{2\pi} e^{n i t} i e^{it} \diff t = i \int_{0}^{2\pi} e^{(n+1)it} \diff t =
	\begin{cases}
		2 \pi i & n = -1, \\
		0 & n \neq -1.
	\end{cases}
	\]
\end{exbox}

\begin{theorem}[Fundamental Theorem of Calculus]\index{fundamental theorem of calculus}
	If $f : U \to \mathbb{C}$ is a continuous function on open $U \subset \mathbb{C}$ with $F' = f$ an antiderivative of $f$ in $U$, then for any curve $\gamma : [a, b] \to U$,
	\[
	\int_{\gamma}f(z) \diff z = F(\gamma(b)) - F(\gamma(a))
	.\]
	In particular, if $\gamma$ is closed then $\int_{\gamma} f = 0$.
\end{theorem}

\begin{proofbox}
	\[
	\int_{\gamma}f(z) \diff z = \int_{a}^{b} f(\gamma'(t)) \gamma'(t) \diff t = \int_{a}^{b} (F \circ \gamma)'(t) \diff t = F(\gamma(b)) - F(\gamma(a))
	.\]
\end{proofbox}

Note that in the $z \mapsto z^{-1}$ integral computation, from the fundamental theorem of calculus there does not exist a branch of the logarithm on any neighbourhood around $0$.

The surprising thing is that the converse of this is true.

\begin{theorem}
	Let $f : D \to C$ be continuous on a domain $D$. If $\int_{\gamma}f = 0$ for all closed curves $\gamma$ in $D$, then there exists a holomorphic $F : D \to \mathbb{C}$ with $F' = f$.
\end{theorem}

\begin{proofbox}
	Fix $a \in D$. If $w \in D$, choose any curve $\gamma_w : [0,1] \to D$ with $\gamma_w(0) = a$, $\gamma_w(1) = w$. Define
	\[
	F(w) = \int_{\gamma_w}f(z) \diff z
	.\]

	Find $r_w > 0$ such that $D(w, r_w) \subset D$. For $|h| < r$, let $\delta_h : [0,1] \to D$ be the line segment from $w$ to $w+h$. Then,
	\[
	F(w+h) = \int_{\gamma_{w+h}}f(z) \diff z = \int_{\gamma_w + \delta_h} f(z) \diff z
	.\]
	So
	\[
		F(w+h) = F(w) + \int_{\delta_h}f(z) \diff z = F(w) + hf(w) + \int_{\delta_h}f(z) - f(w) \diff z
	.\]
	Hence
	\begin{align*}
		\biggl| \frac{F(w+h) - F(w)}{h} - f(w) \biggr| &= \biggl|\frac{1}{h} \int_{\delta_h} f(z) - f(w) \diff z \biggr| \\
							       &\leq \frac{\mathrm{length}(\delta_h)}{|h|} \sup_{\delta_h}|f(z) - f(w)| \\
							       &\leq \sup_{z \in D(w, r_w)} |f(z) - f(w)| \to 0,
	\end{align*}
	as $r_w \to 0$. So $F'(w) = f(w)$.
\end{proofbox}

\begin{definition}
	An open subset $U \subset \mathbb{C}$ is \emph{convex}\index{convex} if for all $a, b \in U$, the line segment between $a$ and $b$ is in $U$. $U$ is \emph{starlike}\index{starlike} (or starshaped) if there exists $a \in U$ such that for all $b \in U$, the line segment from $a$ to $b$ is in $U$.
\end{definition}

Note that disks are a subset of convex sets, which are a subset of starlike sets, which are a subset of domains.

We can simplify the previous theorem as follows:

\begin{lemma}
	Suppose $U$ is a starlike domain, and $f : U \to \mathbb{C}$ is continuous with $\int_{\partial T} f(z) \diff z = 0$ for all triangles $T$ in $U$. Then, $f$ has an antiderivative in $U$.
\end{lemma}

\begin{proofbox}
	This is exactly the same as the previous proof, except we stipulate $\gamma_w$ are straight lines from a basepoint $a$.
\end{proofbox}

\begin{theorem}[Cauchy's theorem for Triangles]
	If $f : U \to \mathbb{C}$ is holomorphic on open $U \subset \mathbb{C}$, and $T \subset U$ is a triangle in $U$, then
	\[
	\int_{\partial T}f(z) \diff z = 0
	.\]
\end{theorem}

We adopt the notion that curves are oriented anticlockwise.

\begin{proofbox}
	We can name
	\[
		\biggl| \int_{\partial T}f(z) \diff z \biggr| = I, \qquad L = \mathrm{length}(\partial T)
	.\]
	We subdivide $T$ by bisecting the sides, to obtain $T_1, T_2, T_3$ and $T_4$. Hence, since
	\[
	\partial T_1 + \partial T_2 + \partial T_3 = \partial T - \partial T_4
	,\]
	we find
	\[
	\int_{\partial T}f(z) \diff z = \sum_{i = 1}^{4} \int_{\partial T_i} f(z) \diff z
	.\]
	By the triangle inequality, there exists $i \in \{1, 2, 3, 4\}$ such that 
	\[
	\biggl| \int_{\partial T_i} f(z) \diff z \biggr| \geq \frac{1}{4} I
	.\]
	Call this triangle $T^{(1)}$ and $\mathrm{length}(\partial T^{(1)}) = \frac{L}{2}$.

	Continuing this way, we get
	\[
		T \supset T^{(1)} \supset T^{(2)} \supset T^{(3)} \supset \cdots
	\]
	These triangles have $\mathrm{length}(T^{(n)}) = \frac{L}{2^{n}} \to 0$, and
	\[
	\biggl| \int_{\partial T^{(n)}} f(z) \diff z \biggr| \geq \frac{1}{4^{n}}I
	.\]
	Since the lengths tend to $0$, we get
	\[
		\bigcap_{n = 1}^{\infty}T^{(n)} = \{w\}
	,\]
	a single point. Note that $z, 1$ have holomorphic derivatives. Hence we can bound
	\[
	\frac{1}{4^{n}}I \leq \biggl| \int_{\partial T^{(n)}} f(z) \diff z \biggr| = \biggl| \int_{\partial T^{(n)}} f(z) - f(w) - (z - w)f'(w) \diff z \biggr|
	.\]
	Since $f$ is differentiable at $w$, there $\delta > 0$ such that for all $\eps > 0$,
	\[
	|w - z| < \delta \implies |f(z) - f(w) - (z - w)f'(w)| < \eps|z - w|
	.\]
	So for $n \gg 1$, we have
	\[
	\biggl| \int_{\partial T^{(n)}} f(z) - f(w) - (z-w)f'(w) \diff z \biggr| \leq \frac{L}{2^{n}} \sup_{z \in \partial T^{(n)}}|z - w| \cdot \eps
	.\]
	So
	\[
	\frac{I}{4^{n}} \leq \frac{L}{2^{n}} \cdot \frac{L}{2^{n}} \eps,\qquad I \leq L^2 \eps
	.\]
	Letting $\eps \to 0$, we get $I = 0$.
\end{proofbox}

\begin{theorem}
	Let $S \subset U$ be a finite set and $f : U \to \mathbb{C}$ be continuous on $U$ and holomorphic on $U \setminus S$. Then $\int_{\partial T} f = 0$ for all triangles $T \in U$.
\end{theorem}

\begin{proofbox}
	Using the triangle subdivision, assume that $S = \{a\}$, for $a \in T$. If $a \in T' \subset T$ for another triangle $T'$, then by the triangular subdivision and the previous theorem,
	\[
	\int_{\partial T} f = \int_{\partial T'} f
	,\]
	since $f$ is holomorphic on $T \setminus T'$. Hence,
	\begin{align*}
		\biggl| \int_{\partial T}f(z) \diff z \biggr| &= \biggl| \int_{\partial T'} f(z) \diff z \biggr| \leq \mathrm{length}(T') \sup_{\partial T'}|f| \\
							      &\leq \mathrm{length}(T') \sup_{T}|f|,
	\end{align*}
	so letting $\mathrm{length}(T') \to 0$, we have $\int_{\partial T} f = 0$.
\end{proofbox}

\begin{theorem}[Cauchy's theorem in a Disk]\index{Cauchy's theorem}
	Let $D$ be any disk (or any starlike domain), and $f : D \to \mathbb{C}$ a continuous function, holomorphic away from at most a finite set of points in $D$. Then, $\int_{\partial_{\gamma}} f = 0$ for any closed curve $\gamma$ in $D$.
\end{theorem}

\begin{proofbox}
	By our previous theorem and the converse of FTC for starlike domains, there exists an antiderivative $F$ for $f$ in $D$. So by the fundamental theorem of calculus, Cauchy's theorem follows.
\end{proofbox}

\begin{theorem}[Cauchy's integral formula]\index{Cauchy's integral formula}
	Let $U \subset \mathbb{C}$ be a domain, $f : U \to \mathbb{C}$ holomorphic, and $\overline{D(a, r)} \subset U$. Then for all $z \in D(a, r)$,
	\[
	f(z) = \frac{1}{2 \pi i} \int_{\partial D(a, r)} \frac{f(w)}{w - z} \diff w
	.\]
\end{theorem}

\begin{proofbox}
	Define an auxiliary function
	\[
	g(w) = 
	\begin{cases}
		\frac{f(w) - f(z)}{w - z} - f'(z) & w \neq z, \\
		0 & w = z.
	\end{cases}
	\]
	Then $g$ is continuous at $z$ and homomorphic on $D(a, r)$, except possibly at $z$. Find $r_1 > 0$ such that $\overline{D(a,r)} \subset D(a, r_1) \subset U$. Applying Cauchy's theorem to $g$ on $D(a, r_1)$ with curve $\gamma = \partial D(a, r)$, we get
	\[
	\int_{\partial D(a, r)}g(w) \diff w = 0 \iff \int_{\partial D(a,r)}\frac{f(w)}{w - z} \diff w = \int_{\partial D(a, r)} \frac{f(z)}{w - z} \diff w
	.\]
	We can expand $\frac{1}{w - z}$ as
	\[
		\frac{1}{w - z} = \frac{1}{(w - a)[1 - \frac{z - a}{w - a}]} = \sum_{n = 0}^{\infty} \frac{(z - a)^{n}}{(w - a)^{n+1}}
	.\]
	Hence we get
	\[
		\int_{\partial D(a, r)}\frac{f(z)}{w - z} \diff w = \sum_{n = 0}^{\infty} \Biggl[ f(z) (z - a)^{n} \int_{\partial D(a, r)} \frac{1}{(w - a)^{n+1}} \diff w \Biggr]
	.\]
	The latter integral vanishes unless $n = 0$, which gives
	\[
	\int_{\partial D(a, r)} \frac{f(w)}{w - z} = 2 \pi i f(z)
	.\]
\end{proofbox}

\begin{corollary}[Mean Value Property]
	If $f : U \to \mathbb{C}$ is holomorphic on a domain $U$, and $\overline{D(a, r)} \subset U$, then
	\[
	f(a) = \int_{0}^{1} f(a + re^{2 \pi i t}) \diff t
	.\]
\end{corollary}

\begin{proofbox}
	We can apply Cauchy's integral formula, with $t \mapsto a + re^{2 \pi i t}$ on $[0, 1]$ for $\partial D(a, r)$.
\end{proofbox}

We can use Cauchy's integral formula to obtain the following:

\begin{corollary}[Local Maximum Principle]
	Let $f : D(a, r) \to \mathbb{C}$ be holomorphic. If $|f(z)| \leq |f(a)|$ for all $z \in D(a, r)$, then $f$ is constant.
\end{corollary}

\begin{proofbox}
	By the mean value property, for all $0 < \rho < r$,
	\[
	|f(a)| = \Biggl| \int_{0}^{1} f(a + \rho e^{2 \pi i t})\diff t \Biggr| \leq \sup_{|z - a| = \rho}|f(z)| = |f(a)|
	.\]
	Since we have equality, we have $|f(z)| = |f(a)|$ for all $|z - a| = \rho$. So $|f|$ is a constant function on $D(a, r)$, hence $f$ is constant on $D(a, r)$.
\end{proofbox}

\begin{theorem}[Liouville's theorem]\index{Liouville's theorem}
	Every bounded entire function is constant.
\end{theorem}

\begin{proofbox}
	Say $|f(z)| \leq M$ for $f$ entire. Take $R \gg 1$, then for any $0 < |z| < \frac{R}{2}$ by Cauchy's integral formula,
	\begin{align*}
	|f(z) - f(0)| &= \frac{1}{2 \pi} \Biggl| \int_{\partial D(0, r)} f(w) \biggl[\frac{1}{w - z} - \frac{1}{w} \biggr] \diff w \Biggr| \\
		      &= \frac{1}{2 \pi} \Biggl| \int_{\partial D(0, r)} f(w) \frac{z}{(w - z)w} \diff w \Biggr| \\
		      &\leq \frac{1}{2 \pi} \cdot 2 \pi R \cdot \sup_{w \in \partial D(0, R)}|f(w)| \cdot |z| \cdot \frac{1}{R \cdot \frac{R}{2}} \\
		      & \leq M |z| \frac{1}{R/2} \to 0
	\end{align*}
	as $R \to \infty$, so $f(z) = f(0)$. Hence $f$ is constant.
\end{proofbox}

\begin{corollary}[Fundamental Theorem of Algebra]\index{fundamental theorem of algebra}
	Every non-constant polynomial with complex coefficients has a root in $\mathbb{C}$.
\end{corollary}

\begin{proofbox}
	If $p(z) = a_d z^{d} + a_{d-1} z^{d-1} + \cdots + a_0$ has no root in $\mathbb{C}$, then $f(z) = \frac{1}{p(z)}$ is entire.

	As $p(z)$ is non-constant, we have $a_d \neq 0$ and $d \geq 1$. So
	\[
	\frac{p(z)}{z^{d}} = a_d + a_{d-1} \cdot \frac{1}{z} + \cdots + a_n \frac{1}{z^{d}}
	\]
	shows that $|p(z)| \to \infty$ as $|z| \to \infty$. Hence $|f(z)| \to 0$ as $|z| \to \infty$ 

	Hence there exists $R > 0$ such that for all $z \not \in D(0, R)$, $|f(z)| \leq 1$, but if $M = \max_{z \in \overline{D(0,R)}} |f(z)|$, $|f|$ is bounded by $\max\{1, M\}$, and so by Liouville's theorem is constant. Therefore $p$ must be constant.
\end{proofbox}

\begin{theorem}
	Let $f : D(a, r) \to \mathbb{C}$ be holomorphic. Then $f$ is represented by convergent power series on $D(a, r)$:
	\[
	f(z) = \sum_{n = 0}^{\infty} c_n (z - a)^{n}
	,\]
	with
	\[
	c_n = \frac{f^{(n)}(a)}{n!} = \frac{1}{2 \pi i} \int_{\partial D(a, \rho)} \frac{f(w)}{(w - a)^{n+1}} \diff w
	,\]
	for $0 < \rho < r$.
\end{theorem}

\begin{proofbox}
	For $|z - a| < \rho < r$, Cauchy's integral formula gives
	\begin{align*}
		f(z) &= \frac{1}{2 \pi i} \int_{\partial D(a, \rho)} \frac{f(w)}{w - z} \diff w \\
		     &= \frac{1}{2 \pi i} \int_{\partial D(a, \rho)}f(w) \sum_{n = 0}^{\infty} \frac{(z - a)^{n}}{(w - a)^{n+1}} \diff w \\
		     &= \sum_{n = 0}^{\infty}\Biggl[ \frac{1}{2 \pi i} \int_{\partial D(a, \rho)}f(w) \frac{1}{(w - a)^{n+1}} \diff w \Biggr] (z - a)^{n},
	\end{align*}
	proving the theorem.
\end{proofbox}

\begin{remark}
	\begin{enumerate}
		\item[]
		\item Holomorphic functions therefore have derivatives of all orders, which are holomorphic themselves.
		\item This shows holomorphic functions are exactly the analytic functions.
	\end{enumerate}
\end{remark}

\begin{corollary}[Morera's theorem]
	Let $D$ be a disk and $f : D \to \mathbb{C}$ be continuous such that $\int_{\gamma} f = 0$ for all closed curves $\gamma$ in $D$. Then $f$ is holomorphic.
\end{corollary}

\begin{proofbox}
	By the converse of the fundamental theorem of calculus, there exists holomorphic $F$ on $D$ with $F' = f$. So $f$ is holomorphic.
\end{proofbox}

\begin{corollary}
	Let $f_u : U \to \mathbb{C}$ be holomorphic functions on a domain $U$, and $f_n \to f$ uniformly on $U$ (note is is sufficient for uniform convergence on compact subsets on $U$). Then $f$ is holomorphic on $U$, and
	\[
	f'(z) = \lim_{n \to \infty} f_n'(x)
	.\]
\end{corollary}

\begin{proofbox}
	Since $U$ is a union of open disks, it suffices to work with $D(z, \eps) \subset U$. Given $\gamma$, a closed curve in $D(z, \eps)$, since $\int_{\gamma}f_n \to \int_{\gamma}f$, and we know $\int_{\gamma} f_n = 0$, we get $\int_{\gamma} f = 0$.

	Since $f$ is continuous on $D(z, \eps)$, Morera's theorem applies, so $f$ is holomorphic on $D(z, \eps)$.

	Recall the Taylor expansion computation for $0 < \rho < \eps$:
	\[
	f^{(n)}(z) = \frac{n!}{2 \pi i} \int_{\partial D(z, \rho)} \frac{f(\xi)}{(\xi - z)^{n+1}} \diff \xi
	.\]
	Hence we get
	\begin{align*}
		|f'(z) - f_n'(z)| &= \frac{1}{2 \pi} \Biggl| \int_{\partial D(x, \rho} \frac{f(\xi)}{(\xi - z)^2} - \frac{f_n(\xi)}{(\xi - z)^2} \diff \xi \Biggr| \\
				  &\leq \rho \cdot \frac{1}{\rho^2} \sup_{\zeta \in \partial D(x, \rho)} |f(\xi) - f_n(\xi)| \to 0,
	\end{align*}
	as $n \to \infty$. Hence $f'(z) = \lim f_n'(z)$.
\end{proofbox}

\begin{remark}
	Note $f$ need not be nonconstant, for example take $f_n(z) = z^{n}$ on $D(0, r)$, with $0 < r < 1$. Then $f_n \to 0$ uniformly.
\end{remark}

\begin{corollary}
	If $f : U \to \mathbb{C}$ is continuous on a domain $U$, and holomorphic on $U \setminus S$ for some finite set $S$, then $f$ is holomorphic on $U$.
\end{corollary}

\begin{proofbox}
	If $a \in S$, find $D(a, r) \subset U$ an open disk. Then by Cauchy's theorem on a disk, $\int_{\gamma} f = 0$ for any closed curve $\gamma$ in $D(a, r)$. By Morera's theorem, $f$ is holomorphic on $D(a, r)$.

	Since this holds for all $a$, $f$ is holomorphic on $U$.
\end{proofbox}

Let $f : D(a, R) \to \mathbb{C}$ be holomorphic, so
\[
f(z) = \sum_{n = 0}^{\infty} c_n(z - a)^{n}
\]
on $D(a, R)$. If $f \not \equiv 9$, then some $c_n$ is non-zero. Let
\[
	m = \min\{n \in \mathbb{N}_{0} \mid c_n \neq 0\}
.\]
If $m > 0$, then we say $f$ has a zero of order $m$ at $a$. In this case, we can write
\[
f(z) = (z - a)^{m} g(z)
,\]
where $g(z)$ is holomorphic on $D(a, R)$, and $g(a) \neq 0$.

\begin{theorem}[Principle of Isolated Zeroes]
	If $f : D(a, R) \to \mathbb{C}$ is holomorphic, and not identically 0, then there exists $0 < r < R$ such that $f(z) \neq 0$ on $0 < |z - a| < r$.
\end{theorem}

\begin{proofbox}
	If $f(a) \neq 0$, then $f(z) \neq 0$ on $D(a, r)$ for some $0 < r < R$ by continuity of $f$.

	If $f$ has a zero of order $m$ at $a$, write $f(z) = (z - a)^{m} g(z)$, where $g(a) \neq 0$ and $g$ is holomorphic.

	By the continuity of $g$, there exists $0 < r < R$ such that $g(z) \neq 0$ for all $z \in D(a, r)$. Hence $f(z) \neq 0$ for all $0 < |z - a| < r$.
\end{proofbox}

\begin{remark}
	\begin{enumerate}
		\item[]
		\item This says there is no accumulation point of the zero set of a holomorphic map inside its domain, unless it is everywhere $0$.
		\item It is possible for the zeroes of a holomorphic map to accumulate outside its domain: consider
			\[
			\sin z = \frac{e^{iz} - e^{-iz}}{2i}
			,\]
			which has zeroes at $z = n \pi$. Hence $\sin (\frac{1}{z})$ has zeroes accumulating at $0$, on the boundary of its domain $\mathbb{C}^{\times}$.
		\item Another application: since $\cos^2 z + \sin^2 z = 1$ holds for all $z \in \mathbb{R}$, then $\cos^2z + \sin^2 z - 1$ is entire with $\mathbb{R}$ contained in its zero set. Hence $\cos^2 z + \sin^2z = 1$ for all $z \in \mathbb{C}$.
	\end{enumerate}
\end{remark}

\begin{proposition}[Identity Theorem for holomorphic functions]
	Let $f, g : U \to \mathbb{C}$ be holomorphic on a domain $U$. Let $S = \{z \in U \mid f(z) = g(z)\}$. If $S$ has a non-isolated point, then $f(z) = g(z)$ for all $z \in U$.
\end{proposition}

\begin{proofbox}
	Define $h(z) = f(z) - g(z)$. This is holomorphic on $U$. Now suppose $w$ is non-isolated in $S$. Then for $\eps > 0$ with $D(w, \eps) \subset U$, by the principle of isolated zeroes, $h = 0$ on $D(w, \eps)$.

	Given $z \in U$, let $\gamma : [0, 1] \to U$ be a path with $\gamma(0) = w$, $\gamma(1) = z$. Consider the set
	\[
		T = \{t \in [0, 1] \mid h^{(n)}(\gamma(t)) = 0 \text{ for all } n \geq 0\}
	.\]
	Note that $T$ is closed by definition. Moreover, since $h = 0$ on $D(w, \eps)$, we get that $T$ is non-empty, as $0 \in T$. Now define
	\[
		t_0 = \sup\{t \in [0, 1] \mid [0,t] \subset T\}
	.\]
	Then $T$ is closed an non-empty, so $t_0 \in T$. Since $h^{(n)}(\gamma(t_0)) = 0$ for all $n \geq 0$, $h \equiv 0$ on a neighbourhood of $\gamma(t_0)$, contradicting the maximality if $t_0$, unless $t_0 = 1$.

	Hence $h(\gamma(1)) = 0$, or $h(z) = 0$, as desired.
\end{proofbox}

\begin{definition}
	Let $U \subset V \subset \mathbb{C}$ be domains, and $f : U \to \mathbb{C}$ is holomorphic. $g : V \to \mathbb{C}$ is an \emph{analytic continuation} of $f$ if:
	\begin{enumerate}
		\item $g$ is holomorphic on $V$, and
		\item $g|_U = f$.
	\end{enumerate}
\end{definition}

\begin{exbox}
	The series
	\[
	\sum_{n \geq 1} = \frac{(-1)^{n+1}}{n} z^{n}
	.\]
	Converges on $D(0, 1)$, and takes the value $\log(1+z)$ on $D(0, 1)$. So $\log(1+z)$ is an analytic continuation of this series, to the domain $\mathbb{C} \setminus (-\infty, -1]$.

	Moreover the series
	\[
	\sum_{n \geq 0} z^{n}
	\]
	has radius of convergence $1$ about $a = 0$, and on $D(0, 1)$ we it takes the value $\frac{1}{1-z}$. Hence $\frac{1}{1-z}$ is an analytic continuation of the series to $\mathbb{C}\setminus\{1\}$.
\end{exbox}

\begin{corollary}[Global maximum principle]
	Let $U \subset \mathbb{C}$ be a bounded domain, and let $\overline{U}$ be its closure. If $f : \overline{U} \to \mathbb{C}$ is continuous and $f$ is holomorphic on $U$, then $|f|$ attains its maximum on $\overline{U} \setminus U$.
\end{corollary}

\begin{proofbox}
	As $U$ is bounded, then $\overline{U}$ is bounded, hence as $f$ is continuous $|f|$ attains a maximum on $\overline{U}$, say $M$.

	If $|f(z_0)| = M$ for $z_0 \in U$, then by the local maximum principle, $f \equiv f(z_0)$ on any disk $D(z_0, r) \subset U$. By the identity theorem, $f \equiv f(z_0)$ on $U$, hence $f \equiv f(z_0)$ on $\overline{U}$.

	Thus $M$ is achieved by $|f|$ on $\overline{U} \setminus U$.
\end{proofbox}

Our goal is to generalize the Cauchy integral formula by allowing more general closed curves for integration.

However, we cannot hope that Cauchy's integral formula works for any closed curve: consider a curve $\gamma'$ given by going around a closed disk $\gamma$ twice. Then immediately
\[
\int_{\gamma'}f \diff z = 2 \int_{\gamma}f \diff z
.\]
Hence we need to deal with the notion of ``winding around'' a point more than once. Quantifying this notion, we will see this is the only issue to generalizing Cauchy's integral formula.

Our first hope is that we can count the crossing of some slit in the plane. However this doesn't work as we can cross infinitely often. But this cannot happen for every direction!

\begin{theorem}
	Let $\gamma : [a, b] \to \mathbb{C} \setminus \{w\}$ be a continuous curve. Then there exists a continuous function $\theta : [a, b] \to \mathbb{R}$ with $\gamma(t) = w + r(t) e^{i \theta (t)}$, with $r(t) = |\gamma(t) - w|$.
\end{theorem}

\begin{proofbox}
	By translation, we can translate to assume $w = 0$. Moreover, since
	\[
	\arg \gamma(t) = \arg \frac{\gamma(t)}{|\gamma(t)|}
	,\]
	so dividing by the modulus of $\gamma$, we can assume $|\gamma(t)| = 1$ for all $t \in [a, b]$.

	Notice that if $\gamma \subset \mathbb{C} \setminus \mathbb{R}_{\leq 0}$, then $t \mapsto \arg (\gamma(t))$ (using the principal argument), gives a continuous choice of $\theta$. More generally, if $\gamma$ lies in any slit plane
	\[
		\mathbb{C} \setminus \{ z \mid z/e^{i\alpha} \subset \mathbb{R}_{\leq 0}\}
	,\]
	then $\theta(t) = \alpha + \arg(z/e^{i\alpha})$ will do.

	Our strategy is to subdivide $\gamma$ so that the pieces lie in slit planes, and so we can make $\theta$ continuous on the pieces.

	Since $\gamma$ is continuous on $[a, b]$, it is uniformly continuous, so there exists $\eps > 0$ such that $|s - t| < \eps$ implies $|\gamma(s) - \gamma(t)| < 2$. Subdividing $a = a_0 < a_1 < \cdots < a_{n-1} < a_n = b$ with $a_{j+1} - a_{j} < 2 \eps$, then
	\[
	\biggl| \gamma(t) - \gamma \biggl( \frac{a_{j+1} - a_j}{2} \biggr) \biggr| < 2
	,\]
	for all $t \in [a_{j}, a_{j+1}]$. Hence $\gamma([a_{j-1}, a_j])$ lies in a slit plane, and we can define $\theta_j$ a continuous choice of argument for $\gamma|_{[a_{j-1}, a_j]}$, for all $j \in \{1, \ldots, n\}$. Then
	\[
	\gamma(a_j) = e^{i \theta_j (a_j)} = e^{i \theta_{j+1}(a_j)}
	.\]
	Since this is true, $\theta_{j+1}(a_j) = \theta_j(a_j) + 2 \pi n_j$ for some $n_j \in \mathbb{C}$. Modifying each of $\theta_j$ for $j \geq 2$ by a suitable integer multiple of $2 \pi$ ensures that the $\theta_j$ fit together to a continuous choice of $\theta$ on $[a, b]$.
\end{proofbox}
\begin{remark}
	$\theta$ is not unique, since $\theta(t) + 2 \pi n$ is also valid for all $n \in \mathbb{Z}$. But if $\theta_1, \theta_2$ are two functions as in the theorem, then $\theta_1 - \theta_2$ is continuous, but takes values in the discrete set $2 \pi \mathbb{Z}$, hence is constant.
\end{remark}

\begin{definition}
	Let $\gamma : [a, b] \to \mathbb{C}$ be a closed curve, and $w \not \in \gamma$. The \emph{winding number}\index{winding number} or \emph{index}\index{index} of $\gamma$ about $w$ is
	\[
	I(\gamma;w) = \frac{\theta(b) - \theta(a)}{2 \pi}
	,\]
	where $\gamma(t) = w + r(t) e^{i \theta(t)}$ with $\theta$ continuous.
\end{definition}

\begin{lemma}
	Let $\gamma : [a, b] \to \mathbb{C} \setminus \{w\}$ be a closed curve. Then
	\[
	I(\gamma; w) = \frac{1}{2 \pi i} \int_{\gamma} \frac{\diff z}{z - w}
	.\]
\end{lemma}

\begin{proofbox}
	Note $\gamma$ is piecewise $C^{1}$, so $r(t)$ and $\theta(t)$ are piecewise $C^{1}$ as well, where $\gamma(t) = w + r(t) e^{i \theta(t)}$. So
	\begin{align*}
		\int_{\gamma} \frac{\diff z}{z - w} &= \int_{a}^{b} \frac{\gamma'(t)}{\gamma(t) - w} \diff t = \int_{a}^{b} \frac{r'(t)}{r(t)} + i \theta'(t) \diff t \\
						    &= [\log r(t) + i \theta(t)]_{t = a}^{t = b} = 2 \pi i I(\gamma, w),
	\end{align*}
	since $\gamma$ is closed and $\theta(b) - \theta(a) = 2 \pi I(\gamma, w)$.
\end{proofbox}

\begin{proposition}
	If $\gamma : [0, 1] \to D(a, R)$ is a closed curve, then for all $w \not \in D(a, R)$, $I(\gamma; w) = 0$.
\end{proposition}

\begin{proofbox}
	Consider the M\"{o}bius maps
	\[
	z \mapsto \frac{z - w}{a - w}
	.\]
	This takes $a \mapsto 1$, $w \mapsto 0$ so $D(a, R) \mapsto D(1, r)$ for some $r < 1$, as it is an affine map. So then $D(a, R)$ is contained in the slit plane
	\[
		\mathbb{C} \setminus \{z \mid \frac{z - w}{a - w} \in \mathbb{R}_{\leq 0} \}
	.\]
	Hence there is a branch of $\arg(z - w)$ defined on $D(a, R)$, and so
	\[
	I(\gamma; w) = \frac{\arg(\gamma(1) - w) - \arg(\gamma(0) - w)}{2 \pi} = 0
	.\]
\end{proofbox}

\begin{definition}
	Let $U \subset \mathbb{C}$ be open. Then a closed curve $\gamma$ in $U$ is \emph{homologous to zero}\index{homologous to zero} in $U$ if, for all $w \not \in U$, $I(\gamma; w) = 0$.

	$U$ is \emph{simply connected}\index{simply connected} if every closed curve in $U$ is homologous to zero.
\end{definition}

\begin{remark}
	For $U$ open, this is equivalent to the homotopy definition of simply connected.
\end{remark}

\begin{exbox}
	\begin{enumerate}
		\item Any disk is simply connected, by the previous proposition.
		\item Any punctured disk $D(a, R) \setminus \{a\}$ is not simply connected, since curves can wind around $a$.
		\item Any annulus is not simply connected.
	\end{enumerate}
\end{exbox}

\begin{theorem}[General Cauchy's Integral Formula]
	Let $f : U \to \mathbb{C}$ be holomorphic on a domain $U$, and $\gamma$ in a closed curve homologous to zero in $U$. Then for all $w \in U \setminus \gamma$,
	\[
	I(\gamma ; w) f(w) = \frac{1}{2 \pi i} \int_{\gamma} \frac{f(z)}{z - w} \diff z
	,\]
	and in particular,
	\[
	\int_{\gamma}f(z) \diff z = 0
	.\]
\end{theorem}

\begin{proofbox}
	Notice applying the first equality to $g(z) = f(z)(z-w)$ gives $\int_{\gamma} f = 0$. So it suffices to prove the first statement.

	We have by the previous lemma that
	\[
	I(\gamma;w)f(w) = \frac{1}{2 \pi i} \int_{\gamma} \frac{f(w)}{z - w} \diff z
	,\]
	so we want to show that
	\[
	\frac{1}{2 \pi i} \int_{\gamma} \frac{f(z) - f(w)}{z - w} \diff z = 0
	,\]
	for all $w \in U\setminus \gamma$. Consider the function
	\[
	g(z, w) =
	\begin{cases}
		\frac{f(z) - f(w)}{z - w} & z \neq w, \\
		f'(w) & z = w.
	\end{cases}
	\]
	This is a continuous function on $U \times U$, and we wish to show that
	\[
	\int_{\gamma}g(z, w) \diff z = 0
	,\]
	for all $w \in U \setminus \gamma$. Consider the auxiliary function $h$ on $\mathbb{C}$,
	\[
	h(w)=
	\begin{cases}
		\int_{\gamma} g(\zeta, w) \diff \zeta & w \in U, \\
		\int_{\gamma} \frac{f(\zeta)}{\zeta - w} \diff \zeta & w \in C\setminus \gamma, I(\gamma;w) = 0.
	\end{cases}
	\]
	Call the latter set $V$. If $w \in U \cap V$, then
	\[
	\int_{\gamma}g(\zeta, w) \diff \zeta = \int_{\gamma} \frac{(f(\zeta) - f(w)}{\zeta - w} \diff \zeta = \int_{\gamma} \frac{f(\zeta)}{\zeta - w} \diff \zeta,
	\]
	so $h$ is well-defined. For any disk $D(0,R)$ with $\gamma \subset D(0,R)$, we have that $I(\gamma;w) = 0$ for all $w \not \in D(0,R)$. In fact, $\gamma$ is homologous to zero in $U$, so $U \cup V = \mathbb{C}$. So we have
	\[
	|h(w)| = \biggl| \int_{\gamma} \frac{f(\zeta)}{\zeta - w} \diff \zeta \biggr| \leq \frac{\len(\gamma) \sup |f(\zeta)|}{|w| - R} \to 0,
	\]
	as $|w| \to \infty$. Then we claim $h$ is holomorphic on $\mathbb{C}$. If so, then $h$ is bounded as $|h(w)| \to 0$. Hence it is constant by Liouville's theorem, taking the value $0$ on the entirety of $\mathbb{C}$, which finishes the proof. We use the following:

	\begin{lemma}
		Let $U \subset \mathbb{C}$ be open, and $\phi : U \times [a, b] \to \mathbb{C}$ be continuous with $z \mapsto \phi(z, s)$ holomorphic on $U$ for every $s \in [a, b]$. Then,
		\[
		g(z) = \int_{a}^{b} \phi(z, s) \diff s
		\]
		is holomorphic on $U$.
	\end{lemma}
	
	The proof of this is using Morera's. Without loss of generality, $U$ is a disk. Then for any closed curve $\gamma : [0,1] \to U$, then
	\begin{align*}
		\int_{\gamma}g(z) \diff z &= \int_{0}^{1} \Biggl[ \int_{a}^{b} \phi(\gamma(t), s) \diff s \Biggr] \gamma'(t) \diff t \\
					  &= \int_{a}^{b} \Biggl[ \int_{0}^{1} \phi(\gamma(t), s)\gamma'(t) \diff t \Biggr] \diff s,
	\end{align*}
	where we swap the order of integration by Fubini's theorem: suppose $f : [a, b] \times [c, d] \to \mathbb{C}$ is a continuous function. Then we have
	\[
	\int_{a}^{b} \Biggl( \int_{c}^{d} f(x,y) \diff y \Biggr) \diff x = \int_{c}^{d}\Biggl( \int_{a}^{b} f(x,y) \diff x \Biggr) \diff y.
	\]
	This clearly holds if $f$ is constant, so it also holds when $f$ is a step function. Since $[a, b] \times [c \times d]$ is closed and bounded, $f$ is uniformly continuous. So $f$ is a uniform limit of step functions, and we can exchange the order as claimed.

	Now, back to our proof of the lemma. We have
	\[
		\int_{\gamma}g(z) \diff z = \int_{a}^{b} \Biggl[ \int_{\gamma} \phi(z, s) \diff z \Biggr] \diff s.
	\]
	Since $z \mapsto \Phi(z, s)$ is holomorphic, this is $0$ by Cauchy's theorem on a disk. So,
	\[
	\int_{\gamma} g(z) \diff z = 0,
	\]
	and by Morera's, $g$ is holomorphic as claimed.

	Therefore, proving this lemma, we get $h$ is holomorphic as claimed and the generalized Cauchy integral formula follows.
\end{proofbox}

\begin{corollary}[Cauchy's theorem for Simply Connected Domains]
	Let $f : U \to \mathbb{C}$ be holomorphic on a simply connected domain $U$. Then for all closed curves $\gamma$ in $U$,
	\[
	\int_{\gamma}f = 0
	\]
\end{corollary}

In fact, if $U \subset \mathbb{C}$ is open, then $U$ is simply connected if and only if the complement of $U$ in $\mathbb{C}_{\infty}$ is connected.

\begin{exbox}
	\begin{enumerate}
		\item $D(a, R) \subset \mathbb{C}$ has disk complement in $C_{\infty}$, so is simply connected.
		\item Convex and starlike sets are simply connected.
		\item The annulus is not simply connected.
	\end{enumerate}	
\end{exbox}

\subsection{Isolated Singularities of Holomorphic Maps}
\label{sub:isolated_singularities_of_holomorphic_maps}

\begin{definition}
	A point $a \in \mathbb{C}$ is an \emph{isolated singularity}\index{isolated singularity} of $f : U \to \mathbb{C}$ holomorphic, if there exists $r > 0$ such that $f$ is holomorphic on $D(a, r) \setminus \{a\}$, denoted $D(a, r)^{\times}$.
\end{definition}

\begin{exbox}
	\begin{enumerate}
		\item Take $a = 0$ and $f(z) = \frac{\sin z}{z}$. Using the identity theorem or expansion of $e^{z}$, we get
			\[
				\sin z = z - \frac{z^3}{3!} + \frac{z^{5}}{5!} - \frac{z^{7}}{7!} + \cdots
			\]
			about $0$. So
			\[
			f(z) = 1 - \frac{z^2}{3!} + \frac{z^4}{5!} - \frac{z^6}{7!} + \cdots
			\]
			about $0$. Thus $f$ is a restriction of a holomorphic function on $\mathbb{C}$, say $f$, and $f(0) = 1$.
		\item Take $a=  0$ and $g = \frac{1}{z^{6}}$. Then $g$ is holomorphic on $\mathbb{C}^{\times}$, and $|g(z)| \to \infty$ as $z \to 0$, so there is no continuous extension at $0$.
		\item Recall the action $w \mapsto e^{w} = e^{\Re w} e^{i \Im w}$.

			The map $h(z) = e^{1/z}$ maps any $D(0, \eps)^{\times}$ to all of $\mathbb{C}^\times$.
	\end{enumerate}
\end{exbox}

%Lecture 9

\begin{theorem}[Laurent Expansion]\index{Laurent expansion}
	Let $f$ be holomorphic on an annulus $A = \{z \in \mathbb{C} \mid r < |z - a| < R\}$, where $0 \leq r < R \leq \infty$. Then,
	\begin{enumerate}[\normalfont(i)]
		\item $f$ has a (unique) convergent expansion on $A$:
			\[
			f(z) = \sum_{n = -\infty}^{\infty} c_n(z - a)^{n},
			\]
			known as the ``Laurent series''\index{Laurent series}.
		\item For any $r < \rho < R$, we have
			\[
			c_n = \frac{1}{2 \pi i} \int_{\partial D(a, \rho)} \frac{f(z)}{(z - a)^{n+1}} \diff z.
			\]
		\item If $r < \rho' \leq \rho < R$, the Laurent series converges uniformly on $\{z \in \mathbb{C} \mid \rho' \leq |z - a| \leq \rho\}$.
	\end{enumerate}
\end{theorem}

\begin{proofbox}
	Fix $w \in A$, and choose $r < \rho_1 < |w-a| < \rho_2 < R$.

	Define two closed curves $\gamma_1, \gamma_2$ by cutting along a diameter of the sub-annulus, labelled such that $I(\gamma_1;w) = 1$ and $I(\gamma_2;w) = 0$.

	Then $\gamma_1, \gamma_2$ are both homologous to zero in $A$, so by the generalized Cauchy's integral formula, we have
	\[
	f(w) = \frac{1}{2 \pi i} \int_{\gamma_1} \frac{f(z)}{z - w} \diff z = \frac{1}{2 \pi i} \int_{\gamma_1 + \gamma_2} \frac{f(z)}{z - w} \diff z.
	\]
	Travelling around $\gamma_1 + \gamma_2$ is the same as travelling $\partial D(a, \rho_2) - \partial D(a, \rho_1)$. So,
	\[
	f(w) = \frac{1}{2 \pi i} \int_{|z - a| = \rho_2} \frac{f(z)}{z - w} \diff z - \frac{1}{2 \pi i} \int_{|z-a|=\rho_1} \frac{f(w)}{z - w} \diff z.
	\]
	Label the first integral as $I_2$, and the second as $I_1$. Using the geometric series for $(1 - \frac{w-a}{z-a})^{-1}$ to compute $I_2$ as a Taylor series gives
	\[
	I_2 = \sum_{n = 0}^{\infty} c_n (w - a)^{n},
	\]
	where the coefficients are
	\[
	c_n = \frac{1}{2 \pi i} \int_{|z - a| = \rho_2} \frac{f(z)}{(z - a)^{n+1}} \diff z,
	\]
	for $n \geq 0$. For $I_1$, since $|z-a| < |w-a|$, using the expansion
	\[
	-\frac{1}{z - w} = \frac{\frac{1}{w-a}}{1 - \frac{z-a}{w - a}} = \sum_{m = 1}^{\infty} \frac{(z - a)^{m+1}}{(w - a)^{m}},
	\]
	we get that
	\[
	I_1 = \sum_{m = 1}^{\infty} d_m (w - a)^{-m},
	\]
	where
	\[
	d_m = \frac{1}{2 \pi i} \int_{|z-a| = \rho_1} \frac{f(z)}{(z - a)^{-m+1}} \diff z.
	\]
	Reindexing with $n = -m$, we obtain the Laurent expansion for $f$.

	To show part (ii) and (iii), suppose that
	\[
	f(z) = \sum_{n = -\infty}^{\infty} c_n(z - a)^{n}
	\]
	on $A$, and let $r < \rho' \leq \rho < R$. The non-negative power series
	\[
	\sum_{n = 0}^{\infty}c_n (z - a)^{n}
	\]
	has radius of convergence greater than or equal to $R$, so it converges uniformly on $D(a, \rho)$. Similarly, if $u = \frac{1}{z - a}$, then the negative part of the Laurent expansion,
	\[
	\sum_{n = 1}^{\infty}c_{-n} u^{n},
	\]
	has radius of convergence greater than or equal to $r^{-1}$, so it converges uniformly on $\{|z - a| \geq \rho'\}$. So the Laurent series converges uniformly on $\rho' \leq |z - a| \leq \rho$.

	So we can integrate term-by-term to get
	\[
	\frac{1}{2 \pi i} \int_{\partial D(a, \rho)} \frac{f(z)}{(z - a)^{m+1}} \diff z = \frac{1}{2 \pi i} \sum_{n = -\infty}^{\infty} c_n \int_{\partial D(a, \rho)} (z - a)^{n - m - 1} \diff z = c_m,
	\]
	since this integral is $0$ unless $n = m$, in which case it is $2 \pi i$.
\end{proofbox}

\begin{remark}
	This shows that $f = f_1 + f_2$, where $f_1$ is holomorphic on $D(a, R)$, and $f_2$ is holomorphic on $|z-a| > r$.
\end{remark}

Applying to the case $r = 0$, we have three possibilities on a punctured disk domain, i.e. an isolated singularity at $a$.
\begin{enumerate}
	\item $c_n = 0$ for all $n < 0$. Then $f$ is the restriction to $D(a, R)^{\times}$ of a function holomorphic on $D(a, R)$. We say $f$ has a \emph{removable singularity}\index{removable singularity} at $a$.

		An example is
		\[
		f(z) = \frac{\sin z}{z},
		\]
		at $a = 0$.
	\item There exists $k < 0$ such that $c_k \neq 0$, but $c_n = 0$ for all $n < k$. Then $(z-a)^{-k}f(z)$ is holomorphic and non-zero at $a$. We say $f$ has a \emph{pole}\index{pole} of order $|k|$ at $a$.

		An example is
		\[
		g(z) = \frac{1}{z^{6}},
		\]
		at $a = 0$, which has a pole of order $6$.
	\item $c_n \neq 0$ for infinitely many $n < 0$. Then we say $f$ has an \emph{essential singularity}\index{essential singularity} at $a$.

		An example is
		\[
		h(z) = \exp \biggl( \frac{1}{z} \biggr),
		\]
		at $a = 0$.
\end{enumerate}

We now look at the local behaviour around each of these kinds of singularities.

\begin{proposition}
	An isolated singularity at $z = a$ for $f$ is removable if and only if
	\[
	\lim_{z \to a} (z - a) f(z) = 0.
	\]
\end{proposition}

\begin{proofbox}
	If $f$ is indeed holomorphic, then $f$ is bounded as it is continuous on a neighbourhood of $a$, hence the limit approaches $0$.

	If the limit converges to $0$, consider
	\[
	g(z) =
	\begin{cases}
		(z - a)^2 f(z) & z \neq a,\\
		0 & z = a.
	\end{cases}
	\]
	Then by definition,
	\[
	g'(a) = \lim_{z \to a}(z - a) f(z) = 0,
	\]
	so $g$ is holomorphic at $a$, with $g(a) = 0$. Therefore
	\[
	g(z) = \sum_{n = 2}^{\infty} c_n(z - a)^{n},
	\]
	so
	\[
	f(z) = \sum_{n = 0}^{\infty} c_{n+2}(z - a)^{n},
	\]
	proving $f$ is holomorphic at $a$.
\end{proofbox}

\begin{proposition}
	An isolated singularity at $z = a$ for $f$ is a pole if and only if $|f(z)| \to \infty$ as $z \to a$.

	Moreover, the following are equivalent:
	\begin{enumerate}[\normalfont(i)]
		\item $f$ has a pole of order $k$ at $z = a$.
		\item $f(z) = (z-a)^{-k}g(z)$, where $g$ is holomorphic at non-zero at $a$.
		\item $f(z) = \frac{1}{h(z)}$, where $h$ is holomorphic at $a$ with a zero of order $k$ at $a$.
	\end{enumerate}
\end{proposition}

\begin{proofbox}
	Immediately, (i) is equivalent to (ii) using the Laurent expansion, and (ii) is equivalent to (iii) as since $g$ is holomorphic at $a$ and non-zero, $\frac{1}{g}$ is holomorphic at $a$.

	Now if $f$ has a pole of order $k$ at $z = a$, then $f(z) = (z - a)^{-k}g(z)$, so $|f(z)| \to \infty$ as $z \to a$.

	Conversely, if $|f(z)| \to \infty$ as $z \to a$, then there exists $r > 0$ such that $f(z) \neq 0$ for all $0 < |z-a| < r$. So $\frac{1}{f}$ is holomorphic on $D(a, r)^{\times}$, and moreover $\frac{1}{f} \to 0$ as $z \to a$, so the singularity at $0$ for $\frac{1}{f}$ is removable, and hence
	\[
	\frac{1}{f(z)} = h(z),
	\]
	where $h$ is holomorphic on $D(a, r)$. Now $h$ has a zero of order $k$ for some $k \geq 1$, so $h(z) = (z - a)^{k}l(z)$ for $l$ holomorphic and non-zero at $a$, so
	\[
	f(z) = (z - a)^{-k}g(z),
	\]
	meaning $f$ has a pole of order $k$ at $z = a$.
\end{proofbox}

\begin{corollary}
	An isolated singularity at $z = a$ is essential if and only if $|f|$ does not approach a limit in $\mathbb{R} \cup \{\infty\}$ as $z \to a$.
\end{corollary}

\newpage

\printindex

\end{document}
