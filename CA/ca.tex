%%% TO DO %%%
% implement siunitx package and change units

\documentclass[12pt]{article}

\usepackage{ishn}

\makeindex[intoc]

\begin{document}

\hypersetup{pageanchor=false}
\begin{titlepage}
	\begin{center}
		\vspace*{1em}
		\Huge
		\textbf{IB Complex Analysis}

		\vspace{1em}
		\large
		Ishan Nath, Lent 2023

		\vspace{1.5em}

		\Large

		Based on Lectures by Prof. Holly Krieger

		\vspace{1em}

		\large
		\today
	\end{center}
	
\end{titlepage}
\hypersetup{pageanchor=true}

\tableofcontents

\newpage

\section{Complex Differentiation}
\label{sec:complex_differentiation}

Our goal in this course is to study the theory of complex-valued differentiable functions in one complex variable. Example include:
\begin{itemize}
	\item Polynomials $p(z) = a_dz^{d} + \cdots + a_1 z + a_0$, with coefficients in $\mathbb{R}, \mathbb{Q}, \mathbb{Z}$ or $\mathbb{C}$.
	\item The infinite series
		\[
		\sum_{n = 1}^{\infty} \frac{1}{n^{z}}
		,\]
		which we showed convergence for $z$ having real part greater than $1$.
	\item Harmonic functions\index{harmonic function} $u(x, y) : \mathbb{R}^2 \to \mathbb{R}$, $u_{xx} + u_{yy} = 0$.
\end{itemize}

In this course, we make the convention that $\theta = \arg(z) \in [0, 2\pi)$.

\subsection{Basic Notions}
\label{sub:basic_notions}

\begin{itemize}
	\item $U \subset \mathbb{C}$ is \emph{open}\index{open} if for all $u \in U$, there exists $\eps > 0$ such that
		\[
			D(x, \eps) = \{z \in \mathbb{C} \mid |z - u| < \eps\} \subset U
		.\]
	\item A \emph{path}\index{path} in $U \subset \mathbb{C}$ is a continuous map $\gamma : [a, b] \to U$. We say the path is $C^{1}$ if $\gamma'$ exists and is continuous (we take one-sided derivatives at the endpoints).

		$\gamma$ is \emph{simple}\index{simple} if it is injective.
	\item $U \subset \mathbb{C}$ is \emph{path-connected}\index{path-connected} if for all $z, w \in U$, there exists a path in $U$ with endpoints at $z, w$.
\end{itemize}
\begin{remark}
	If $U$ is open, and $z, w \in U$ are connected by a path $\gamma$ in $U$, then there exists a path $\gamma$ in $U$ connected $z, w$ consisting of finitely many horizontal and vertical segments.
\end{remark}

\begin{definition}
	A \emph{domain}\index{domain} is a non-empty, open, path-connected subset of $\mathbb{C}$.
\end{definition}

\begin{definition}
	\begin{enumerate}[(i)]
		\item[]
		\item $f : U \to \mathbb{C}$ is \emph{differentiable}\index{complex differentiable} at $u \in U$ if
			\[
			f'(u) = \lim_{z \to u} \frac{f(z) - f(u)}{z - u}
			\]
			exists.
		\item $f : U \to \mathbb{C}$ is \emph{holomorphic}\index{holomorphic} at $u \in U$ if there exists $\eps > 0$ such that $f$ is differentiable at $z$, for all $z \in D(u, \eps)$. We may also call such a function \emph{analytic}\index{analytic}.
		\item $f : \mathbb{C} \to \mathbb{C}$ is \emph{entire} if it is holomorphic everywhere.
	\end{enumerate}
\end{definition}

\begin{remark}
	All differentiation rules (sum, products, ...) in $\mathbb{R}$ hold, by the same proofs.
\end{remark}

Identifying $\mathbb{C}$ with $\mathbb{R}^2$, we may write $f : U \to \mathbb{C}$ as $f(x+iy) = u(x,y) + iv(x, y)$, where $u, v$ are the real and imaginary parts of $f$.

From analysis and topology, recall that $u : U \to \mathbb{R}$ as a function of two real variables if $(\mathbb{R}^2)$ differentiable at $(c, d) \in \mathbb{R}^2$with $Du|_{(c,d)} = (\lambda, \mu)$ if
\[
	\frac{u(x,y) - u(c,d) - [\lambda(x-c) + \mu(y-d)]}{\sqrt{(x-c)^2 + (y-d)^2}} \to 0
,\]
as $(x, y) \to (c, d)$. However, \textbf{this is a weaker condition} than differentiability over $\mathbb{C}$.

\begin{proposition}[Cauchy-Riemann equations]\index{Cauchy-Riemann equations}
	Let $f : U \to \mathbb{C}$ on an open set $U \subset \mathbb{C}$. Then $f$ is differentiable at $w = c+id \in U$ if and only if, writing $f = u + iv$, we have $u, v$ are $\mathbb{R}^2$-differentiable at $(c, d)$, and
	\begin{align*}
		u_{x} &= v_y, & u_y &= -v_x.
\end{align*}
\end{proposition}

\begin{proofbox}
	$f$ is differentiable at $w$ if and only if $f'(w) = p + iq$ exists, so
	\[
	\lim_{z \to w}\frac{f(z) - f(w) - (z - w)(p + iq)}{|z - w| = 0}
	.\]
	Writing $f = u + iv$ and considering the real and imaginary parts in the quotient above, this holds if and only if
	\[
		\lim_{(x, y) \to (c, d)} \frac{u(x, y) - u(c, d) - [p(x-c) - q(y-d)]}{\sqrt{(x - c)^2 + (y - d)^2}} = 0
	,\]
	and
	\[
		\lim_{(x, y) \to (c, d)} \frac{v(x, y) - v(c, d) - [q(x-c) + p(y-d)]}{\sqrt{(x-c)^2+(y-d)^2}} = 0
	.\]
	This holds if and only if $u, v$ are $\mathbb{R}^2$-differentiable at $(c, d)$, and $u_x = v_y$, $u_y = -v_x$.
\end{proofbox}

\begin{remark}
	If the partial $u_x, u_y, v_x, v_y$ exist and are continuous on $U$, then $u, v$ are differentiable on $U$. So it suffices to check the partials exist and are continuous, and the Cauchy-Riemann equations hold to deduce complex differentiability.
\end{remark}

\begin{exbox}
	\begin{enumerate}
		\item Take $f(z) = \overline{z}$. Then $f$ has $u(x, y) = x$ and $v(x, y) = -y$, so $u_x = 1$, $v_y = -1$. So $f(z) = \overline{z}$ is not holomorphic or differentiable anywhere.
		\item Any polynomial $p(z) = a_d z^{d} + \cdots + a_1z + a_0$, with $a_i \in \mathbb{C}$ is entire.
		\item Rational function\index{rational functions}, which are quotients of polynomials $\frac{p(z)}{q(z)}$ are holomorphic on the open set $\mathbb{C}\setminus\{\text{zeroes of } q\}$.
	\end{enumerate}
\end{exbox}

Note that $f = u + iv$ satisfying the Cauchy-Riemann equations at a point does not mean it is differentiable at that point.

Some proofs in regular analysis have natural extensions to complex analysis. For example, if $f : U \to \mathbb{C}$ on a domain $U$ with $f'(z) = 0$ on $U$, then $f$ is constant on $U$.

Now we ask: why are we interested in complex analysis?
\begin{itemize}
	\item Unlike $\mathbb{R}^2$ differentiable functions, holomorphics functions are very constrained. For example, if $f$ is entire and bounded (so $|f(z)| < M$ for all $z \in \mathbb{C}$), then $f$ is constant. Contrast with $\sin$, for example.
	\item We will see that $f$ holomorphic on a domain $U$ has holomorphic derivative on $U$. This implies that $f$ is infinitely differentiable, as are $u$ and $v$.
\end{itemize}
In particular, we can differentiate the Cauchy-Riemann equations to get
\[
u_{xx} = v_{yx} = v_{xy} = -u_{yy}
,\]
so $u_{xx} + u_{yy} = 0$, and similarly $v_{xx} + v_{yy} = 0$. Hence the real and imaginary parts of a holomorphic function are harmonic.

Let $f : U \to \mathbb{C}$ be a holomorphic function on an open set $U_1$ and $w \in U$ with $f'f(w) \neq 0$. We want to look at the geometric behaviour of $f$ at $w$.

In fact, we claim $f$ is \emph{conformal}\index{conformal} at $w$. Let $\gamma_1, \gamma_2$ be $C^{1}$-paths through $w$, say $\gamma_1, \gamma_2 : [-1,1] \to U_1$, such that $\gamma_1(0) = \gamma_2(0) = w$, and $\gamma_i'(0) \neq 0$. If we write $\gamma_j(t) = w + r_j(t) = e^{i \theta_j(t)}$, then we have
\[
\arg(\gamma_j'(z)) = \theta_j(0)
,\]
and the argument of the image line is
\[
\arg((f \circ \gamma_j)'(0)) = \arg(\gamma_j'(0) f'(\gamma_j(0))) = \arg(\gamma_j'(0)) + \arg(f'(w)) + 2 \pi n
,\]
where crucially we use $\gamma_j'(0) f'(\gamma_j(0)) \neq 0$, so the direction of $\gamma_j$ at $w$ under the application of $f$ is rotated by $\arg(f'(w))$. This is independent of $\gamma_j$. Since the angle between $\gamma_1$ and $\gamma_2$ is the difference of the arguments $f$ preserves the angle. This is what it means to be conformal.

\begin{definition}
	Let $U, V$ be domains in $\mathbb{C}$. A map $f : U \to V$ is a \emph{conformal equivalence}\index{conformal equivalence} of $U$ and $V$ if $f$ is a bijective holomorphic map with $f'(z) \neq 0$, for all $z \in U$.
\end{definition}

\begin{remark}
	\begin{enumerate}[1.]
		\item[]
		\item Using the real inverse function theorem, one can show if $f : U \to V$ is a holomorphic bijection of open sets with $f'(z) \neq 0$ for all $z \in U$, then the inverse of $f$ is also holomorphic, so also conformal by the chain rule. So conformally equivalent domains are equal from the perspective of the functions $f$.
		\item We will later see than being injective and holomorphic on a domain implies $f'(z) \neq 0$ for all $z \in U$, so this requirement is redundant.
	\end{enumerate}	
\end{remark}

\begin{exbox}
	\begin{enumerate}[1.]
		\item Any change of coordinates: on $\mathbb{C}$, take $f(z) = az + b$, for $a \neq 0$ and $b$, which is a conformal equivalence $\mathbb{C} \to \mathbb{C}$. More generally, a M\"{o}bius map
			\[
			f(z) = \frac{az + b}{cz + d},
			\]
			for $ad - bc \neq 0$, is a conformal equivalence from the Riemann sphere to itself. This can eb seen as adding a point at infinity to make a sphere $\mathbb{C}_{\infty}$ (or gluing two copies of the unit disc with coordinates $z$ and $\frac{1}{z}$).

			If $f : \mathbb{C}_{\infty} \to \mathbb{C}_{\infty}$ is continuous, then
			\begin{itemize}
				\item if $f(\infty) = \infty$, then $f$ is holomorphic at $\infty$ if and only if $g(z) = \frac{1}{f(\frac{1}{z})}$ is holomorphic at $0$.
				\item If $f(\infty) \neq \infty$, then $f$ is homolorphic at $\infty$ if and only if $f(\frac{1}{z})$ is holomorphic at $0$.
				\item If $f(a) = \infty$ for $a \in \mathbb{C}$, then $f$ is holomorphic at $a$ if and only if $\frac{1}{f(z)}$ is holomorphic at $a$.
			\end{itemize}
			We can then think of M\"{o}bius maps as change of coordinates for the sphere.

			Choosing $z_1 \to 0$, $z_2 \to \infty$, $z_3 \to 1$ defined a M\"{o}bius map
			\[
			f(z) = \frac{z - z_1}{z - z_2} \cdot \frac{z_3 - z_2}{z_3 - z_1}
			,\]
			for distinct $z_1, z_2, z_3 \in \mathbb{C}$.
		\item For $n \in \mathbb{N}$, $f(z) = z^{n}$ is a conformal equivalence from the sector $\{z \in \mathbb{C}^{\times} \mid 0 < \arg z < \frac{\pi}{n}\}$ to the upper half plane $\mathbb{H} = \{z \in \mathbb{C} \mid \Im z > 0\}$.
		\item The M\"{o}bius map $f(z) = \frac{z-i}{z+i}$ is a conformal equivalence between $\mathbb{H}$ and $D(0,1)$. We can compute $f'(z) \neq 0$ on $\mathbb{H}$, and
			\begin{align*}
				z \in \mathbb{H} & \iff |z-i| < |z+i| \iff |f(z)) < 1.
			\end{align*}
			Note that $f^{-1}(w) = -i \frac{w+1}{w-1}$.
		\item We can use these examples to write down conformal equivalences. Let $U_1$ be the upper half semicircle, and $U_2$ the lower half plane. Considering $g(z) = \frac{z+1}{z-1}$, we know that sends $D(0,1)$ to the left half-plane, so it sends $U_1$ to the upper left quadrant.

			Then, the upper left quadrant if mapped by the squaring map to $U_2$. So $f(z) = (\frac{z+1}{z-1})^2$ is a conformal equivalence from $U_1 \to U_2$.
	\end{enumerate}
\end{exbox}

These are all examples of the deep \emph{Riemann mapping theorem}\index{Riemann mapping theorem}:

\begin{theorem}[Riemann mapping theorem]
	Let $U \subset \mathbb{C}$ be a proper domain which is simply connected. Then there exists a conformal equivalence between $U$ and $D(0,1)$.
\end{theorem}

Here, \emph{simply connected}\index{simply connected} means a subset $U \subset \mathbb{C}$ which is path-connected, and contractible: any loop in $U$ can be contracted to a point. So any continuous path $\gamma : S^{1} \to U$ extends to a continuous map $\hat \gamma : D(0,1) \to U_1$ with $\hat \gamma|_{S_1} = \gamma$.

In fact any domain bounded by a simple closed curve is simply connected, so all of these are conformally equivalent to $D(0,1)$.

\begin{exbox}
	We look at a domains in the Riemann sphere, with bounded and connected complement. This is simply connected as a subset of $\mathbb{C}_{\infty}$.

	Now, the Mandelbrot set is bounded and connected, so the complement of the Mandelbrot set is simply connected in $\mathbb{C}_{\infty}$.
\end{exbox}

Recall the following facts about functions defined by power series, or sequences of functions:

\begin{enumerate}[1.]
	\item A sequence $(f_n)$ of functions \emph{converges uniformly}\index{uniform convergence} to a function $f$ on some set $S$ if for all $\eps > 0$, there exists $N \in \mathbb{N}$ such that for all $n \geq N$ and for all $x \in S$,
		\[
		|f_n(x) - f(x)| < \eps
		.\]
	\item The uniform limit of continuous functions is continuous.
	\item The \emph{Weierstrass M-test}\index{Weierstrass M-test}: if there exists $M_n \in \mathbb{R}$ for all $n$ such that $0 \leq |f_n(x)| \leq M_n$ for all $x \in S$, then
		\[
			\sum_{n = 1}^{\infty}M_n  \infty \implies \sum_{n = 1}^{\infty}f_n(x) \text{ converges uniformly on $S$ as $N \to \infty$}
		.\]
	\item Let $(c_n)$ be complex numbers, and fix $a \in \mathbb{C}$. Then there exists unique $R \in [0, \infty]$ such that the function
		\[
		z \mapsto \sum_{n = 1}^{\infty}c_n (z - a)^{n}
		\]
		converges absolutely if $|z - a| < R$, and diverges if $|z-a| > R$. If $0 < r < R$, then the series converges uniformly in $D(a, r)$. $R$ is the \emph{radius of convergence}\index{radius of convergence} of the series. We can compute
		\[
			R = \sup\{r \geq 0 \mid |c_n|r^{n} \to 0\}
		,\]
		or
		\[
		R = \frac{1}{\lambda}, \qquad \lambda = \limsup_{n \to \infty} |c_n|^{1/n}
		.\]
\end{enumerate}

\begin{theorem} 
	\[
	f(z) = \sum_{n = 0}^{\infty} c_n(z - a)^{n}
	\]
	is a complex power series with radius of convergence $R$. Then,
	\begin{enumerate}[\normalfont(i)]
		\item $f$ is holomorphic on $D(a, R)$.
		\item $f$ has derivative
			\[
			f'(z) = \sum_{n = 1}^{\infty} n c_n (z - a)^{n-1}
			,\]
			with radius of convergence $R$ about $a$.
		\item $f$ has derivatives of all orders on $D(a, R)$, and $f^{(n)}(a) = n! c_n$.
	\end{enumerate}
\end{theorem}

\begin{proofbox}
	We can let $a = 0$ by change of variables $z \to z - a$. Consider the series
	\[
	\sum_{n = 1}^{\infty}n c_n z^{n-1}
	.\]
	Since $|n c_n| \geq |c_n|$, the radius of convergence of this series is no larger than $R$. If $0 < R_1 < R$, then for $|z| < R_1$, we have
	\[
	|n c_n z^{n-1}| = n |c_n| R_1^{n-1} \frac{|z|^{n-1}}{R_1^{n-1}}
	,\]
	and
	\[
	n \biggl( \frac{|z|}{R_1} \biggr)^{n-1} \to 0
	.\]
	Applying the M-test with $M_n = c_n R_1^{n-1}$, we have the convergence of the series. So the series has radius of convergence $R$.

	Now for $|z|, |w| < R$, we need to consider
	\[
	\frac{f(z) - f(w)}{z - w}
	.\]
	Taking the partial sums,
	\[
		\sum_{n = 0}^{N} c_n \frac{z^{n} - w^{n}}{z - w} = \sum_{n = 0}^{N} c_n \Biggl( \sum_{j = 0}^{n-1} z^{j}w^{n-1-j}\Biggr)
	.\]
	For $|z|, |w| < \rho < R$, we have
	\[
	\biggl| c_n \Biggl( \sum_{j = 0}^{n-1} z^{j} w^{n-1-j} \Biggr) \biggr| \leq |c_n| n \rho^{n-1}
	.\]
	Hence the partial sums converge uniformly on $\{(z, w) \mid |z|, |w| < \rho\}$. So the series converges to a continuous limit on $\{|z|, |w| < R\}$, say $g(z, w)$. When $z \neq w$, we know
	\[
	g(z, w) = \frac{f(z) - f(w)}{z - w}
	.\]
	When $z = w$, we have
	\[
	g(w, w) = \sum_{n = 0}^{\infty} n c_n w^{n-1}
	.\]
	Hence by the continuity of $g$, this proves (i) and (ii). Then (iii) follows from a simple induction.
\end{proofbox}

\begin{corollary}
	Suppose $0 < \rho < R$, where $R$ is the radius of convergence of the complex power series
	\[
	f(z) = \sum_{n = 0}^{\infty}c_n (z - a)^{n}
	,\]
	and $f(z) = 0$ for all $z \in D(a, \rho)$. Then $f \equiv 0$ on $D(a, R)$.
\end{corollary}

\begin{proofbox}
	Since $f \equiv 0$ on $D(a, \rho)$, we have $f^{(n)}(a) = 0$ for all $n$. Hence $c_n = 0$ for all $n$, so $f \equiv 0$ on $D(a, R)$.
\end{proofbox}

\subsection{Exponential and Logarithm}
\label{sub:exponential_and_logarithm}

We define the complex exponential
\[
e^{z} = \exp(z) = \sum_{n = 0}^{\infty} \frac{z^{n}}{n!}
.\]

The complex exponential has the following properties:
\begin{enumerate}[1.]
	\item It has radius of convergence $\infty$, so the function is entire, and we have $\frac{\diff}{\diff z} e^{z} = e^{z}$.
	\item For all $z, w \in \mathbb{C}$, $e^{z+w} = e^{z} e^{w}$, and $e^{z} \neq 0$.

		This follows from setting $F(z) = e^{z+w}e^{-z}$, then taking the derivative,
		\[
		F'(z) = e^{z+w}e^{-z} - e^{z+w}e^{-z} = 0
		,\]
		so $F$ is constant. Since $e^{0} = 1$, $F(z) = e^{w}$, and $e^{z+w} = e^{z}e^{w}$. Since $e^{z}e^{-z} = e^{0} = 1$, $e^{z} \neq 0$.
	\item Let $z = x + iy$. Then $e^{z} = e^{x + iy} = e^{x}e^{iy}$. But $e^{iy} = \cos y + i \sin y$, and note that $|e^{iy}| = 1$, so
		\[
		e^{z} = e^{x} (\cos y + i \sin y)
		,\]
		and $|e^{z}| = e^{x}$, so $e^{z} = 1$ if and only if $x = 0$ and $y = 2 \pi k$ for $k \in \mathbb{Z}$. In fact, for all $w \in \mathbb{C}^{\times}$, there exist infinitely many $z \in \mathbb{C}$ such that $e^{z} = w$, differing by integer multiples of $2 \pi i$.
\end{enumerate}

\begin{definition}
	Let $U \subset \mathbb{C}^{\times}$ be an open set. We say a continuous function $\lambda : U \to \mathbb{C}$ is a \emph{branch of the logarithm}\index{branch of the logarithm} if for all $z \in U$, $\exp(\lambda(z)) = z$.
\end{definition}

\begin{exbox}
	Let $U = \mathbb{C} \setminus \mathbb{R}_{\leq 0}$. Define $\log : U \to \mathbb{C}$ by
	\[
	\log(z) = \ln|z| + i \theta
	,\]
	where $\theta = \arg (z)$, and $\theta \in (-\pi, \pi)$. This is the \emph{principal branch of the logarithm}\index{principal branch of the logarithm}.
\end{exbox}

\begin{proposition}
	$\log(z)$ is holomorphic on $\mathbb{C}\setminus \mathbb{R}_{\leq 0}$ with derivative $\frac{1}{z}$. Moreover, if $|z| < 1$, then
	\[
	\log(1+z) = \sum_{n = 1}^{\infty} \frac{(-1)^{n-1} z^{n}}{n}
	.\]
\end{proposition}

\begin{proofbox}
	As an inverse to $e^{z}$ and by the chain rule, we have $\log z$ is holomorphic with $\frac{\diff}{\diff z}\log z = \frac{1}{z}$, We have
	\[
	\frac{\diff}{\diff z} \log (1 + z) = \frac{1}{z+1} = 1 - z + z^2 - z^3 + z^{4} - \cdots
	,\]
	which is the derivative of
	\[
	\sum_{n = 1}^{\infty} \frac{(-1)^{n-1} z^{n}}{n}
	.\]
	So $\log(1+z)$ agrees with this series up to a constant. Since $\log(1) = 0$, the equality holds.
\end{proofbox}

If $\alpha \in \mathbb{C}$, we can define $z^{\alpha} = \exp(\alpha \log z)$. This gives a definition of $z^{\alpha}$ on $\mathbb{C}\setminus \mathbb{R}_{\leq 0}$. We can compute that $\frac{\diff}{\diff z} z^{\alpha} = \alpha z^{\alpha - 1}$.

It is not necessarily true that $z^{\alpha}w^{\alpha} = (zw)^{\alpha}$. Take $\alpha = \frac{1}{2}$, then
\[
z^{1/2} = \exp \biggl( \frac{1}{2} \log z \biggr) = \exp \biggl( \frac{1}{2} \ln|z| + \frac{1}{2} i \theta \biggr)
,\]
for $\theta \in (-\pi, \pi)$. Hence the argument of $z^{1/2}$ is in $(-\frac{\pi}{2}, \frac{\pi}{2})$.

\subsection{Contour Integration}
\label{sub:contour_integration}

If $f : [a, b] \to \mathbb{C}$ is continuous, we define
\[
\int_{a}^{b} f(t) \diff t = \int_{a}^{b} \Re(f(t)) \diff t + i \int_{a}^{b} \Im(f(t)) \diff t
.\]

\begin{proposition}
	Let $f : [a, b] \to \mathbb{C}$ be continuous. Then,
	\[
	\biggl| \int_{a}^{b} f(t) \diff t \biggr| \leq (b - a)\sup_{a \leq t \leq b} |f(t)|
	,\]
	with equality if and only if $f$ is constant.
\end{proposition}

\begin{proofbox}
	Write $M = \sup_{a \leq t \leq b}|f(t)|$, and $\theta = \arg(\int_{a}^{b} f(t) \diff t)$. Then
	\begin{align*}
		\biggl| \int_{a}^{b}f(t) \diff t \biggr| &= e^{-i \theta} \int_{a}^{b} f(t) \diff t = \int_{a}^{b} e^{-i \theta} f(t) \diff t \\
							 &= \int_{a}^{b} \Re(e^{-i \theta}f(t)) \diff t \\
							 &\leq \int_{a}^{b} |f(t)| \diff t \leq M(b - a).
	\end{align*}
	If we have equality, then $|f(t)| = M$, and $\arg f(t) = \theta$, so $f$ is constant.
\end{proofbox}

\begin{definition}
	Let $\gamma : [a, b] \to \mathbb{C}$ be a $C^{1}$-smooth curve. Then we define the arc-length of $\gamma$ to be
	\[
		\rm{length}(\gamma) = \int_{a}^{b} |\gamma'(t)| \diff t
	.\]
	We say $\gamma$ is \emph{simple}\index{simple} if $\gamma(t_1) = \gamma(t_2) \iff t_1 = t_2$ or $\{t_1, t_2\} = \{a, b\}$. If $\gamma$ is simple, then $\rm{length}(\gamma)$ is the length of the image of $\gamma$.
\end{definition}

\begin{definition}
	Let $f : U \to \mathbb{C}$ be continuous, with $U$ open, and $\gamma : [a, b] \to U$ be a $C^{1}$-smooth curve. Then the integral of $f$ along $\gamma$ is
	\[
	\int_{\gamma} f(z) \diff z = \int_{a}^{b} f(\gamma(t)) \gamma'(t) \diff t
	.\]
\end{definition}

This integral satisfies the following properties:
\begin{enumerate}[1.]
	\item Linearity:
		\[
		\int_{\gamma} c_1 f_1 + c_2 f_2 \diff z = c_1 \int_{\gamma}f_1 \diff z + c_2 \int_{\gamma} f_2 \diff z
		.\]
	\item Additivity: if $a < a' < b$, then
		\[
			\int_{\gamma|_{[a,a']}}f(z) \diff z + \int_{\gamma|_{[a',b]}}f(z) \diff z = \int_{\gamma} f(z) \diff z
		.\]
	\item Inverse path: if $(- \gamma)(t) = \gamma(-t)$ on $[-b, -a]$, then
		\[
		\int_{-\gamma}f(z) \diff z = - \int_{\gamma} f(z) \diff z
		.\]
	\item Independence of parametrization: if $\phi : [a', b'] \to [a, b]$ is $C^{1}$-smooth with $\phi(a') = a$, $\phi(b') = b$ and $\delta = \gamma \circ \phi$, then
		\[
		\int_{\delta}f(z) \diff z = \int_{\gamma} f(z) \diff z
		.\]
		This lets us assume that $\gamma : [0, 1] \to U$.
\end{enumerate}

We can loosen the restriction that $\gamma$ is $C^{1}$-smooth and allow it to be piecewise $C^{1}$-smooth, i.e. there exist $a = a_0 < a_1 < \cdots < a_n = b$ such that $\gamma_i = \gamma|_{[a_{i-1},a_i]}$ is $C^{1}$-smooth. Define then
\[
\int_{\gamma}f(z) \diff z = \sum_{i = 1}^{n} \int_{\gamma_i}f(z) \diff z
.\]

\begin{remark}
	Any piecewise $C^{1}$-smooth curve can be reparametrized to be $C^{1}$: for such a $\gamma$ as above, replace $\gamma_i$ by $\gamma_i \circ h_i$ where $h_i$ is monotonic $C^{1}$-smooth bijection with endpoint derivative $0$.

	So $C^{1}$-smooth paths can have corners, for example
	\[
	\gamma(t) =
	\begin{cases}
		1 + i \sin (\pi t) & t \in [0, \frac{1}{2}], \\
		\sin(\pi t) + i & t \in [\frac{1}{2}, 1].
	\end{cases}
	\]
\end{remark}

We say a ``curve'' is a piecewise $C^{1}$-smooth path, and a ``contour'' is a simple \emph{closed} piecewise $C^{1}$-smooth path, where closed means the endpoints are equal.

\begin{proposition}
	For any continuous $f : U \to \mathbb{C}$ with $U$ open, and any curve $\gamma : [a, b] \to U$,
	\[
		\biggl| \int_{\gamma}f(z) \diff z \biggr| \leq \rm{length}(\gamma) \sup_{z \in \gamma}|f(z)|
	.\]
\end{proposition}

\begin{proofbox}
	
	\begin{align*}
		\biggl| \int_{\gamma}f(z) \diff z \biggr| &= \biggl| \int_{a}^{b} f(\gamma(t)) \gamma'(t) \diff t \biggr| \\
							  &\leq \int_{a}^{b} |f(\gamma(t)) \gamma'(t)| \diff t \\
							  &\leq \sup_{z \in \gamma}|f(z)| \rm{length}(\gamma).
	\end{align*}
\end{proofbox}

\begin{proposition}
	If $f_n : U \to \mathbb{C}$ for $n \in \mathbb{N}$ and $f : U \to \mathbb{C}$ are continuous, and $\gamma : [a, b] \to U$ is a curve in $U$ with $f_n \to f$ uniformly on $\gamma$, then
	\[
	\int_{\gamma}f_n(z) \diff z \to \int_{\gamma}f(z) \diff z
	,\]
	as $n \to \infty$.
\end{proposition}

\begin{proofbox}
	By uniform convergence, $\sup_{z \in \gamma}|f(z) - f_n(z)| \to 0$ as $n \to \infty$. So by the previous proposition,
	\begin{align*}
		\biggl| \int_{\gamma}f(z) \diff z - \int_{\gamma}f_n(z) \diff z \biggr| &\leq \rm{length}(\gamma) \sup_{\gamma}|f - f_n| \\
											&\to 0
	\end{align*}
	as $n \to \infty$.
\end{proofbox}

\begin{exbox}
	Let $f_n(z) = z^{n}$ for $n \in \mathbb{Z}$ on $C^{\times} = U$, and $\gamma : [0, 2\pi] \to U$ with $\gamma(t) = e^{it}$. Then,
	\[
	\int_{\gamma}f_n(z) \diff z = \int_{0}^{2\pi} e^{n i t} i e^{it} \diff t = i \int_{0}^{2\pi} e^{(n+1)it} \diff t =
	\begin{cases}
		2 \pi i & n = -1, \\
		0 & n \neq -1.
	\end{cases}
	\]
\end{exbox}

\begin{theorem}[Fundamental Theorem of Calculus]\index{fundamental theorem of calculus}
	If $f : U \to \mathbb{C}$ is a continuous function on open $U \subset \mathbb{C}$ with $F' = f$ an antiderivative of $f$ in $U$, then for any curve $\gamma : [a, b] \to U$,
	\[
	\int_{\gamma}f(z) \diff z = F(\gamma(b)) - F(\gamma(a))
	.\]
	In particular, if $\gamma$ is closed then $\int_{\gamma} f = 0$.
\end{theorem}

\begin{proofbox}
	\[
	\int_{\gamma}f(z) \diff z = \int_{a}^{b} f(\gamma'(t)) \gamma'(t) \diff t = \int_{a}^{b} (F \circ \gamma)'(t) \diff t = F(\gamma(b)) - F(\gamma(a))
	.\]
\end{proofbox}

Note that in the $z \mapsto z^{-1}$ integral computation, from the fundamental theorem of calculus there does not exist a branch of the logarithm on any neighbourhood around $0$.

The surprising thing is that the converse of this is true.

\begin{theorem}
	Let $f : D \to C$ be continuous on a domain $D$. If $\int_{\gamma}f = 0$ for all closed curves $\gamma$ in $D$, then there exists a holomorphic $F : D \to \mathbb{C}$ with $F' = f$.
\end{theorem}

\begin{proofbox}
	Fix $a \in D$. If $w \in D$, choose any curve $\gamma_w : [0,1] \to D$ with $\gamma_w(0) = a$, $\gamma_w(1) = w$. Define
	\[
	F(w) = \int_{\gamma_w}f(z) \diff z
	.\]

	Find $r_w > 0$ such that $D(w, r_w) \subset D$. For $|h| < r$, let $\delta_h : [0,1] \to D$ be the line segment from $w$ to $w+h$. Then,
	\[
	F(w+h) = \int_{\gamma_{w+h}}f(z) \diff z = \int_{\gamma_w + \delta_h} f(z) \diff z
	.\]
	So
	\[
		F(w+h) = F(w) + \int_{\delta_h}f(z) \diff z = F(w) + hf(w) + \int_{\delta_h}f(z) - f(w) \diff z
	.\]
	Hence
	\begin{align*}
		\biggl| \frac{F(w+h) - F(w)}{h} - f(w) \biggr| &= \biggl|\frac{1}{h} \int_{\delta_h} f(z) - f(w) \diff z \biggr| \\
							       &\leq \frac{\mathrm{length}(\delta_h)}{|h|} \sup_{\delta_h}|f(z) - f(w)| \\
							       &\leq \sup_{z \in D(w, r_w)} |f(z) - f(w)| \to 0,
	\end{align*}
	as $r_w \to 0$. So $F'(w) = f(w)$.
\end{proofbox}

\begin{definition}
	An open subset $U \subset \mathbb{C}$ is \emph{convex}\index{convex} if for all $a, b \in U$, the line segment between $a$ and $b$ is in $U$. $U$ is \emph{starlike}\index{starlike} (or starshaped) if there exists $a \in U$ such that for all $b \in U$, the line segment from $a$ to $b$ is in $U$.
\end{definition}

Note that disks are a subset of convex sets, which are a subset of starlike sets, which are a subset of domains.

We can simplify the previous theorem as follows:

\begin{lemma}
	Suppose $U$ is a starlike domain, and $f : U \to \mathbb{C}$ is continuous with $\int_{\partial T} f(z) \diff z = 0$ for all triangles $T$ in $U$. Then, $f$ has an antiderivative in $U$.
\end{lemma}

\begin{proofbox}
	This is exactly the same as the previous proof, except we stipulate $\gamma_w$ are straight lines from a basepoint $a$.
\end{proofbox}

\begin{theorem}[Cauchy's theorem for Triangles]
	If $f : U \to \mathbb{C}$ is holomorphic on open $U \subset \mathbb{C}$, and $T \subset U$ is a triangle in $U$, then
	\[
	\int_{\partial T}f(z) \diff z = 0
	.\]
\end{theorem}

We adopt the notion that curves are oriented anticlockwise.

\begin{proofbox}
	We can name
	\[
		\biggl| \int_{\partial T}f(z) \diff z \biggr| = I, \qquad L = \mathrm{length}(\partial T)
	.\]
	We subdivide $T$ by bisecting the sides, to obtain $T_1, T_2, T_3$ and $T_4$. Hence, since
	\[
	\partial T_1 + \partial T_2 + \partial T_3 = \partial T - \partial T_4
	,\]
	we find
	\[
	\int_{\partial T}f(z) \diff z = \sum_{i = 1}^{4} \int_{\partial T_i} f(z) \diff z
	.\]
	By the triangle inequality, there exists $i \in \{1, 2, 3, 4\}$ such that 
	\[
	\biggl| \int_{\partial T_i} f(z) \diff z \biggr| \geq \frac{1}{4} I
	.\]
	Call this triangle $T^{(1)}$ and $\mathrm{length}(\partial T^{(1)}) = \frac{L}{2}$.

	Continuing this way, we get
	\[
		T \supset T^{(1)} \supset T^{(2)} \supset T^{(3)} \supset \cdots
	\]
	These triangles have $\mathrm{length}(T^{(n)}) = \frac{L}{2^{n}} \to 0$, and
	\[
	\biggl| \int_{\partial T^{(n)}} f(z) \diff z \biggr| \geq \frac{1}{4^{n}}I
	.\]
	Since the lengths tend to $0$, we get
	\[
		\bigcap_{n = 1}^{\infty}T^{(n)} = \{w\}
	,\]
	a single point. Note that $z, 1$ have holomorphic derivatives. Hence we can bound
	\[
	\frac{1}{4^{n}}I \leq \biggl| \int_{\partial T^{(n)}} f(z) \diff z \biggr| = \biggl| \int_{\partial T^{(n)}} f(z) - f(w) - (z - w)f'(w) \diff z \biggr|
	.\]
	Since $f$ is differentiable at $w$, there $\delta > 0$ such that for all $\eps > 0$,
	\[
	|w - z| < \delta \implies |f(z) - f(w) - (z - w)f'(w)| < \eps|z - w|
	.\]
	So for $n \gg 1$, we have
	\[
	\biggl| \int_{\partial T^{(n)}} f(z) - f(w) - (z-w)f'(w) \diff z \biggr| \leq \frac{L}{2^{n}} \sup_{z \in \partial T^{(n)}}|z - w| \cdot \eps
	.\]
	So
	\[
	\frac{I}{4^{n}} \leq \frac{L}{2^{n}} \cdot \frac{L}{2^{n}} \eps,\qquad I \leq L^2 \eps
	.\]
	Letting $\eps \to 0$, we get $I = 0$.
\end{proofbox}

\begin{theorem}
	Let $S \subset U$ be a finite set and $f : U \to \mathbb{C}$ be continuous on $U$ and holomorphic on $U \setminus S$. Then $\int_{\partial T} f = 0$ for all triangles $T \in U$.
\end{theorem}

\begin{proofbox}
	Using the triangle subdivision, assume that $S = \{a\}$, for $a \in T$. If $a \in T' \subset T$ for another triangle $T'$, then by the triangular subdivision and the previous theorem,
	\[
	\int_{\partial T} f = \int_{\partial T'} f
	,\]
	since $f$ is holomorphic on $T \setminus T'$. Hence,
	\begin{align*}
		\biggl| \int_{\partial T}f(z) \diff z \biggr| &= \biggl| \int_{\partial T'} f(z) \diff z \biggr| \leq \mathrm{length}(T') \sup_{\partial T'}|f| \\
							      &\leq \mathrm{length}(T') \sup_{T}|f|,
	\end{align*}
	so letting $\mathrm{length}(T') \to 0$, we have $\int_{\partial T} f = 0$.
\end{proofbox}

\begin{theorem}[Cauchy's theorem in a Disk]\index{Cauchy's theorem}
	Let $D$ be any disk (or any starlike domain), and $f : D \to \mathbb{C}$ a continuous function, holomorphic away from at most a finite set of points in $D$. Then, $\partial_{\gamma} f = 0$ for any closed curve $\gamma$ in $D$.
\end{theorem}

\begin{proofbox}
	By our previous theorem and the converse of FTC for starlike domains, there exists an antiderivative $F$ for $f$ in $D$. So by the fundamental theorem of calculus, Cauchy's theorem follows.
\end{proofbox}


\newpage

\printindex

\end{document}
