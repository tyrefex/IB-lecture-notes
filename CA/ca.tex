%%% TO DO %%%
% implement siunitx package and change units

\documentclass[12pt]{article}

\usepackage{ishn}

\makeindex[intoc]

\begin{document}

\hypersetup{pageanchor=false}
\begin{titlepage}
	\begin{center}
		\vspace*{1em}
		\Huge
		\textbf{IB Complex Analysis}

		\vspace{1em}
		\large
		Ishan Nath, Lent 2023

		\vspace{1.5em}

		\Large

		Based on Lectures by Prof. Holly Krieger

		\vspace{1em}

		\large
		\today
	\end{center}
	
\end{titlepage}
\hypersetup{pageanchor=true}

\tableofcontents

\newpage

\section{Complex Differentiation}
\label{sec:complex_differentiation}

Our goal in this course is to study the theory of complex-valued differentiable functions in one complex variable. Example include:
\begin{itemize}
	\item Polynomials $p(z) = a_dz^{d} + \cdots + a_1 z + a_0$, with coefficients in $\mathbb{R}, \mathbb{Q}, \mathbb{Z}$ or $\mathbb{C}$.
	\item The infinite series
		\[
		\sum_{n = 1}^{\infty} \frac{1}{n^{z}}
		,\]
		which we showed convergence for $z$ having real part greater than $1$.
	\item Harmonic functions\index{harmonic function} $u(x, y) : \mathbb{R}^2 \to \mathbb{R}$, $u_{xx} + u_{yy} = 0$.
\end{itemize}

In this course, we make the convention that $\theta = \arg(z) \in [0, 2\pi)$.

\subsection{Basic Notions}
\label{sub:basic_notions}

\begin{itemize}
	\item $U \subset \mathbb{C}$ is \emph{open}\index{open} if for all $u \in U$, there exists $\eps > 0$ such that
		\[
			\Delta(x, \eps) = \{z \in \mathbb{C} \mid |z - u| < \eps\} \subset U
		.\]
	\item A \emph{path}\index{path} in $U \subset \mathbb{C}$ is a continuous map $\gamma : [a, b] \to U$. We say the path is $C^{1}$ if $\gamma'$ exists and is continuous (we take one-sided derivatives at the endpoints).

		$\gamma$ is \emph{simple}\index{simple} if it is injective.
	\item $U \subset \mathbb{C}$ is \emph{path-connected}\index{path-connected} if for all $z, w \in U$, there exists a path in $U$ with endpoints at $z, w$.
\end{itemize}
\begin{remark}
	If $U$ is open, and $z, w \in U$ are connected by a path $\gamma$ in $U$, then there exists a path $\gamma$ in $U$ connected $z, w$ consisting of finitely many horizontal and vertical segments.
\end{remark}

\begin{definition}
	A \emph{domain}\index{domain} is a non-empty, open, path-connected subset of $\mathbb{C}$.
\end{definition}

\begin{definition}
	\begin{enumerate}[(i)]
		\item[]
		\item $f : U \to \mathbb{C}$ is \emph{differentiable}\index{complex differentiable} at $u \in U$ if
			\[
			f'(u) = \lim_{z \to u} \frac{f(z) - f(u)}{z - u}
			\]
			exists.
		\item $f : U \to \mathbb{C}$ is \emph{holomorphic}\index{holomorphic} at $u \in U$ if there exists $\eps > 0$ such that $f$ is differentiable at $z$, for all $z \in \Delta(u, \eps)$. We may also call such a function \emph{analytic}\index{analytic}.
		\item $f : \mathbb{C} \to \mathbb{C}$ is \emph{entire}if it is holomorphic everywhere.
	\end{enumerate}
\end{definition}

\begin{remark}
	All differentiation rules (sum, products, ...) in $\mathbb{R}$ hold, by the same proofs.
\end{remark}

Identifying $\mathbb{C}$ with $\mathbb{R}^2$, we may write $f : U \to \mathbb{C}$ as $f(x+iy) = u(x,y) + iv(x, y)$, where $u, v$ are the real and imaginary parts of $f$.

From analysis and topology, recall that $u : U \to \mathbb{R}$ as a function of two real variables if $(\mathbb{R}^2)$ differentiable at $(c, d) \in \mathbb{R}^2$with $Du|_{(c,d)} = (\lambda, \mu)$ if
\[
	\frac{u(x,y) - u(c,d) - [\lambda(x-c) + \mu(y-d)]}{\sqrt{(x-c)^2 + (y-d)^2}} \to 0
,\]
as $(x, y) \to (c, d)$. However, \textbf{this is a weaker condition} than differentiability over $\mathbb{C}$.

\begin{proposition}[Cauchy-Riemann equations]\index{Cauchy-Riemann equations}
	Let $f : U \to \mathbb{C}$ on an open set $U \subset \mathbb{C}$. Then $f$ is differentiable at $w = c+id \in U$ if and only if, writing $f = u + iv$, we have $u, v$ are $\mathbb{R}^2$-differentiable at $(c, d)$, and
	\begin{align*}
		u_{x} &= v_y, & u_y &= -v_x.
\end{align*}
\end{proposition}

\begin{proofbox}
	$f$ is differentiable at $w$ if and only if $f'(w) = p + iq$ exists, so
	\[
	\lim_{z \to w}\frac{f(z) - f(w) - (z - w)(p + iq)}{|z - w| = 0}
	.\]
	Writing $f = u + iv$ and considering the real and imaginary parts in the quotient above, this holds if and only if
	\[
		\lim_{(x, y) \to (c, d)} \frac{u(x, y) - u(c, d) - [p(x-c) - q(y-d)]}{\sqrt{(x - c)^2 + (y - d)^2}} = 0
	,\]
	and
	\[
		\lim_{(x, y) \to (c, d)} \frac{v(x, y) - v(c, d) - [q(x-c) + p(y-d)]}{\sqrt{(x-c)^2+(y-d)^2}} = 0
	.\]
	This holds if and only if $u, v$ are $\mathbb{R}^2$-differentiable at $(c, d)$, and $u_x = v_y$, $u_y = -v_x$.
\end{proofbox}

\begin{remark}
	If the partial $u_x, u_y, v_x, v_y$ exist and are continuous on $U$, then $u, v$ are differentiable on $U$. So it suffices to check the partials exist and are continuous, and the Cauchy-Riemann equations hold to deduce complex differentiability.
\end{remark}

\begin{exbox}
	\begin{enumerate}
		\item Take $f(z) = \overline{z}$. Then $f$ has $u(x, y) = x$ and $v(x, y) = -y$, so $u_x = 1$, $v_y = -1$. So $f(z) = \overline{z}$ is not holomorphic or differentiable anywhere.
		\item Any polynomial $p(z) = a_d z^{d} + \cdots + a_1z + a_0$, with $a_i \in \mathbb{C}$ is entire.
		\item Rational function\index{rational functions}, which are quotients of polynomials $\frac{p(z)}{q(z)}$ are holomorphic on the open set $\mathbb{C}\setminus\{\text{zeroes of } q\}$.
	\end{enumerate}
\end{exbox}

Note that $f = u + iv$ satisfying the Cauchy-Riemann equations at a point does not mean it is differentiable at that point.

Some proofs in regular analysis have natural extensions to complex analysis. For example, if $f : U \to \mathbb{C}$ on a domain $U$ with $f'(z) = 0$ on $U$, then $f$ is constant on $U$.

Now we ask: why are we interested in complex analysis?
\begin{itemize}
	\item Unlike $\mathbb{R}^2$ differentiable functions, holomorphics functions are very constrained. For example, if $f$ is entire and bounded (so $|f(z)| < M$ for all $z \in \mathbb{C}$), then $f$ is constant. Contrast with $\sin$, for example.
	\item We will see that $f$ holomorphic on a domain $U$ has holomorphic derivative on $U$. This implies that $f$ is infinitely differentiable, as are $u$ and $v$.
\end{itemize}
In particular, we can differentiate the Cauchy-Riemann equations to get
\[
u_{xx} = v_{yx} = v_{xy} = -u_{yy}
,\]
so $u_{xx} + u_{yy} = 0$, and similarly $v_{xx} + v_{yy} = 0$. Hence the real and imaginary parts of a holomorphic function are harmonic.

Let $f : U \to \mathbb{C}$ be a holomorphic function on an open set $U_1$ and $w \in U$ with $f'f(w) \neq 0$. We want to look at the geometric behaviour of $f$ at $w$.

In fact, we claim $f$ is \emph{conformal}\index{conformal} at $w$. Let $\gamma_1, \gamma_2$ be $C^{1}$-paths through $w$, say $\gamma_1, \gamma_2 : [-1,1] \to U_1$, such that $\gamma_1(0) = \gamma_2(0) = w$, and $\gamma_i'(0) \neq 0$. If we write $\gamma_j(t) = w + r_j(t) = e^{i \theta_j(t)}$, then we have
\[
\arg(\gamma_j'(z)) = \theta_j(0)
,\]
and the argument of the image line is
\[
\arg((f \circ \gamma_j)'(0)) = \arg(\gamma_j'(0) f'(\gamma_j(0))) = \arg(\gamma_j'(0)) + \arg(f'(w)) + 2 \pi n
,\]
where crucially we use $\gamma_j'(0) f'(\gamma_j(0)) \neq 0$, so the direction of $\gamma_j$ at $w$ under the application of $f$ is rotated by $\arg(f'(w))$. This is independent of $\gamma_j$. Since the angle between $\gamma_1$ and $\gamma_2$ is the difference of the arguments $f$ preserves the angle. This is what it means to be conformal.

\begin{definition}
	Let $U, V$ be domains in $\mathbb{C}$. A map $f : U \to V$ is a \emph{conformal equivalence}\index{conformal equivalence} of $U$ and $V$ if $f$ is a bijective holomorphic map with $f'(z) \neq 0$, for all $z \in U$.
\end{definition}

\begin{remark}
	\begin{enumerate}[1.]
		\item[]
		\item Using the real inverse function theorem, one can show if $f : U \to V$ is a holomorphic bijection of open sets with $f'(z) \neq 0$ for all $z \in U$, then the inverse of $f$ is also holomorphic, so also conformal by the chain rule. So conformally equivalent domains are equal from the perspective of the functions $f$.
		\item We will later see than being injective and holomorphic on a domain implies $f'(z) \neq 0$ for all $z \in U$, so this requirement is redundant.
	\end{enumerate}	
\end{remark}

\begin{exbox}
	\begin{enumerate}[1.]
		\item Any change of coordinates: on $\mathbb{C}$, take $f(z) = az + b$, for $a \neq 0$ and $b$, which is a conformal equivalence $\mathbb{C} \to \mathbb{C}$. More generally, a M\"{o}bius map
			\[
			f(z) = \frac{az + b}{cz + d},
			\]
			for $ad - bc \neq 0$, is a conformal equivalence from the Riemann sphere to itself. This can eb seen as adding a point at infinity to make a sphere $\mathbb{C}_{\infty}$ (or gluing two copies of the unit disc with coordinates $z$ and $\frac{1}{z}$).

			If $f : \mathbb{C}_{\infty} \to \mathbb{C}_{\infty}$ is continuous, then
			\begin{itemize}
				\item if $f(\infty) = \infty$, then $f$ is holomorphic at $\infty$ if and only if $g(z) = \frac{1}{f(\frac{1}{z})}$ is holomorphic at $0$.
				\item If $f(\infty) \neq \infty$, then $f$ is homolorphic at $\infty$ if and only if $f(\frac{1}{z})$ is holomorphic at $0$.
				\item If $f(a) = \infty$ for $a \in \mathbb{C}$, then $f$ is holomorphic at $a$ if and only if $\frac{1}{f(z)}$ is holomorphic at $a$.
			\end{itemize}
			We can then think of M\"{o}bius maps as change of coordinates for the sphere.

			Choosing $z_1 \to 0$, $z_2 \to \infty$, $z_3 \to 1$ defined a M\"{o}bius map
			\[
			f(z) = \frac{z - z_1}{z - z_2} \cdot \frac{z_3 - z_2}{z_3 - z_1}
			,\]
			for distinct $z_1, z_2, z_3 \in \mathbb{C}$.
		\item For $n \in \mathbb{N}$, $f(z) = z^{n}$ is a conformal equivalence from the sector $\{z \in \mathbb{C}^{\times} \mid 0 < \arg z < \frac{\pi}{n}\}$ to the upper half plane $\mathbb{H} = \{z \in \mathbb{C} \mid \Im z > 0\}$.
		\item The M\"{o}bius map $f(z) = \frac{z-i}{z+i}$ is a conformal equivalence between $\mathbb{H}$ and $D(0,1)$. We can compute $f'(z) \neq 0$ on $\mathbb{H}$, and
			\begin{align*}
				z \in \mathbb{H} & \iff |z-i| < |z+i| \iff |f(z)) < 1.
			\end{align*}
			Note that $f^{-1}(w) = -i \frac{w+1}{w-1}$.
		\item We can use these examples to write down conformal equivalences. Let $U_1$ be the upper half semicircle, and $U_2$ the lower half plane. Considering $g(z) = \frac{z+1}{z-1}$, we know that sends $D(0,1)$ to the left half-plane, so it sends $U_1$ to the upper left quadrant.

			Then, the upper left quadrant if mapped by the squaring map to $U_2$. So $f(z) = (\frac{z+1}{z-1})^2$ is a conformal equivalence from $U_1 \to U_2$.
	\end{enumerate}
\end{exbox}

These are all examples of the deep \emph{Riemann mapping theorem}\index{Riemann mapping theorem}:

\begin{theorem}[Riemann mapping theorem]
	Let $U \subset \mathbb{C}$ be a proper domain which is simply connected. Then there exists a conformal equivalence between $U$ and $D(0,1)$.
\end{theorem}

Here, \emph{simply connected}\index{simply connected} means a subset $U \subset \mathbb{C}$ which is path-connected, and contractible: any loop in $U$ can be contracted to a point. So any continuous path $\gamma : S^{1} \to U$ extends to a continuous map $\hat \gamma : D(0,1) \to U_1$ with $\hat \gamma|_{S_1} = \gamma$.

In fact any domain bounded by a simple closed curve is simply connected, so all of these are conformally equivalent to $D(0,1)$.

\begin{exbox}
	We look at a domains in the Riemann sphere, with bounded and connected complement. This is simply connected as a subset of $\mathbb{C}_{\infty}$.

	Now, the Mandelbrot set is bounded and connected, so the complement of the Mandelbrot set is simply connected in $\mathbb{C}_{\infty}$.
\end{exbox}



\newpage

\printindex

\end{document}
