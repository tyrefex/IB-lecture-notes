\documentclass[12pt]{article}
\usepackage{amsmath}
\usepackage[a4paper]{geometry}
\usepackage{fancyhdr}
\usepackage{tikz}
\usepackage{amssymb}
\usepackage{graphicx}
\usepackage{amsthm}
\usepackage{import}
\usepackage{xifthen}
\usepackage{pdfpages}
\usepackage{transparent}
\usepackage{adjustbox}
\usepackage[shortlabels]{enumitem}
\usepackage{parskip}
\makeatletter
\newcommand{\@minipagerestore}{\setlength{\parskip}{\medskipamount}}
\makeatother
\usepackage{imakeidx}

\DeclareMathOperator{\Ker}{Ker}
\DeclareMathOperator{\Img}{Im}
\DeclareMathOperator{\rank}{rank}
\DeclareMathOperator{\nullity}{null}
\DeclareMathOperator{\spn}{span}
\DeclareMathOperator{\tr}{tr}
\DeclareMathOperator{\adj}{adj}
\DeclareMathOperator{\id}{id}
\DeclareMathOperator{\Sym}{Sym}
\DeclareMathOperator{\Orb}{Orb}
\DeclareMathOperator{\Stab}{Stab}
\DeclareMathOperator{\ccl}{ccl}
\DeclareMathOperator{\Aut}{Aut}
\DeclareMathOperator{\Syl}{Syl}
\DeclareMathOperator{\sgn}{sgn}
\DeclareMathOperator{\Fit}{Fit}
\DeclareMathOperator{\Ann}{Ann}


\newcommand{\incfig}[1]{%
	\def\svgwidth{\columnwidth}
	\import{./figures/}{#1.pdf_tex}
}

\setlength\parindent{0pt}

\newcommand{\course}{GRM }
\newcommand{\lecnum}{}

\newtheorem{theorem}{Theorem}[section]
\newtheorem{corollary}{Corollary}[section]
\newtheorem{lemma}{Lemma}[section]
\newtheorem{proposition}{Proposition}[section]

\theoremstyle{definition}
\newtheorem{definition}{Definition}[section]

\theoremstyle{remark}
\newtheorem*{remark}{Remark}

\pagestyle{fancy}
\fancyhf{}
\rhead{\leftmark}
\lhead{Page \thepage}
\setlength{\headheight}{15pt}

\newcommand{\mapsfrom}{\mathrel{\reflectbox{\ensuremath{\mapsto}}}}

\makeindex[intoc]

\usepackage{hyperref}
\hypersetup{
    colorlinks,
    citecolor=black,
    filecolor=black,
    linkcolor=black,
    urlcolor=black
    pdfauthor={Ishan Nath}
}

\begin{document}

\hypersetup{pageanchor=false}
\begin{titlepage}
	\begin{center}
		\vspace*{1em}
		\Huge
		\textbf{IB Groups, Rings \& Modules}

		\vspace{1em}
		\large
		Ishan Nath, Lent 2022

		\vspace{1.5em}

		\Large

		Based on Lectures by Dr. Rong Zhou

		\vspace{1em}

		\large
		\today
	\end{center}
	
\end{titlepage}
\hypersetup{pageanchor=true}

\tableofcontents

\newpage

\part{Groups}%
\label{prt:groups}

\section{Revision and Basic Theory}%
\label{sec:revision_and_basic_theory}

\subsection{Basic Definitions}%
\label{sub:basic_definitions_groups}

\begin{definition}
	A \textbf{group}\index{group} is a pair $(G, \cdot)$, where $G$ is a set and $\cdot : G \times G \to G$ is a binary operation satisfying:
	\begin{itemize}
		\item $a \cdot (b \cdot c) = (a \cdot b) \cdot c$.
		\item There exists $e \in G$ such that $e \cdot g = g \cdot e = g$ for all $g \in G$.
		\item For all $g \in G$, there exists $g^{-1} \in G$ such that $g \cdot g^{-1} =  g^{-1} \cdot g = e$.
	\end{itemize}
\end{definition}

\begin{remark}
\begin{enumerate}[label = (\roman*)]
	\item[]
	\item In addition to check $\cdot$ is well defined, we also need to check \textbf{closure}, i.e. $g \cdot h \in G$.
	\item If using additive (or multiplicative) notation, we often write 0 (or 1) for the identity.
\end{enumerate}
\end{remark}

\begin{definition}
	A subset $H \subseteq G$ is a \textbf{subgroup}\index{subgroup} (written $H \leq G$) if it is closed under $\cdot$, and $(H, \cdot)$ is a group.
\end{definition}

\begin{remark}
	A non-empty subset $H$ of $G$ is a subgroup if $g, h \in H$ implies $gh^{-1} \in H$.
\end{remark}

\textbf{Examples:} 

\begin{enumerate}[label = (\roman*)]
	\item $(\mathbb{Z}, +) \leq (\mathbb{Q}, +) \leq (\mathbb{R}, +) \leq (\mathbb{C}, +)$.
	\item Cyclic and dihedral groups. $C_n$ is the cyclic group of order $n$, and $D_{2n}$ is the dihedral group of order $2n$, where $C_n \leq D_{2n}$.\index{cyclic group}\index{dihedral group}
	\item \textbf{Abelian groups} satisfy $gh = hg$ for every $g, h \in G$.
	\item Symmetric and alternating groups. $S_n$ is the permutations of $\{1, \ldots, n\}$ under composition, and $A_n \leq S_n$ is the subgroup of even permutations.\index{symmetric group}\index{alternating group}
	\item The quaternion group $Q_8 = \{\pm 1, \pm i, \pm j, \pm k\}$.
	\item General and special linear groups. $GL_n(\mathbb{R})$ is the set of invertible $n \times n$ matrices over $\mathbb{R}$ under matrix multiplication, and $SL_n(\mathbb{R}) \leq GL_n(\mathbb{R})$ is the subgroup of matrices with determinant 1.
\end{enumerate}

\begin{definition}
	The \textbf{direct product}\index{group!direct product} of groups $G$ and $H$ is the set $G \times H$ with operation
	\[
		(g_1, h_1) \cdot (g_2, h_2) = (g_1 \cdot_G g_2, h_1 \cdot_H h_2)
	.\]
\end{definition}

Let $H \leq G$. The left cosets\index{cosets} of $H$ in $G$ are the sets $gH = \{gh \mid h \in H\}$. These partition $G$, and each has the same cardinality.

\begin{theorem}[Lagrange's theorem]\index{Lagrange's theorem}
	Let $G$ be a finite group and $H \leq G$. Then $|G| = |H| \cdot [G:H]$, where $[G:H]$ is the number of left cosets of $H$, known as the \textbf{index} of $H$ in $G$.
\end{theorem}

\begin{remark}
	We can also copy this with right cosets. Lagrange's implies the number of left cosets equals the number of right cosets.
\end{remark}

\begin{definition}
	Let $g \in G$. If there exists $n \geq 1$ such that $g^{n} = 1$, then the least such $n$ is the order of $g$. Otherwise $g$ has infinite order.\index{order}
\end{definition}

\begin{remark}
	If $g$ has order $d$, then;
	\begin{enumerate}[label = (\roman*)]
		\item $g^{n} = 1 \implies d \mid n$.
		\item $\{e, g, \ldots, g^{d - 1}\} \leq G$ and so if $G$ is finite, then $d \mid |G|$, by Lagrange.
	\end{enumerate}
\end{remark}
 
\begin{definition}
	A subgroup $N \leq G$ is \textbf{normal} if $g^{-1}Ng = N$ for all $g \in G$. We write $N \lhd G$.\index{normal subgroup}
\end{definition}

\begin{proposition}
	If $H \lhd G$ then the set $G/H$ of left cosets of $H$ in $G$ is a group, called the \textbf{quotient group}, with operation\index{quotient group}
	\[
	g_1H \cdot g_2H = g_1g_2H
	.\]
\end{proposition}

\begin{adjustbox}{minipage = \columnwidth - 25.5pt, margin=1em, frame=1pt, margin=0em}
\textbf{Proof:} We need to check $\cdot$ is well defined. Suppose $g_1H = g_1'H$ and $g_2H = g_2'H$. Then $g_1' = g_1h_1$ and $g_2' = g_2h_2$ for some $h_1, h_2 \in H$. Therefore,
\begin{align*}
	g_1'g_2' &= g_1h_1g_2h_2 = g_1g_2(g_2^{-1}h_1g_2)h_2 = g_1g_2 h,
\end{align*}
where $h = (g_2^{-1} h_1 g_2) h_2 \in H$ by our normality criterion. Therefore, $g_1'g_2'H = g_1g_2H$.

Associativity is inherited from $G$, the identity is $H = eH$, and the inverse of $gH$ is $g^{-1}H$.

\end{adjustbox}

\subsection{Homomorphisms}%
\label{sub:homomorphisms}

\begin{definition}
	For groups $G, H$, a function $\phi : G \to H$ is a \textbf{group homomorphism} if $\phi(g_1g_2) = \phi(g_1)\phi(g_2)$.\index{group!homomorphism}
\end{definition}
It has kernel
\[
	\Ker(\phi) = \{g \in G \mid \phi(g) = e_H\} \leq G
\]
and image
\[
	\Img(\phi) = \{\phi(g) \mid g \in G\} \leq H
.\]
In fact, $\Ker(\phi) \lhd G$, since if $h \in \Ker(\phi)$ and $g \in G$, then
\[
	\phi(g^{-1}hg) = \phi(g^{-1})\phi(h)\phi(g) = \phi(g^{-1})\phi(g) = \phi(g^{-1}g) = \phi(e_G) = e_H
.\]
Therefore $g^{-1}hg \in \Ker(\phi)$.

\begin{definition}
	An isomorphism of groups is a group homomorphism that is also a bijection. We say $G$ and $H$ are isomorphic (written $G \cong H$) if there exists an isomorphism $\phi : G \to H$.
\end{definition}
It can be shown that $\phi^{-1} : H \to G$ is also a group homomorphism.

\subsection{Isomorphism Theorems}%
\label{sub:isomorphism_theorems_groups}\index{group!isomorphism theorems}

\begin{theorem}[First Isomorphism Theorem]
	Let $\phi : G \to H$ be a group homomorphism. Then $\Ker(\phi) \lhd G$, and
	\[
		G / \Ker(\phi) \cong \Img (\phi)
	.\]
\end{theorem}

\begin{adjustbox}{minipage = \columnwidth - 25.5pt, margin=1em, frame=1pt, margin=0em}
\textbf{Proof:} Let $K = \Ker(\phi)$. We have already shown $K$ is normal in $G$, so we only need to show $G/K \cong \Img(\phi)$. Define
\begin{align*}
	\Phi : G/K &\to \Img(\phi) \\
	gK &\mapsto \phi(g)
\end{align*}
We will show $\Phi$ is an isomorphism. To show it is well-defined and injective,
\begin{align*}
	g_1K = g_2K &\iff g_2^{-1}g_1 \in K \iff \phi(g_2^{-1}g_1) = 1 \\
		    &\iff \phi(g_1) = \phi(g_2).
\end{align*}
\end{adjustbox}

\begin{adjustbox}{minipage = \columnwidth - 25.5pt, margin=1em, frame=1pt, margin=0em}
To show it is a group homomorphism,
\begin{align*}
	\Phi(g_1Kg_2K) = \Phi(g_1g_2K) = \phi(g_1g_2) = \phi(g_1)\phi(g_2) = \Phi(g_1K)\Phi(g_2K).
\end{align*}
Finally, it is surjective as if $x \in \Img(\phi)$, then $x = \phi(g)$ for some $g \in G$, so $x = \Phi(gK)$.

\end{adjustbox}

\textbf{Example:} Consider $\phi : \mathbb{C} \to \mathbb{C}^{\times} = \mathbb{C} \setminus \{0\}$, given by $z \mapsto e^{z}$. Then,
\begin{align*}
	\Ker(\phi) &= \{z \in \mathbb{C} \mid e^{z} = 1\} = 2 \pi i \mathbb{Z}, \\
	\Img(\phi) &= \mathbb{C}^{\times}.
\end{align*}
Thus $\mathbb{C} / 2 \pi i \mathbb{Z} \cong \mathbb{C}^{\times}$.

\begin{theorem}[Second Isomorphism Theorem]
	Let $H \leq G$ and $K \lhd G$. Then $HK = \{hk \mid h \in H, k \in K\} \leq G$, $H \cap K \lhd H$, and
	\[
	HK / K \cong H / H \cap K
	.\] 
\end{theorem}

\begin{adjustbox}{minipage = \columnwidth - 25.5pt, margin=1em, frame=1pt, margin=0em}
\textbf{Proof:} Let $h_1k_1, h_2k_2 \in HK$. Then,
\begin{align*}
	h_1k_1(h_2k_2)^{-1} = h_1k_1k_2^{-1}h_2^{-1} = h_1h_2^{-1}(h_2k_1k_2^{-1}h_2^{-1}) \in HK,
\end{align*}
since $K \lhd G$. Thus $HK \leq G$. Consider
\begin{align*}
	\phi : H &\to G/K \\
	h &\mapsto hK
\end{align*}
This is the composition of $H \to G$ and quotient map $G \to G/K$, thus $\phi$ is a group homomorphism.
\begin{align*}
	\Ker(\phi) &= \{h \in H \mid hK = K\} = H \cap K \lhd H, \\
	\Img(\phi) &= \{hK \mid h \in H\} = HK/K.
\end{align*}
Therefore we are done by the first isomorphism theorem.

\end{adjustbox}

\begin{remark}
	Suppose $K \lhd G$. There is a bijection
	\begin{align*}
		\{\text{subgroups of } G/K\} &\mapsto \{\text{subgroups of } G \text{ containing }K\}, \\
		X &\mapsto \{g \in G : gK \in X\}, \\
		H/K &\mapsto H.
	\end{align*}
	This map takes normal subgroups of $G/K$ to normal subgroups of $G$ containing $K$.
\end{remark}

\begin{theorem}[Third Isomorphism Theorem]
	Let $K \leq H \leq G$ be normal subgroups of $G$. Then
	\[
		(G/K)/(H/K) = G/H
	.\]
\end{theorem}

\begin{adjustbox}{minipage = \columnwidth - 25.5pt, margin=1em, frame=1pt, margin=0em}
\textbf{Proof:} Define
\begin{align*}
	\phi : G/K &\to G/H \\
	gK &\mapsto gH
\end{align*}
If $g_1K = g_2K$, then $g_2^{-1}g_1 \in K \leq H$, so $g_1H = g_2H$. Therefore $\phi$ is well-defined, and is a surjective group homomorphism with kernel $H/K$, so we are finished by the first isomorphism theorem.

\end{adjustbox}

If $K \lhd G$, then studying the groups $K$ and $G/K$ gives some information about $G$. This approach is not always available.

\subsection{Simple Groups}%
\label{sub:simple_groups}

\begin{definition}\index{simple group}
	A group $G$ is \textbf{simple} if it has no non-trivial proper normal subgroups.
\end{definition}

We do not consider the trivial group to be a simple group.

\begin{lemma}
	Let $G$ be an abelian group. $G$ is simple iff $G \cong C_p$ for some prime $p$.
\end{lemma}

\begin{adjustbox}{minipage = \columnwidth - 25.5pt, margin=1em, frame=1pt, margin=0em}
\textbf{Proof:} Let $H \leq C_p$. Lagrange's theorem says $|H| \mid |C_p| = p$, so $|H| = 1$ or $p$. But this implies $H$ is either the trivial group, or $C_p$, thus $C_p$ is simple.

\end{adjustbox}

Let $1 \neq g \in G$. Then $G$ contains the subgroup $\langle g \rangle$, which is normal since $G$ is normal. Since $G$ is simple, $\langle g \rangle = G$. If $G$ is infinite, then $G \cong (\mathbb{Z}, +)$, but then $2 \mathbb{Z}$ is a normal subgroup.

Otherwise $G \cong C_n$ for some $n$. If $m \mid n$, then $g^{n/m}$ generates a subgroup of order $m$. So if $C_n$ is simple, then the only factors of $n$ can be $1$ and $n$, which implies $n$ is prime.

\newpage

\begin{lemma}\index{composition series}
	If $G$ is a finite group, then $G$ has a composition series
	\[
	1 \cong G_0 \lhd G_1 \lhd \ldots \lhd G_n = G
	,\]
	with each quotient $G_i / G_{i-1}$ simple. (Note $G_i$ may not be normal in $G$).
\end{lemma}

\begin{adjustbox}{minipage = \columnwidth - 25.5pt, margin=1em, frame=1pt, margin=0em}
\textbf{Proof:} We induct on $|G|$. If $|G| = 1$, we are done. Otherwise, if $|G| > 1$, let $G_{m-1}$ be a normal subgroup of largest possible order not equal to $G$. Since normal subgroups of $G/G_{m-1}$ biject with normal subgroups of $G$ containing $G_{m-1}$, we get that $G/G_{m-1}$ is simple, and we can induct.

\end{adjustbox}

\newpage

\section{Group Actions}%
\label{sec:group_actions}

\subsection{Definitions and Permutation Groups}%
\label{sub:definitions_and_permutation_groups}

\begin{definition}
	For a set $X$, let $\Sym(X)$ be the group of all bijections $X \to X$ under composition.
\end{definition}

\begin{definition}\index{permutation group}
	A group $G$ is a permutation group of degree $n$ if $G \leq \Sym(X)$ with $|X| = n$.
\end{definition}

For example, $S_n$ is a permutation group of degree $n$, as is $A_n$ and $D_{2n}$.

\begin{definition}\index{group!action}
	An action of a group $G$ on a set $X$ is a function $\ast : G \times X \to X$, satisfying
	\begin{enumerate}[label = (\roman*)]
		\item $e \ast x = x$,
		\item $(g_1g_2) \ast x = g_1 \ast (g_2 \ast x)$.
	\end{enumerate}
\end{definition}

\begin{proposition}
	An action of a group $G$ on a set $X$ is equivalent to specifying a group homomorphism $\phi : G \to \Sym(X)$.
\end{proposition}

\begin{adjustbox}{minipage = \columnwidth - 25.5pt, margin=1em, frame=1pt, margin=0em}
\textbf{Proof:} For each $g \in G$, let
\begin{align*}
	\phi_g : X &\to X \\
	x &\mapsto gx
\end{align*}
We have
\[
\phi_{g_1g_2}(x) = (g_1g_2) \ast x = g_1 \ast (g_2 \ast x) = \phi_{g_1} (\phi_{g_2} (x)).
\]
Thus $\phi_{g_1g_2} = \phi_{g_1} \circ \phi_{g_2}$. In particular, $\phi_{g} \circ \phi_{g^{-1}} = \phi_{g^{-1}} \circ \phi_{g} = \phi_{e} = \id_{X}$. Thus $\phi_g \in \Sym(X)$. Define
\begin{align*}
	\Phi : G &\to \Sym(X) \\
	g &\mapsto \phi_g
\end{align*}
This is a group homomorphism from the above. Conversely, let $\phi : G \to \Sym(X)$ be a group homomorphism. Define
\begin{align*}
	\ast : G \times X &\to X \\
	(g, x) &\mapsto \phi(g)(x)
\end{align*}
\end{adjustbox}

\begin{adjustbox}{minipage = \columnwidth - 25.5pt, margin=1em, frame=1pt, margin=0em}
Then $e \ast x = \phi(e)(x) = \id(x) = x$, and
\[
	(g_1 g_2) \ast x = \phi(g_1g_2)(x) = \phi(g_1)(\phi(g_2)(x)) = g_1 \ast (g_2 \ast x)
.\]
\end{adjustbox}


\begin{definition}
	We say $\phi : G \to \Sym(X)$ is a permutation representation of $G$.
\end{definition}

\subsection{Orbits and Stabilizers}%
\label{sub:orbits_and_stabilizers}

\begin{definition}
	Let $G$ act on a set $X$.
	\begin{enumerate}[(i)]
		\item The orbit of $x \in X$ is $\Orb_{G}(x) = \{g \ast x \mid g \in G\} \subseteq X$.\index{orbit}
		\item The stabilizer of $x \in X$ is
			\[
				G_x = \{g \in G \mid g \ast x = x\} \leq G
			.\]\index{stabilizer}
	\end{enumerate}
\end{definition}

\begin{theorem}[Orbit-Stabilizer Theorem]
	There is a bijection $\Orb_G(x) \leftrightarrow (G:G_x)$. In particular, if $G$ is finite, then
	\[
		|G| = |\!\Orb_{G}(x)||G_x|
	.\]
\end{theorem}

For example, if $G$ is the group of symmetries of a cube, and $X$ is the set of vertices, then $|\!\Orb_G(x)| = 8$, $|G_x| = 6$, so $|G| = 48$.

\begin{remark}
	\begin{enumerate}[label = (\roman*)]
		\item[]
		\item $\displaystyle \Ker \phi = \bigcap_{x \in X} G_x$ is called the kernel of the group action.
		\item The orbits partition $X$. We say the action is \textbf{transitive} if there is only one orbit.
		\item $G_{g \ast x} = g G_{x} g^{-1}$, so if $x, y \in X$, then their stabilizers are conjugate.
	\end{enumerate}
\end{remark}

\subsection{Examples of Group Actions}%
\label{sub:examples_of_group_actions}

\begin{enumerate}[label = (\roman*)]
	\item Let $G$ act on itself by left multiplication, i.e. $g \ast x = gx$. The kernel of this action is $\{g \in G \mid gx = x \, \forall x\} = \{e\}$. Thus $G$ has an injection into $\Sym(G)$, proving the following:	
\end{enumerate}

\begin{theorem}[Cayley's Theorem]\index{Cayley's theorem}
			Any finite group is isomorphic to a subgroup of $S_n$ for some $n$.
\end{theorem}

\begin{enumerate}[resume, label = (\roman*)]
	\item Let $H \leq G$, and $G$ acts on $(G:H)$ by left multiplication, i.e. $g \ast xH = gxH$. This action is transitive, with
		\[
			G_{xH} = \{g \in G \mid gxH = xH\} = \{g \in G \mid x^{-1}gx \in H\} = xHx^{-1}
		.\]
		Thus $\displaystyle \Ker(\phi) = \bigcap_{x \in G} xHx^{-1}$, which is the largest normal subgroup of $G$ that is contained in $H$.
\end{enumerate}

\begin{theorem}
	Let $G$ be a non-abelian simple group, and $H \leq G$ a subgroup of index $n > 1$. Then $n \geq 5$ and $G$ is isomorphic to a subgroup of $A_n$.
\end{theorem}

\begin{adjustbox}{minipage = \columnwidth - 25.5pt, margin=1em, frame=1pt, margin=0em}
\textbf{Proof:} Let $G$ act on $X = (G:H)$ by left multiplication, and $\phi : G \to \Sym(X)$ be the associated permutation representation.

As $G$ is simple, $\Ker(\phi) = e$ or $G$. However, if $\Ker(\phi) = G$, then $\Img(\phi) = \id$, but this is a contradiction as we know $G$ acts transitively on $X$ and $|X| > 1$. Thus $\Ker(\phi) = e$ and $G \cong \Img(\phi) \leq S_n$.

Since $G \leq S_n$ and $A_n \lhd S_n$, the second isomorphism theorem gives
\[
	G \cap A_n \lhd G \text{ and } G/G \cap A_n \cong GA_n/A_n \leq S_n/A_n \cong C_2
.\]
Since $G$ is simple, $G \cap A_n = e$ or $G$. However if $G \cap A_n = e$, then $G \leq C_2$, which is abelian. Thus, we have $G \cap A_n = G$, meaning $G \leq A_n$.

Finally, if $n \leq 4$, then $A_n$ has no non-abelian simple subgroups.

\end{adjustbox}



\begin{enumerate}[resume, label = (\roman*)]
	\item Let $G$ act on itself by conjugation, i.e. $g \ast x = g x g^{-1}$. In this case,
		\[
			\Orb_G(x) = \{gxg^{-1} \mid g \in G\} = \ccl_G(x)
		,\]
		the \textbf{conjugacy class}\index{conjugacy class} of $x$ in $G$. The stabilizer is
		\[
			G_x = \{g \in G \mid gx = xg\} = C_G(x) \leq G
		,\]
		the \textbf{centralizer}\index{centralizer} of $x$ in $G$. Finally,
		\[
			\ker(\phi) = \{g \in G \mid gx = xg\} = Z(G)
		\]
		is the \textbf{centre}\index{centre} of $G$.
\end{enumerate}

\begin{remark}
	The function $\phi$ mapping $h \mapsto ghg^{-1}$ satisfies
	\[
		\phi(g)(h_1h_2) = gh_1h_2g^{-1} = gh_1g^{-1}gh_2g^{-1} = \phi(g)(h_1) \phi(g)(h_2)
	.\]
	Therefore, $\phi(g)$ is a group homomorphism, and also a bijection, so $\phi(g)$ is an isomorphism.
\end{remark}

\begin{definition}
	We define
	\[
		\Aut(G) = \{f : G \to G \mid f \text{ is an isomorphism}\}
	.\]
	Then $\Aut(G) \leq \Sym(G)$ and $\phi : G \to \Sym(G)$ has image in $\Aut(G)$.
\end{definition}

\begin{enumerate}[resume, label = (\roman*)]
	\item Let $X$ be the set of all subgroups of $G$ then $G$ acts on $X$ be conjugation, i.e.
		\[
		g \ast H = g H g^{-1}
		.\]
		The stabilizer of $H$ is
		\[
			\{g \in G \mid gHg^{-1} = H\} = N_G(H)
		,\]
		the \textbf{normalizer}\index{normalizer} of $H$ in $G$. This is the largest subgroup of $G$ containing $H$ as a normal subgroup. In particular,
		\[
			H \lhd G \iff N_G(H) = G
		.\]
\end{enumerate}

\newpage

\section{Alternating Groups}%
\label{sec:alternating_groups}\index{alternating group}

We have seen that in $S_n$, elements are conjugate if and only if they have the same cycle type. For example, in $S_5$,

\begin{center}
\begin{tabular}{c|c|c}
	cycle type & \# elements & sign \\
	\hline
	$\id$ & 1 & + \\
	$(\bullet \bullet)$ & 10 & - \\
	$(\bullet \bullet)(\bullet \bullet)$ & 15 & + \\
	$(\bullet \bullet \bullet)$ & 20 & + \\
	$(\bullet \bullet)(\bullet \bullet \bullet)$ & 20 & - \\
	$(\bullet \bullet \bullet \bullet)$ & 30 & - \\
	$(\bullet \bullet \bullet \bullet \bullet)$ & 24 & + \\
	\hline
	Total & 120 &  \\
\end{tabular}
\end{center}

Let $g \in A_n$. Then $C_{A_n}(g) = C_{S_n}(g) \cap A_n$. If there is an odd permutation commuting with $g$, then
\[
	|C_{A_n}(g)| = \frac{1}{2}|C_{S_n}(g)| \text{ and } |\!\ccl_{A_n}(g)| = |\!\ccl_{S_n}(g)|
.\]
Otherwise,
\[
	|C_{A_n}(g)| = |C_{S_n}(g)| \text{ and } |\!\ccl_{A_n}(g)| = \frac{1}{2}|\!\ccl_{S_n}(g)|
.\]
In $S_5$, the double transpositions and 3-cycles commute with transpositions, so their conjugacy class stay the same. However the 5-cycles only commutes with even elements, so its conjugacy class splits.

Thus $A_5$ has conjugacy classes of sizes $1, 15, 20, 12, 12$. If $H \lhd A_5$, then $H$ is a union of conjugacy classes, so
\[
	|H| = 1 + 15a + 20b + 12c, \quad a, b \in \{0, 1\}, \; c \in \{0, 1, 2\}
,\]
and $|H| \mid 60$ by Lagrange's. Thus $|H| = 1$ or 60, i.e. $A_5$ is simple.

\begin{lemma}
	$A_n$ is generated by 3-cycles.
\end{lemma}

\begin{adjustbox}{minipage = \columnwidth - 25.5pt, margin=1em, frame=1pt, margin=0em}
\textbf{Proof:} Each $\sigma \in A_n$ is a product of an even number of transpositions. Thus it suffices to write the product of any two transpositions as a product of 3-cycles.

For $a, b, c, d$ distinct, two transpositions can be written as
\[
\begin{cases}
	(a\; b)(b\; c) = (a\; b\; c) \\
	(a\; b)(c\; d) = (a\; c\; b)(a\; c\; d)
\end{cases}
\]

\end{adjustbox}

\begin{lemma}
	If $n \geq 5$ then all 3-cycles in $A_n$ are conjugate.
\end{lemma}

\begin{adjustbox}{minipage = \columnwidth - 25.5pt, margin=1em, frame=1pt, margin=0em}
\textbf{Proof:} We claim that every 3-cycle in $A_n$ is conjugate to $(1\; 2\; 3)$. Indeed if $(a\; b\; c)$ is a 3-cycle then
\[
	(a\; b\; c) = \sigma(1\; 2\; 3)\sigma^{-1}
\]
for some $\sigma \in S_n$. If $\sigma \not \in A_n$, then replace $\sigma$ by $\sigma(4\; 5)$.

\end{adjustbox}


\subsection{Simplicity of \texorpdfstring{$A_n$}{An}}%
\label{sub:simplicity_of_a_n_}

\begin{theorem}
	$A_n$ is simple for all $n \geq 5$.
\end{theorem}

\textbf{Proof:} Let $\id \neq N \lhd A_n$. It suffices to show that $N$ contains a 3-cycle from the previous lemmas. Take $\id \neq \sigma \in N$ and with $\sigma$ as a product of disjoint cycles.

\begin{enumerate}[label = Case \arabic*:]
	\item $\sigma$ contains a cycle of length $r \geq 4$. Say 
		\[
			\sigma = (1\; 2\; \ldots\; r) \tau
		.\]
		Let $\delta = (1\; 2\; 3)$, and consider
		\[
			\sigma^{-1} \delta^{-1} \sigma \delta = (r\; \ldots\; 2\; 1)(1\; 3\; 2)(1\; 2\; \ldots\; r)(1\; 2\; 3) = (2\; 3\; r)
		.\]
		This implies $N$ contains a 3-cycle.
	\item $\sigma$ contains two 3-cycles.  Say
		\[
			\sigma = (1\; 2\; 3)(4\; 5\; 6) \tau
		.\]
		Let $\delta = (1\; 2\; 4)$. Then,
		\[
			\sigma^{-1} \delta^{-1} \sigma \delta = (1\; 3\; 2)(4\; 6\; 5)(1\; 2\; 3)(3\; 4\; 5)(1\; 2\; 4) = (1\; 2\; 4\; 3\; 6)
		.\]
		Now we are done by case 1.
	\item $\sigma$ contains two 2-cycles. Say
		\[
			\sigma = (1\; 2)(3\; 4)\tau
		.\]
		Let $\delta = (1\; 2\; 3)$, and
		\[
			\sigma^{-1} \delta^{-1} \sigma \delta = (1\; 2)(3\; 4)(1\; 3\; 2)(1\; 2)(3\; 4)(1\; 2\; 3) = (1\; 4)(2\; 3) = \pi
		.\]
		Let $\epsilon = (2\; 3\; 5)$. Then
		\[
			\pi^{-1} \epsilon^{-1} \pi \epsilon = (1\; 4)(2\; 3)(2\; 5\; 3)(1\; 4)(2\; 3)(2\; 3\; 5) = (2\; 5\; 3)
		.\]
		Thus $N$ contains a 3-cycle.
\end{enumerate}
It remains to consider $\sigma$ with at most 1 3-cycle, and at most 1 transposition. However, if it has 1 transposition, it is odd, so $\sigma$ must be a 3-cycle.

\newpage

\section{\texorpdfstring{$p$}{p}-groups and \texorpdfstring{$p$}{p}-subgroups}%
\label{sec:_p_groups_and_p_subgroups}

\begin{definition}\index{$p$-group}
	Let $p$ be a prime. A finite group $G$ is a $p$-group if $|G| = p^{n}$, $n \geq 1$.
\end{definition}

\begin{theorem}
	If $G$ is a $p$-group, then $Z(G) \neq 1$.
\end{theorem}

\begin{adjustbox}{minipage = \columnwidth - 25.5pt, margin=1em, frame=1pt, margin=0em}
\textbf{Proof:} For $g \in G$, we have
\[
	|\!\ccl_G(g)||C_G(g)| = |G| = p^{n}
,\]
so each conjugacy class has size a power of $p$. Since $G$ is a disjoint union of conjugacy classes,
\[
	|G| = \# \text{ of conjugacy classes of size 1} \pmod p
.\]
Therefore,
\[
	0 = |Z(G)| \pmod p
,\]
and hence $Z(G) \neq 1$.

\end{adjustbox}

\begin{corollary}
	The only simple $p$-group is $C_p$.
\end{corollary}

\begin{adjustbox}{minipage = \columnwidth - 25.5pt, margin=1em, frame=1pt, margin=0em}
\textbf{Proof:} Let $G$ be a simple $p$-group. Since $Z(G) \lhd G$, we have $Z(G) = 1$ or $G$. But the first case is impossible, so $Z(G) = G$, meaning $G$ is abelian, so $G \cong C_p$.

\end{adjustbox}

\begin{corollary}
	Let $G$ be a $p$-group of order $p^{n}$. Then $G$ has a subgroup of order $p^{r}$ for all $0 \leq r \leq n$.
\end{corollary}

\begin{adjustbox}{minipage = \columnwidth - 25.5pt, margin=1em, frame=1pt, margin=0em}
\textbf{Proof:} Consider the composition series of $G$:
\[
1 = G_0 \lhd G_1 \lhd \cdots \lhd G_{m-1} \lhd G_m = G
\]
with each $G_i/G_{i-1}$ simple. However since $G$ is a $p$-group, $G_i/G_{i-1}$ is a $p$-group, so $G_i/G_{i-1} \cong C_p$. Then by induction, $|G_i| = p^{i}$, and $m = n$.

\end{adjustbox}


\begin{lemma}
	For $G$ a group, if $G/Z(G)$ is cyclic, then $G$ is abelian (so in fact $G/Z(G) \cong 1$).
\end{lemma}

\begin{adjustbox}{minipage = \columnwidth - 25.5pt, margin=1em, frame=1pt, margin=0em}
\textbf{Proof:} Let $gZ(G)$ be a generator for $G/Z(G)$. Then each coset is of the form $g^{r}Z(G)$ for some $r \in \mathbb{Z}$ Thus
\[
	G = \{g^{r}z \mid r \in \mathbb{Z}, z \in Z(G)\}
.\]
Then,
\[
g^{r_1}z_1 g^{r_2}z_2 = g^{r_1}g^{r_2}z_1z_2 = g^{r_2}g^{r_1}z_2z_1 = g^{r_2}z_2g^{r_1}z_1
.\]
Therefore any two elements of the group commute, so $G$ is abelian.

\end{adjustbox}

\begin{corollary}
	If $|G| = p^2$, then $G$ is abelian.
\end{corollary}

\begin{adjustbox}{minipage = \columnwidth - 25.5pt, margin=1em, frame=1pt, margin=0em}
\textbf{Proof:} We have either $|Z(G)| = 1, p$ or $p^2$. However the centre is nontrivial, so $|Z(G)| \neq 1$. If $|Z(G)| = p$, then $|G/Z(G)| = p$, so $G/Z(G)$ is cyclic, and $|Z(G)| = |G|$, contradiction. Therefore $|Z(G)| = p^2$, so $G$ is abelian.

\end{adjustbox}


\subsection{Sylow Theorems}%
\label{sub:sylow_theorems}

\begin{theorem}[Sylow Theorems]\index{Sylow theorems}
	Let $G$ be a finite group of order $p^{a}m$ where $p$ is a prime and $p \nmid m$. Then,
	\begin{enumerate}[label = (\roman*)]
		\item The set $\Syl_p(G) = \{P \leq G \mid  |P| = p^{a}\}$ is non-empty.
		\item All elements of $\Syl_p(G)$ are conjugate.
		\item $n_p = |\!\Syl_p(G)|$ satisfies
			\[
				n_p \equiv 1 \pmod p \text{ and } n_p \mid |G|
			\]
			and so $n_p \mid m$.
	\end{enumerate}
\end{theorem}

\begin{corollary}
	If $n_p = 1$, then the unique Sylow $p$-subgroup is normal.
\end{corollary}

This follows by letting $g \in G$ and $P \in \Syl_p(G)$. Then $gPg^{-1} \in \Syl_p(G)$, meaning $gPg^{-1} = P$. Thus $P \lhd G$.

\begin{adjustbox}{minipage = \columnwidth - 25.5pt, margin=1em, frame=1pt, margin=0em}
\textbf{Examples:}

Let $|G| = 1000$. Then $n_5 \equiv 1 \pmod 5$ and $n_5 \mid 8$, so $n_5 = 1$. Thus the unique Sylow 5-group is normal and hence $G$ is not simple.

Let $|G| = 132$. Then $n_{11} \equiv 1 \pmod 11$ and $n_{11} \mid 12$. Thus $n_{11} = 1$ or $12$. If $G$ is simple, then $n_{11} \neq 1$, so it must be equal to 12. Now $n_3 \equiv 1 \pmod 3$ and $n_3 \mid 44$, so $n_3 = 1$, $4$ or $22$. Note $n_3 \neq 1$, and if $n_3 = 4$, then letting $G$ act on $\Syl_3(G)$ by conjugation gives a group homomorphism $\phi$. Thus,
\[
	\Ker(\phi) \lhd G \implies \Ker(\phi) = 1 \text{ or } G
.\]
However if $\Ker(\phi) = G$ then $\Img(\phi) = \id$, which is a contradiction to the second Sylow theorem. Therefore $\Ker(\phi) = 1$, so $G \leq S_4$, which is a contradiction as $|G| > |S_4|$.

Thus $n_3 = 22$ and $n_{11} = 12$. So $G$ has $22 \cdot (3 - 1) = 44$ elements of order 3, and $12 \cdot (11 - 1) = 120$ elements of order 11. But $44 + 120 > 132 = |G|$.
\end{adjustbox}

\textbf{Proof of Sylow Theorems:}

\begin{enumerate}[label = (\roman*)]
	\item Let $\Omega$ be the set of all \textbf{subsets} of $G$ of order $p^{a}$. Then
		\[
			|\Omega| = \binom{p^{a}m}{p^{a}} = \frac{p^{a}m}{p^{a}} \frac{p^{a}m - 1}{p^{a} - 1} \cdots \frac{p^{a}m - p^{a} + 1}{1}
		.\]
		From this representation, we can see that $|\Omega|$ is coprime to $p$. Let $G$ act on $|\Omega|$ by left multiplication, i.e. for $g \in G$ and $X \in \Omega$,
		\[
			g \ast X = \{gx \mid x \in X\} \subset \Omega
		.\]
		For any $X \in \Omega$ we have
		\[
			|G_X| |\!\Orb_G(X)| = |G| = p^{a}m
		.\]
		However since $|\Omega|$ is coprime to $p$, there exists $X$ such that $|\!\Orb_G(X)|$ is coprime to $p$, and thus $p^{a} \mid |G_X|$. On the other hand, if $g \in G$ and $x \in X$ then $g \in (gx^{-1}) \ast X$ and hence
		\[
			G = \bigcup_{g \in G} g \ast X = \bigcup_{Y \in \Orb_G(X)} Y
		.\]
		Hence $|G| \leq |\!\Orb_G(X)| \cdot |X|$, so
		\[
			|G_X| = \frac{|G|}{|\!\Orb_G(X)|} \leq |X| = p^{a}
		.\]
		Combining these facts, $|G_X| = p^{a}$, i.e. $G_X \in \Syl_P(G)$.
	\item We prove a stronger result.
		\begin{lemma}
			If $P \in \Syl_p(G)$ and $Q \leq G$ is a $p$-subgroup, then $Q \leq gPg^{-1}$ for some $g \in G$.
		\end{lemma}
		To prove this, let $Q$ act on the set of left cosets $(G:P)$ by left multiplication, i.e. $q \ast gP = qgP$. By orbit-stabilizer, each orbit has size $|Q|$, so either $1$ or a multiple of $p$. But since $|(G:P)|$ is coprime to $p$, there is an orbit of size 1, i.e.
		\[
		qgP = gP \quad \forall q \in Q \implies g^{-1}qg \in P \quad \forall q \in Q
		.\]
		Thus $Q \leq gPg^{-1}$, as required.
	\item Let $G$ act on $\Syl_p(G)$ by conjugation. From the above, this action is transitive, so by orbit-stabilizer,
		\[
			n_p = |\!\Syl_p(G)| \; \big\vert\; |G|
		.\]
		Now let $P \in \Syl_p(G)$. Then $P$ acts on $\Syl_p(G)$ by conjugation. The orbits have size dividing $|P|$, so either 1 or a multiple of $p$. Since $P$ is fixed by this action, it suffices to show that $\{P\}$ is the unique orbit of size 1. If $\{Q\}$ is an orbit of size 1, then $P$ normalizes $Q$, hence $P \leq N_G(Q)$. But then $P$ and $Q$ are both Sylow $p$-subgroups of $N_G(Q)$, hence they are conjugate, but since $Q \lhd N_G(Q)$, $P = Q$.
\end{enumerate}

\newpage

\section{Matrix Groups}%
\label{sec:matrix_groups}

If $F$ is a field, let $GL_n(F)$\index{general linear group} denote the $n \times n$ matrices with elements in $F$, with non-zero determinant. Then
\[
	SL_n(F) = \ker(GL_n(F) \to F^{\times})
\]\index{special linear group}
under the determinant operation. Hence $SL_n(F) \lhd GL_n(F)$. Let $Z \lhd GL_n(F)$ be the subgroup of scalar matrices, then define
\[
	PGL_n(F) = GL_n(F) / Z
.\]\index{projective general linear group}
We can then define
\[
	PSL_n(F) = SL_n(F) / (Z \cap SL_n(F)) \cong Z \cdot SL_n(F) / Z \leq PGL_n(F)
.\]\index{projective special linear group}

If $G = GL_n(\mathbb{Z}/p \mathbb{Z})$, then a list of $n$ vectors in $(\mathbb{Z}/p \mathbb{Z})^{n}$ are the columns of some $A \in G$ if and only if they are linearly independent. Hence
\[
	|G| = (p^{n}-1)(p^{n} - p) \cdots (p^{n} - p^{n-1}) = p^{\binom{n}{2}} \prod_{i = 1}^{n} (p^{i} - 1)
.\]
So the Sylow $p$-subgroups have size $p^{\binom{n}{2}}$, Let $U$ be the set of upper-triangular matrices with 1's on the diagonal. Then $U \in \Syl_p(G)$, since there $\binom{n}{2}$ entries, and each can take $p$ values.

Just as $PGL_2(\mathbb{C})$ acts on $\mathbb{C} \cup \{\infty\}$ via M\"{o}bius transformations, $PGL_2(\mathbb{Z}/ p \mathbb{Z})$ acts on $\mathbb{Z} / p \mathbb{Z} \cup \{\infty\}$ via
\[
\begin{pmatrix}
	a & b \\
	c & d
\end{pmatrix}
: z \mapsto \frac{az + b}{cz + d}
.\]
Since scalar matrices act trivially, we obtain an action of $PGL_2(\mathbb{Z} / p \mathbb{Z})$.

\begin{lemma}
	The permutation representation
	\[
		PGL_2(\mathbb{Z} / p \mathbb{Z}) \to S_{p+1}
	\]
	is injective (in fact an isomorphism if $p = 2$ or $3$).
\end{lemma}

\begin{adjustbox}{minipage = \columnwidth - 25.5pt, margin=1em, frame=1pt, margin=0em}
\textbf{Proof:} Suppose
\[
\frac{az + b}{cz + d} = z
\]
for all $z \in \mathbb{Z} / p \mathbb{Z} \cup \{\infty\}$. Then $z = 0$ gives $b = 0$, $z = \infty$ gives $c = 0$, and $z = 1$ gives $a = d$, hence the matrix is scalar, so trivial in $PGL_2(\mathbb{Z} / p \mathbb{Z})$.
\end{adjustbox}

\begin{lemma}
	If $p$ is an odd prime, then
	\[
		|PSL_2(\mathbb{Z}/p \mathbb{Z})| = \frac{p(p-1)(p+1)}{2}
	.\]
\end{lemma}

\begin{adjustbox}{minipage = \columnwidth - 25.5pt, margin=1em, frame=1pt, margin=0em}
	\textbf{Proof:} Note $|GL_2(\mathbb{Z} / p \mathbb{Z})| = p(p^2-1)(p-1)$. The homomorphism $GL_2(\mathbb{Z}/p \mathbb{Z}) \to (\mathbb{Z} / p \mathbb{Z})^{\times}$ is surjective. Thus $|SL_2(\mathbb{Z} / p \mathbb{Z})| = p(p-1)(p+1)$. Note
	\[
	\begin{pmatrix}
		\lambda & 0 \\
		0 & \lambda
	\end{pmatrix}
	\in SL_2(\mathbb{Z} / p \mathbb{Z}) \iff \lambda = \pm 1 \pmod p
	.\]
	Thus $Z \cap SL_2(\mathbb{Z} / p \mathbb{Z}) = \{\pm I\}$, so modding out by $Z$,
	\[
		|PSL_2(\mathbb{Z}/ p \mathbb{Z})| = \frac{1}{2} |SL_2(\mathbb{Z} / p \mathbb{Z})| = \frac{p(p-1)(p+1)}{2}
	.\]
\end{adjustbox}

\textbf{Example:} If we let $G = PSL_2(\mathbb{Z} / 5 \mathbb{Z})$, then $|G| = 60$. Let $G$ act on $\mathbb{Z} / 5 \mathbb{Z} \cup \{\infty\}$. This is injective from lemma 5.1. Thus $\phi : G \to S_6$ is injective. In fact, $\Img(\phi) \leq A_6$, as if $\psi = \sgn \circ \phi$, then
\[
	\psi(h^{m}) = 1 \iff \psi(h) = 1
\]
for odd $m$. Thus it suffices to show $\psi(g) = 1$ for all $g \in G$ with order a power of 2. However every such $g$ belongs to a Sylow 2-subgroup, so it suffices to show $\psi(H) = 1$ for a Sylow 2-subgroup $H$, since all Sylow 2-subgroups are conjugate. Take
\[
H = \langle
\begin{pmatrix}
	2 & 0 \\
	0 & 3
\end{pmatrix}
,
\begin{pmatrix}
	0 & 1 \\
	-1 & 0
\end{pmatrix}
\rangle \leq G
.\]
Now compute
\begin{align*}
	\phi \left(
		\begin{pmatrix}
			2 & 0 \\
			0 & 3
		\end{pmatrix}
	\right)
	&= (1\; 4)(2\; 3) \quad z \mapsto -z, \\
	\phi \left(
		\begin{pmatrix}
			0 & 1 \\
			-1 & 0
		\end{pmatrix}
	\right)
	&= (0\; \infty)(1\; 4) \quad z \mapsto -\frac{1}{z}.
\end{align*}
Thus $\psi(H) = 1$. Now we know that if $G \leq A_6$ and $|G| = 60$, then $G \cong A_5$.

\begin{remark}
	\begin{itemize}
		\item[]
		\item $PSL_n(\mathbb{Z} / p \mathbb{Z})$ is a simple group for all $n \geq 2$, and $p$ a prime, known as the finite groups of Lie type.
		\item The smallest non-abelian simple groups are $A_5 \cong PSL_2(\mathbb{Z} / 5 \mathbb{Z})$ with order 60 and $PSL_2(\mathbb{Z} / 7 \mathbb{Z}) \cong GL_3(\mathbb{Z} / 2 \mathbb{Z})$ with order 168.
	\end{itemize}
\end{remark}

\newpage

\section{Finite Abelian Groups}%
\label{sec:finite_abelian_groups}

The main theorem is as follows:

\begin{theorem}
	Every finite abelian group is isomorphic to a product of cyclic groups.
\end{theorem}

Note such an isomorphism is not unique.

\begin{lemma}
	If $m, n \in \mathbb{N}$ are coprime, then $C_m \times C_n \cong C_{mn}$.
\end{lemma}

Using this, we can edit our first theorem to get uniqueness.

\begin{corollary}
	Let $G$ be a finite abelian group. Then
	\[
	G \cong C_{n_1} \times C_{n_2} \times \cdots \times C_{n_k}
	,\]
	where $n_i$ is a prime power.
\end{corollary}

We can also refine theorem 6.1 as follows:

\begin{theorem}
	Let $G$ be a finite abelian group. Then
	\[
	G \cong C_{d_1} \times C_{d_2} \times \cdots \times C_{d_l}
	\]
	for some $d_1 \mid d_2 \mid \cdots \mid d_l$.
\end{theorem}

\begin{remark}
	The integers $n_1, \ldots, n_k$ are unique (up to ordering), and $d_1, \ldots, d_l$ are unique (assuming $d_1 > 1$).
\end{remark}

\begin{definition}
	The \textbf{exponent}\index{exponent} of a group $G$ is the least integer $n \geq 1$ such that $g^{n} = 1$ for all $g \in G$, i.e. the lowest common multiple of all the orders of the elements of $G$.
\end{definition}

In $A_4$, the exponent is 6, but there is no element of order 6.

\begin{corollary}
	Every finite abelian group contains an element whose order is the exponent of the group.
\end{corollary}

This is direct from theorem 6.2.

\newpage

\part{Rings}%
\label{prt:rings}

\section{Definitions and Examples}%
\label{sec:definitions_and_examples_rings}

\begin{definition}
	A ring\index{ring} is a triple $(R, +, \cdot)$ consisting of a set $R$ and two binary operations $+ : R \times R \to R$ and $\cdot : R \times R \to R$ satisfying
	\begin{enumerate}[label = (\roman*)]
		\item $(R, +)$ is an abelian group with identity $0$ ($=0_{R}$).
		\item Multiplication is associative and has an identity, i.e. $x \cdot (y \cdot z) = (x \cdot y) \cdot z$, and there exists $1 \in R$ such that $x \cdot 1 = 1 \cdot x = x$.

			We say that $R$ is a \textbf{commutative ring} if $x \cdot y = y \cdot x$ for all $x, y \in R$. We will only consider commutative rings.
		\item Addition distributes over multiplication: $x \cdot (y + z) = x \cdot y + x \cdot z$ and $(y + z) \cdot x = y \cdot x + z \cdot x$.
	\end{enumerate}
\end{definition}
\begin{remark}
	\begin{enumerate}[label = (\roman*)]
		\item[]
		\item As in the case of groups, we need to check closure.
		\item For $x \in R$, we write $-x$ for the inverse of $x$ under $+$, and abbreviate $x + (-y)$ as $x - y$.
		\item $0 \cdot x = (0 + 0)\cdot x = 0 \cdot x + 0 \cdot x$, so $0 \cdot x = 0$, and similarly $x \cdot 0 = 0$.
		\item $0 = 0 \cdot x = (1 - x) \cdot x = 1 \cdot x + (-1) \cdot x = x + (-1) \cdot x$. Thus, $(-1) \cdot x = -x$ for all $x \in R$.
	\end{enumerate}
\end{remark}

\begin{definition}\index{subring}
	A subset $S \subseteq R$ is a subring (written $S \leq R$) if it is a ring under $+$ and $\cdot$, with the same identity elements $0$ and $1$.
\end{definition}

\textbf{Examples:} 

\begin{enumerate}[label = (\roman*)]
	\item $\mathbb{Z} \leq \mathbb{Q} \leq \mathbb{R} \leq \mathbb{C}$.
	\item $\mathbb{Z}[i] = \{a + bi \mid a, b \in \mathbb{Z}\} \leq \mathbb{C}$.
	\item $\mathbb{Q}[\sqrt{2}] = \{a + b\sqrt{2} \mid a, b \in \mathbb{Q}\} \leq \mathbb{R}$.
	\item $\mathbb{Z} / n \mathbb{Z}$.
	\item For $R, S$ rings, the ring $R \times S$\index{ring!direct product} is a ring, where
		\begin{itemize}
			\item $(r_1, s_1) + (r_2, s_2) = (r_1 + r_2, s_1 + s_2)$.
			\item $(r_1, s_1) \cdot (r_2, s_2) = (r_1r_2, s_1s_2)$.
			\item $0_{R \times S} = (0_{R}, 0_{S})$ and $1_{R \times S} = (1_{R}, 1_{S})$. (Note $R \times \{0\}$ is not a subring of $R \times S$).
		\end{itemize}
	\item For $R$ a ring, a polynomial $f$ over $R$ is an expression
		\[
		f = a_0 + a_1X + \cdots + a_nX^{n}, \quad a_i \in R
		.\]
		The degree of $f$ is the largest $n \in \mathbb{N}$ such that $a_n \neq 0$. We write $R[X]$ for the set of all polynomials over $R$. Then if $g = b_0 + b_1X + \cdots + b_nX^{n}$, then
		\begin{align*}
			f + g &= \sum_{i}(a_i + b_i)X^{i}, \\
			f \cdot g &= \sum_{i} \left( \sum_{j}^{i} a_j b_{i - j} \right) X^{i}. 
		\end{align*}
		Then $R[X]$ is a ring with identities $0_{R}$ and $1_{R}$, which are constant polynomials. Note we can identify $R$ with the subring of $R[X]$ of constant polynomials.\index{polynomial ring}
\end{enumerate}

\begin{definition}
	An element $r \in R$ is a \textbf{unit}\index{unit} if it has an inverse under multiplication, i.e. there exists $s \in R$ such that $s \cdot r = 1$. The units in $R$ form a group $(R^{\times}, \cdot)$ under multiplication.
\end{definition}

For example,
\begin{itemize}
	\item $\mathbb{Z}^{\times} = \{\pm 1\}$,
	\item $\mathbb{Q}^{\times} = \mathbb{Q} \setminus \{0\}$.
\end{itemize}

\begin{definition}
	A \textbf{field}\index{field} is a ring with $0 \neq 1$, such that every non-zero element is a unit.
\end{definition}

For example, we have $\mathbb{Q}$ and $\mathbb{Z} / p \mathbb{Z}$ for $p$ prime.

\begin{remark}
	If $R$ is a ring where $0 = 1$, then $x = 1 \cdot x = 0 \cdot x = 0$, so $R = \{0\}$ is the trivial ring.
\end{remark}

\begin{proposition}
	Let $f, g \in R[X]$. Suppose the leading coefficient of $g$ is a unit. Then there exist $q, r \in R[X]$ such that
	\[
		f(x) = q(x)g(x) + r(x)
	,\]
	where $\deg(r) < \deg(g)$.
\end{proposition}

\begin{adjustbox}{minipage = \columnwidth - 25.5pt, margin=1em, frame=1pt, margin=0em}
\textbf{Proof:} We will induct on $n = \deg(f)$. Write
\begin{align*}
	f(X) &= a_nX^{n} + a_{n-1}X^{n-1} + \cdots + a_0, \\
	g(X) &= b_nX^{m} + b_{m-1}X^{m-1} + \cdots + b_0.
\end{align*}
If $n < m$, then put $q = 0$ and $r = f$. Otherwise, we have $n \geq m$ and we set
\[
	f_1(X) = f(X) - a_nb_m^{-1}g(X)X^{n - m}
.\]
The coefficient of $X^{n}$ in $f_1$ vanishes, and so $\deg(f_1) < n$. By the inductive hypothesis, there exists $q_1, r \in R[X]$ where
\[
	f_1(X) = q_1(X)g(X) + r(X)
,\]
where $\deg(r) < \deg (g)$. Therefore,
\[
	f(x) = (q_1(X) + a_nb_m^{-1}X^{n - m})g(X) + r(X)
.\]
\end{adjustbox}

\begin{remark}
	If $R$ is a field, then we only need $g \neq 0$.
\end{remark}

\textbf{Further Examples:} 

\begin{enumerate}[resume*]
	\item If $R$ is a ring and $X$ is a set, then the set of all functions $X \to R$ is a ring under pointwise operations, e.g.
		\[
			(f + g)(x) = f(x) + g(x), \quad (f \cdot g)(x) = f(x) \cdot g(x)
		.\]
		Further interesting examples appear as subrings, for example continuous functions $\mathbb{R} \to \mathbb{R}$, and polynomial functions $\mathbb{R} \to \mathbb{R} = \mathbb{R}[X]$.
	\item Power series ring:\index{power series ring}
		\[
			R[[X]] = \{a_0 + a_1X + a_2X^2 + \cdots \mid a_i \in R\}
		.\]
	\item Laurent polynomials:\index{Laurent polynomials}
		\[
			R[X, X^{-1}] = \left\{ \sum_{i \in \mathbb{Z}} a_i X_i \mid a_i \in R, a_i \neq 0 \text{ for finitely many } i\right\}
		.\]
\end{enumerate}

\newpage

\section{Homomorphisms, Ideals and Quotients}%
\label{sec:homomorphisms_ideals_and_quotients}

\subsection{Definitions}%
\label{sub:definitions}

\begin{definition}\index{ring!homomorphism}
	Let $R$ and $S$ be rings. A function $\phi : R \to S$ is called a ring homomorphism if
	\begin{enumerate}[label = (\roman*)]
		\item $\phi(r_1 + r_2) = \phi(r_1) + \phi(r_2)$,
		\item $\phi(r_1 \cdot r_2) = \phi(r_1)\cdot\phi(r_2)$,
		\item $\phi(1_{R}) = 1_{S}$.
	\end{enumerate}
	A ring homomorphism that is also a bijection is called an isomorphism.

	The kernel of $\phi$ is $\Ker(\phi) = \{r \in R \mid \phi(r) = 0_{S}\}$.
\end{definition}

\begin{lemma}
	A ring homomorphism $\phi : R \to S$ is injective if and only if $\Ker(\phi) = \{0_{R}\}$.
\end{lemma}
This follows by taking the corresponding result of groups on the additive group of $R$.

\begin{definition}\index{ideal}
	A subset $I \subseteq R$ is an \textbf{ideal}, written $I \lhd R$, if
	\begin{enumerate}[label = (\roman*)]
		\item $I$ is a subgroup of $(R, +)$.
		\item If $r \in R$ and $x \in I$, then $r \cdot x \in I$.
	\end{enumerate}
	We say that $I$ is proper if $I \neq R$.
\end{definition}

\begin{lemma}
	If $\phi : R \to S$ is a ring homomorphism, then $\Ker(\phi)$ is an ideal of $R$.
\end{lemma}

\begin{adjustbox}{minipage = \columnwidth - 25.5pt, margin=1em, frame=1pt, margin=0em}
	\textbf{Proof:} Since $\phi$ is a group homomorphism, $\Ker(\phi)$ is a subgroup of $(R, +)$. Now if $r \in R$, $x \in \Ker(\phi)$,
	\[
		\phi(rx) = \phi(r)\phi(x) = \phi(r) \cdot 0_{S} = 0_{S}
	.\]
	Thus $rx \in \Ker(\phi)$.
\end{adjustbox}

\begin{remark}
	If $I$ contains a unit, then $1_{R} \in I$, so $I = R$. Thus if $I$ is a proper ideal, $1_{R} \not \in I$, so $I$ is not a subring of $R$.
\end{remark}

\begin{lemma}
	The ideals in $\mathbb{Z}$ are
	\[
		n \mathbb{Z} = \{\ldots, -2n, -n, 0, n, 2n, \ldots\}
	\]
	for $n \geq 0$.
\end{lemma}

\begin{adjustbox}{minipage = \columnwidth - 25.5pt, margin=1em, frame=1pt, margin=0em}
\textbf{Proof:} Certainly these are all ideals. Now let $I \lhd \mathbb{Z}$ be a non-zero ideal, and let $n$ be the smallest positive integer in $I$. Then $n \mathbb{Z} \subseteq I$.

If $m \in I$ with $m = qn + r$, $q, r \in \mathbb{Z}$ with $0 \leq r < n$. Then $r = m - qn \in I$, which contradicts our choice of $n$ unless $r = 0$. Thus $m = qn$ for all $m \in I$, i.e.
\[
I = n \mathbb{Z}
.\]
\end{adjustbox}

\begin{definition}
	For $a \in R$, write $(a) = \{ra \mid r \in R\} \lhd R$. This is the \textbf{ideal generated by} $a$. More generally, if $a_1, \ldots, a_n \in R$, we write
	\[
		(a_1, \ldots, a_n) = \{r_1a_1 + r_2a_2 + \cdots + r_na_n \mid r_i \in R\} \lhd R
	.\]
\end{definition}

\begin{definition}\index{ideal!principal}
	Let $I \lhd R$. We say $I$ is \textbf{principal} if $I = (a)$ for $a \in R$.
\end{definition}

\begin{theorem} 
	If $I \lhd R$, then the set $R/I$ of cosets of $I$ in $(R, +)$ forms a ring (called the quotient ring) with the operations
	\begin{enumerate}[label = (\roman*)]
	\item $(r_1 + I) + (r_2 + I) = r_1 + r_2 + I$,
	\item $(r_1 + I) \cdot (r_2 + I) = r_1r_2 + I$,
	\item $0_{R/I} = 0_{R} + I = I$ and $1_{R/I} = 1_{R} + I$.
	\end{enumerate}
	Moreover, the map
	\begin{align*}
		R &\to R/I \\
		r &\mapsto r + I
	\end{align*}
	is a ring homomorphism (called the quotient map) with kernel $I$.
\end{theorem}

\begin{adjustbox}{minipage = \columnwidth - 25.5pt, margin=1em, frame=1pt, margin=0em}
	\textbf{Proof:} We already know $(R/I, +)$ is a group. Now if $r_1 + I = r_1' + I$ and $r_2 + I = r_2' + I$, then $r_1' = r_1 + a_1$, $r_2' = r_2 + a_2$ for $a_1, a_2 \in I$. Then,
	\[
		r_1'r_2' = (r_1 + a_1)(r_2 + a_2) = r_1r_2 + r_1a_2 + a_1r_2 + a_1a_2
	.\]
	Thus $r_1r_2 + I = r_1'r_2' + I$. The remaining properties follow from those for $R$, and the last part follows from these properties.
\end{adjustbox}
\newpage

\textbf{Examples:}

\begin{enumerate}[label = (\roman*)]
	\item $n \mathbb{Z} \lhd \mathbb{Z}$, with quotient ring $\mathbb{Z} / n \mathbb{Z}$. Then $\mathbb{Z} / n \mathbb{Z}$ has elements
		\[
			\{0 + n \mathbb{Z}, 1 + n \mathbb{Z}, \ldots, (n-1) + n \mathbb{Z} \}
		.\]
		Addition and multiplication are carried out modulo $n$.
	\item Consider $(X) \lhd \mathbb{C}[X]$. These are the set of polynomials with constant term 0. Then $f(X) + (X) = a_0 + (X)$. There is a bijection
		\begin{align*}
			\mathbb{C}[X]/(X) &\to \mathbb{C} \\
			f(X) + (X) &\mapsto f(0)
		\end{align*}
		These maps are ring homomorphisms, thus $\mathbb{C}[X]/(X) \cong \mathbb{C}$.
	\item Consider $(X^2 + 1) \lhd \mathbb{R}[X]$. By proposition 7.1, $f(X) = q(X)(X^2 + 1) + r(X)$, with $\deg(r) < 2$, i.e. $r$ is affine. Thus
		\[
			\mathbb{R}[X]/(X^2 + 1) = \{a + bX + (X^2 + 1) \mid a, b \in \mathbb{R}\}
		.\]
		These representatives are unique, by subtracting and looking at degrees. Now consider the bijection
		\begin{align*}
			\phi : \mathbb{R}[X]/(X^2 + 1) &\to \mathbb{C} \\
			a + bX + (X^2 + 1) &\mapsto a + bi
		\end{align*}
		This is a bijection, so we show $\phi$ is a ring homomorphism. It preserves addition and maps $1 + (X^2 + 1)$ to $1$. Now
		\begin{align*}
			\phi((a + bX + I)(c + dX + I)) &= \phi((a + bX)(c + dX) + I) \\
						       &= \phi(ac + (ad + bc)X + bd(X^2 + 1) - bd + I) \\
			&= ac - bd + (ad + bc)i = (a + bi)(c + di) \\
			&= \phi(a + bX + I) \phi(c + dx + I),
		\end{align*}
		so $\phi$ is a ring isomorphism.
\end{enumerate}

\subsection{Isomorphism Theorems}%
\label{sub:isomorphism_theorems_rings}

\begin{theorem}[First Isomorphism Theorem]\index{ring!isomorphism theorems}
	Let $\phi: R \to S$ be a ring homomorphism. Then, $\Ker(\phi) \lhd R$, $\Img(\phi) \leq S$, and
	\[
		R/\Ker(\phi) \cong \Img(\phi)
	.\]
\end{theorem}

\begin{adjustbox}{minipage = \columnwidth - 25.5pt, margin=1em, frame=1pt, margin=0em}
	\textbf{Proof:} We have already seen $\Ker(\phi) \lhd R$ and $\Img(\phi)$ is a subgroup of $(S, +)$. To show it is a subring, we need to show it is closed under multiplication and contains $1$. Now,
	\[
		\phi(r_1)\phi(r_2) = \phi(r_1r_2) \in \Img(\phi)
	,\]
	and $1_S = \phi(1_R) \in \Img(\phi)$. Let $K = \Ker(\phi)$, and define
	\begin{align*}
		\Phi : R/K &\to \Img(\phi) \\
		r + K &\mapsto \phi(r)
	\end{align*}
	By the 1st isomorphism theorem for groups, this is well-defined, a bijection and a group homomorphism under addition. Also
	\[
		\Phi(1_R + K) = \phi(1_R) = 1_S
	,\]
	and
	\begin{align*}
		\Phi((r_1 + K)(r_2 + K)) &= \Phi(r_1r_2 + K) = \phi(r_1r_2) \\
					 &= \phi(r_1)\phi(r_2) = \Phi(r_1 + K)\Phi(r_2 + K).
	\end{align*}
	Thus $\Phi$ is a ring isomorphism.
\end{adjustbox}

\begin{theorem}[Second Isomorphism Theorem]
	Let $R \leq S$ and $J \lhd S$. Then $R \cap J \lhd R$, $R + J = \{r + j \mid r \in R, j \in J\} \leq S$ and
	\[
		R/(R \cap J) \cong (R+J)/J \leq S/J
	.\]
\end{theorem}

\begin{adjustbox}{minipage = \columnwidth - 25.5pt, margin=1em, frame=1pt, margin=0em}
	\textbf{Proof:} By the second isomorphism theorem for groups, $R + J$ is a subgroup of $(S, +)$, and also $1_S \in R + J$. Now for $r_1, r_2 \in R$, $j_1, j_2 \in J$,
	\[
		(r_1 + j_1)(r_2 + j_2) = r_1r_2 + r_1j_2 + r_2j_1 + j_1j_2 = r_3 + j_3
	,\]
	where $r_3 = r_1r_2$ and $j_3 = r_1j_2 + r_2j_1 + j_1j_2$. So $R + J$ is a subring of $S$. Let
	\begin{align*}
		\phi : R &\to S/J \\
		r &\mapsto r + J
	\end{align*}
\end{adjustbox}

\begin{adjustbox}{minipage = \columnwidth - 25.5pt, margin=1em, frame=1pt, margin=0em}
	This is a composition of the inclusion $R \leq S$ and $S \to S/J$, hence $\phi$ is a ring homomorphism. Now
	\[
		\Ker(\phi) = \{r \in R \mid r + J = J\} = R + J \lhd R
	,\]
	\[
		\Img(\phi) = \{r + J \mid r \in R\} = (R+J)/J \leq S/J
	.\]
	We finish by applying the first isomorphism theorem.

\end{adjustbox}

\begin{remark}
	Let $I \lhd R$. There is a bijection between ideals in $R/I$, and ideals of $R$ containing $I$, given by
	\begin{align*}
		K &\mapsto \{r \in R \mid r + I \in K\} \\
			J/I &\mapsfrom I
		\end{align*}
\end{remark}

\begin{theorem}[Third Isomorphism Theorem]
	Let $I \lhd R$, $J \lhd R$ with $I \lhd J$. Then $J/I \lhd R/I$, and
	\[
		(R/I)/(J/I) \cong R/J
	.\]
\end{theorem}

\begin{adjustbox}{minipage = \columnwidth - 25.5pt, margin=1em, frame=1pt, margin=0em}
\textbf{Proof:} Consider $\phi : R/I \to R/J$, given by $r + I \mapsto r + J$. This is surjective ring homomorphism, and
\[
	\Ker(\phi) = \{r + I \mid r \in J\} = J/I \lhd R/I
.\]
Now we finish by applying the first isomorphism theorem.
\end{adjustbox}

We can now prove that $R[X]/(X^2 + 1) \cong \mathbb{C}$ using the first isomorphism theorem. Define
\begin{align*}
	\phi : \mathbb{R}[X] &\to \mathbb{C} \\
	f(x) = \sum a_k x^k &\mapsto f(i) = \sum a_k i^{k}
\end{align*}
Then $\Ker(\phi) = (X^2 + 1)$ by the division algorithm, and $\Img(\phi) = \mathbb{C}$, since $\phi(a + bX) = a + bi$. Thus,
\[
	R[X] / (X^2 + 1) \cong \mathbb{C}
.\]

\begin{adjustbox}{minipage = \columnwidth - 25.5pt, margin=1em, frame=1pt, margin=0em}
\textbf{Example:} Let $R$ be a ring. Then there exists a unique ring homomorphism $i : \mathbb{Z} \to R$, given by
\begin{align*}
	0 &\mapsto 0_R, \\
	1 &\mapsto 1_R, \\
	n &\mapsto \underbrace{1_R + \cdots + 1_R}_{n \text{ times}}, \\
	-n &\mapsto -(1_R + \cdots + 1_R).
\end{align*}
Since $\Ker(i) \lhd \mathbb{Z}$, we have $\Ker(i) = n \mathbb{Z}$ for some $n \in \mathbb{Z}_{\geq 0}$. By the first isomorphism theorem,
\[
	\mathbb{Z} / n \mathbb{Z} \cong \Img(i) \leq R
.\]
\end{adjustbox}

\begin{definition}\index{characteristic}
	We call $n$ the \textbf{characteristic} of $R$.
\end{definition}

For example, $\mathbb{Z}, \mathbb{Q}, \mathbb{R}, \mathbb{C}$ all have characteristic 0, while $\mathbb{Z}/p \mathbb{Z}$ and $\mathbb{Z}/p \mathbb{Z}[X]$ have characteristic $p$.

\newpage

\section[Integral Domains and Ideals]{Integral Domains, Maximal Ideals and Prime Ideals}%
\label{sec:integral_domains_maximal_ideals_and_prime_ideals}

\subsection{Integral Domains}%
\label{sub:integral_domains}

\begin{definition}\index{integral domain}
	An \textbf{integral domain} is a ring with $0 \neq 1$ and such that for $a, b \in R$, $ab = 0 \implies a = 0$ or $b = 0$.
\end{definition}

A \textbf{zero-divisor} in a ring $R$ is a non-zero element $a \in R$ such that $ab = 0$ for some $0 \neq b \in R$. So an integral domain is a ring with no zero-divisors.\index{zero-divisor}

\textbf{Examples:}
\begin{enumerate}[label= (\roman*)]
	\item All fields are integral domains (if $ab = 0$, multiplying by $b^{-1}$ gives $a = 0$).
	\item Any subring of an integral domain is an integral domain.
	\item $\mathbb{Z} \times \mathbb{Z}$ is not an integral domain, since $(1, 0) \cdot (0, 1) = (0, 0)$.
\end{enumerate}

\begin{lemma}
	Let $R$ be an integral domain. Then $R[X]$ is an integral domain.
\end{lemma}

\begin{adjustbox}{minipage = \columnwidth - 25.5pt, margin=1em, frame=1pt, margin=0em}
	\textbf{Proof:} Write $f(X) = a_mX^{m} + \cdots + a_0$, $g(X) = b_nX^{n} + \cdots + b_0$, with $a_m \neq 0$ and $b_n \neq 0$. Then
	\[
		f(X)g(X) = a_mb_nX^{m + n} + \cdots
	.\]
	However $a_mb_n \neq 0$ since $a_m, b_n \in R$ and $R$ is an integral domain. Thus $\deg(fg) = m + n = \deg(f) + \deg(g)$, so we can conclude $fg \neq 0$.
\end{adjustbox}

\begin{definition}
	A polynomial $f(x) = a_nX^{n} + a_{n-1}X^{n-1} + \cdots + a_0 \in R[X]$ is monic if $a_n = 1_{R}$.
\end{definition}

\begin{lemma}
	Let $R$ be an integral domain and $0 \neq f \in R[X]$. Let $\text{\normalfont Roots}(f) = \{a \in R \mid f(a) = 0\}$. Then,
	\[
		|\text{\normalfont Roots}(f)| \leq \deg(f)
	.\]
\end{lemma}

\begin{theorem}
	Let $F$ be a field. Then any finite subgroup $G \leq (F^{\times}, \cdot)$ is cyclic.
\end{theorem}

\begin{adjustbox}{minipage = \columnwidth - 25.5pt, margin=1em, frame=1pt, margin=0em}
	\textbf{Proof:} Note $G$ is a finite abelian group. If $G$ is not cyclic, then there exists $H \leq G$ such that $H \cong C_{d_1} \times C_{d_1}$, where $2 \leq d_1$.

	If we then consider the polynomial
	\[
		f(x) = x^{d_1} - 1
	,\]
	then this has degree $d_1$ and at least $d_1^2$ roots, which is a contradiction.
\end{adjustbox}

This implies that $(\mathbb{Z} / p \mathbb{Z})^{\times}$ is cyclic.

\begin{proposition}
	Any finite integral domain is a field.
\end{proposition}

\begin{adjustbox}{minipage = \columnwidth - 25.5pt, margin=1em, frame=1pt, margin=0em}
	\textbf{Proof:} Let $R$ be a finite integral domain. Let $0 \neq a \in R$. Consider map
	\begin{align*}
		\phi : R &\to R \\
		x &\mapsto ax
	\end{align*}
	If $\phi(x) = \phi(y)$, then $a \cdot (x - y) = 0$, so $x = y$, since $R$ is an integral domain. Thus $\phi$ is injective, so it must be surjective since $R$ is finite. Thus there exists $b$ such that $ab = 1$, hence $a$ is a unit, so $R$ is a field.
\end{adjustbox}

\begin{theorem}
	Let $R$ be an integral domain. Then there exists a field $F$ such that
	\begin{enumerate}[\normalfont(i)]
		\item $R \leq F$.
		\item Every element of $F$ can be written in the form $a \cdot b^{-1}$, where $a, b \in R$ with $b \neq 0$.
	\end{enumerate}
	$F$ is called the \textbf{field of fractions} of $R$.\index{field of fractions}
\end{theorem}

\begin{adjustbox}{minipage = \columnwidth - 25.5pt, margin=1em, frame=1pt, margin=0em}
	\textbf{Proof:} Consider the set $S = \{(a, b) \in R^2, b \neq 0\}$ and the equivalence relation on $S$ given by
	\[
		(a, b) \sim (c, d) \iff ad - bc = 0
	.\]
	This is clearly reflexive and symmetric. For transitivity, if $(a, b) \sim (c, d) \sim (e, f)$, then
	\begin{align*}
		(ad)f = (bc)f = b(cf) &= b(de) \\
		\implies d(af - bc) &= 0.
	\end{align*}
\end{adjustbox}

\begin{adjustbox}{minipage = \columnwidth - 25.5pt, margin=1em, frame=1pt, margin=0em}
	Since $R$ is an integral domain and $d \neq 0$, this gives $af - be = 0$, i.e. $(a, b) \sim (e, f)$. So let $F = S/\sim$, and write $a/b$ for $[(a, b)]$.
	We can now define operations on $F$. Define
	\[
		\frac{a}{b} + \frac{c}{d} = \frac{ad + bc}{bd} \quad \text{and} \quad \frac{a}{b} \cdot \frac{c}{d} = \frac{ac}{bd}
	.\]
	It can be checked that these equations are well defined and make $F$ into a ring, with
	\[
		0_{F} = \frac{0_{R}}{1_{R}} \quad \text{and} \quad 1_{F} = \frac{1_{R}}{1_{R}}
	.\]
	Now if $a/b \neq 0_{F}$, then $a \neq 0_{R}$ and
	\[
	\frac{a}{b} \cdot \frac{b}{a} = \frac{ab}{ab} = \frac{1_{R}}{1_{R}} = 1_{F}
	.\]
	So $F$ is a field. Checking our conditions, we get that
	\[
		R \cong \left\{ \frac{r}{1_{R}} \mid r \in R \right\} \leq F
	.\]
	Moreover, by definition $a/b = a \cdot b^{-1}$ as required.
\end{adjustbox}

Examples of this are $\mathbb{Z}$, with field of fractions $\mathbb{Q}$, and $\mathbb{C}[X]$, with field of fractions $\mathbb{C}(X)$, the field of rational functions in $X$.

\subsection{Maximal Ideals}%
\label{sub:maximal_ideals}

\begin{definition}\index{ideal!maximal}
	An ideal $I \lhd R$ is \textbf{maximal} if $I \neq R$, and if $I \subseteq J \lhd R$, then $J = I$ or $R$.
\end{definition}

\begin{lemma}
	A (non-zero) ring $R$ is a field if and only if its only ideals are $(0)$ and $R$.
\end{lemma}

\begin{adjustbox}{minipage = \columnwidth - 25.5pt, margin=1em, frame=1pt, margin=0em}
	\textbf{Proof:} If $0 \neq I \lhd R$, then $I$ contains a unit so $I = R$. Now if $0 \neq x \in R$, then the ideal $(x)$ is non-zero, hence $(x) = R$ so $x$ is a unit.
\end{adjustbox}

\begin{proposition}
	Let $I \lhd R$ be an ideal. Then $I$ is maximal if and only if $R/I$ is a field.
\end{proposition}

\begin{adjustbox}{minipage = \columnwidth - 25.5pt, margin=1em, frame=1pt, margin=0em}
	\textbf{Proof:} Note $R/I$ is a field if and only if $I/I$ and $R/I$ are the only ideals in $R/I$. However, we have seen this implies that $I$ and $R$ are the only ideals in $R$ containing $I$, so $I \lhd R$ is maximal.
\end{adjustbox}

\subsection{Prime Ideals}%
\label{sub:prime_ideals}

\begin{definition}\index{ideal!prime}
	An ideal $I \lhd R$ is prime if $I \neq R$ and whenever $a, b \in R$ with $ab \in I$, we have either $a \in I$ or $b \in I$.
\end{definition}

Note the ideal $n \mathbb{Z} \lhd \mathbb{Z}$ is prime if and only if $n = 0$ or $n = p$ is a prime number.

\begin{proposition}
	Let $I \lhd R$ be an ideal. Then $I$ is prime if and only if $R/I$ is an integral domain.
\end{proposition}

\begin{adjustbox}{minipage = \columnwidth - 25.5pt, margin=1em, frame=1pt, margin=0em}
	\textbf{Proof:} $I$ is prime if and only if, whenever $a, b \in R$ with $ab \in I$, we have either $a \in I$ or $b \in I$. However this means whenever $a + I, b + I \in R/I$ with $(a + I)(b + I) = 0 + I$, then either $a + I = 0 + I$ or $b + I = 0 + I$. This is equivalent to $R/I$ being an integral domain.
\end{adjustbox}

\begin{remark}
	\begin{itemize}
		\item[]
		\item Since any field is an integral domain, a maximal ideal is always a prime ideal.
		\item If the characteristic of $R$ is $n$, then $\mathbb{Z} / n \mathbb{Z} \leq R$. So if $R$ is an integral domain, then $\mathbb{Z} / n \mathbb{Z}$ is an integral domain,
		\[
				n \mathbb{Z} \lhd \mathbb{Z} \text{ a prime ideal } \implies n = 0 \text{ or } p \text{ prime}.
		\]
		In particular, a field has characteristic 0 (and so contains $\mathbb{Q}$ ) or has characteristic $p$ (and contains $\mathbb{Z} / p \mathbb{Z} = \mathbb{F}_p$).	
	\end{itemize}
\end{remark}

\newpage
\section{Factorisation in Integral Domains}%
\label{sec:factorisation_in_integral_domains}

In this section $R$ is an integral domain.

\begin{definition}
	\begin{enumerate}[label = (\roman*)]
		\item[]
		\item $a \in R$ is a unit if there exists $b \in R$ with $ab = 1$ (equivalently $(a) = R$). We let $R^{\times}$ be the units in $R$.\index{unit}
		\item $a \in R$ divides $b \in R$ if there exists $c \in R$ such that $b = ac$ (equivalently $(b) \subseteq (a)$).
		\item $a, b \in R$ are associates if $a = bc$ for some unit $c \in R$ (equivalently $(b) = (a)$).\index{associates}
		\item $r \in R$ is irreducible if $r \neq 0$, $r$ is not a unit and $r = ab \implies a$ or $b$ is a unit.\index{irreducible}
		\item $r \in R$ is prime if $r \neq 0$, $r$ is not a unit and $r \mid ab \implies r \mid a$ or $r \mid b$.\index{prime}
	\end{enumerate}
\end{definition}

\begin{remark}
	These properties depend on the ring $R$, for example $2$ is prime and irreducible in $\mathbb{Z}$, but it is a unit in $\mathbb{Q}$. Moreover $2X$ is irreducible in $\mathbb{Q}[X]$, but not in $\mathbb{Z}[X]$.
\end{remark}

\begin{lemma}
	$(r) \lhd R$ is a prime ideal if and only if $r = 0$ or $r$ is a prime element.
\end{lemma}

\begin{adjustbox}{minipage = \columnwidth - 25.5pt, margin=1em, frame=1pt, margin=0em}
	\textbf{Proof:} Suppose $(r)$ is a prime ideal and $r \neq 0$. Since prime ideals are proper, $r$ is not a unit. Now if $r \mid ab$, then $ab \in (r)$, so $a \in (r)$ or $b \in (r)$, meaning $r \mid a$ or $r \mid b$, so $r$ is prime.

	If $r = 0$, then $(0) \lhd R$ is a prime ideal since $R$ is integral. Let $r \in R$ be prime. Then $(r) \neq R$ since $r \not \in R^{\times}$. Now if $ab \in (r)$, then $r \mid ab$, so $r \mid a$ or $r \mid b$. Hence either $a \in (r)$ or $b \in (r)$, so $(r)$ is a prime ideal.
\end{adjustbox}

\begin{lemma}
	If $r \in R$ is prime, then it is irreducible.
\end{lemma}

\begin{adjustbox}{minipage = \columnwidth - 25.5pt, margin=1em, frame=1pt, margin=0em}
\textbf{Proof:} Since $r$ is a prime, $r \neq 0$ and $r \not \in R^{\times}$. Suppose $r = ab$. Then $r \mid ab$ so $r \mid a$ or $r \mid b$. Say that $r \mid a$, so $a = rc$ for some $c \in R$. But then
\[
	r = ab = rcb \implies r(1 - bc) = 0
.\]
Since we assumed $r \neq 0$, then $bc = 1$ since $R$ is an integral domain. Hence $b$ is a unit.
\end{adjustbox}

Note that the converse does not hold in general. For example, let $R = \mathbb{Z}[\sqrt{-5}]$. Since $R$ is a subring of a field, it is an integral domain. We can define a norm
\begin{align*}
	N : R &\to \mathbb{Z} \\
	a + b \sqrt{-5} &\mapsto a^2 + 5b^2
\end{align*}
Note that $N(z_1z_2) = N(z_1)(z_2)$. Now if $r \in R^{\times}$, i.e. $rs = 1$, then
\[
	N(r)N(s) = N(1) = 1
.\]
But then $a^2 + 5b^2 = 1$, i.e. $r = \pm 1$. Then 2 is irreducible since $N(2) = 4$ and there are no elements $r$ such that $N(r) = 2$. Similarly, we can show $3$, $1 + \sqrt{-5}$, $1 - \sqrt{-5}$ are irreducible. Now,
\[
	 (1 + \sqrt{-5}) \cdot (1 - \sqrt{-5}) = 6 = 2 \cdot 3
.\]
Thus $2 \mid (1 + \sqrt{-5})(1 - \sqrt{-5})$ but $2$ doesn't divide either of these elements, so $2$ is not prime.

\subsection{Principal Ideal Domains}%
\label{sub:principal_ideal_domains}

\begin{definition}\index{principal ideal domain}\index{PID}
	An integral domain $R$ is a \textbf{principal ideal domain} (PID) if any ideal is principal, i.e. if $I \lhd R$, then $I = (r)$.
\end{definition}

\begin{proposition}
	Let $R$ be a PID. Then every irreducible element of $R$ is prime.
\end{proposition}

\begin{adjustbox}{minipage = \columnwidth - 25.5pt, margin=1em, frame=1pt, margin=0em}
	\textbf{Proof:} Let $r \in R$ be irreducible, $r \mid ab$. Since $R$ is a PID, $(a, r) = (d)$ for some $d \in R$, i.e. $r = cd$ for some $c \in R$. Since $r$ is irreducible, this says either $c$ or $d$ is a unit.

	If $c$ is a unit, then $(a, r) = (r)$, so $r \mid a$.

	If $d$ is a unit, then $(a, r) = R$. So there exists $s, t \in R$ such that
	\[
	sa + tr = 1
	.\]
	Then multiplying by $b$, we get
	\[
	b = sab + trb
	.\]
	Since $r$ divides both terms on the right hand side, $r \mid b$, as desired.
\end{adjustbox}

\begin{lemma}
	Let $R$ be a PID, and $0 \neq r \in R$. Then $r$ is irreducible if and only if $(r)$ is a maximal ideal.
\end{lemma}

\begin{adjustbox}{minipage = \columnwidth - 25.5pt, margin=1em, frame=1pt, margin=0em}
	\textbf{Proof:} Suppose $r$ is irreducible. Then $r \not \in R^{\times}$, so $(r) \neq R$. Suppose $(r) \subseteq I \subseteq R$, where $I \lhd R$. However since $R$ is a PID,
	\[
		I = (a) \implies r = ab
	.\]
	Since $r$ is irreducible, either $a = R^{\times}$, so $I = R$, or $b$ is a unit, so $a$ and $b$ are associates, meaning $I = (a) = (r)$.

	Now suppose $(r)$ is maximal. We have assumed $r \neq 0$ and since $(r)$ is proper, $r \not \in R^{\times}$. Suppose that $r = ab$. Then
	\[
		(r) \subseteq (a) \subseteq R
	.\]
	Since $(r)$ is maximal, either $(a) = R$, meaning $a$ is a unit, or $(a) = (r)$, meaning $a$ and $r$ are associates, so $b$ is a unit.
\end{adjustbox}

\begin{remark}
	\begin{enumerate}[label = (\roman*)]
		\item[]
		\item The reverse direction holds without assuming $R$ is a PID.
		\item Let $R$ be a PID. Then
			\[
				(r) \text{ maximal } \iff r \text{ irreducible } \iff r \text{ prime } \iff (r) \text{ prime}
			.\]
			Therefore there is a bijection between non-zero prime ideals and non-zero maximal ideals.
	\end{enumerate}
\end{remark}

\subsection{Euclidean Domains}%
\label{sub:euclidean_domains}

\begin{definition}\index{euclidean domain}\index{ED}
	An integral domain is a \textbf{Euclidean domain} if there exists a function $\phi : \mathbb{R} \setminus \{0\} \to \mathbb{Z}_{> 0}$ such that
	\begin{enumerate}[label = (\roman*)]
		\item If $a \mid b$, then $\phi(a) \leq \phi(b)$.
		\item If $a, b \in R$ with $b \neq 0$, then there exists $q, r \in R$ with $a = bq + r$, and either $r = 0$, or $\phi(r) < \phi(b)$.
	\end{enumerate}
\end{definition}

For example, $\mathbb{Z}$ is a Euclidean Domain with function $\phi(n) = |n|$.

\begin{proposition}
	If $R$ is a Euclidean domain, then it is a PID.
\end{proposition}

\begin{adjustbox}{minipage = \columnwidth - 25.5pt, margin=1em, frame=1pt, margin=0em}
	\textbf{Proof:} Let $R$ have Euclidean function $\phi$, and let $I \lhd R$ be non-zero. Choose $b \in I \setminus \{0\}$, with $\phi(b)$ minimal. Then $(b) \subseteq I$.

	For $a \in I$, write $a = bq + r$ with $q, r \in R$ and either $r = 0$ or $\phi(r) < \phi(b)$. Since $r = bq - a \in I$, we cannot have $\phi(r) < \phi(b)$ by the minimality of $b$. So $r = 0$, and so $a = bq$ for element $a \in I$. Therefore, $I = (b)$.
\end{adjustbox}

\begin{remark}
	In the proof, we only use (ii). Property (i) allows us to describe the units of $R$ as
			\[
				R^{\times} = \{u \in R \setminus\{0\} \mid \phi(a) = \phi(1)\}
			.\]
			In fact, if there exists a function satisfying (ii), then we can find a function satisfying both (i) and (ii).
\end{remark}

\begin{adjustbox}{minipage = \columnwidth - 25.5pt, margin=1em, frame=1pt, margin=0em}
\textbf{Examples:}
\begin{enumerate}[label = (\roman*)]
	\item If $F$ is a field, then $F[X]$ is an ED with Euclidean function $\phi(f) = \deg f$.
	\item $R = \mathbb{Z}[i]$ is an ED with Euclidean function $\phi(a + bi) = N(a + bi) = a^2 + b^2$. Since $N$ is multiplicative, property (i) holds, and for property (ii), let $z_1, z_2 \in \mathbb{Z}[i]$ with $z_2 \neq 0$. Consider $z_1/z_2 \in \mathbb{C}$. Then there exists $q \in \mathbb{Z}[i]$ such that
		\[
		\left| \frac{z_1}{z_2} - q \right| < 1
		.\]
		Setting $r = z_1 - z_2 q \in \mathbb{Z}[i]$, then $z_1 = z_2q + r$, and
		\[
			\phi(r) = |r|^2 = |z_1 - z_2q|^2 < |z_2|^2 = \phi(z_2)
		.\]
\end{enumerate}
Therefore both $\mathbb{Z}[i]$ and $F[X]$ for a field $F$ are PID's.
\end{adjustbox}

\textbf{Applications:}

\begin{enumerate}[(1)]
	\item Let $A \in M_n(F)$, and let
\[
	I = \{f \in F[X] \mid f(A) = 0\}
.\]
Note $I \subseteq F[X]$ is an ideal, and so $I = (f)$ for some $f \in F[X]$, since $F[X]$ is a PID. We may assume $f$ is monic by multiplying by a unit. Then for $g \in F[X]$,
\[
	g(A) = 0 \iff g = I \iff g = (f) \iff f \mid g
.\]
Therefore $f$ is the minimal polynomial of $A$.
	\item Let $\mathbb{F}_2 = \mathbb{Z} / 2 \mathbb{Z}$. Then let
\[
	f(X) = X^3 + X + 1 \in \mathbb{F}_2[X]
.\]
If $f(X) = g(X)h(X)$ with $q, h \in \mathbb{F}_2[X]$ and $\deg(g), \deg(h) > 0$, then either $\deg(g) = 1$ or $\deg(h) = 1$. However, this is equivalent to $f$ having a root, but
\[
	f(0) = 1 \neq 0, \quad f(1) = 1 \neq 0
.\]
Since $\mathbb{F}_2[X]$ is a PID, the ideal generated by $f$ is maximal ideal, hence
\[
	F = \mathbb{F}_2[X] / (f)
\]
is a field. However, note
\[
	F = \{aX^2 + bX + c \mid a, b, c \in \mathbb{F}_2\}
.\]
Thus this is a field of order 8.
	\item $\mathbb{Z}[X]$ is not a PID, by considering $(2, X)$. Indeed, suppose $(2, X) = I = (f)$ for some $f \in \mathbb{Z}[X]$. Then $\deg (f) = \deg (g) = 0$, and so $f \in \mathbb{Z}$, meaning $f = \pm 1$ or $\pm 2$. However in both these cases, $f$ does not divide $X$.
\end{enumerate}

\subsection{Unique Factorisation Domains}%
\label{sub:unique_factorisation_domains}

\begin{definition}\index{unique factorisation domain}\index{UFD}
	An integral domain is a unique factorisation domain (UFD) if
	\begin{enumerate}[label = (\roman*)]
		\item Every non-zero, non-unit is a product of irreducibles.
		\item If $p_1\ldots p_n = q_1 \ldots q_m$, where $p_i, q_j$ are irreducible, then $m = n$ and we can reorder such that $p_i$ is an associate of $q_i$ for all $i$.
	\end{enumerate}
\end{definition}

\begin{proposition}
	Let $R$ be an integral domain satisfying $(i)$. Then $R$ is an irreducible if and only if every irreducible is prime.
\end{proposition}

\begin{adjustbox}{minipage = \columnwidth - 25.5pt, margin=1em, frame=1pt, margin=0em}
\textbf{Proof:} If $p$ is irreducible, then $p$ is prime by unique factorisation. So suppose $p_1\ldots p_n = q_1 \ldots q_m$ with $p_i, q_j$ irreducible. Since $p_1$ is prime, $p_1 = q_1 u$ for $u \in R$. Since $q_1$ is irreducible, $u$ is a unit and so $p_1$ and $q_1$ are associates. Then the result follows by induction.
\end{adjustbox}

\begin{lemma}
	Let $R$ be a PID an let $I_1 \subseteq I_2 \subseteq I_3 \subseteq \cdots$ a nested sequence of ideal. Then there exists $N$ such that $I_n = I_{n + 1}$ for all $n \geq N$.
\end{lemma}

\begin{adjustbox}{minipage = \columnwidth - 25.5pt, margin=1em, frame=1pt, margin=0em}
	\textbf{Proof:} Let $I = \bigcup I_i$. Then since $R$ is a PID, $I = (a)$ for some $a \in R$. However considering $(a)$,
	\[
		(a) \subseteq I_N \subseteq I_n \subseteq I = (a)
	,\]
	so $I_n = I$.
\end{adjustbox}

\begin{theorem}
	If $R$ is a PID, then $R$ is a UFD.
\end{theorem}

\begin{adjustbox}{minipage = \columnwidth - 25.5pt, margin=1em, frame=1pt, margin=0em}
	\textbf{Proof:} We will check (i) and (ii). If $x \in R$ and $x$ is not a product of irreducibles, then $x$ is not irreducible, so $x = x_1 y_1$ where one of $x_1, y_1$ are not a product of irreducibles. As a result,
	\[
		(x) \subset (x_1) \subset \ldots
	\]	
	This is a contradiction. Now since irreducibles are primes in PIDs, we conclude.
\end{adjustbox}

\begin{definition}
	Let $R$ be an integral domain.
	\begin{enumerate}[label = (\roman*)]
		\item $d \in R$ is the greatest common divisor of $a_1, \ldots, a_n \in R$ if $d \mid a_i$ for all $i$, and if $d' \mid a_i$ for all $i$, then $d' \mid d$.
		\item $m \in R$ is a least common multiple of $a_1, \ldots, a_n \in R$ if $a_i \mid m$, and if $a_i \mid m'$ for all $i$, then $m \mid m'$.
	\end{enumerate}
	
\end{definition}

\begin{proposition}
	In a UFD, both GCD's and LCM's exist.
\end{proposition}

\begin{adjustbox}{minipage = \columnwidth - 25.5pt, margin=1em, frame=1pt, margin=0em}
\textbf{Proof:} Let $a_i = u_i \prod p_j^{n_{ij}}$, where $u_i$ is a unit and the $p_i$ are irreducible and not associate to each other.

We claim that $d = \prod p_j^{m_j}$, where
\[
	m_j = \min n_{ij}
.\]
Certainly $d \mid a_i$, and if $d' \mid a_i$, then $d' = u \prod p_j^{t_j}$, where $t_j \leq m_j$. Therefore $d' \mid d$, as required.

The argument for LCM's is similar.
\end{adjustbox}


\newpage

\section{Factorisation in Polynomial Rings}%
\label{sec:factorisation_in_polynomial_rings}

The main theorem is the following:

\begin{theorem}
	If $R$ is a UFD, then $R[X]$ is also a UFD.
\end{theorem}

We will say that $R$ is a UFD with a field of fractions $F$. Then $R[X] \leq F[X]$. Moreover $F[X]$ is a ED, hence a PID and a UFD.

\begin{definition}
	The \textbf{content} of $f = a_n X^{n} + \cdots + a_1 X + a_0 \in R[X]$ is\index{content}
	\[
		c(f) = \gcd (a_0, \ldots, a_n)
	.\]
	This is well defined up to multiplication by a unit. We say that $f$ is \textbf{primitive} if $c(f)$ is a unit.\index{primitive polynomial}
\end{definition}

\begin{lemma}
	\begin{enumerate}[\normalfont(i)]
		\item[]
		\item If $f, g \in R[X]$ are primitive, then $fg$ is primitive.
		\item If $f, g \in R[X]$, then $c(fg) = c(f)c(g)$, up to a unit.
	\end{enumerate}
\end{lemma}

\begin{adjustbox}{minipage = \columnwidth - 25.5pt, margin=1em, frame=1pt, margin=0em}
	\textbf{Proof:} Let $f = a_nX^{n} + \cdots + a_0$, $g = b_mX^{m} + \cdots + b_0$. If $fg$ is not primitive, then $c(fg)$ is not a unit, so there is some prime $p$ such that $p \mid c(fg)$. Since $f$ and $g$ are primitive, $p \nmid c(f)$ and $p \nmid c(g)$. Now suppose
	\[
	p \mid a_0, p \mid a_1, \ldots , p \nmid a_k,
	\]
	\[
	p \mid b_0, p \mid b_1, \ldots, p \nmid b_l
	.\]
	Then the coefficient of $X^{k + l}$ in $fg$ is
	\[
	\sum_{i + j = k + l}a_i b_j = \cdots + a_{k - 1}b_{l + 1} + a_kb_l + a_{k + 1}b_{l - 1} + \cdots
	.\]
	Hence $p \mid a_kb_l$, so $p \mid a_k$ or $p \mid b_l$, since $p$ is prime, contradiction.

	To prove the second part, let $f = c(f) \cdot f_0$, $g = c(g) \cdot g_0$, where $f_0, g_0$ are primitive. Then,
	\[
		c(fg) = c(c(f)c(g) f_0 g_0) = c(f)c(g) c(f_0g_0) = c(f)c(g)
	.\]
\end{adjustbox}

\begin{corollary}
	Let $p \in R$ be prime. Then $p$ is prime in $R[X]$.
\end{corollary}

\begin{adjustbox}{minipage = \columnwidth - 25.5pt, margin=1em, frame=1pt, margin=0em}
	\textbf{Proof:} $R[X]^{\times} = R^{\times}$, so $p$ is not a unit in $R[X]$. Let $f \in R[X]$. Then
	\[
		p \mid f \iff p \mid c(f)
	.\]
	Therefore, if $p \mid gh$ in $R[X]$, we have
	\[
		p \mid c(gh) = c(g)c(h) \implies p \mid c(g) \text{ or } p \mid c(h) \implies p \mid g \text{ or } p \mid h
	.\]
	Thus $p$ is prime in $R[X]$.
\end{adjustbox}

\begin{lemma}
	Let $f, g \in R[X]$ with $g$ primitive. If $g \mid f$ in $F[X]$, then $g \mid f$ in $R[X]$.
\end{lemma}

\begin{adjustbox}{minipage = \columnwidth - 25.5pt, margin=1em, frame=1pt, margin=0em}
	\textbf{Proof:} Let $f = gh$, where $h \in F[X]$. Let $0 \neq a \in R$ such that $ah \in R[X]$, and write $ah = c(ah)h_0$, with $h_0$ primitive. Then
	\[
		af = c(ah)h_0 g
	.\]
	Taking the contents, we find that $a \mid c(ah)$. However this means that $h \in R[X]$, and $g \mid f$ in $R[X]$.
\end{adjustbox}

\subsection{Gauss' Lemma}%
\label{sub:gauss_lemma}

\begin{lemma}[Gauss' Lemma]\index{Gauss' lemma}
	Let $f \in R[X]$ be primitive. Then $f$ irreducible in $R[X]$ implies $f$ irreducible in $F[X]$.
\end{lemma}

\begin{adjustbox}{minipage = \columnwidth - 25.5pt, margin=1em, frame=1pt, margin=0em}
	\textbf{Proof:} Since $f$ is irreducible and primitive, we have $\deg (f) > 0$, otherwise $f$ is a unit.

	Suppose that $f$ is not irreducible in $F[X]$, say $f = gh$, where $g, h \in F[X]$ with $\deg(g), \deg(h) > 0$. Then we may take $\lambda \in F^{\times}$ such that $\lambda^{-1} g \in R[X]$ is primitive by first clearing out the denominator, then dividing by the content. Upon replacing $g$ by $\lambda^{-1}g$ and $h$ with $\lambda h$, we may assume $g \in R[X]$ is primitive.

	However the previous lemma implies that $h \in R[X]$, and so $f = gh$ in $R[X]$, with $\deg g, \deg h > 0$.
\end{adjustbox}

\begin{lemma}
	Let $g \in R[X]$ be primitive. Then $g$ prime in $F[X]$ implies that $g$ is prime in $R[X]$.
\end{lemma}

\begin{adjustbox}{minipage = \columnwidth - 25.5pt, margin=1em, frame=1pt, margin=0em}
	\textbf{Proof:} Suppose $f_1, f_2 \in R[X]$, and $g \mid f_1 f_2$ in $R[X]$. Then
	\begin{align*}
		g \text{ prime in } F[X] \implies g \mid f_1 \text{ or } g \mid f_2 \text{ in } F[X] \implies g \mid f_1 \text{ or } g \mid f_2 \text{ in } R[X].
	\end{align*}
	Thus $g$ is prime in $R[X]$.
\end{adjustbox}

Now we go back and prove our main theorem. Let $f \in R[X]$, and write $f = c(f)f_0$, with $f_0 \in R[X]$ primitive. Since $R$ is a UFD, $c(f)$ is a product of irreducibles in $R$ (which are also irreducible in $R[X]$).

If $f_0$ is not irreducible, say $f_0 = gh$, then $\deg g, \deg h > 0$, since $f_0$ is primitive, and so $g, h$ are also primitive. By induction, $f_0$ is a product of irreducibles in $R[X]$, establishing part (i) in the definition of the UFD.

It suffices to show that if $f \in R[X]$ is irreducible, then $f$ is prime. Write $f = c(f)f_0$, where $f_0 \in R[X]$ is primitive. Then $f$ is irreducible implies $f$ is constant or primitive.
\begin{itemize}
	\item[Case 1:] $f$ is constant. Then $f$ is irreducible in $R[X]$ implies $f$ is irreducible in $R$, however then $f$ is prime in $R$ since $R$ is a UFD, and this implies $f$ is prime in $R[X]$, from what we have seen.
	\item [Case 2:] $f$ is primitive. Then $f$ is irreducible in $R[X]$ implies $f$ is irreducible in $F[X]$. However then $f$ is prime in $F[X]$, since $F[X]$ is a UFD, and this implies $f$ is prime in $R[X]$, from our previous lemma.
\end{itemize}

This concludes the proof.

\begin{remark}
	We may show that $f$ primitive and irreducible in $R[X]$ if and only if $f$ is irreducible in $F[X]$, using the fact if $(f)$ is prime in $F[X]$, then it is prime in $R[X]$ (provided $f$ is irreducible).
\end{remark}

\subsection{Applications}%
\label{sub:Applications}

\begin{enumerate}[label = (\roman*)]
	\item Since $\mathbb{Z}$ is a UFD, $\mathbb{Z}[X]$ is a UFD.
	\item If $R$ is a UFD, then applying 11.1 inductively, $R[X_1, \ldots, X_n]$ is a UFD.
\end{enumerate}

\begin{proposition}[Eisenstein's Criterion]\index{Eisenstein's criterion}
	Let $R$ be a UFD and $f(x) = a_nX^{n} + \cdots + a_1 X + a_0 \in R[X]$ is primitive. Suppose there exists $p \in R$ irreducible such that
	\begin{itemize}
		\item $p \nmid a_n$,
		\item $p \mid a_i$ for all $0 \leq i \leq n-1$,
		\item $p^2 \nmid a_0$.
	\end{itemize}
	Then $f$ is irreducible in $R[X]$.
\end{proposition}

\begin{adjustbox}{minipage = \columnwidth - 25.5pt, margin=1em, frame=1pt, margin=0em}
	\textbf{Proof:} Suppose $f = gh$, where $g, h \in R[X]$ are not units. Then $f$ primitive implies $\deg g , \deg h > 0$. Letting $g = r_k X^{k} + \cdots + r_0$, $h = s_l X^{l} + \cdots + s_0$, with $k + l = n$. Then
	\[
		p \nmid a_n = r_k s_l \implies p \nmid r_k \text{ and } p \nmid s_l
	,\]
	\[
		p \mid a_0 = r_0 s_0 \implies p \mid r_0 \text{ or } p \mid s_0
	.\]
	 Assume $p \mid r_0$, then there exists $j \leq k$ such that
	 \[
	 p \mid r_0, p \mid r_1, \ldots, p \mid r_{j - 1}, p \nmid r_{j}
	 .\]
	 Then
	 \[
	 a_j = r_0s_j + r_1s_{j - 1} + \cdots + r_{j - 1}s_1 + r_js_0
	 .\]
	 $p$ divides $a_j$ and the first $j$ terms on the right hand side, so it must divide $r_j s_0$. Thus, since $p$ prime and $p \nmid r_j$, $p \mid s_0$. But then
	 \[
	 p^2 \mid r_0 s_0 = a_0
	 ,\]
	 which is a contradiction.
\end{adjustbox}

\begin{enumerate}[resume, label = (\roman*)]
	\item Let $f(X) = X^3 + 2X + 5 \in \mathbb{Z}[X]$, then assuming $f$ is not irreducible for contradiction, this implies
		\[
			f(x) = (X + a)(X^2 + bx + c)
		.\]
		Thus $ac = 5$, but none of $\pm 1, \pm 5$ are units of $f$. Thus by Gauss' Lemma, $f$ is irreducible in $\mathbb{Q}[X]$, and so
		\[
			\mathbb{Q}[X] / (f)
		\]
		is a field.
	\item Let $p \in \mathbb{Z}$ be prime. Then by Eisenstein's criterion, $X^{n} - p$ is irreducible in $\mathbb{Z}[X]$, hence in $\mathbb{Q}[X]$, by Gauss' Lemma.
	\item Let $f(x) = X^{p - 1} + X^{p - 2} \cdots + X + 1 \in \mathbb{Z}[X]$, where $p \in \mathbb{Z}$ is prime. Then Eisenstein's does not apply directly, but note that
		\[
			f(x) = \frac{X^{p} - 1}{X - 1}
		.\]
		Substituting $Y = X - 1$, this gives
		\[
			f(Y + 1) = \frac{(Y + 1)^{p} - 1}{(Y + 1) - 1} = Y^{p - 1} + \binom{p}{1} Y^{p - 2} + \cdots + \binom{p}{p-2} Y + \binom{p}{p-1}
		.\]
		Now using Eisenstein's with $p$, this implies $f(Y + 1)$ is irreducible in $\mathbb{Z}[Y]$, so $f(X)$ is irreducible in $\mathbb{Z}[X]$.
\end{enumerate}

\newpage

\section{Algebraic Integers}%
\label{sec:algebraic_integers}

\subsection{Primes in \texorpdfstring{$\mathbb{Z}[i]$}{Z[i]}}%
\label{sub:primes_in_z_i_}

Recall that $\mathbb{Z}[i] = \{a + bi \mid a, b \in \mathbb{Z}\} \leq \mathbb{C}$ is the ring of Gaussian integers\index{Gaussian integers}, with norm $N(a + bi) = a^2 + b^2$. We have shown this norm makes $\mathbb{Z}[i]$ a ED, hence it is a PID and a UFD, so the primes are the irreducibles in $\mathbb{Z}[i]$.

We know that the units in $\mathbb{Z}[i]$ are $\pm 1, \pm i$ are the units, as these are the only elements with norm 1. We wish to find the primes in $\mathbb{Z}[i]$. Indeed, note
\[
	2 = (1 + i)(1 - i), \quad 5 = (1 - 2i)(1 + 2i)
,\]
so neither 2 nor 3 are prime. However 3 is a prime, since if $3 = ab$, then
\[
	9 = N(3) = N(a)N(b)
.\]
However there are no elements in $\mathbb{Z}[i]$ with norm 3, so one of $a$, $b$ is a unit. Similarly, 7 is a prime. In fact, we can classify exactly when $p$ is prime in $\mathbb{Z}[i]$.

\begin{proposition}
	Let $p \in \mathbb{Z}$ be a prime number. The following are equivalent:
	\begin{enumerate}[label = (\roman*)]
		\item $p$ is not a prime in $\mathbb{Z}[i]$.
		\item $p = a^2 + b^2$ for some $a, b \in \mathbb{Z}$.
		\item $p = 2$ or $p = 1 \pmod 4$.
	\end{enumerate}
\end{proposition}

\begin{adjustbox}{minipage = \columnwidth - 25.5pt, margin=1em, frame=1pt, margin=0em}
	\textbf{Proof:} We show (i) implies (ii). Indeed, if $p = xy$, where $xy$ are not units, then
	\[
		p^2 = N(p) = N(x)N(y)
	.\] 
	Since $N(x), N(y) > 1$, these must both equal $p$. But if $x = a + bi$, then
	\[
		p = N(x) = a^2 + b^2
	.\]
	Now (ii) implies (iii) since the squares mod 4 are 0 and 1, so if $p = a^2 + b^2$, then $p \not \equiv 3 \pmod 4$.

	Now (iii) implies (i), since we have already seen that 2 is not prime. Since $(\mathbb{Z} / p \mathbb{Z})^{\times}$ is cyclic, if $p \equiv 1 \pmod 4$, then it contains an element $x$ of order 4, meaning $x^{4} \equiv 1$ but $x^2 \not \equiv 1$. Thus, $x^2 \equiv -1$, so
	\[
		p \mid x^2 + 1 = (x + i)(x - i)
	.\]
	But $p \nmid x + i$ and $p \nmid x - i$, hence $p$ is not prime in $\mathbb{Z}[i]$.
\end{adjustbox}

\begin{theorem}
	The primes in $\mathbb{Z}[i]$ are (up to associates):
	\begin{enumerate}[label = (\roman*)]
		\item $a + bi$, where $a, b \in \mathbb{Z}$ and $a^2 + b^2 = p$ is a prime number with $p = 2$ or $p \equiv 1 \pmod 4$.
		\item Prime numbers $p \in \mathbb{Z}$ with $p \equiv 3 \pmod 4$.
	\end{enumerate}
\end{theorem}

\begin{adjustbox}{minipage = \columnwidth - 25.5pt, margin=1em, frame=1pt, margin=0em}
\textbf{Proof:} First we check these are prime. If $a^2 + b^2 = p$, then
\[
	p = N(a + bi)
.\]
Hence if $a + bi = uv$, then $N(u) = 1$ or $N(v) = 1$, so one of $u, v$ is a unit. $p \equiv 3 \pmod 4$ we have shown are prime.

Now let $z \in \mathbb{Z}[i]$ be prime. Then $\overline{z} \in \mathbb{Z}[i]$ is prime and
\[
	N(z) = z \overline{z}
\]
is a factorisation into irreducibles. Suppose $p \mid N(z)$, where $p \equiv 3 \pmod 4$. Then $p$ is an associate of either $z$, or of $\overline{z}$, which still means $p$ is an associate of $z$ by conjugation.

Otherwise for all $p \mid N(z)$ we have $p = 2$ or $p \equiv 1 \pmod 4$. This means $p = (a + bi)(a - bi)$ for some primes $a + bi, a - bi$. Therefore, either $z$ is an associate of $a + bi$ or of $a - bi$, as required.
\end{adjustbox}

\begin{remark}
	In the above theorem, if $p = a^2 + b^2$, then $a + bi$ and $a - bi$ are not conjugates, unless $p = 2$.
\end{remark}

\begin{corollary}
	An integer $n \geq 1$ is the sum of two squares if and only if every prime factor $p$ of $n$ with $p \equiv 3 \pmod 4$ divides $n$ to an even power.
\end{corollary}

\begin{adjustbox}{minipage = \columnwidth - 25.5pt, margin=1em, frame=1pt, margin=0em}
\textbf{Proof:}
\begin{align*}
	n = a^2 + b^2 &\iff n = N(z) \iff n \text{ is a product of norms}.
\end{align*}
The above theorem implies that the norms of primes in $\mathbb{Z}[i]$ are the primes $p \in \mathbb{Z}$ with $p \not \equiv 1 \pmod 4$, and squares of primes $p \in \mathbb{Z}$ with $p \equiv 3 \pmod 4$.
\end{adjustbox}

\newpage

\subsection{Algebraic Numbers}%
\label{sub:algebraic_numbers}

\begin{definition}
	\begin{enumerate}[label = (\roman*)]
		\item[]
		\item $\alpha \in \mathbb{C}$ is an algebraic number if there exists a non-zero $f \in \mathbb{Q}[X]$ with $f(\alpha) = 0$.\index{algebraic number}
		\item $\alpha \in \mathbb{C}$ is an algebraic integer if there exists monic $f \in \mathbb{Z}[X]$ with $f(\alpha) = 0$.\index{algebraic integer}
	\end{enumerate}
\end{definition}

In the following, let $R$ be a subring of $S$, and $a \in S$. We write $R[\alpha]$ as the smallest subring of $S$ containing $R$ and $\alpha$.

Let $\alpha$ be an algebraic number, and let
\begin{align*}
	\phi : \mathbb{Q}[X] &\to \mathbb{C} \\
	g(x) &\mapsto g(\alpha)
\end{align*}
$\mathbb{Q}[X]$ is a PID, so $\Ker(\phi) = (f)$ for some $f \in \mathbb{Q}[X]$. Then $f \neq 0$, and upon multiplying $f$ by a unit, we may assume that $f$ is monic.

\begin{definition}\index{minimal polynomial}
	$f$ is the minimal polynomial of $\alpha$.
\end{definition}

By first isomorphism theorem,
\[
	\mathbb{Q}[X] / (f) \cong \mathbb{Q}[\alpha] \leq \mathbb{C}
.\]
Thus $\mathbb{Q}[\alpha]$ is an integral domain, so $f$ is irreducible in $\mathbb{Q}[X]$, and therefore $\mathbb{Q}[\alpha]$ is a field.

\begin{proposition}
	Let $\alpha$ be an algebraic integer and $f \in \mathbb{Q}[X]$ its minimal polynomial. Then $f \in \mathbb{Z}[X]$ and $(f) = \Ker(\theta) \lhd \mathbb{Z}[X]$ where
	\begin{align*}
		\theta : \mathbb{Z}[X] &\to \mathbb{C} \\
		g(x) &\mapsto g(\alpha)
	\end{align*}
\end{proposition}

\begin{adjustbox}{minipage = \columnwidth - 25.5pt, margin=1em, frame=1pt, margin=0em}
	\textbf{Proof:} Let $\lambda \in \mathbb{Q}^{\times}$ be such that $f \lambda \in \mathbb{Z}[X]$ is primitive. Then $\lambda f(\alpha) = 0$, so $\lambda f \in \Ker \theta$.

	If $g \in \Ker \theta$, then $g \in \Ker \phi$ and so $\lambda f \mid g$ in $\mathbb{Q}[X]$, so $\lambda f \mid g$ in $\mathbb{Z}[X]$.

	Since $\alpha$ is an algebraic integer, there exists $g \in \Ker \theta$ monic. Then $\lambda f \mid g$ implies $\lambda = \pm 1$, hence  $f \in \mathbb{Z}[X]$ and $(f) = \Ker \theta$.
\end{adjustbox}

Let $\alpha \in \mathbb{C}$ be an algebraic integer. Applying the first isomorphism theorem gives us
\[
	\mathbb{Z}[X] / (f) \cong \mathbb{Z}[\alpha]
.\]
For example,
\[
	\mathbb{Z}[i] \cong \mathbb{Z}[X] / (X^2 + 1)
.\]

\begin{corollary}
	If $\alpha$ is an algebraic integer and $\alpha \in \mathbb{Q}$, then $\alpha \in \mathbb{Z}$.
\end{corollary}

If $\alpha \in \mathbb{Q}$, then the minimal polynomial is linear, so it must be of the form $X - \alpha$, so $\alpha \in \mathbb{Z}$.

\newpage

\section{Noetherian Rings}%
\label{sec:noetherian_rings}

\subsection{Definitions}%
\label{sub:definitions_noetherian}

We showed that any PID $R$ satisfies the ``ascending chain condition'' (ACC):

\begin{center}\index{ascending chain condition}\index{ACC}
If $I_1 \subseteq I_2 \subseteq \ldots$ are ideals in $R$, then there exists $N \in \mathbb{N}$ such that $I_n = I_{n + 1}$.
\end{center}

More generally, we have the following result:

\begin{lemma}
	Let $R$ be a ring. Then
	\[
		R \text{ satisfies ACC} \implies \text{All ideals in } R \text{ are finitely generated}
	.\]
\end{lemma}

\begin{adjustbox}{minipage = \columnwidth - 25.5pt, margin=1em, frame=1pt, margin=0em}
	\textbf{Proof:} We prove finitely generated rings satisfy the ACC. Indeed, let
	\[
	I_1 \subseteq I_2 \subseteq \ldots
	\]
	be a chain of ideals, and then let $I = \bigcup I_i$, which is again an ideal. By assumption,
	\[
		I = (a_1, \ldots, a_m)
	.\]
	These elements belong to a nested union, so there exists $N \in \mathbb{N}$ such that
	\[
	a_1, \ldots, a_m \in I_N
	.\]
	But then for all $n \geq N$, $I_n = I$.

	Now assume $J \lhd R$ is not finitely generated. Choose $a_1 \in J$, then since $J \neq (a_1)$, we may pick $a_2 \in J \setminus (a_1)$. Inductively, we can define $a_i$ such that
	\[
		(a_1) \subset (a_1, a_2) \subset (a_1, a_2, a_3) \subset \ldots
	\]
	Thus $R$ does not satisfy the ACC.
\end{adjustbox}

\begin{definition}\index{Noetherian ring}
	A ring satisfying the ACC is called Noetherian.
\end{definition}

\newpage

\subsection{Hilbert's Basis Theorem}%
\label{sub:hilbert_s_basis_theorem}

\begin{theorem}[Hilbert's Basis Theorem]\index{Hilbert's basis theorem}
	If $R$ is a Noetherian ring, then $R[X]$ is a Noetherian ring.
\end{theorem}

\begin{adjustbox}{minipage = \columnwidth - 25.5pt, margin=1em, frame=1pt, margin=0em}
	\textbf{Proof:} Suppose $J \lhd R[X]$ is not finitely generated and $R$ is Noetherian. We can choose $f_1 \in J$ of minimal degree. Then inductively we pick $f_i \in J \setminus (f_1, \ldots, f_{i - 1})$ of minimal degree. We obtain a sequence
	\[
		f_1, f_2, f_3, \ldots \in R[X]
	,\]
	where the degrees are non-decreasing. Let $a_i$ be the leading coefficient of $f_i$, then we obtain
	\[
		(a_1) \subseteq (a_1, a_2) \subseteq (a_1, a_2, a_3) \subseteq \ldots
	,\]
	a chain of ideals in $R$. Since $R$ is Noetherian, there exists $m$ such that
	\[
	a_{m + 1} = \sum_{i = 1}^{m} \lambda_i a_i
	,\]
	and then we can consider
	\[
	g = \sum_{i = 1}^{m} \lambda_i X^{\deg f_{m + 1} - \deg f_i} f_i
	.\]
	Then $\deg f_{m + 1} = \deg g$, and they have the same leading coefficient $a_{m + 1}$, so $f_{m + 1} - g \in J$ has smaller degree than $f_{m + 1}$, so $f_{m + 1} - g \in (f_1, \ldots, f_m)$, but then $f_{m + 1} \in (f_1, \ldots, f_m)$, contradiction.

	So all $J$ is finitely generated, and so $R[X]$ is Noetherian.
\end{adjustbox}

\begin{corollary}
	$\mathbb{Z}[X_1, \ldots, X_n]$ and $F[X_1, \ldots, X_n]$ are Noetherian, where $F$ is a field.
\end{corollary}

An application is to varieties on $R = \mathbb{C}[X_1, \ldots, X_n]$. $V \subseteq \mathbb{C}^{n}$ is a subset of the form
\[
	V = \{(a_1, \ldots, a_n) \in \mathbb{C}^{n} \mid f(a_1, \ldots, a_n) = 0 \, \forall f \in \mathcal{F}\}
,\]
where $\mathcal{F} \subseteq R$ is a possibly infinite set of polynomial. By considering
\[
	I = \{ \sum_{i = 1}^{m} \lambda_i f_i \mid m \in \mathbb{N}, \lambda_i \in R, f_i \in \mathcal{F}\}
.\]
Then since $I \lhd R$ and $R$ is Noetherian, $I = (g_1, \ldots g_r)$, so we can redefine
\[
	V = \{(a_1, \ldots, a_n) \in \mathbb{C}^{n} \mid g(a_1, \ldots, a_n) = 0, i = 1, \ldots, r\}
.\]

\begin{lemma}
	Let $R$ be a Noetherian ring and $I \lhd R$. Then $R/I$ is Noetherian.
\end{lemma}

\begin{adjustbox}{minipage = \columnwidth - 25.5pt, margin=1em, frame=1pt, margin=0em}
\textbf{Proof:} Let $J_1' \subseteq J_2', \ldots$ be a chain of ideal in $R/I$. By the usual correspondence, we have $J_i' = J_i/I$, where
\[
J_1 \subseteq J_2 \subseteq \ldots
.\]
If $R$ is Noetherian, then there exists $N$ such that $J_n = J_{n + 1}$, so $J_{n}' = J_{n + 1}'$, so $R/I$ is Noetherian.
\end{adjustbox}

This proves that if $R[X]$ is Noetherian, then $R[X]/(x) \cong R$ is Noetherian.

\paragraph{Non-Noetherian Rings}%
\label{par:non_noetherian_rings}

\begin{enumerate}[label = (\roman*)]
	\item $R = \mathbb{Z}[X_1, X_2, \ldots]$, i.e. the polynomials in countably many variables, does not obey the ascending chain condition since
		\[
			(X_1) \subset (X_1, X_2) \subset (X_1, X_2, X_3) \subset \ldots
		.\]
	\item Let $R = \{f \in \mathbb{Q}[X] \mid f(0) \in \mathbb{Z}\} \leq \mathbb{Q}[X]$, i.e. the polynomials with integer constant term. Then
		\[
			(X) \subset \left(\frac{1}{2}X\right) \subset \left(\frac{1}{4} X\right) \subset \ldots
		.\]
\end{enumerate}

\newpage

\part{Modules}%
\label{prt:modules}

\section{Definitions and Examples}%
\label{sec:definitions_and_examples_modules}

\subsection{Basic Definitions}%
\label{sub:basic_definitions_modules}

\begin{definition}
	Let $R$ be a ring. A module\index{module} over $R$ is a triple $(M, +, \cdot)$, consisting of a set $M$ and two operations,
	\[
	+ : M \times M \to M, \qquad \cdot : R \times M \to M
	,\]
	such that
	\begin{enumerate}[label = (\roman*)] 
		\item $(M, +)$ is an abelian group with identity.
		\item $+$ distributes over $\cdot$, and also
			\[
				r_1 \cdot (r_2 \cdot m) = (r_1r_2) \cdot m, \qquad 1_R \cdot m = m
			.\]
	\end{enumerate}
\end{definition}

\begin{remark}
	As with groups and rings, we must check closure as well.
\end{remark}

\textbf{Examples:} 

\begin{enumerate}[label = (\roman*)]
	\item Let $R = F$ be a field. Then an $F$-module is \textbf{precisely the same as} a vector space over $F$.
	\item Let $R = \mathbb{Z}$. Then a $\mathbb{Z}$-module is \textbf{precisely the same as} an abelian group, where
		\begin{align*}
			\cdot : \mathbb{Z} \times A &\to A \\
			(n, a) &\mapsto
			\begin{cases}
				\underbrace{a + \cdots + a}_{n \text{ times}} & \text{if } n > 0, \\
				0  &\text{if } n = 0, \\
				\underbrace{-(a + \cdots + a)}_{n \text{ times}} & \text{if } n < 0.
			\end{cases}
		\end{align*}
	\item If $F$ is a field, $V$ is a vector space over $F$ and $\alpha : V \to V$ is a linear map, we can make $V$ into an $F[X]$ module via
		\begin{align*}
			\cdot : F[X] \times V &\to V \\
			(f, v) &\mapsto (f(\alpha))(v)
		\end{align*}
\end{enumerate}
\begin{remark}
	Difference choices of $\alpha$ make $V$ into different $F[X]$ modules. We sometimes write $V = V_\alpha$ to make this clear.
\end{remark}
We also give some more general constructions of modules.

\begin{enumerate}[resume, label = (\roman*)]
	\item For any ring $R$, $R^{n}$ is an $R$-module via
		\[
			r \cdot (r_1, \ldots, r_n) = (rr_1, \ldots, rr_n)
		.\]
		Therefore $R$ is an $R$-module.
	\item If $I \lhd R$, then $I$ is an $R$-module by restricting the usual multiplication in $R$. Similarly, $R/I$ is an $R$-module, where
		\[
			r \cdot (s + I) = rs + I
		.\]
	\item Let $\phi : R \to S$ be a ring homomorphism. Then an $S$-module $M$ may be regarded as an $R$-module, via
		\begin{align*}
			\cdot : R \times M & \to M \\
			(r, m) &\mapsto \phi(r)m
		\end{align*}
		In particular, if $R \leq S$, then any $S$-module may be viewed as an $R$-module.
\end{enumerate}

\begin{definition}\index{submodule}
	If $M$ is an $R$-module, then $N \subseteq M$ is an $R$-submodule (written $N \leq M$) if it is a subgroup of $(M, +)$ and $r \cdot n \in N$ for all $r \in R, n \in N$.
\end{definition}

\textbf{Examples:} 

\begin{enumerate}[label = (\roman*)]
	\item A subset of $R$ is an $R$-submodule \textbf{precisely} when it is an ideal.
	\item When $R = F$ is a field, since a module is a vector space, a submodule is a vector subspace.
\end{enumerate}

\begin{definition}
	If $N \leq M$ is an $R$-submodule, the quotient $M/N$ is the quotient of groups under addition, with
	\[
		r \cdot (m + N) = rm + N
	.\]
	This is well-defined, and makes $M/N$ an $R$-module.
\end{definition}

\begin{definition}\index{module!homomorphism}
	Let $M, N$ be $R$-modules. A function $f : M \to N$ is an $R$-module homomorphism if it is a homomorphism of abelian groups, and
	\[
		f(r \cdot m) = r \cdot f(m)
	.\]
\end{definition}

For $R = F$ a field, an $F$-module homomorphism is just a linear map.

\subsection{Isomorphism Theorems}%
\label{sub:isomorphism_theorems_modules}\index{module!isomorphism theorems}

\begin{theorem}[First Isomorphism Theorem]
	Let $f : M \to N$ be an $R$-module homomorphism. Then $\Ker f = \{m \in M \mid f(m) = 0\} \leq M$, $\Img f = \{f(m) \in N \mid m \in M\} \leq N$, and
	\[
		M / \Ker(f) \cong \Img(f)
	.\]
\end{theorem}

The proof is an exercise.

\begin{theorem}[Second Isomorphism Theorem]
	Let $A, B \leq M$ be $R$-submodules. Then $A + B = \{a + b \mid a \in A, b \in B\} \leq M$, $A \cap B \leq M$, and
	\[
		A / (A \cap B) \cong (A + B) / B
	.\]
\end{theorem}

This follows by applying the first isomorphism theorem to the homomorphism
\[
A \to M \to M/B
.\]

\begin{remark}
	Let $M \leq N$. There is a bijection between submodules in $M/N$, and submodules of $M$ containing $N$, given by
	\begin{align*}
		K &\mapsto \{m \in M \mid m + K \in N\} \\
			M/L &\mapsfrom L
		\end{align*}
\end{remark}

\begin{theorem}[Third Isomorphism Theorem]
	If $N \leq L \leq M$ are $R$-submodules, then
	\[
		(M/N)/(L/N) \cong M/L
	.\]
\end{theorem}

In particular, these all apply to vector spaces.

Let $M$ be an $R$-module. If $m \in M$, we write
\[
	Rm = \{rm \in M \mid r \in R\}
.\]
This is the submodule generated by $m$.

\begin{definition}\index{finitely generated module}
	$M$ is finitely generated if there exist $m_1, \ldots, m_n \in M$ such that
	\[
	M = Rm_1 + \cdots + Rm_{n}
	.\]
\end{definition}

\begin{lemma}
	$M$ is finitely generated if and only if there is a surjective $R$-module homomorphism $f : R^{n} \to M$ for some $n \in \mathbb{N}$.
\end{lemma}

\begin{adjustbox}{minipage = \columnwidth - 25.5pt, margin=1em, frame=1pt, margin=0em}
\textbf{Proof:} If $M = Rm_1 + \cdots + Rm_n$, then define
\begin{align*}
	f : R^{n} &\to M \\
	(r_1, \ldots, r_n) &\mapsto \sum r_i m_i
\end{align*}
By definition this is surjective.

For the other direction, let $e_i = (0, \ldots, 0, 1, 0, \ldots, 0) \in R^{n}$. Given $f : R^{n} \to M$ surjective, we can set $m_i = f(e_i)$. Then any $m \in M$ is of the form
\[
	m = f(r_1, \ldots, r_n) = f( \sum r_i e_i) = \sum r_i f(e_i) = \sum r_i m_i
.\]
Thus $M = Rm_1 + \cdots + Rm_n$.
\end{adjustbox}

\begin{corollary}
	Let $N \leq M$ be an $R$-submodule. If $M$ is finitely generated, then $M/N$ is finitely generated.
\end{corollary}

This follows by taking 
\[
R^{n} \to M \to M/N
.\]

However, a submodule of a finitely generated module need not be finitely generated. Let $R$ be a non-Noetherian ring and $I \lhd R$ be a non-finitely generated ideal. Then $R$ is a finitely generated $R$-module, and $I$ is a submodule which is not finitely generated.

\begin{remark}
	A submodule of a finitely generated module over a Noetherian ring is finitely generated.
\end{remark}

\begin{definition}
	Let $M$ be an $R$-module.
	\begin{enumerate}[label = (\roman*)]
		\item An element $m \in M$ is \textbf{torsion} if there exists $0 \neq r \in R$ with $r \cdot m = 0$.\index{torsion}
		\item $M$ is a \textbf{torsion module} if every $m \in M$ is torsion.\index{torsion module}
		\item $M$ is \textbf{torsion-free} if every $0 \neq m \in M$ is not torsion.\index{torsion-free}
	\end{enumerate}
\end{definition}

In particular, the torsion elements in a $\mathbb{Z}$-module are the elements of finite order.

\newpage

\section{Direct Sums and Free Modules}%
\label{sec:direct_sums_and_free_modules}

\begin{definition}\index{module!direct sum}
	Let $M_1, \ldots, M_n$ be $R$-modules. The direct sum $M_1 \oplus \cdots \oplus M_n$ is the set $M_1 \times \cdots \times M_n$ with operations
	\[
		(m_1, \ldots, m_n) + (m_1', \ldots, m_n') = (m_1 + m_1', \ldots, m_n + m_n')
	,\]
	\[
		r \cdot (m_1, \ldots, m_n) = (r m_1, \ldots, rm_n)
	.\]
	Then $M_1 \oplus \cdots \oplus M_n$ is an $R$-module
\end{definition}

For example, our usual $R^{n} = \underbrace{R \oplus \cdots \oplus R}_{n \text{ times}}$.

\begin{lemma} 
	If
	\[
	M = \bigoplus_{i = 1}^{n} M_i
	\]
	and $N_i \leq M_i$ for all $i$, then setting
	\[
	N = \bigoplus_{i = 1}^{n} N_i \leq M
	,\]
	we have
	\[
	M / N \cong \bigoplus_{i = 1}^{n} M_i / N_i
	.\]
\end{lemma}

\begin{adjustbox}{minipage = \columnwidth - 25.5pt, margin=1em, frame=1pt, margin=0em}
	\textbf{Proof:} Apply the first isomorphism theorem to the surjective $R$-module homomorphism
\begin{align*}
	M &\to \bigoplus_{i = 1}^{n} M_i/N_i \\
	(m_1, \ldots, m_n) &\mapsto (m_1 + N_1, \ldots, m_n + N_n).
\end{align*}
This has kernel $N$.
\end{adjustbox}

\begin{definition}\index{independent set}
	Let $m_1, \ldots, m_n \in M$. The set $\{m_1, \ldots, m_n\}$ is independent if
	\[
	\sum_{i = 1}^{n} r_i m_i = 0 \iff r_1 = r_2 = \cdots = r_n = 0
	.\]
\end{definition}

\begin{definition}\index{free basis}
	A subset $S \subseteq M$ generates $M$ freely if
	\begin{enumerate}[label = (\roman*)]
		\item $S$ generates $M$, i.e. for all $m \in M$,
			\[
			m = \sum r_i s_i
			,\]
			for $r_i \in R$, $s_i \in S$.
		\item Any function $\psi : S \to N$, where $N$ is an $R$-module, extends to an $R$-module homomorphism $\theta : M \to N$.
	\end{enumerate}
\end{definition}
An $R$-module which is freely generated by some subset $S \subseteq M$ is called \textbf{free} and $S$ is called a \textbf{free basis}.\index{free}

\begin{proposition}
	For a subset $S = \{m_1, \ldots, m_n\} \subseteq M$, the following are equivalent:
	\begin{enumerate}[label = \normalfont(\roman*)]
		\item $S$ generates $M$ freely.
		\item $S$ generates $M$ and $S$ is independent.
		\item Every element of $M$ can be written uniquely as $r_1m_1 + \cdots + r_nm_n$ for some $r_1, \ldots, r_n \in R$.
		\item The $R$-module homomorphism
			\begin{align*}
				R^{n} &\to M \\
				(r_1, \ldots, r_n) &\mapsto \sum r_i m_i
			\end{align*}
			is an isomorphism.
	\end{enumerate}
\end{proposition}

\begin{adjustbox}{minipage = \columnwidth - 25.5pt, margin=1em, frame=1pt, margin=0em}
	\textbf{Proof:} To prove (i) implies (ii), suppose $S$ generates $M$ freely. If $S$ is not independent, then there exists $r_1, \ldots, r_n \in R$ with
	\[
	r_1 m_1 + \cdots + r_n m_n = 0
	,\]
	where not all of the $r_i$ are 0. Define $\psi : S \to R$ by
	\[
	m_i \mapsto
	\begin{cases}
		1 & \text{if } i = j, \\
		0 & \text{otherwise}.
	\end{cases}
	\]
	This extends to an $R$-module homomorphism $\theta : M \to R$. Then,
	\[
		0 = \theta(0) = \theta\left( \sum r_i m_i\right) = \sum r_i \theta(m_i) = r_i
	.\]
	Thus $S$ is independent.

	The other implications are exercises.
\end{adjustbox}

\textbf{Examples:} 

\begin{enumerate}[label = (\roman*)]
	\item A non-trivial finite abelian group is not a free $\mathbb{Z}$-module.
	\item The set $\{2, 3\}$ generates $\mathbb{Z}$ as a $\mathbb{Z}$-module, but they are not independent since $(3) \cdot 2 + (-2) \cdot 3 = 0$.

		Furthermore no subset of $\{2, 3\}$ is a free basis since $(2)$, $(3)$ do not generate $\mathbb{Z}$.
\end{enumerate}

\begin{proposition}[Invariance of Dimension]
	Let $R$ be a non-zero ring. If $R^{m} \cong R^{n}$ as $R$-modules, then $m = n$.
\end{proposition}

\begin{adjustbox}{minipage = \columnwidth - 25.5pt, margin=1em, frame=1pt, margin=0em}
\textbf{Proof:} First we introduce a general construction. Let $I \lhd R$ and $M$ an $R$-module. Define
\[
	IM = \left\{ \sum a_i m_i \mid a_i \in I, m_i \in M \right\} \leq M
.\]
The quotient $M / IM$ is an $R/I$ module via
\[
	(r + I) \cdot (m + IM) = rm + IM
.\]
Suppose $R^{m} \cong R^{n}$. Choose $I \lhd R$ a maximal ideal, using Zorn's lemma. By the above, we get an isomorphism of $R/I$-modules
\[
	(R/I)^{m} \cong R^{m} / IR^{m} \cong R^{n} / IR^{n} \cong (R/I)^{n}
.\]
But $I \lhd R$ is maximal, so $R/I$ is a field. Thus $m = n$ by the invariance of dimension for vector spaces.
\end{adjustbox}

\newpage

\section{The Structure Theorem and Applications}%
\label{sec:the_structure_theorem_and_applications}

\subsection{Smith Normal Form}%
\label{sub:smith_normal_form}

For now, we assume $R$ is a Euclidean domain with $\phi : R \setminus{0} \to \mathbb{Z}_{> 0}$ a Euclidean function.

Let $A$ be an $m \times n$ matrix with entries in $R$.

\begin{definition}\index{elementary row operations}
	The elementary row operations are:
	\begin{enumerate}[label = (ER\arabic*)]
		\item We can add $\lambda$ times the $i$'th row to the $j$'th row ($\lambda \in R$, $i \neq j$).
		\item We can swap the $i$'th and $j$'th rows.
		\item We can multiply the $i$'th row by $u \in R^{\times}$.
	\end{enumerate}
	Each of these can be realised by left multiplication by an $m \times m$ invertible matrix. In particular, these operations are reversible.

	Similarly, we can define elementary column operations\index{elementary column operations} realized by right multiplication by an $n \times n$ matrix.
\end{definition}

\begin{definition}\index{equivalent matrices}
	Two $m \times n$ matrices are equivalent if there exists a sequence of elementary row and column operations taking $A$ to $B$.
\end{definition}

\begin{theorem}[Smith Normal Form]\index{Smith normal form}
	An $m \times n$ matrix $A = (a_{ij})$ over a Euclidean domain $R$ is equivalent to a diagonal matrix
	\[
	\begin{pmatrix}
		d_1 & \cdots & 0 & 0 & \cdots \\
		\vdots & \ddots & \vdots & \vdots & \ddots \\
		0 & \cdots & d_t & 0 & \cdots \\
		0 & \cdots & 0 & 0 & \cdots \\
		\vdots & \ddots & \vdots & \vdots & \ddots
	\end{pmatrix}
	,\]
	where $d_1 \mid d_2 \mid \ldots \mid d_t$.
\end{theorem}

The $d_i$ are called \textbf{invariant factors}\index{invariant factors}. We will show that they are unique up to associates.

\begin{adjustbox}{minipage = \columnwidth - 25.5pt, margin=1em, frame=1pt, margin=0em}
	\textbf{Proof:} If $A = 0$, we are done. Otherwise, upon swapping rows and columns, we may assume $a_{11} \neq 0$. We reduce $\phi(a_{11})$ as much as possible, as follows:
	\begin{itemize}
		\item If $a_{11} \nmid a_{1j}$ for some $j \geq2$, then we can write $a_{1j} = q a_{11} + r$, where $\phi(r) < \phi(a_{11})$. Subtracting $q$ times column 1 from column $j$, and swapping these, $a_{11} = r$, which is smaller.
		\item We can do a similar thing if $a_{11} \nmid a_{i1}$.
	\end{itemize}
	So we can decrease $\phi(a_{11})$, until $a_{11} \mid a_{1j}$, $a_{11} \mid a_{i1}$. Subtracting multiples of the first row/column leaves
	\[
	A =
	\begin{pmatrix}
		a_{11} & 0 \\
		0 & A'
	\end{pmatrix}
	,\]
	where $A'$ is $(m-1) \times (n-1)$. Now if $a_{11} \nmid a_{ij}$, then adding the $i$'th row to the first row, we get $a_{11} \nmid a_{1j}$, so we may do the above process. Thus $a_{11} \mid a_{ij}$ for all $i, j$, and we can repeat this process on $A'$ to get the Smith Normal Form.
\end{adjustbox}

To prove the uniqueness of the invariant factors, we introduce the notion of minors.

\begin{definition}\index{minor}
	A $k \times k$ minor of $A$ is the determinant of a $k \times k$ submatrix.
\end{definition}

\begin{definition}\index{Fitting ideal}
	The $k$'th Fitting ideal $\Fit_k(A) \lhd R$ is the ideal generated by the $k \times k$ minors of $A$.
\end{definition}

\begin{lemma}
	If $A$ and $B$ are equivalent matrices, then $\Fit_{k}(A) = \Fit_{k}(B)$ for all $k$.
\end{lemma}
\begin{adjustbox}{minipage = \columnwidth - 25.5pt, margin=1em, frame=1pt, margin=0em}
	\textbf{Proof:} We show that $(ER1 - ER3)$ don't change $\Fit_{k}(A)$:

	For (ER1), add $\lambda$ times the $j$'th row to the $i$'th row, to take $A$ to $A'$. Let $C$ be a $k \times k$ submatrix of $A$ and $C'$ the corresponding submatrix of $A'$.
			\begin{itemize}
				\item If we did not choose the $i$'th row, then $C = C'$, so $\det C = \det C'$.
				\item If we chose both of the rows $i$ and $j$, then $C$ and $C'$ differ by a row operation, so $\det C = \det C'$.
				\item If we chose the $i$'th row but not the $j$'th row, then by expanding along the $i$'th row,
					\[
					\det C' = \det C + \lambda \det D
					,\]
					where $D$ is a $k \times k$ matrix obtained by choosing the $j$'th row instead of the $i$'th row for $C$.
			\end{itemize}
\end{adjustbox}

\begin{adjustbox}{minipage = \columnwidth - 25.5pt, margin=1em, frame=1pt, margin=0em}
 Thus $\det C' \in \Fit_{k}(A)$, so $\Fit_{k}(A') \subset \Fit_{k}(A)$, but since (ER1) is reversible we get equality. (ER2) and (ER3) are similar. Now if $A$ has Smith Normal Form
\[
	\begin{pmatrix}
		d_1 & \cdots & 0 & 0 & \cdots \\
		\vdots & \ddots & \vdots & \vdots & \ddots \\
		0 & \cdots & d_t & 0 & \cdots \\
		0 & \cdots & 0 & 0 & \cdots \\
		\vdots & \ddots & \vdots & \vdots & \ddots
	\end{pmatrix}
	,\]
	then $\Fit_{k}(A) = (d_1 d_2 \ldots d_k) \lhd R$. Thus the products $d_1, \ldots, d_k$ depend only on $A$.
\end{adjustbox}

\subsection{Structure Theorem and Corollaries}%
\label{sub:structure_theorem_and_corollaries}

\begin{lemma}
	Let $R$ be an ED. Any submodule of $R^{m}$ is generated by at most $m$ elements.
\end{lemma}

This is proven by induction on $m$. For $N \lhd R^{m}$, consider
\[
	I = \{r \in R \mid \exists\, r_2, \ldots, r_n, \; (r, r_2, \ldots, r_m) \in N\} \lhd R
.\]
Since $R$ is a PID, $I = (a)$, so we can find an $n = (a, a_2, \ldots, a_m)$. Then by subtracting multiples of $n$, we can reduce $N$ to a $R^{m-1}$ submodule, and induct.

\begin{theorem}
	Let $R$ be an $ED$ and $N \leq R^{m}$. There is a free basis $x_1, \ldots, x_m$ for $R^{m}$ such that $N$ is generated by $d_1x_1, \ldots, d_tx_t$ for some $t \leq m$ and $d_1, \ldots, d_t \in R$ with $d_1 \mid \ldots \mid d_t$.
\end{theorem}

\begin{adjustbox}{minipage = \columnwidth - 25.5pt, margin=1em, frame=1pt, margin=0em}
\textbf{Proof:} We have $N = Ry_1 + \cdots + Ry_n$ for some $n \leq m$. Each $y_i$ belongs to $R^{m}$, so we can form an $m \times n$ matrix
\[
	A =
	\begin{pmatrix}
		y_1 & y_2 & \cdots & y_n
	\end{pmatrix}
.\]
Then $A$ is equivalent to 
\[
	\begin{pmatrix}
		d_1 & \cdots & 0 & 0 & \cdots \\
		\vdots & \ddots & \vdots & \vdots & \ddots \\
		0 & \cdots & d_t & 0 & \cdots \\
		0 & \cdots & 0 & 0 & \cdots \\
		\vdots & \ddots & \vdots & \vdots & \ddots
	\end{pmatrix}
,\]

\end{adjustbox}

\begin{adjustbox}{minipage = \columnwidth - 25.5pt, margin=1em, frame=1pt, margin=0em}
Each row operation changes our choice of free basis for $R^{m}$, and each column operation changes our set of generators for $N$. Thus after changing free basis of $R^{m}$ to $x_1, \ldots, x_m$, the sub module $N$ is generated by $d_1x_1, \ldots, d_tx_t$ as claimed.
\end{adjustbox}

\begin{theorem}[Structure Theorem]\index{structure theorem}
	Let $R$ be an ED and $M$ a finitely generated $R$-module. Then
	\[
		M \cong R/(d_1) \oplus R/(d_2) \oplus \cdots \oplus R/(d_t) \oplus R^{k}
	,\]
	for some $0 \neq d_i \in R$ with $d_1 \mid \ldots \mid d_t$, and $k \geq 0$. The $d_i$ are called invariant factors.
\end{theorem}

\begin{adjustbox}{minipage = \columnwidth - 25.5pt, margin=1em, frame=1pt, margin=0em}
	\textbf{Proof:} Since $M$ is finitely generated, there exists a surjective $R$-module homomorphism $\phi : R^{m} \to M$, for some $m$. 

	By the first isomorphism theorem, $M \cong R^{m} / \Ker \phi$, and we can choose a free basis $x_1, \ldots, x_m$ for $R^{m}$ such that $\Ker \phi$ is generated by $d_1x_1, \ldots, d_tx_t$ with $d_1 \mid \ldots \mid d_t$. Thus,
	\[
		M \cong \frac{R \oplus R \oplus \cdots \oplus R \oplus R \oplus \cdots \oplus R}{d_1R \oplus d_2R \oplus \cdots \oplus d_tR \oplus 0 \oplus \cdots \oplus 0} \cong \frac{R}{(d_1)} \oplus \frac{R}{(d_2)} \oplus \cdots \oplus \frac{R}{(d_t)} \oplus R^{m - t}
	.\]
\end{adjustbox}

\begin{remark}
	After deleting the $d_i$ which are units, the module $M$ is uniquely determined up to associates.
\end{remark}

\begin{corollary}
	Let $R$ be an ED. Then any finitely generated torsion-free module is free.
\end{corollary}

\begin{theorem}[Structure Theorem for Finitely Generated Abelian Groups]
	Any finitely generated abelian group $G$ is isomorphic to
	\[
	\frac{\mathbb{Z}}{d_1 \mathbb{Z}} \oplus \cdots \oplus \frac{\mathbb{Z}}{d_t \mathbb{Z}} \oplus \mathbb{Z}^{r}
	.\]
\end{theorem}

This follows from taking $R = \mathbb{Z}$ in the structure theorem, and this proves the special case for $G$ finite. We also saw that any finite abelian group could be written as a product of $C_{p^{i}}$'s. We can generalise this:

\begin{lemma}
	let $R$ be a PID and $a, b \in R$ with $\gcd(a, b) = 1$. Then
	\[
		\frac{R}{(ab)} \cong \frac{R}{(a)} \oplus \frac{R}{(b)}
	\]
	as $R$-modules.
\end{lemma}

\begin{adjustbox}{minipage = \columnwidth - 25.5pt, margin=1em, frame=1pt, margin=0em}
	\textbf{Proof:} Since $R$ is a PID, $(a, b) = (d)$ for some $d \in R$. But since $\gcd(a, b) = 1$, $(a, b) = (1)$, so there are $r, s \in R$ with $ra + sb = 1$. Define an $R$-module homomorphism by
	\begin{align*}
		\phi : R &\to \frac{R}{(a)} \oplus \frac{R}{(b)} \\
		x &\mapsto (x + (a), x + (b)).
	\end{align*}
	Then $\phi(sb) = (1 + (a), 0 + (b))$, $\phi(ra) = (0 + (a), 1 + (b))$, so $\phi(sbx + ray) = (x + (a), y + (b))$, meaning $\phi$ is surjective. Moreover, $(ab) \subseteq \Ker \phi$, and if $x \in \Ker \phi$, then $x \in (a) \cap (b)$, and
	\[
		x = x(ra + sb) = r(ax) + s(xb) \in (ab)
	.\]
	Thus $\Ker \phi = (ab)$, so
	\[
		\frac{R}{(ab)} \cong \frac{R}{(a)} \oplus \frac{R}{(b)}
	.\]
\end{adjustbox}

\begin{theorem}[Primary Decomposition Theorem]\index{primary decomposition theorem}
	Let $R$ be an ED and $M$ a finitely generated $R$-module. Then,
	\[
		M \cong \frac{R}{(p_1^{n_1})} \oplus \cdots \oplus \frac{R}{(p_k^{n_k})} \oplus R^{m}
	,\]
	where $p_1, \ldots, p_k$ are primes and $m \geq 0$.
\end{theorem}

\begin{adjustbox}{minipage = \columnwidth - 25.5pt, margin=1em, frame=1pt, margin=0em}
\textbf{Proof:} By the structure theorem,
\[
	M \cong \frac{R}{(d_1)} \oplus \cdots \oplus \frac{R}{(d_t)} \oplus R^{m}
.\]
So it suffices to consider $M \cong R/(d_i)$. Then $d_i = u p_1^{\alpha_1} \cdots p_r^{\alpha_r}$, where $u$ is a unit and $p_1, \ldots, p_r$ are distinct. From the previous lemma, this holds.
\end{adjustbox}

\subsection{Rational Canonical Form and Jordan Normal Form}%
\label{sub:rational_canonical_form_and_jordan_normal_form}

Let $V$ be a vector space over a field $F$, and let $\alpha : V \to V$ be a linear map. Let $V_\alpha$ denote the $F[X]$ module $V$, where
\begin{align*}
	F[X] \times V &\to V \\
	(f(X), V) &\mapsto f(\alpha)(V).
\end{align*}

\begin{lemma}
	If $V$ is finite dimensional, then $V_{\alpha}$ is a finitely generated $F[X]$-module.
\end{lemma}

This follows since if $v_1, \ldots, v_n$ generate $V$ as a $F$-vector space, then they generate $V_{\alpha}$ as an $F[X]$-module.

\begin{adjustbox}{minipage = \columnwidth - 25.5pt, margin=1em, frame=1pt, margin=0em}

\textbf{Examples:} 

\begin{enumerate}[label = (\roman*)]
	\item Suppose $V_{\alpha} \cong F[X] / (X^{n})$ as an $F[X]$-module. Then $1, X, \ldots, X^{n-1}$ is a basis for $F[X]/(X^{n})$ as an $F$-vector space, and with respect to this basis $\alpha$ has matrix
		\[
		\begin{pmatrix}
			0 & 0 & 0 & \cdots & 0 & 0 \\
			1 & 0 & 0 & \cdots & 0 & 0 \\
			0 & 1 & 0 & \cdots & 0 & 0 \\
			\vdots & \vdots & \vdots & \ddots & \vdots & \vdots \\
			0 & 0 & 0 & \cdots & 1 & 0 \\




		\end{pmatrix}
		.\]
	\item Suppose $V_{\alpha} \cong F[X]/(X - \lambda)^{n}$, as an $F[X]$-module. Then this has basis $1, X - \lambda, \ldots, (X - \lambda)^{n-1}$ with respect to $F$, and $\alpha$ has matrix
		\[
		\begin{pmatrix}
			\lambda & 0 & 0 & \cdots & 0 & 0 \\
			1 & \lambda & 0 & \cdots & 0 & 0 \\
			0 & 1 & \lambda & \cdots & 0 & 0 \\
			\vdots & \vdots & \ddots & \ddots & \vdots & \vdots \\
			0 & 0 & 0 & \ddots & \lambda & 0 \\
			0 & 0 & 0 & \cdots & 1 & \lambda
		\end{pmatrix}
		.\]
	\item Suppose $V_{\alpha} \cong F[X]/(f)$, where $f(x) = X^{n} + a_{n-1}X^{n-1} + \cdots + a_0$. Then with respect to the basis $1, X, \ldots, X^{n-1}$, $a$ has matrix
		\[
		\begin{pmatrix}
			0 & 0 & \cdots & 0 & -a_0 \\
			1 & 0 & \cdots & 0 & -a_1 \\
			0 & 1 & \cdots & 0 & -a_2 \\
			\vdots & \vdots & \ddots & \vdots & \vdots \\
			0 & 0 & \cdots & 1 & -a_{n-1}
		\end{pmatrix}
		.\]
\end{enumerate}

\end{adjustbox}

\begin{theorem}[Rational Canonical Form]\index{rational canonical form}
	Let $\alpha : V \to V$ be an endomorphism of a finite dimensional vector space, where $F$ is any field. The $F[X]$-module $V_{\alpha}$ decomposes as
	\[
		V_{\alpha} \cong \frac{F[X]}{(f_1)} \oplus \cdots \oplus \frac{F[X]}{(f_t)}
	,\]
	where $f_i \in F[X]$. Moreover, with respect to a suitable basis for $V$, $\alpha$ has matrix
	\[
	\begin{pmatrix}
		c(f_1) & 0 & \cdots & 0 \\
		0 & c(f_2) & \cdots & 0 \\
		\vdots & \vdots & \ddots & \vdots \\
		0 & 0 & \cdots & c(f_t)
	\end{pmatrix}
	.\]
\end{theorem}

\begin{adjustbox}{minipage = \columnwidth - 25.5pt, margin=1em, frame=1pt, margin=0em}
	\textbf{Proof:} We known $V_{\alpha}$ is finitely generated, and since $F[X]$ is an ED, we can use the structure theorem. Since $V$ is finite dimensional, $m = 0$, so upon multiplying each $f_i$ be a unit, we can assume $f_i$ are monic, and use the above.
\end{adjustbox}

\begin{remark}
	\begin{enumerate}[label = (\roman*)]
		\item[]
		\item If $\alpha$ is represented by an $n \times n$ matrix $A$, then the theorem says that $A$ is similar to a matrix of companion matrices.
		\item The minimum polynomial of $\alpha$ is $f_t$, and the characteristic polynomial of $\alpha$ is $\prod f_i$. This implies the Cayley-Hamilton theorem.\index{Cayley-Hamilton theorem}
	\end{enumerate}
\end{remark}

If $\dim V = 2$, then the sum of the degrees of $f_i$ is 2, so
\[
	V_{\alpha} \cong \frac{F[X]}{(X - \lambda)} \oplus \frac{F[X]}{(X - \mu)} \text{ or } \frac{F[X]}{(f)}
.\]

\begin{corollary}
	Let $A, B \in GL_2(F)$ be non-scalar matrices. Then $A$ and $B$ are similar if and only if they have the same characteristic polynomial.
\end{corollary}

\begin{definition}\index{annihilator}
	The annihilator of an $R$-module $M$ is
	\[
		\Ann_R(M) = \{r \in R \mid rm = 0 \, \forall m \in M \} \lhd R
	.\]
\end{definition}

\begin{enumerate}[label = (\roman*)]
	\item If $I \lhd R$, then $\Ann_{R}(R/I) = I$.
	\item If $A$ is a finite abelian group, then $\Ann_{\mathbb{Z}}(A) = (e)$, where $e$ is the exponent of $A$.
	\item If $V_{\alpha}$ is as above, then $\Ann_{F[X]}(V_{\alpha})$ is the ideal generated by the minimal polynomial of $\alpha$, $(f)$.
\end{enumerate}

\begin{lemma}
	The primes in $\mathbb{C}[X]$ are the polynomials $X - \lambda$, for $\lambda \in \mathbb{C}$.
\end{lemma}

\begin{adjustbox}{minipage = \columnwidth - 25.5pt, margin=1em, frame=1pt, margin=0em}
	\textbf{Proof:} By the Fundamental Theorem of Algebra, any non-constant polynomial in $\mathbb{C}[X]$ has a root in $\mathbb{C}$, so has a factor of $X - \lambda$. Hence the irreducibles have degree 1.
\end{adjustbox}

\begin{theorem}[Jordan Normal Form]\index{Jordan normal form}
	Let $\alpha : V \to V$ be an endomorphism of a finite dimensional $\mathbb{C}$-vector space. Let $V_{\alpha}$ be $V$ regarded as a $\mathbb{C}[X]$-module with $X$ acting as $\alpha$. Then there is an isomorphism of $\mathbb{C}[X]$-modules
	\[
		V_{\alpha} \cong \frac{\mathbb{C}[X]}{((X-\lambda_1)^{n_1})} \oplus \cdots \oplus \frac{\mathbb{C}[X]}{((X - \lambda_t)^{n_t})}
	,\]
	where $\lambda_1, \ldots, \lambda_t \in \mathbb{C}$. In particular, there exists a basis for $V$ such that $\alpha$ has matrix
	\[
	\begin{pmatrix}
		J_{n_1}(\lambda_1) & & \\
				   & \ddots & \\
				   & & J_{n_t}(\lambda_t)
	\end{pmatrix}
	,\]
	where the Jordan blocks are
	\[
		J_n(\lambda) =
		\begin{pmatrix}
			\lambda & & & \\
			1 & \ddots & & \\
			  & \ddots & \ddots & \\
			  & & 1 & \lambda
		\end{pmatrix}
	.\]
\end{theorem}

\begin{adjustbox}{minipage = \columnwidth - 25.5pt, margin=1em, frame=1pt, margin=0em}
	\textbf{Proof:} $\mathbb{C}[X]$ is an ED, and $V_{\alpha}$ is finitely generated as a $\mathbb{C}[X]$-module. We apply the primary decomposition theorem, noting that the primes in $\mathbb{C}[X]$ are linear. Since $V$ is finite dimensional, there are no copies of $\mathbb{C}[X]$.

	Now to deduce Jordan Normal Form, note by taking a basis of $1, (X - \lambda), \cdots, (X - \lambda)^{n-1}$ on
	\[
		\frac{\mathbb{C}[X]}{((X - \lambda)^{n})}
	,\]
	this gives us our Jordan block. Then taking a union of these bases for all factors gives us the JNF.
\end{adjustbox}

\begin{remark}
	\begin{enumerate}[label = (\roman*)]
		\item[]
		\item If $\alpha$ is represented by matrix $A$, then this theorem says $A$ is similar to a matrix in Jordan Normal Form.
		\item The Jordan blocks\index{Jordan blocks} are uniquely determined up to reordering. This can be proved by considering the dimensions of the generalized eigenspaces:
			\[
				\Ker ((A - \lambda I)^{m}), \quad m = 1, 2, \ldots
			.\]
		\item The minimal polynomial\index{minimal polynomial} of $\alpha$ is
			\[
				\prod_{\lambda} (X - \lambda)^{c_{\lambda}}
			\]
			over all eigenvectors $\lambda$, where $c_{\lambda}$ is the size of the largest $\lambda$ block.
		\item The characteristic polynomial\index{characteristic polynomial} of $\alpha$ is
			\[
				\prod_{\lambda} (X - \lambda)^{a_{\lambda}}
			,\]
			where $a_{\lambda}$ is the sum of the sizes of the $\lambda$ blocks.
		\item The number of $\lambda$ blocks is the dimension of the $\lambda$-eigenspace.
	\end{enumerate}
\end{remark}

\newpage

\section{Modules over PID's}%
\label{sec:modules_over_pid_s}

The Structure Theorem\index{structure theorem} also holds for PID's. There are a few ideas which go into this proof.

\begin{theorem}
	Let $R$ be a PID. Then any finitely generated torsion-free $R$-module is free.
\end{theorem}

\begin{lemma}
	Let $R$ be a PID and $M$ an $R$-module. Let $r_1, r_2 \in R$ not both zero and let $d = \gcd(r_1, r_2)$.
	\begin{enumerate}[label = \normalfont(\roman*)]
		\item There exists $A \in SL_{2}(R)$ such that
			\[
			A
			\begin{pmatrix}
				r_1 \\
				r_2
			\end{pmatrix}
			=
			\begin{pmatrix}
				d \\
				0
			\end{pmatrix}
			.\]
		\item If $x_1, x_2 \in M$, then there exists $x_1', x_2' \in M$ such that $Rx_1 + Rx_2 = Rx_1' + Rx_2'$, and $r_1x_1 + r_2x_2 = dx_1' + 0 \cdot x_2'$.
	\end{enumerate}
	
\end{lemma}

\begin{adjustbox}{minipage = \columnwidth - 25.5pt, margin=1em, frame=1pt, margin=0em}
	\textbf{Proof:} Since $R$ is a PID, $(r_1, r_2) = (d)$. Therefore, there exists $\alpha, \beta \in R$ such that $\alpha r_1 + \beta r_2 = d$. Write $r_1 = s_1d$ and $r_2 = s_2d$ for some $s_1, s_2 \in R$. Then $\alpha s_1 + \beta s_2 = 1$. Then
	\[
	\begin{pmatrix}
		\alpha & \beta \\
		-s_2 & s_1
	\end{pmatrix}
	\begin{pmatrix}
		r_1 \\
		r_2
	\end{pmatrix}
	=
	\begin{pmatrix}
		d \\
		0
	\end{pmatrix}
	.\]
	Then by construction, $\det A = \alpha s_1 + \beta s_2 = 1$, so $A \in SL_2(R)$.

	For the second part, let $x_1' = s_1x_1 + s_2x_2$ and $x_2' = -\beta x_1 + \alpha x_2$. Then $Rx_1' + Rx_2' \subseteq Rx_1 + Rx_2$, and the reverse holds since $\det A = 1$, so $A$ is invertible and we can write $x_1, x_2$ in terms of $x_1', x_2'$. Thus $Rx_1 + Rx_2 = Rx_1' + Rx_2'$.

	Finally, $r_1x_1 + r_2x_2 = d(s_1x_1 + s_2x_2) = dx_1' + 0 \cdot x_2'$.
\end{adjustbox}

Finally, we prove the theorem from the beginning of the section. Let $M = Rx_1 + \cdots + Rx_n$, with $n$ as small as possible. If $x_1, \ldots, x_n$ are independent, then $M$ is free, and we are done. Otherwise, there exists $r_1, \ldots, r_n \in R$ such that
\[
r_1x_1 + \cdots + r_nx_n = 0
.\]
Wlog $r_1 \neq 0$, then replacing $x_1$ and $x_2$ by $x_1'$ and $x_2'$, we can assume that $r_1 \neq 0$ and $r_2 = 0$. Repeating this process, we can assume $r_1 \neq 0$ and $r_i = 0$ for all $i > 1$. But then $r_1x_1' = 0 \implies x_1' = 0$, since $M$ is torsion-free.

Thus $M = Rx_2' + \cdots + Rx_n'$, contradicting our assumption $n$ was minimal.

\newpage

\printindex
\end{document}

