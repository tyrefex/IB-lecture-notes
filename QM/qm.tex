\documentclass[12pt]{article}
\usepackage{amsmath}
\usepackage[a4paper]{geometry}
\usepackage{fancyhdr}
\usepackage{tikz}
\usepackage{amssymb}
\usepackage{graphicx}
\usepackage{amsthm}
\usepackage{import}
\usepackage{xifthen}
\usepackage{pdfpages}
\usepackage{transparent}
\usepackage{adjustbox}
\usepackage[shortlabels]{enumitem}
\usepackage{parskip}
\makeatletter
\newcommand{\@minipagerestore}{\setlength{\parskip}{\medskipamount}}
\makeatother
\usepackage{imakeidx}

\DeclareMathOperator{\Ker}{Ker}
\DeclareMathOperator{\Img}{Im}
\DeclareMathOperator{\rank}{rank}
\DeclareMathOperator{\nullity}{null}
\DeclareMathOperator{\spn}{span}
\DeclareMathOperator{\tr}{tr}
\DeclareMathOperator{\adj}{adj}
\DeclareMathOperator{\id}{id}
\DeclareMathOperator{\Sym}{Sym}
\DeclareMathOperator{\Orb}{Orb}
\DeclareMathOperator{\Stab}{Stab}
\DeclareMathOperator{\ccl}{ccl}
\DeclareMathOperator{\Aut}{Aut}
\DeclareMathOperator{\Syl}{Syl}
\DeclareMathOperator{\sgn}{sgn}
\DeclareMathOperator{\Fit}{Fit}
\DeclareMathOperator{\Ann}{Ann}


\newcommand{\incfig}[1]{%
	\def\svgwidth{\columnwidth}
	\import{./figures/}{#1.pdf_tex}
}

\setlength\parindent{0pt}

\newcommand{\course}{QM }
\newcommand{\lecnum}{}

\newtheorem{theorem}{Theorem}[section]
\newtheorem{corollary}{Corollary}[section]
\newtheorem{lemma}{Lemma}[section]
\newtheorem{proposition}{Proposition}[section]

\theoremstyle{definition}
\newtheorem{definition}{Definition}[section]

\theoremstyle{remark}
\newtheorem*{remark}{Remark}

\pagestyle{fancy}
\fancyhf{}
\rhead{\leftmark}
\lhead{Page \thepage}
\setlength{\headheight}{15pt}

\newcommand{\mapsfrom}{\mathrel{\reflectbox{\ensuremath{\mapsto}}}}

\makeindex[intoc]

\usepackage{hyperref}
\hypersetup{
    colorlinks,
    citecolor=black,
    filecolor=black,
    linkcolor=black,
    urlcolor=black
    pdfauthor={Ishan Nath}
}

\begin{document}

\hypersetup{pageanchor=false}
\begin{titlepage}
	\begin{center}
		\vspace*{1em}
		\Huge
		\textbf{IB Quantum Mechanics}

		\vspace{1em}
		\large
		Ishan Nath, Michaelmas 2022

		\vspace{1.5em}

		\Large

		Based on Lectures by Prof. Maria Ubiali

		\vspace{1em}

		\large
		\today
	\end{center}
	
\end{titlepage}
\hypersetup{pageanchor=true}

\tableofcontents

\newpage

\section{Historical Introduction}%
\label{sec:historical_introduction}

\subsection{Particles and Waves in Classical Mechanics}%
\label{sub:particles_and_waves_in_classical_mechanics}

These are the basic concepts of particle mechanics. We begin by looking at particles.

\begin{definition}
	A point particle\index{point particle} is an object carrying energy $E$ and momentum $p$ in an infinitesimally small point of space.
\end{definition}

A particle is defined by its position $\mathbf{x}$ and velocity $\mathbf{v} = \mathbf{\dot v} = \frac{d}{dt} \mathbf{x}$. From Newton's second law, we have
\[
	\mathbf{F}(\mathbf{x}(t), \mathbf{\dot x}(t)) = m \mathbf{\ddot x}(t)
.\]
Solving this determines $\mathbf{x}(t), \mathbf{\dot x}(t)$ for all $t$ once the initial conditions $\mathbf{x}(t_0), \mathbf{\dot x}(t_0)$ are known.

Particles do not interfere with each other.

\begin{definition}
	A wave\index{wave} is any real or complex-valued function with periodicity in time or space.
\end{definition}

If we take a function of time $t$, such that $f(t + T) = f(t)$, where $T$ is the period, then $\nu = 1/T$ is the frequency, and the angular frequency is $\omega = 2 \pi \nu = 2 \pi/T$. Examples of such functions are $f(t) = \sin \omega t, \cos \omega t, e^{i \omega t}$.

If we take a function of space $x$, such that $f(x + \lambda) = f(x)$, where $\lambda$ is the wave length, then $k = 2 \pi/\lambda$ is the wave number. Some examples are $f(x) = \cos \omega x, \sin \omega x, e^{i \omega x}$.

In one dimension, an EM wave obeys the equation\index{wave equation}
\[
	\frac{\partial^2 f(x, t)}{\partial t^2} - c^2 \frac{\partial^2 f(x, t)}{\partial x^2} = 0
,\]
where $c \in \mathbb{R}$. This has solutions
\[
	f_{\pm}(x, t) = A_{\pm} \exp(\pm ikx - i\omega t)
,\]
provided that the wavelength and frequency are related by $\omega = ck$ or $\lambda \nu = c$. Here $A_{\pm}$ is the amplitude of the wave, and $\omega = ck$ is the dispersion relation\index{dispersion relation}.

In three dimensions, an EM wave obeys the equation
\[
	\frac{\partial^2 f(\mathbf{x}, t)}{\partial t^2} - c^2 \nabla^2 f(\mathbf{x}, t) = 0
.\]
Here we need $f(x, t_0)$ and $\frac{df}{dt} (x, t_0)$ to determine a unique solution. The periodic solutions are
\[
	f(\mathbf{x}, t) = A \exp(i \mathbf{k} \cdot \mathbf{x} - i \omega t)
,\]
where $\omega = c |\mathbf{k}|$.

\begin{remark}
	\begin{itemize}[(i)]
		\item[]
		\item Other kind of waves arise as solution of other governing equations provided a different dispersion relation.
		\item If the governing equation is linear, the superposition principle holds, stating if $f_1, f_2$ are solutions, then $f = f_1 + f_2$ is a solution.
	\end{itemize}
	
\end{remark}

\subsection{Particle-like behaviour of waves}%
\label{sub:particle_like_behaviour_of_waves}

\subsubsection{Black-body radiation}%
\label{subsub:black_body_radiation}\index{black-body radiation}

When a body is heated at temperature $T$, it radiates light at different frequencies. The classical prediction is that $E = k_B T$, where $E$ is the energy of the wave and $k_B$ is the Boltzmann constant. This gives
\[
	I(\omega) \propto k_B T \frac{\omega^2}{\pi^2 c^3}
.\]
This diverges as $\omega \to \infty$. Planck's model stated
\[
	I(\omega) \propto \frac{\omega^2}{\pi^2 c^3} \frac{\hbar \omega}{\exp(\hbar \omega / k_B T) - 1}
.\]
Here $\hbar = h/2\pi$ is the reduced Planck constant, with $h \approx 6.6 \cdot 10^{-34} \text{Joule} \times \text{sec}$. This only makes sense if $E = \hbar \omega$.

\subsubsection{Photoelectric effect}%
\label{subsub:photoelectric_effect}\index{photoelectric effect}

The photoelectric effect is a result of an experimental phenomena, where light hitting a metal surface caused electrons to emit from the surface.

This experiment took place as the intensity $I$ and angular frequency $\omega$ of the incident light changed.

The classical expectation is as follows:
\begin{enumerate}[(i)]
	\item Since the energy of the incident light is proportional to $I$, as $I$ increases, there will be enough energy to break the bonds of the electrons with the atoms.
	\item The emission rate should be constant as $I$ increases.
\end{enumerate}

The experiment drew a number of surprising facts:
\begin{enumerate}[1.]
	\item Below $\omega_{min}$, there was not electron emission.
	\item The maximum energy of the electrons depended on $\omega$ and not $I$.
	\item The emission rate increased as $I$ increased.
\end{enumerate}

In 1905, Einstein developed Planck's idea to explain this phenomena.
\begin{itemize}
	\item Light was quantized in small quanta, called photons.
	\item Each photon carries $E = \hbar \omega$, $p = \hbar k$.
	\item The phenomenon of electron emission comes from scattering of a single photon off of a single electron.
\end{itemize}

Then for the electron to leave, we must have
\[
E_{min} = 0  = \hbar \omega_{min} - \phi
,\]
where $\phi$ is the binding energy of the electron with the metal atoms. Then moreover,
\[
E_{max} = \hbar \omega_{max} - \phi
.\]
Finally, as $I$ increases, there is a greater number of photons, so this leads to a higher electron emission rate.

\subsubsection{Compton scattering}%
\label{subsub:compton_scattering}\index{Compton scattering}

In 1923, Compton studied X-rays scattering off free electrons. Here, the binding energy of the electrons was much smaller than the incoming energy, so the electrons were essentially free.

The expectation was that, given an X-ray of frequency $\omega$, the resulting frequency $\omega'$ after the impact would follow a Gaussian centred at $\omega$. This could be done by analysing the intensity of the outgoing light.

The result was a very narrow Gaussian centred around $\omega$, but there also was another peak at another frequency $\varphi$.

In fact, we can find that the angle of the outgoing X-ray via
\[
2 \sin^2 \frac{\theta}{2} = \frac{mc}{|q|} - \frac{mc}{|p|}
,\]
where $p, q$ are the momenta of the ingoing and outgoing photons. Then, since $p = \hbar k$ and $q = \hbar k'$, we get
\[
	|p| = \hbar k = \hbar \frac{\omega}{c}, \quad |q| = \hbar \frac{\omega'}{c}, \quad \frac{1}{\omega'} = \frac{1}{\omega} + \frac{\hbar}{mc}(1 - \cos \theta)
.\]

\subsection{Atomic Spectra}%
\label{sub:atomic_spectra}

In 1897, Thompson formulated the plum-pudding model, where the atom has uniformly distributed charge.

In 1909, Rutherford conducted the gold foil experiment, showing the majority of the atom was vacuum. This resulted in the Rutherford model. However, this did not work because:
\begin{enumerate}[(i)]
	\item If the electron moves on a circular orbit, it would radiate.
	\item The electrons would collapse on the nucleus due to the Coulomb force.
	\item The model did not explain the measured spectra.
\end{enumerate}

In 1913, Bohr explained these problems by assuming the electron orbits around the nucleus are quantized so that the orbital angular momentum $L$ takes discrete values
\[
L_n = n \hbar
.\]
\begin{proposition}
	If $L$ is quantized, then $r, v, E$ are quantized.
\end{proposition}

\begin{adjustbox}{minipage = \columnwidth - 25.5pt, margin=1em, frame=1pt, margin=0em}
\textbf{Proof:} Since $L = m_e v r$, this implies
\[
v = \frac{L}{m_0 r} \implies v_n = n \frac{\hbar}{m_e r}
.\]
The Coulomb Force shows
\[
	F = \frac{e^2}{4 \pi \epsilon_0} \frac{1}{r^2} = m_e \frac{v^2}{r}
.\]
This gives
\[
	r = r_n = n^2 \left( \frac{4 \pi \epsilon_0}{m_e e^2}\hbar ^2 \right) = n^2 a_0
,\]
where $a_0$ is the Bohr radius.
\end{adjustbox}

\begin{adjustbox}{minipage = \columnwidth - 25.5pt, margin=1em, frame=1pt, margin=0em}
As a result of the quantization of the radius and velocity, the energy is also quantized. The energy is
\[
E_n = \frac{1}{2} m_e v_n^2 - \frac{e^2}{4 \pi \varepsilon_0} \frac{1}{r_n} = - \frac{e^2}{8 \pi \varepsilon_0 a_0} \frac{1}{n^2} = - \frac{e^{4} m_e}{32 \pi^2 \varepsilon_0^2 \hbar^2} \frac{1}{n^2} = \frac{E_1}{n^2}
.\]
Here $E_1$ is the lowest possible energy state, or ground state, of the Bohr atom.

\end{adjustbox}


The energy emitted by transition from the $m$-th to the $n$-th orbital is $E_{mn} = E_m - E_n$. Using $E_{mn} = \hbar \omega_{mn}$, we get
\[
	\omega_{mn} = 2 \pi c R_0 \left( \frac{1}{n^2} - \frac{1}{m^2} \right)
,\]
where $R_0$ agrees with the Rydberg constant.

\subsection{Wave-like Behaviour of Particles}%
\label{sub:wave_like_behaviour_of_particles}

In 1923, De Broglie hypothesised that any particles of any mass can be associated with a wave having
\[
\omega = \frac{E}{\hbar}, \quad k = \frac{p}{\hbar}
.\]
In 1927, Davisson and Germer scattered electrons off of crystals. The interference pattern was consistent with the De Broglie hypothesis.


\newpage

\printindex

\end{document}
