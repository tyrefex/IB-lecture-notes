\documentclass[12pt]{article}

\usepackage{ishn}

\makeindex[intoc]

\begin{document}

\hypersetup{pageanchor=false}
\begin{titlepage}
	\begin{center}
		\vspace*{1em}
		\Huge
		\textbf{IB Quantum Mechanics}

		\vspace{1em}
		\large
		Ishan Nath, Michaelmas 2022

		\vspace{1.5em}

		\Large

		Based on Lectures by Prof. Maria Ubiali

		\vspace{1em}

		\large
		\today
	\end{center}
	
\end{titlepage}
\hypersetup{pageanchor=true}

\tableofcontents

\newpage

\section{Historical Introduction}%
\label{sec:historical_introduction}

\subsection{Particles and Waves in Classical Mechanics}%
\label{sub:particles_and_waves_in_classical_mechanics}

These are the basic concepts of particle mechanics. We begin by looking at particles.

\begin{definition}
	A point particle\index{point particle} is an object carrying energy $E$ and momentum $p$ in an infinitesimally small point of space.
\end{definition}

A particle is defined by its position $\mathbf{x}$ and velocity $\mathbf{v} = \mathbf{\dot v} = \frac{d}{dt} \mathbf{x}$. From Newton's second law, we have
\[
	\mathbf{F}(\mathbf{x}(t), \mathbf{\dot x}(t)) = m \mathbf{\ddot x}(t)
.\]
Solving this determines $\mathbf{x}(t), \mathbf{\dot x}(t)$ for all $t$ once the initial conditions $\mathbf{x}(t_0), \mathbf{\dot x}(t_0)$ are known.

Particles do not interfere with each other.

\begin{definition}
	A wave\index{wave} is any real or complex-valued function with periodicity in time or space.
\end{definition}

If we take a function of time $t$, such that $f(t + T) = f(t)$, where $T$ is the period, then $\nu = 1/T$ is the frequency, and the angular frequency is $\omega = 2 \pi \nu = 2 \pi/T$. Examples of such functions are $f(t) = \sin \omega t, \cos \omega t, e^{i \omega t}$.

If we take a function of space $x$, such that $f(x + \lambda) = f(x)$, where $\lambda$ is the wave length, then $k = 2 \pi/\lambda$ is the wave number. Some examples are $f(x) = \cos \omega x, \sin \omega x, e^{i \omega x}$.

In one dimension, an EM wave obeys the equation\index{wave equation}
\[
	\frac{\partial^2 f(x, t)}{\partial t^2} - c^2 \frac{\partial^2 f(x, t)}{\partial x^2} = 0
,\]
where $c \in \mathbb{R}$. This has solutions
\[
	f_{\pm}(x, t) = A_{\pm} \exp(\pm ikx - i\omega t)
,\]
provided that the wavelength and frequency are related by $\omega = ck$ or $\lambda \nu = c$. Here $A_{\pm}$ is the amplitude of the wave, and $\omega = ck$ is the dispersion relation\index{dispersion relation}.

In three dimensions, an EM wave obeys the equation
\[
	\frac{\partial^2 f(\mathbf{x}, t)}{\partial t^2} - c^2 \nabla^2 f(\mathbf{x}, t) = 0
.\]
Here we need $f(x, t_0)$ and $\frac{df}{dt} (x, t_0)$ to determine a unique solution. The periodic solutions are
\[
	f(\mathbf{x}, t) = A \exp(i \mathbf{k} \cdot \mathbf{x} - i \omega t)
,\]
where $\omega = c |\mathbf{k}|$.

\begin{remark}
	\begin{itemize}[(i)]
		\item[]
		\item Other kind of waves arise as solution of other governing equations provided a different dispersion relation.
		\item If the governing equation is linear, the superposition principle holds, stating if $f_1, f_2$ are solutions, then $f = f_1 + f_2$ is a solution.
	\end{itemize}
	
\end{remark}

\subsection{Particle-like behaviour of waves}%
\label{sub:particle_like_behaviour_of_waves}

\subsubsection{Black-body radiation}%
\label{subsub:black_body_radiation}\index{black-body radiation}

When a body is heated at temperature $T$, it radiates light at different frequencies. The classical prediction is that $E = k_B T$, where $E$ is the energy of the wave and $k_B$ is the Boltzmann constant. This gives
\[
	I(\omega) \propto k_B T \frac{\omega^2}{\pi^2 c^3}
.\]
This diverges as $\omega \to \infty$. Planck's model stated
\[
	I(\omega) \propto \frac{\omega^2}{\pi^2 c^3} \frac{\hbar \omega}{\exp(\hbar \omega / k_B T) - 1}
.\]
Here $\hbar = h/2\pi$ is the reduced Planck constant\index{Planck constant}, with $h \approx 6.6 \cdot 10^{-34} \text{Joule} \times \text{sec}$. This only makes sense if $E = \hbar \omega$.

\subsubsection{Photoelectric effect}%
\label{subsub:photoelectric_effect}\index{photoelectric effect}

The photoelectric effect is a result of an experimental phenomena, where light hitting a metal surface caused electrons to emit from the surface.

This experiment took place as the intensity $I$ and angular frequency $\omega$ of the incident light changed.

The classical expectation is as follows:
\begin{enumerate}[(i)]
	\item Since the energy of the incident light is proportional to $I$, as $I$ increases, there will be enough energy to break the bonds of the electrons with the atoms.
	\item The emission rate should be constant as $I$ increases.
\end{enumerate}

The experiment drew a number of surprising facts:
\begin{enumerate}[1.]
	\item Below $\omega_{min}$, there was not electron emission.
	\item The maximum energy of the electrons depended on $\omega$ and not $I$.
	\item The emission rate increased as $I$ increased.
\end{enumerate}

In 1905, Einstein developed Planck's idea to explain this phenomena.
\begin{itemize}
	\item Light was quantized in small quanta, called photons.
	\item Each photon carries $E = \hbar \omega$, $p = \hbar k$.
	\item The phenomenon of electron emission comes from scattering of a single photon off of a single electron.
\end{itemize}

Then for the electron to leave, we must have
\[
E_{min} = 0  = \hbar \omega_{min} - \phi
,\]
where $\phi$ is the binding energy of the electron with the metal atoms. Then moreover,
\[
E_{max} = \hbar \omega_{max} - \phi
.\]
Finally, as $I$ increases, there is a greater number of photons, so this leads to a higher electron emission rate.

\subsubsection{Compton scattering}%
\label{subsub:compton_scattering}\index{Compton scattering}

In 1923, Compton studied X-rays scattering off free electrons. Here, the binding energy of the electrons was much smaller than the incoming energy, so the electrons were essentially free.

The expectation was that, given an X-ray of frequency $\omega$, the resulting frequency $\omega'$ after the impact would follow a Gaussian centred at $\omega$. This could be done by analysing the intensity of the outgoing light.

The result was a very narrow Gaussian centred around $\omega$, but there also was another peak at another frequency $\varphi$.

In fact, we can find that the angle of the outgoing X-ray via
\[
2 \sin^2 \frac{\theta}{2} = \frac{mc}{|q|} - \frac{mc}{|p|}
,\]
where $p, q$ are the momenta of the ingoing and outgoing photons. Then, since $p = \hbar k$ and $q = \hbar k'$, we get
\[
	|p| = \hbar k = \hbar \frac{\omega}{c}, \quad |q| = \hbar \frac{\omega'}{c}, \quad \frac{1}{\omega'} = \frac{1}{\omega} + \frac{\hbar}{mc}(1 - \cos \theta)
.\]

\subsection{Atomic Spectra}%
\label{sub:atomic_spectra}\index{atomic spectra}

In 1897, Thompson formulated the plum-pudding model, where the atom has uniformly distributed charge.

In 1909, Rutherford conducted the gold foil experiment, showing the majority of the atom was vacuum. This resulted in the Rutherford model. However, this did not work because:
\begin{enumerate}[(i)]
	\item If the electron moves on a circular orbit, it would radiate.
	\item The electrons would collapse on the nucleus due to the Coulomb force.
	\item The model did not explain the measured spectra.
\end{enumerate}

In 1913, Bohr explained these problems by assuming the electron orbits around the nucleus are quantized so that the orbital angular momentum $L$ takes discrete values
\[
L_n = n \hbar
.\]
\begin{proposition}
	If $L$ is quantized, then $r, v, E$ are quantized.
\end{proposition}

\begin{proofbox}
Since $L = m_e v r$, this implies
\[
v = \frac{L}{m_0 r} \implies v_n = n \frac{\hbar}{m_e r}
.\]
The Coulomb Force shows
\[
	F = \frac{e^2}{4 \pi \epsilon_0} \frac{1}{r^2} = m_e \frac{v^2}{r}
.\]
This gives
\[
	r = r_n = n^2 \left( \frac{4 \pi \epsilon_0}{m_e e^2}\hbar ^2 \right) = n^2 a_0
,\]
where $a_0$ is the Bohr radius\index{Bohr radius}.

As a result of the quantization of the radius and velocity, the energy is also quantized. The energy is
\[
E_n = \frac{1}{2} m_e v_n^2 - \frac{e^2}{4 \pi \varepsilon_0} \frac{1}{r_n} = - \frac{e^2}{8 \pi \varepsilon_0 a_0} \frac{1}{n^2} = - \frac{e^{4} m_e}{32 \pi^2 \varepsilon_0^2 \hbar^2} \frac{1}{n^2} = \frac{E_1}{n^2}
.\]
Here $E_1$ is the lowest possible energy state, or ground state, of the Bohr atom.
\end{proofbox}


The energy emitted by transition from the $m$-th to the $n$-th orbital is $E_{mn} = E_m - E_n$. Using $E_{mn} = \hbar \omega_{mn}$, we get
\[
	\omega_{mn} = 2 \pi c R_0 \left( \frac{1}{n^2} - \frac{1}{m^2} \right)
,\]
where $R_0$ agrees with the Rydberg constant.

\subsection{Wave-like Behaviour of Particles}%
\label{sub:wave_like_behaviour_of_particles}

In 1923, De Broglie hypothesised that any particles of any mass can be associated with a wave having
\[
\omega = \frac{E}{\hbar}, \quad k = \frac{p}{\hbar}
.\]
In 1927, Davisson and Germer scattered electrons off of crystals. The interference pattern was consistent with the De Broglie hypothesis.

\newpage

\section{Foundation of Quantum Mechanics}%
\label{sec:foundation_of_quantum_mechanics}

Quantum mechanics is founded in linear algebra:

\begin{itemize}
	\item The vector $\mathbf{v}$ in LA becomes the state $\psi$ in QM.
	\item The bases $\{e_i\}$ in LA becomes the bases $\mathbf{x}$ in QM.
	\item The coordinate representation
		\[
		\mathbf{v} \to
		\begin{pmatrix}
			v_1 \\
			\vdots \\
			v_n
		\end{pmatrix}
		\]
		becomes the wavefunction $\psi(\mathbf{x}, t)$.
	\item The vector space $V$ becomes the wavefunction space $L^2(\mathbb{R}^3)$.
	\item The inner product $\langle -, - \rangle$ becomes the inner product
		\[
			(\psi, \phi) = \int_{\mathbb{R}^3} \psi^{\ast}(\mathbf{x}, t) \phi(\mathbf{x}, t)\Diff3 x
		.\]
	\item The linear map $V \to V$ represented by a matrix $T$ becomes the linear maps between $L^2(\mathbb{R}^3) \to L^2(\mathbb{R}^3)$, given by operators $\hat O$.
\end{itemize}

\subsection{Wavefunctions and Probabilistic Interpretation}%
\label{sub:wavefunctions_and_probabilistic_interpretation}\index{wavefunction}

In classical mechanics, the dynamics of a particle is determined by $\mathbf{x}$ and $\mathbf{\dot x}$. In quantum mechanics, we have a similar idea.

\begin{definition}
	$\psi$ is the state\index{quantum state} of the particle.
\end{definition}

\begin{definition}
	$\psi(\mathbf{x}, t) : \mathbb{R}^3 \to \mathbb{C}$ is a complex-valued function satisfying mathematical properties dictated by physical interpretation.
\end{definition}

\begin{proposition}[Born's rule]\index{Born's rule}
	The probability density for a particle to sit at $\mathbf{x}$ at given time $t$ is $\rho(\mathbf{x}, t) \propto |\psi(\mathbf{x}, t)|^2$. Then, $\rho(\mathbf{x}, t)\diff V$ is the probability that the volume sits in a small volume centred around $\mathbf{x}$, which is proportional to the squared modulus of $\psi(\mathbf{x}, t)$.
\end{proposition}

\begin{enumerate}[(i)]
	\item Because the particle has to be somewhere, the wavefunction has to be normalisable (or square-integrable\index{square-integrable function}\index{normalisable function}) in $\mathbb{R}^3$, so
		\[
			\int_{\mathbb{R}^3} \psi^{\ast}(\mathbf{x}, t) \psi(\mathbf{x}, t)\Diff3 x = \int_{\mathbb{R}^3} |\psi(\mathbf{x}, t)|^2\Diff3 x = \mathcal{N} < \infty
		,\]
		with $\mathcal{N} \in \mathbb{R}$ and $\mathcal{N} \neq 0$.
	\item Because the total probability has to be 1, we consider the normalised wavefunction\index{normalised wavefunction}
		\[
			\bar \psi(\mathbf{x}, t) = \frac{1}{\sqrt{\mathcal{N}}} \psi(\mathbf{x}, t)
		.\]
		Then we have
		\[
			\int_{\mathbb{R}^3}|\bar \psi (\mathbf{x}, t)|^2\Diff3 x = 1
		,\]
		so $\rho(\mathbf{x}, t) = |\bar \psi(\mathbf{x}, t)|^2$. We often write wavefunctions as $\psi$, and then normalise at the end.
	\item If $\tilde \psi (\mathbf{x}, t) = e^{i\alpha} \psi (\mathbf{x}, t)$ with $\alpha \in \mathbb{R}$, then $|\tilde \psi(\mathbf{x}, t)|^2 = |\psi(\mathbf{x}, t)|^2$, so $\psi$ and $\tilde \psi$ are equivalent states.
\end{enumerate}

The state $\psi$ corresponds to rays in the vector space, which are equivalence classes of wavefunctions under the equivalence relation $\psi_1 \sim \psi_2 \iff \psi_1 = e^{i \alpha}\psi_2$.

\subsection{Hilbert Space}%
\label{sub:hilbert_space}\index{Hilbert space}

\begin{definition}
	The set of all square-integrable functions in $\mathbb{R}^3$ is called a Hilbert space $\mathcal{H}$ or $L^2(\mathbb{R}^3)$.
\end{definition}

\begin{theorem}
	If $\psi_1(\mathbf{x}, t), \psi_2(\mathbf{x}, t) \in \mathcal{H}$, then $\psi(\mathbf{x}, t) = \alpha_1 \psi_1(\mathbf{x}, t) + \alpha_2 \psi_2(\mathbf{x}, t) \in \mathcal{H}$.
\end{theorem}

\begin{proofbox}
Since $\psi_1, \psi_2 \in \mathcal{H}$, we can say
\[
	\int_{\mathbb{R}^3} |\psi_1(\mathbf{x}, t)|^2\Diff3 x = \mathcal{N}_1, \quad \int_{\mathbb{R}^3} |\psi_2 (\mathbf{x}, t)|^2\Diff3 x = \mathcal{N}_2
.\]
Note the triangle inequality: if $z_1, z_2 \in \mathbb{C}$, then $|z_1 + z_2| < |z_1| + |z_2|$. Let $z_1 = \alpha_1 \psi_1(\mathbf{x}, t)$, $z_2 = \alpha_2 \psi_2 (\mathbf{x}, t)$. Then
\begin{align*}
	\int_{\mathbb{R}^3}|\psi(\mathbf{x}, t)|^2\Diff3 x &= \int_{\mathbb{R}^3} |\alpha_1 \psi_1(\mathbf{x}, t) + \alpha_2 \psi_2(\mathbf{x}, t)|^2\Diff3 x \\
							   &\leq \int_{\mathbb{R}^3} (|\alpha_1 \psi_1(\mathbf{x}, t)| + |\alpha_2 \psi_2(\mathbf{x}, t)|)^2\Diff3 x \\
							   &= \int_{\mathbb{R}^3}(|\alpha_1 \psi_1(\mathbf{x}, t)|^2 + |\alpha_2 \psi_2(\mathbf{x}, t)|^2 + 2|\alpha_1 \psi_1||\alpha_2\psi_2|)\Diff3 x \\
							   &\leq \int_{\mathbb{R}^3} 2|\alpha_1 \psi_1(\mathbf{x}, t)|^2 + 2|\alpha_2 \psi_2(\mathbf{x}, t)|^2\Diff3 x \\
							   &= 2|a_1|^2\mathcal{N}_1 + 2|a_2|^2\mathcal{N}_2 < \infty.
\end{align*}
\end{proofbox}

\subsection{Inner Product}%
\label{sub:inner_product}

\begin{definition}
	Define the inner product\index{inner product} in $\mathcal{H}$ as
	\[
		(\psi, \phi) = \int_{\mathbb{R}^3} \psi^{\ast}(\mathbf{x}, t) \phi(\mathbf{x}, t)\Diff3 x
	.\]
\end{definition}

\begin{theorem}
	If $\psi, \phi \in \mathcal{H}$, then the inner product exists.
\end{theorem}

\begin{proofbox}
Let the square integrals of $\psi$ and $\phi$ be $\mathcal{N}_1$ and $\mathcal{N}_2$, respectively. Then, we use the Schwarz inequality as follows:
\begin{align*}
	|(\psi, \phi)| &= \left| \int_{\mathbb{R}^3} \psi^{\ast}(\mathbf{x}, t) \phi(\mathbf{x}, t) \Diff3 x \right| \\
		       &\leq \sqrt{\int_{\mathbb{R}^3} |\psi(\mathbf{x}, t)|^2\Diff3 x \cdot \int_{\mathbb{R}^3}|\phi(\mathbf{x}, t)|^2\Diff3 x} \\
		       &= \sqrt{\mathcal{N}_1\mathcal{N}_2} < \infty.
\end{align*}
\end{proofbox}

\subsubsection{Properties of the Inner Product}%
\label{subsub:properties_of_the_inner_product}

\begin{enumerate}[(i)]
	\item $(\psi, \phi) = (\phi, \psi)^{\ast}$,
	\item It is antilinear in the first entry, and linear in the second entry:
		\begin{align*}
			(a_1 \psi_1 +a_2 \psi_2, \phi) &= a_1^{\ast}(\psi_1, \phi) + a_2^{\ast}(\psi_2, \phi), \\
			(\psi, a_1 \phi_1 + a_2 \phi_2) &= a_1(\psi, \phi_1) + a_2(\psi, \phi_2).
		\end{align*}
	\item The inner product of $\psi \in \mathcal{H}$ with itself is non-negative:
		\[
			(\psi, \psi) = \int_{\mathbb{R}^3} |\psi(\mathbf{x}, t)|^2 \Diff3 x > 0
		.\]
\end{enumerate}

\begin{definition}
	The norm of the wavefunction $\psi$ is the real number $\|\psi\| = \sqrt{(\psi, \psi)}$. We say $\psi$ is normalized if $\|\psi\| = 1$.
\end{definition}

\begin{definition}
	Two wavefunctions $\psi, \phi \in \mathcal{H}$ are orthogonal if $(\psi, \phi) = 0$, and a set of wavefunctions $\{\psi_n\}$ is orthonormal if
	\[
		(\psi_m, \psi_n) = \delta_{mn}
	.\]
\end{definition}

\begin{definition}
	A set of wavefunctions $\{\psi_n\}$ is complete if all $\phi \in \mathcal{H}$ can be written as a linear combination of the $\{\psi_n\}$:
	\[
	\phi = \sum_{n = 0}^{\infty}c_n \psi_n
	.\]
\end{definition}

\begin{lemma}
	If $\{\psi_n\}$ form a complete orthonormal basis of $\mathcal{H}$, then $c_n = (\psi_n, \phi)$.
\end{lemma}

\begin{proofbox}	
\begin{align*}
	(\psi_n, \phi) &= \left( \psi_n, \sum_{m = 0}^{n} c_m \psi_m \right) \\
		       &= \sum_{m = 0}^{\infty} c_m (\psi_n, \psi_m) = \sum_{m = 0}^{\infty} c_m \delta_{mn} \\
		       &= c_n.
\end{align*}
\end{proofbox}


\subsection{Time-dependent Schr\"{o}dinger Equation}%
\label{sub:time_dependent_schrodinger_equation}

The first postulate of quantum mechanics that we have encountered is Born's rule:
\[
	\rho(\mathbf{x}, t) \propto |\psi(\mathbf{x}, t)|^2
.\]
The second is the time-dependent Schr\"{o}dinger equation\index{time-dependent Schr\"{o}dinger equation}:
\[
	i \hbar \frac{\partial \psi}{\partial t} (\mathbf{x}, t) = - \frac{\hbar^2}{2m} \nabla^2 \psi(\mathbf{x}, t) + U(\mathbf{x}) \psi(\mathbf{x}, t)
.\] 
Here $U(\mathbf{x}) \in \mathbb{R}$ is the potential. Looking at the equation, we spot the following:
\begin{itemize}
	\item There is a first derivative in time: once $\psi(x, t_0)$ is known, then we know $\psi(x, t)$ at all times.
	\item There is an asymmetry in time and space. This implies the TDSE is a non-relativistic equation.
\end{itemize}

Heuristically, this comes from the observation of electron diffraction, which leads to the thought that electrons believe like waves. Thus, we can think of a function
\[
	\psi(\mathbf{x}, t) \propto \exp[i(\mathbf{k} \cdot \mathbf{x} - \omega t)]
\]
that describes the dynamics of the electron. From De Broglie, we get
\[
\mathbf{k} = \frac{\mathbf{p}}{\hbar}, \quad \omega = \frac{E}{\hbar}
.\]
For a free particle, we get
\[
E = \frac{|\mathbf{p}|^2}{2m} \implies \omega = \frac{|\mathbf{p}|^2}{2 m \hbar} = \frac{\hbar}{2m} |\mathbf{k}|^2
.\]
The dispersion relation for a particle-wave is
\[
\omega \propto |\mathbf{k}|^2
.\]
For a light-wave, as the energy equation is different, we get
\[
\omega \propto |\mathbf{k}|
.\]
From dimensional analysis, we see that the wave equation must have a single derivative with respect to time, and a double derivative with respect to space.

To apply the Schr\"{o}dinger equation, we need to ensure the wavefunction remains normalized throughout time. Hence we look at the following properties:

\begin{enumerate}[(i)]
	\item The squared integral
		\[
			\int_{\mathbb{R}^3}|\psi(\mathbf{x}, t)|^2\Diff3 x = \mathcal{N}
		,\]
		is independent of time.
\end{enumerate}

\begin{proofbox}	
We have
	\[
		\frac{\diff \mathcal{N}}{\diff t} = \frac{\diff}{\diff t} \int_{\mathbb{R}^3} |\psi(\mathbf{x}, t)|^2\Diff 3 x = \int_{\mathbb{R}^3} \frac{\partial}{\partial t}|\psi(\mathbf{x}, t)|^2\Diff3 x
	.\]
	But the partial derivative
	\[
		\frac{\partial}{\partial t}(\psi^{\ast}(\mathbf{x}, t) \psi(\mathbf{x}, t)) = \psi^{\ast} \frac{\partial \psi}{\partial t} + \frac{\partial \psi^{\ast}}{\partial t}\psi
	.\]
	From the TDSE and its conjugate,
	\begin{align*}
		\frac{\partial \psi}{\partial t} &= \frac{i \hbar}{2m} \nabla^2 \psi - i \frac{U}{\hbar}\psi, \\
		\frac{\partial \psi^{\ast}}{\partial t} &= - \frac{i\hbar}{2m} \nabla^2\psi^{\ast} + i \frac{U}{\hbar} \psi^{\ast} \\
		\implies \frac{\partial}{\partial t}(\psi^{\ast}\psi) &= \nabla \cdot \left[\frac{i\hbar}{2m} (\psi^{\ast}\nabla\psi - \psi\nabla\psi^{\ast}) \right] \\
		\implies \frac{\diff \mathcal{N}}{\diff t} &= \int_{\mathbb{R}^3} \nabla \cdot \left[\frac{i \hbar}{2m} (\psi^{\ast}\nabla \psi - \psi \nabla \psi^{\ast})\right] = 0,
	\end{align*}
	because $\psi, \psi^{\ast}$ are such that $|\psi|, |\psi^{\ast}| \to 0$ as $|\mathbf{x}| \to \infty$.
\end{proofbox}
\begin{enumerate}[resume*]
	\item The probability is conserved with respect to time:
		\[
			\frac{\partial \rho}{\partial t} (\mathbf{x}, t) + \nabla \cdot J = 0
		,\]
		where the probability current is\index{probability current}
		\[
			J(\mathbf{x}, t) = - \frac{i \hbar}{2m} \left[ \psi^{\ast}\nabla \psi - \psi\nabla\psi^{\ast}\right]
		.\]
\end{enumerate}

\subsection{Expectation Values and Operators}%
\label{sub:expectation_values_and_operators}

We have seen that all information is stored within the wavefunction, but we want to know how to extract information from $\psi$.

\begin{definition}
	An observable\index{observable} is any property of the particle described by $\psi$ that can be measured.
\end{definition}

\subsubsection{Heuristic Interpretation}%
\label{subsub:heuristic_interpretation}

Suppose we want to measure the position of a particle. The expectation is
\[
	\langle x \rangle = \int_{-\infty}^{\infty}x |\psi(x, t)|^2\diff x = \int_{-\infty}^{\infty}\psi^{\ast}(x, t) x \psi(x, t)\diff x
.\]
Hence the operator with respect to $x$ is
\[
\mathcal{O}_x \to \hat x \to x
.\]
The expectation value of an observable is the mean of an infinite series of measurements performed on particles on the same state. Performing the same measurement on one particle will collapse the wavefunction, and subsequent measurements will give the same result.

As time goes on $\langle x \rangle$ will change. Thus we might be interested in knowing the momentum. Using our usual definition of momentum as mass times velocity, we get
\begin{align*}
	\langle p \rangle &= m \frac{\diff \langle x \rangle}{\diff t} = m \frac{\diff}{\diff t} \int_{-\infty}^{\infty} \psi^{\ast} x \psi \diff x = m \int_{-\infty}^{\infty} x \frac{\partial}{\partial t}(\psi^{\ast} \psi)\diff x \\
			  &= \frac{i \hbar m}{2m}\int_{-\infty}^{\infty} x \frac{\partial}{\partial x}\left(\psi^{\ast} \frac{\partial \psi}{\partial x} - \psi \frac{\partial \psi^{\ast}}{\partial x}\right) \diff x \\
			  &= - \frac{i \hbar}{2} \int_{-\infty}^{\infty} \left( \psi^{\ast} \frac{\partial \psi}{\partial x} - \psi \frac{\partial \psi^{\ast}}{\partial x} \right) \diff x \\
			  &= - i \hbar \int_{-\infty}^{\infty} \psi^{\ast} \frac{\partial \psi}{\partial x}\diff x = \int_{-\infty}^{\infty} \psi^{\ast} \left(-i\hbar \frac{\partial}{\partial x} \right)\psi \diff x.
\end{align*}
Hence we can see the operator $\hat x = x$ represents position, and the operator $\hat p = - i \hbar \partial/\partial x$ represents momentum.

\subsubsection{Hermitian Operators}%
\label{subsub:hermitian_operators}

In a $\mathbb{C}^{n}$ linear map, we generally have $w = Tv$, where $T$ is a complex matrix of size $n$. In quantum mechanics, the linear maps are from $\mathcal{H} \to \mathcal{H}$, given by $\hat O : \psi \to \tilde \psi$.

\begin{definition}
	An operator\index{operator} $\hat O$ is any linear map $\mathcal{H} \to \mathcal{H}$ such that
	\[
		\hat O(a_1 \psi_1 + a_2 \psi_2) = a_1 \hat O \psi_1 + a_2 \hat O \psi_2
	,\]
	with $a_1, a_2 \in \mathbb{C}$, $\psi_1, \psi_2 \in \mathcal{H}$.
\end{definition}

Some examples of operators are:
\begin{itemize}
	\item Finite differential operators, given by
		\[
			\sum_{n = 0}^{N} p_n(x) \frac{\partial^{n}}{\partial x^{n}}
		.\]
	\item Translation operators
		\[
			S_a : \psi(x) \to \psi(x - a)
		.\]
	\item Parity operator
		\[
			P : \psi(x) \to \psi(-x)
		.\]
\end{itemize}

\begin{definition}
	The Hermitian conjugate\index{Hermitian conjugate} $\hat O^{\dagger}$ of an operator $\hat O$ is the operator such that
	\[
		(\hat O^{\dagger} \psi_1, \psi_2) = (\psi_1, \hat O \psi_2)
	.\]
\end{definition}
We can verify that
\begin{itemize}
	\item $(a_1 \hat A_1 + a_2 \hat A_2)^{\dagger} = a_1^{\ast}\hat A_1^{\dagger} + a_2^{\ast} \hat A_2^{\dagger}$,
	\item $(\hat A \hat B)^{\dagger} = \hat B^{\dagger} \hat A^{\dagger}$.
\end{itemize}

\begin{definition}
	An operator $\hat O$ is Hermitian if\index{Hermitian operator}
	\[
		\hat O = \hat O^{\dagger} \iff (\hat O \psi_1, \psi_2) = (\psi_1, \hat O \psi_2)
	.\]
	All physics quantities in quantum mechanics are represented by Hermitian operators.
\end{definition}

\begin{exbox}
	\begin{enumerate}[(i)]
		\item $\hat x : \psi(x, t) \to x \psi(x, t)$ is Hermitian as
			\[
				\int_{-\infty}^{\infty}(x\psi_1)^{\ast}\psi_1\diff x = \int_{-\infty}^{\infty}\psi_1^{\ast} x \psi_2 \diff x
			.\]
		\item $\hat p : \psi(x, t) \to -i\hbar \frac{\partial \psi}{\partial x}(x, t)$ is Hermitian as
			\begin{align*}
				(\hat p \psi_1, \psi_2) &= \int_{-\infty}^{\infty} \left(-i\hbar \frac{\partial \psi_1}{\partial x}\right)^{\ast} \psi_2 \diff x = i \hbar \int_{-\infty}^{\infty}\frac{\partial \psi_1^{\ast}}{\partial x}\psi_2 \diff x \\
							&= i \hbar \left[\psi_1^{\ast} \psi_2\right]_{-\infty}^{\infty} - i \hbar \int_{-\infty}^{\infty} \psi_1^{\ast} \frac{\partial \psi_2}{\partial x}\diff x \\
							&= \int_{-\infty}^{\infty} \psi_1^{\ast} \left(-i\hbar \frac{\partial \psi_2}{\partial x}\right) \diff x = (\psi_1, \hat p \psi_2).
			\end{align*}
		\item Kinetic energy
			\[
				\hat T : \psi(x, t) \to \frac{\hat p^2}{2m} \psi(x, t) = - \frac{\hbar^2}{2m} \frac{\partial^2 \psi}{\partial x^2}(x, t)
			.\]
		\item Potential energy
			\[
				\hat U : \psi(x, t) \to U(\hat x)\psi(x, t) = U(x)\psi(x, t)
			.\]
		\item Total energy
			\[
				\hat H : \psi(x, t) \to (\hat T + \hat U)\psi(x, t) = \left(- \frac{\hbar^2}{2m} \frac{\partial^2}{\partial x^2} + U(x) \right) \psi(x, t)
			.\]
	\end{enumerate}
\end{exbox}

\begin{theorem}
	The eigenvalues of Hermitian operators are real.
\end{theorem}

\begin{proofbox}
Let $\hat A$ be a hermitian operator with eigenvalue $a$, eigenfunction $\|\psi\| = 1$. Then
\[
	(\psi, \hat A \psi) = (\psi, a \psi) = a(\psi, \psi) = a
,\]
but since $\hat A$ is Hermitian, this equals
\[
	(\hat A \psi, \psi) = (a \psi, \psi) = a^{\ast} (\psi, \psi) = a^{\ast}
.\]
So $a^{\ast} = a$, and $a \in \mathbb{R}$.
\end{proofbox}


\begin{theorem}
	If $\hat A$ is a Hermitian operator, and $\psi_1, \psi_2$ are normalised eigenfunctions of $\hat A$ with distinct eigenvalues $a_1, a_2$, then $\psi_1, \psi_2$ are orthogonal.
\end{theorem}

\begin{proofbox}	
We have $\hat A \psi_1 = a_1\psi_1$, $\hat A \psi_2 = a_2 \psi_2$. Then
\begin{align*}
	a_1(\psi_1, \psi_2) &= a_1^{\ast}(\psi_1, \psi_2) = (a_1 \psi_1, \psi_2) = (\hat A\psi_1, \psi_2) \\
			    &= (\psi_1, A \psi_2) = (\psi_1, a_2 \psi_2) = a_2(\psi_1, \psi_2).
\end{align*}
Since $a_1 \neq a_2$, we get $(\psi_1, \psi_2) = 0$.
\end{proofbox}


\begin{theorem}
	The discrete (or continuous) set of eigenfunctions of any Hermitian operator together form a complete orthonormal basis of $\mathcal{H}$, so
	\[
		\psi(x, t) = \sum_{i \in I}c_i \psi_i(x, t)
	.\]
\end{theorem}

\subsubsection{Expectation values and operators}%
\label{subsub:expectation_values_and_operators}

So far, we have seen every quantum observable is represented by a Hermitian operator $\hat O$. We define the following postulates for the operators:

\begin{enumerate}[1.]
	\item The possible outcomes of a measurement of the observable $O$ are the eigenvalues of $\hat O$.
	\item If $\hat O$ has a discrete set of normalized eigenfunctions $\{\psi_i\}$ with distinct eigenvalues $\{\lambda_i\}$, the measurement of $O$ on a particle described by $\psi$ has probability
		\[
			\mathbb{P}(O = \lambda_i) = |a_i|^2
		,\]
		where $\psi = \sum a_i \psi_i$.
	\item If $\{\psi_i\}$ are the set of orthonormal eigenfunctions of $\hat O$, and $\{\psi_i\}_{i \in I}$ is the complete set of orthonormal eigenfunctions with eigenvalue $\lambda$, then
		\[
			\mathbb{P}(O = \lambda) = \sum_{i \in I}|a_i|^2
		.\]
		Indeed, if $\psi$ is normalized, we can check
		\[
			\sum_{i = 1}^{N} |a_i|^2 = \sum_{i = 1}^{N}(a_i \psi_i, a_i \psi_i) = \sum_{i, j = 1}^{N} (a_i \psi_i, a_j \psi_j) = (\psi, \psi) = 1
		.\]
	\item The projection postulate\index{projection postulate}: If $O$ is measured on $\psi$ at time $t$ and the outcome of the measurement of $\lambda_i$, the wavefunction of $\psi$ instantaneously becomes $\psi_i$. If $\hat O$ has degenerate eigenvalues with the same eigenvalue, the wavefunction becomes
		\[
		\psi = \sum_{i \in I}a_i \psi_i
		.\]
\end{enumerate}

\begin{definition}[Projection Operator]\index{projection operator}
	Given $\psi = \sum a_i \psi_i = \sum (\psi_i, \psi) \psi$, define
	\[
		\hat P_i : \psi \mapsto (\psi_i, \psi) \psi_i
	.\]
\end{definition}

We can now define the expectation\index{expectation} value of an observable measured on state $\psi$:
\begin{align*}
	\langle O \rangle_{\psi} &= \sum_{i} \lambda_i \mathbb{P}(O = \lambda_i) = \sum_{i} \lambda_i |a_i|^2 = \sum_{i} \lambda_i |(\psi_i, \psi)|^2 \\
				 &= \Biggl( \sum_{i}(\psi_i, \psi)\psi_i, \sum_{j} \lambda_j(\psi_j, \psi) \psi_j \Biggr) = (\psi, \hat O \psi) \\
				 &= \int \psi^{\ast}(x, t) \hat O \psi(x, t) \diff x.
\end{align*}
The expectation satisfies linearity:
\[
	\langle a \hat A + b \hat B \rangle_{\psi} = a \langle \hat A \rangle_{\psi} + b \langle \hat B \rangle_{\psi}
.\]

\begin{remark}
	\begin{itemize}
		\item[]
		\item The physics implication of the projection postulate is that if $O$ is measured twice, the outcome of the second measurement is the same as the first with probability $1$, if the difference between measurement times is small.
		\item Born's rule says if $\phi(\mathbf{x}, t)$ is the state that gives the desired outcome om a measurement on a state $\psi(\mathbf{x}, t)$, the probability of such an outcome is given by
			\[
				|(\psi, \phi)|^2 = \Biggl| \int_{-\infty}^{\infty} \psi^{\ast}(x, t) \phi(x, t) \diff x \Biggr|^2
			.\]
	\end{itemize}
\end{remark}

\subsection{Time-independent Schr\"{o}dinger equation}%
\label{sub:time_independent_schr"_o_dinger_equation}

Recall the one-dimensional time-dependent Schr\"{o}dinger equation
\[
	i \hbar \frac{\partial \psi}{\partial t}(x, t) = -\frac{\hbar^2}{2m} \frac{\partial^2 \psi}{\partial x^2} (x, t) + U(x) \psi(x, t) = \hat H \psi(x, t)
.\] 
We try separation of variables:
\[
	\psi(x, t) = T(t) \chi(x)
.\]
Substituting, this gives
\[
	i \hbar \frac{\partial T}{\partial t}(t) \chi(x) = T(t) \hat H \chi(x)
.\]
Dividing by $T(t)\chi(x)$, we get
\[
	\frac{1}{T(t)} i \hbar \frac{\partial T (t)}{\partial t}(t) = \frac{\hat H \chi (x)}{\chi(x)} = E
.\]
Solving for time, we get
\[
	\frac{1}{T(t)} i \hbar \frac{\partial T(t)}{\partial t} = E \implies T(t) = e^{-i E t/\hbar}
.\]
The time-independent Schr\"{o}dinger equation is the equation for $\chi(x)$:
\[
	\hat H \chi(x) = E \chi(x) \iff - \frac{\hbar^2}{2m} \frac{\partial^2 \chi(x)}{\partial x^2} + U(x) \chi(x) = E \chi(x)
.\]
Note that the time-independent Schr\"{o}dinger equation is an eigenvalue equation for the $\hat H$ operator (Hamiltonian)\index{Hamiltonian}, and the eigenvalues of $\hat H$ are all possible outcomes for the measurement of the energy of the state $\psi$.

\subsection{Stationary states}%
\label{sub:stationary_states}

We have find a particular solution to the time-dependent Schr\"{o}dinger equation
\[
	\Psi(x, t) = \chi(x) e^{-iEt/\hbar}
.\]

\begin{definition}
	These solutions are called \textbf{stationary states}.\index{stationary states}
\end{definition}

This is because the probability distribution of these states satisfies
\[
	\rho(x, t) = |\Psi(x, t)|^2 = |\chi(x)|^2
,\]
meaning they are independent of time.

Due to completeness of eigenfunctions, applying this to $\hat O = \hat H$ gives the following result.

\begin{theorem}
	Every solution of the time-dependent Schr\"{o}dinger equation can be written as a linear combination of stationary states.
\end{theorem}

For systems with a discrete set of eigenvalues of $\hat H$, namely, $E_n = E_1, E_2, \ldots$, then
\[
	\psi(x, t) = \sum_{n} a_n \chi_n(x) e^{-iE_n t/\hbar}
.\]

For systems with a continuous set of eigenvalues of $\hat H$,
\[
	\psi(x, t) = \int_{\Delta}A(\alpha)\chi_{\alpha}(x) e^{-iE_{\alpha}t/\hbar} \diff \alpha
.\]
Here, the probability of measuring the energy to be $E_n = E(\alpha)$ is $|a_n|^2$ in the discrete case, or $|A(\alpha)|^2 \diff \alpha$ in the continuous case.

Consider a system with only two energy eigenvalues $E_1 \neq E_2$. Then we can write the state $\psi$ as a combination
\[
	\psi(x, t) = a_1 \chi_1(x) e^{-i E_1 t/\hbar} + a_2 \chi_2(x) e^{-i E_2 t/\hbar}
.\]
This gives $\psi(x, 0) = a_1 \chi_1(x) + a_2 \chi_2(x)$. If $a_1 = 0$, then $\psi(x, t) = a_2 \chi_2(x) e^{-i E_2 t/\hbar}$, which implies $\psi$ is a stationary state.

If $a_1 \neq 0$ and $a_2 \neq 0$, then
\begin{align*}
	|\psi(x, t)|^2 &= |a_1 \chi_1 e^{-iE_1 t/\hbar} + a_2 \chi_2 e^{-iE_2 t/\hbar}|^2 \\
		       &= a_1^2|\chi_1^2| + a_2^2|\chi_2|^2 + 2a_1a_2 \chi_1(x)\chi_2(x) \cos \biggl( \frac{(E_1 - E_2)t}{\hbar} \biggr).
\end{align*}
Hence $\psi$ is not a stationary state.
\newpage

\section[4D Solutions]{4D Solutions of Schr\"{o}dinger Equation}%
\label{sec:4d_solutions_of_schr"_o_dinger_equation}

The time-independent Schr\"{o}dinger equation is
\[
	\hat H \chi(x) = E \chi(x)
,\]
\[
	- \frac{\hbar^2}{2m}\chi''(x) + U(x) \chi(x) = E \chi(x)
.\]
We want to solve the TISE for 3 cases:
\begin{enumerate}[1.]
	\item Bound states;
	\item Free particles;
	\item Scattering states.
\end{enumerate}

\subsection{Bound States}%
\label{sub:bound_states}

\subsubsection{Infinite potential well}%
\label{subsub:infinite_potential_well}

Consider the potential function
\[
	U(x) =
	\begin{cases}
		0 & |x| \leq a, \\
		+\infty & |x| > a,
	\end{cases}
\]
For $|x| > a$, we must have $\chi(x) = 0$, otherwise $U \cdot \chi = \infty$. Hence we have boundary conditions $\chi(\pm a) = 0$.

For $|x| \leq a$, we look for solutions of
\[
\begin{dcases}
	- \frac{\hbar^2}{2m} \chi''(x) = E \chi(x), \\
	\chi(\pm a) = 0.
\end{dcases}
\]
This has solutions $\chi(x) = A \sin (k x) + B \cos (k x)$, where $k = \sqrt{2mE/\hbar^2}$. Using the boundary conditions, we get $A \sin (k a) = B \cos (k a) = 0$.
\begin{enumerate}[(i)]
	\item $A = 0$ and $\cos (k a) = 0$, so $k_n = n \pi/(2a)$ for odd integers $a$.
	\item $B = 0$ and $\sin (k a) = 0$, so $k_n = n \pi/(2a)$ for even integers $a$.
\end{enumerate}

We can determine $A$ and $B$ by requiring normalization of the electron:
\[
	\int_{-a}^{a} |\chi_n(x)|^2\diff x = 1 \implies A = B = \sqrt{\frac{1}{a}}
.\]
\begin{remark}
	\begin{enumerate}[(i)]
		\item The ground state $\chi_1$ has nonzero energy.
		\item As $n \to \infty$, $|\chi_n(x)|^2$ approaches a constant.
	\end{enumerate}
\end{remark}

\begin{proposition}
	If the quantum system has non-degenerate eigenstates, then if $U(x) = U(-x)$, the eigenfunctions of $\hat H$ are either odd or even.
\end{proposition}

\begin{proofbox}	
If $U(x) = U(-x)$, then the TISE is invariant under $x \mapsto -x$. Hence if $\chi(x)$ is a solution with eigenvalue $\lambda$, $\chi(-x)$ is also a solution with eigenvalue $\lambda$ and $\chi(-x) = \alpha \chi(x)$.

	Hence $\chi(x) = \chi(-(-x)) = \alpha^2 \chi(x)$, so $\alpha^2 = 1$. Hence either $\alpha = 1$, and $\chi$ is even, or $\alpha = -1$, and $\chi$ is odd.
\end{proofbox}

\subsubsection{Finite Potential Well}%
\label{subsub:finite_potential_well}

Consider the more physical problem of a finite potential well:
\[
	U(x) =
	\begin{cases}
		0 & |x| \leq a, \\
		U_0 & |x| > a.
	\end{cases}
\]
Consider $E > 0$ and $E < U_0$. We look for odd and even eigenfunctions.
\begin{enumerate}[(i)]
	\item For even parity bounded states, $\chi(-x) = \chi(x)$, so we solve
		\[
		\begin{cases}
			-\frac{\hbar^2}{2m} \chi''(x) = E \chi(x) & |x| \leq a,\\
			-\frac{\hbar^2}{2m}\chi''(x) = (E - U_0)\chi(x) & |x| > a.
		\end{cases}
		\]
		The first equation gives $\chi''(x) + k^2 \chi(x) = 0$, where $k = \sqrt{2mE/\hbar^2}$, and so $\chi(x) = A \sin (kx) + B \cos (kx)$. But $A = 0$, since $\chi$ is even, so $\chi(x) = B \cos (kx)$.

		The second equation gives $\chi''(x) - \bar k^2 \chi(x) = 0$, with $\bar k = \sqrt{2m(U_0 - E)/\hbar^2}$. Then $\chi(x) = ce^{+\bar k x} + D e^{- \bar k x}$. Imposing normalisability, we get $c = 0$ for $x > a$ and $d = 0$ for $x < -a$. Then, imposing evenness, we get
		\[
			\chi(x) =
			\begin{cases}
				C e^{\bar k r} & x < -a, \\
				B \cos (kx) & |x| \leq a, \\
				C e^{-\bar k r} &  x > a.
			\end{cases}
		\]
		Continuity of $\chi$ and $\chi'$ at $x = \pm a$ implies
		\[
			c e^{- \bar k a} = B \cos (k a)
		,\]
		\[
			- \bar k c e^{- \bar k a} = - k B \sin (k a)
		.\]
		This gives $k \tan (ka) = \bar k$, and $k^2 + \bar k^2 = (2mU_0)/\hbar^2$. Let $\xi = k a, \eta = \bar k a$. Then we have $\xi \tan \xi = \eta$ and $\xi^2 + \eta^2 = r_0^2$.

		The eigenvalues of the Hamiltonian correspond to the points of intersection of these graphs.

		As $U_0 \to \infty$, we get the same result for the infinite potential well.
	\item The odd part is found in the example sheets.
\end{enumerate}

\subsubsection{Harmonic Oscillator}%
\label{subsub:harmonic_oscillator}

Consider the harmonic oscillator\index{harmonic oscillator}, which models many physical situations.
\[
	U(x) = \frac{1}{2}k x^2 = \frac{1}{2}m \omega^2 x^2
.\]
Here, we say $k$ is the elastic constant\index{elastic constant} and $\omega = \sqrt{k/m}$ is the angular frequency of the harmonic oscillator. In classical mechanics, we model this system as $\ddot x (t) = - \omega^2 x(t)$, so
\[
	x(t) = A \sin (\omega t) + B \cos (\omega t)
,\]
with $T = 2\pi/\omega$ being the period. In quantum mechanics, we have the Schr\"{o}dinger equation
\[
	- \frac{\hbar^2}{2m} \chi''(x) + \frac{1}{2} m \omega^2 x^2 \chi(x) =  E \chi(x)
.\]

We expect to have a discrete set of eigenvalues, with either even or odd eigenfunctions. Defining the variables
\[
\xi^2 = \frac{m \omega}{\hbar} x^2, \quad \eps = \frac{2E}{\hbar \omega}
,\]
under a change of variables, our equation becomes
\[
	- \frac{\Diff2 \chi(\xi)}{\diff \xi^2} + \xi^2 \chi(\xi) = \eps \chi(\xi)
.\]
We solve this by starting from a particular solution $\eps = 1$, which corresponds to $E_0 = \hbar \omega/2$. Then,
\[
	\chi_0(\xi) = - e^{-\xi^2/2}
.\]
Hence we have found an eigenvalue $E_0 = \hbar \omega/2$. To find other eigenfunctions of $\hat H$, we take the function
\[
	\chi(\xi) = f(\xi) e^{-\xi^2/2}
.\]
Then plugging in $\chi$, we get $f$ must satisfy
\[
	- \frac{\Diff2 f}{\diff \xi^2} + 2 \xi \frac{\diff f}{\diff \xi} + (1 - \eps)f = 0
.\]
We can represent $f$ as a power series: $f(\xi) = \sum a_n \xi^{n}$. Then,
\begin{align*}
	\xi \frac{\diff f}{\diff \xi} &= \sum_{n = 0}^{\infty} n a_n \xi^{n}, \\
	\frac{\Diff2 f}{\diff \xi^2} &= \sum_{n = 0}^{\infty}n(n-1)a_n \xi^{n-1} = \sum_{n = 0}^{\infty}(n+1)(n+2)a_n \xi^{n}.
\end{align*}

Plugging these values in, we get
\[
	\sum_{n = 0}^{\infty} [(n+1)(n+2)a_{n+2} = 2 na_n + (\eps - 1)a_n]\xi^{n} = 0 \implies a_{n+2} = \frac{(2n - \eps + 1)}{(n+1)(n+2)}
.\]

Because of the parity of eigenfunctions, we either have $a_n = 0$ for odd $n$, giving $f(\xi) = f(-\xi)$, or $a_n = 0$ for even $n$, giving $f(\xi) = -f(-\xi)$.

\begin{proposition}
	If the series $\sum a_n \xi^{n}$ does not terminate, then the eigenfunction of $\hat H$ would not be normalisable.
\end{proposition}

\begin{proofbox}
	Suppose that the series does not terminate. We examine the behaviour of the series:
	\[
	\frac{a_{n+2}}{a_n} \to \frac{2}{n}
	.\]
	This has the same asymptotic behaviour as
	\[
		y(\xi) = e^{\xi ^2} = \sum_{m = 0}^{\infty} \frac{\xi^{2m}}{m!} = \sum_{n = 0}^{\infty} b_m \xi^{m}
	.\]
	So if $e^{\xi^2/2}$ and $f(\xi)$ have the same asymptotic behaviour, then
	\[
		\chi(\xi) \sim e^{\xi^2} \cdot e^{- \xi^2/2} = e^{\xi^2/2}
	.\]
	However, this is non-normalisable.
\end{proofbox}

Given that the series $\sum a_n \xi^{n}$ terminates, there must exist $N$ such that $a_{N + 2} = 0$, with $a_N \neq 0$, i.e. $\eps = 2N+1$.

Thus $E_N = (N+1/2)\hbar \omega$. Note in particular $E_{N+1} - E_N = \hbar \omega$, and the eigenfunctions are $\chi(x) = f_N(\xi) e^{-\xi^2/2}$. Here, the polynomials $f_N$ satisfy
\[
	f_N(\xi) = (-1)^{N} e^{\xi^2} \frac{\Diff N}{\diff \xi^{N}}(e^{-\xi^2})
,\]
and are known as the Hermite polynomials.\index{Hermite polynomials}. The first few values are

\begin{center}
\begin{tabular}{c|c|c}
	$N$ & $E_N$ & $f_N(\xi)$ \\
	\hline
	$0$ & $\hbar \omega/2$ & $1$ \\
	$1$ & $3\hbar\omega/2$ & $\xi$ \\
	$2$ & $5\hbar\omega/2$ & $(1 - 2\xi^2)$ \\
	$3$ & $7\hbar\omega/2$ & $(\xi - 2\xi^2/3)$ \\
\end{tabular}
\end{center}

\subsection{The Free Particle}%
\label{sub:the_free_particle}

We take the time-independent Schr\"{o}dinger equation with $U \equiv 0$:
\[
	-\frac{\hbar^2}{2m} \chi''(x) = E\chi(x)
.\]
This has solutions
\[
	\chi(x) = e^{ikx}
,\]
where $k = \sqrt{2mE/\hbar^2}$, and the energy is then $\hbar^2k^2/2m$. Hence,
\[
	\psi_k(x) = \chi_k(x) e^{-iE_kt/\hbar} = \exp \left( i \left(kx - \frac{\hbar k^2}{2m} \right) \right)
.\]
However, this wavefunction is not square integrable, as
\[
	\int_{-\infty}^{\infty} |\phi_k(x, t)|^2\diff x = \int_{-\infty}^{\infty}1\diff x = +\infty
.\]
To resolve this, we have two options:
\begin{enumerate}[1.]
	\item Build a linear superposition of non-normalisable states that is normalisable;
	\item Ignore the problem but change the interpretation.
\end{enumerate}

\subsubsection{Gaussian Wavepacket}%
\label{subsub:gaussian_wavepacket}

Here we will build a superposition of non-normalisable states to make a normalisable state. Take
\[
	\psi(x, t) = \int_{-\infty}^{\infty} A(k) \psi_k(x, t) \diff k
.\]
A possible option is the \textit{Gaussian wavepacket}\index{Gaussian wavepacket}:
\[
	A(k) = A_{GP}(k) = \exp \left[ - \frac{\sigma}{2} (k - k_0)^2 \right]
,\]
where $\sigma$ is positive and $k_0 \in \mathbb{R}$.

Substituting these values,
\begin{align*}
	\psi_{GP}(x, t) &= \int_{-\infty}^{\infty} \exp \left[ - \frac{\sigma}{2}(k - k_0)^2 + i k x - \frac{i \hbar k^2}{2m} \right]\diff k \\
			&= \int_{-\infty}^{\infty} \left[-\frac{1}{2}\alpha k^2 + \beta k + \delta\right]\diff k \\
			&= \int_{-\infty}^{\infty} \left[ - \frac{\alpha}{2}\left(k - \frac{\beta}{\alpha}\right)^2 + \frac{\beta^2}{2\alpha} + \delta\right] \diff k \\
			&= \exp\left[ \frac{\beta^2}{2\alpha} + \delta \right] \int_{-\infty}^{\infty} \exp \left[- \frac{\alpha}{2} \left(k - \frac{\beta}{\alpha}\right)^2\right]\diff k \\
			&= \exp \left[ \frac{\beta^2}{2\alpha} + \delta \right] \int_{-\infty - iu}^{\infty - iu} \exp\left( - \frac{\alpha}{2} \tilde k^2\right) \diff k \\
			&= \sqrt{\frac{2\pi}{\alpha}} \exp\left[\frac{\beta^2}{2\alpha} + \delta\right].
\end{align*}
Now $\beta^2$ has a real factor of $-x^2$, hence $\psi_{GP}$ is normalisable to $\bar \psi_{GP}$. Hence
\[
	\rho_{GP}(x,t) = |\bar\psi_{GP}(x, t)|^2 = \sqrt{\frac{\sigma}{\pi(\sigma^2 + \frac{\hbar^2t^2}{m^2})}} \exp\left[\frac{\sigma(x - \frac{\hbar k_0}{m}t)^2}{\sigma^2 + \frac{\hbar^2t^2}{m^2}} \right]
.\]
The centre of the distribution is $\langle x \rangle_{\psi_{GP}}$, which is
\[
	\langle x \rangle_{\psi_{GP}} = \int_{-\infty}^{\infty} \bar \psi^{\ast}_{GP}(x, t) x \bar \psi_{GP}^{\ast}(x, t) \diff x = \int_{-\infty}^{\infty} x \rho_{GP}(x, t) \diff x = \frac{\hbar k_0}{m}t
.\]
Here $\hbar k_0/m$ can be though of as the velocity of the particle. The error of the position of the particle is
\[
	\Delta x = \sqrt{\langle x^2 \rangle_{\psi_{GP}} - \langle x \rangle^2_{\psi_{GP}}} = \sqrt{\frac{1}{2} \left(\sigma + \frac{\hbar^2t^2}{m^2 \sigma}\right)}
.\]
Physically, this means the longer the particle is left to travel, the more uncertain we are about its position. We can also calculate the momentum
\[
	\langle p \rangle_{\psi_{GP}} = \int_{-\infty}^{\infty} \bar \psi_{GP}^{\ast}(x, t) \left( - i \hbar \frac{\diff}{\diff x} \bar \psi_{GP}(x, t) \right) \diff x = \hbar k_0
,\]
which we expect given the velocity of the particle. The error of the momentum is
\[
	\Delta p = \sqrt{\langle p^2 \rangle_{\psi_{GP}} - \langle p \rangle_{\psi_{GP}}^{2}} = \frac{\hbar}{\sqrt{2 \sigma}}
.\]
%At $t = 0$, $\Delta p = \hbar \sqrt{2/\sigma}$, but as  $t \to \infty$, $\Delta p$ decreases.

%What we have found is that
%\begin{itemize}
	%\item As $t\to \infty$, $\Delta x \to \infty$ but $\Delta p \to 0$.
	%\item $\Delta x \cdot \Delta p = \frac{\hbar}{2}$, which is the minimal such value.
%\end{itemize}

We see that at $t = 0$, $\Delta x \cdot \Delta p = \hbar/2$, and in fact this is minimal.

The Gaussian wavepacket is a state of minimum uncertainty at $t = 0$. Other $A(k)$ may give a normalisable state, but if you compute $\Delta x \cdot \Delta p$ the value will be greater they $\frac{\hbar}{2}$.

For the De Broglie wave $\psi_k(x, t)$, we have $\Delta x = \infty$ and $\Delta p = 0$.

\subsubsection{Beam Interpretation}%
\label{subsub:beam_interpretation}

In the following, we ignore the normalisation problem, and take $\chi_k = e^{ikx}$ as an eigenfunction of $\hat H$, so
\[
	\psi_k(x, t) = A e^{ikx} e^{-i \frac{\hbar^2 k^2}{2m} t}
,\]
but instead of $\chi_n(x)$ describing a single particle, they describe a beam of particles with
\[
	P_n = \hbar k, \quad E_k = \frac{\hbar^2 k^2}{2m}
,\]
and probability density
\[
	\rho_k(x, t) = |A|^2
,\]
representing the constant average density of particles. We can compute the probability current as
\begin{align*}
	j_k(x, t) &= -\frac{i\hbar}{2m} \biggl( \psi_k^{\ast} \frac{\partial \psi_k}{\partial x} - \psi_k \frac{\partial \psi^{\ast}_k}{\partial x} \biggr) \\
		  &= |A|^2 \frac{\hbar k}{m} = |A|^2 \frac{p}{m},
\end{align*}
the average flux of particles.

\subsection{Scattering States}%
\label{sub:scattering_states}

Suppose we have a free particle, thrown at a potential barrier, between $0$ and $a$. Typically, if $E > U_0$, the particle will make it over, otherwise it will be reflected. For quantum states, this is more complicated.

\begin{definition}
	The probability that the particle is reflected is given by the reflection coefficient\index{reflection coefficient}
	\[
		R = \lim_{t \to \infty}\int_{-\infty}^{0} |\psi_{GP}(x, t)|^2\diff x
	.\]

	Similarly, the probability that the particle is transmitted is given by the transmission coefficient\index{transmission coefficient}
	\[
		T = \lim_{t \to \infty} \int_{0}^{\infty} |\psi_{GP}(x, t)|^2\diff x
	.\]
	From construction, $R + T = 1$.
\end{definition}

It is possible to solve scattering problems using the Gaussian wavepacket, however the beam interpretation gives the same results for $R$ and $T$, and it easier, hence we will use it.

\subsubsection{Scattering off Potential Step}%
\label{subsub:scattering_off_potential_step}

Consider the potential
\[
	U(x) =
	\begin{cases}
		0 & x \leq 0, \\
		U_0 & x > 0.
	\end{cases}
\]
To find $\chi_k(x)$, we solve the time-independent Schr\"{o}dinger equation on both regions:
\[
	- \frac{\hbar^2}{2m} \chi_k''(x) + U(x) \chi_k(x) = E \chi_k(x)
.\]
\begin{itemize}
	\item On $x \leq 0$, $U \equiv 0$, so the TISE becomes
		\[
			\chi_k''(x) + K^2 \chi_k(x) = 0
		,\]
		where $k = \sqrt{2mE/\hbar^2} > 0$. We know the general solution is
		\[
			\chi_k(x) = Ae^{ikx} + Be^{-ikx}
		.\]
	\item For $x > 0$, $U(x) = U_0$, so
		\[
			\chi_{\bar k}''(x) + \bar k^2 \chi_{\bar k}(x) = 0
		,\]
		where $\bar k = \sqrt{2m(E - U_0)/\hbar^2}$. If $E > U_0$, then
		\[
			\chi_{\bar k}(x) = Ce^{i \bar k x} + De^{i \bar k x}
		.\]
		Now $D = 0$, since initially there is no beam coming from the right.

		If $E < U_0$, we get
		\[
			\chi_{\bar k}(x) = C e^{- \nu x} + D e^{\nu x}
		.\]
		Similarly, $D = 0$, otherwise $\chi_{\bar k}$ diverges.
\end{itemize}
We look at the case when $E \geq U_0$. Putting the solutions together,
\[
	\chi_{k, \bar k}(x) =
	\begin{cases}
		A e^{ikx} + Be^{-ikx} & x \leq 0, \\
		Ce^{i\bar k x} & x > 0.
	\end{cases}
.\]
We impose the continuity of $\chi(x)$ and $\chi'(x)$ to get
\[
\begin{cases}
	A + B = C \\
	ikA - ikB = i \bar k C
\end{cases}
.\]
Hence, we get
\[
B = \frac{k - \bar k}{k + \bar k} A, \quad C = \frac{2k}{k + \bar k} A
.\]
In terms of the particle flux $J(x, t)$, we get
\[
	J(x, t) =
	\begin{dcases}
		\frac{\hbar k}{m}(|A|^2 - |B|^2) & x < 0, \\
		\frac{\hbar \bar k}{m} |C|^2 & x \geq 0.
	\end{dcases}
\]

Hence we have incoming, reflected and transmitted flux
\[
	j_{inc}(x) = \frac{\hbar k}{m}|A|^2, \quad j_{ref}(x) = \frac{\hbar k}{m} \biggl(\frac{k - \bar k}{k + \bar k}\biggr)^2 |A|^2, \quad j_{tr} = \frac{\hbar \bar k}{m} \frac{4k^2}{(k + \bar k)^2}|A|^2
.\]
Hence we can calculate the reflection and transmission coefficients
\[
	R = \frac{j_{ref}}{j_{inc}} = \frac{|B|^2}{|A|^2} = \biggl( \frac{k - \bar k}{k + \bar k}\biggr)^2
,\]
\[
	T = \frac{j_{tr}}{j_{inc}} = \frac{|C|^2}{|A|^2}\frac{\bar k}{k} = \frac{4 k \bar k}{(k + \bar k)^2}
.\]

Note $R + T = 1$, as $E \to U_0$, then $\bar k \to 0$ so $T \to 0$ and $R \to 1$, and as $E \to \infty$, $T \to 1$ and $R \to 0$.

Now we can look at the case when $E < U_0$. Then, we can calculate
\[
	j_{in} = \frac{\hbar k}{m}|A|^2, \quad j_{tr} = 0, \quad j_{ref}(x, t) = \frac{\hbar k}{m}|B|^2
.\]
Hence $R = 1$ and $T = 0$, but $\chi_{\bar k}(x) \neq 0$ for all $x > 0$.

\subsubsection{Scattering off Potential Barrier}%
\label{subsub:scattering_off_potential_barrier}

Consider the potential
\[
	U(x) =
	\begin{cases}
		0 & x \leq 0, x \geq a, \\
		U_0 & 0 < x < a.
	\end{cases}
\]
We consider when $E < U_0$, then let $k = \sqrt{2mE/\hbar^2} > 0$, $\eta = \sqrt{2m(U_0 - E)/\hbar^2} > 0$. The solution of the TISE is
\[
	\chi(x) =
	\begin{cases}
		e^{ikx} + Ae^{-ikx} & x \leq 0, \\
		B e^{-\nu x} + C e^{\nu x} & 0 < x < a, \\
		De^{ikx} + E^{-ikx} & x \geq a.
	\end{cases}
\]
Now $E = 0$ as there are no incoming waves from the right. The continuity of $\chi, \chi'$ at $x = 0, a$ gives four equations, from which we can solve for $A, B, C$ and $D$:
\[
\begin{cases}
	1 + A = B + C, \\
	ik - ikA = - \eta B + \eta C, \\
	Be^{-\eta a} + Ce^{\eta a} = De^{ika}, \\
	-\eta Be^{-\eta a} + \eta Ce^{\eta a} = ikDe^{ika}.
\end{cases}
\]

Solving, we can find
\[
	D = - \frac{4 i \eta k}{(\eta - ik)^2 \exp[(\eta + ik)a] - (\eta + ik)^2\exp[-(\eta - ik)a]}
.\]
Hence the transmission coefficient is
\[
	T = |D|^2 = \frac{4 k^2 \eta^2}{(k^2 + \eta^2)^2 \sinh^2(\eta a) + 4 k^2 \eta^2}
.\]

Taking the limit $U_0 \gg E$, then $\eta a \gg 1$, so
 \[
	 T \to \frac{16k^2 \eta^2}{(\eta^2 + k^2)^2} e^{-2 \eta a}
.\]

\newpage

\section{Simultaneous Measurements}%
\label{sec:simultaneous_measurements}

\subsection{Commutators}%
\label{sub:commutators}

\begin{definition}
	The \textit{commutator} of two operators $\hat A$, $\hat B$ is the operator $[\hat A, \hat B] = \hat A \hat B - \hat B \hat A$.\index{commutator}
\end{definition}

The commutator has the properties:
\begin{itemize}
	\item $[\hat A, \hat B] = - [\hat B, \hat A]$;
	\item $[\hat A, \hat A] = 0$;
	\item $[\hat A, \hat B \hat C] = [\hat A, \hat B] \hat C + \hat B[\hat A, \hat C]$;
	\item $[\hat A \hat B, \hat C] = \hat A[\hat B, \hat C] + [\hat A, \hat C]\hat B$.
\end{itemize}

\begin{exbox}
	We compute $[\hat x, \hat p]$ in one dimension. Take $\psi \in \mathcal{H}$. Then
	\[
		\hat x \hat p \psi = x \biggl( - i \hbar \frac{\partial}{\partial x} \biggr) \psi(x) = - i \hbar x \frac{\partial \psi}{\partial x}(x),
	\]
	\[
		\hat p \hat x \psi = - i \hbar \frac{\partial}{\partial x} (x \psi(x)) = - i \hbar \psi(x) - i \hbar x \frac{\partial \psi}{\partial x}(x)
	,\]
	\[
		\implies [\hat x, \hat p]\psi = i \hbar \psi, \quad [\hat x, \hat p] = i \hbar \hat I
	.\]
\end{exbox}

\begin{definition}
	Two Hermitian operators $\hat A$ and $\hat B$ are simultaneously diagonalizable in $\mathcal{H}$ if there exists a complete basis of joint eigenfunctions $\{\psi_i\}$ such that
	\[
	\hat A \psi_i = a_i \psi_i, \quad \hat B \psi_i = b_i \psi_i
	.\]
\end{definition}

\begin{theorem}
	Two Hermitian operators $\hat A$ and $\hat B$ are simultaneously diagonalizable if and only if $[\hat A, \hat B] = 0$.
\end{theorem}

\begin{proofbox}
	If $\hat A, \hat B$ are simultaneously diagonalizable then there exists a set of joint eigenfunctions $\{\psi_i\}$ which are a complete basis of $x$. Now,
	\[
		[\hat A, \hat B] \psi_i = \hat A \hat B \psi_i - \hat B \hat A \psi_i = (a_i b_i - a_i b_i)\psi_i = 0
	.\]
	Take $\psi \in \mathcal{H}$. Then we can write it as a sum of $\psi_i$, so
	\[
		[\hat A, \hat B] \psi = \sum c_i [\hat A, \hat B] \psi_i = 0
	.\]
	Now if $[\hat A, \hat B] = 0$ and $\psi_i$ is an eigenfunction of $\hat A$ with eigenvalue $a_i$, then
	\[
		0 = [\hat A, \hat B] \psi_i = \hat A \hat B \psi_i - \hat B \hat A \psi_i = \hat A \hat B \psi_i - a_i \hat B \psi_i
	\]
	\[
		\implies \hat A (\hat B \psi_i) = a_i (\hat B \psi_i)
	.\]
	Thus $\hat B$ maps the eigenspace $E_i$ of eigenfunctions of $\hat A$ with eigenvalues $a_i$ into itself, so $\hat B|_{E_i}$ is a Hermitian operator on $E_i$, and we can diagonalize it.

	Since this holds for all eigenspaces $E_i$ of $\hat A$, we can find a complete basis of simultaneous eigenfunctions of $\hat A$ and $\hat B$.
\end{proofbox}

\subsection{Heisenberg's Uncertainty Principle}%
\label{sub:heisenberg_s_uncertainty_principle}

\begin{definition}
	The uncertainty in a measurement of an observable $A$ on a state $\psi$ is defined as
	\[
		\Delta_{\psi}A = \sqrt{(\Delta_{\psi}A)^2}
	,\]
	where
	\[
		(\Delta_{\psi}A)^2 = \langle (\hat A - \langle \hat A \rangle_{\psi} \hat I)^2 \rangle_{\psi} = \langle \hat A^2 \rangle_{\psi} - (\langle \hat A \rangle_{\psi})^2
	.\]
\end{definition}

Note these two definitions are equivalent, as
\begin{align*}
	\langle( \hat A - \langle \hat A \rangle_{\psi} \hat I)^2 \rangle_{\psi} &= \int_{\mathbb{R}^3} \psi^{\ast} (\hat A - \langle \hat A \rangle_{\psi}\hat I)^2 \psi \Diff3 x \\
										 &= \int_{\mathbb{R}^3} \psi^{\ast} \hat A^2 \psi \Diff3x + (\langle \hat A \rangle_{\psi})^2 \int_{\mathbb{R}^3} \psi^{\ast}\psi \Diff3 - 2 \langle \hat A \rangle_{\psi} \int_{\mathbb{R}^3} \psi^{\ast} \hat A \psi \Diff3 x \\
										 &= \langle \hat A \rangle^2_{\psi} - (\langle \hat A \rangle_{\psi})^2.
\end{align*}

\begin{lemma}
	$(\Delta_{\psi} A)^2 \geq 0$, with equality if and only if $\psi$ is an eigenfunction of $A$.
\end{lemma}

\begin{proofbox}
	\begin{align*}
		(\Delta_{\psi}A)^2 &= \langle (\hat A - \langle \hat A \rangle_{\psi} \hat I)^2 \rangle_{\psi} \\
				   &= (\psi, (\hat A - \langle \hat A \rangle_{\psi} \hat I)^2 \psi) = ((\hat A - \langle \hat A \rangle_{\psi} \hat I)\psi, (\hat A - \langle \hat A \rangle_{\psi} \hat I)\psi) \\
				   &= (\phi, \phi) \geq 0.
	\end{align*}
	Now we prove that $(\Delta_{\psi}A)^2 = 0 \iff \phi = 0$. Indeed, if $(\Delta_{\psi}A)^2 = 0$, then $(\phi, \phi) = 0$, so $\hat A \psi = \langle \hat A \rangle\_{psi} \psi$.

		Alternatively, is $\psi$ is an eigenfunction of $\hat A$ with real eigenvalue $a \in \mathbb{R}$, then $\langle \hat A \rangle_{\psi} = (\psi, \hat A \psi) = a (\psi, \psi) = a$, and similarly $\langle \hat A^2 \rangle_{\psi} = a^2$.

		Then $(\Delta_{\psi}A)^2 = \langle \hat A^2 \rangle_{\psi} - (\langle \hat A \rangle_{\psi})^2 = 0$.
\end{proofbox}

\begin{lemma}
	If $\psi, \phi \in \mathcal{H}$, then $|(\psi, \phi)|^2 \leq (\psi, \psi)(\phi, \phi)$, with equality if and only if $\phi = a \psi$ for some $a \in \mathbb{C}$.
\end{lemma}

This is the Schwarz inequality.

\begin{theorem}[Generalized Uncertainty Theorem]
	If $A$ and $B$ are observables, then
	\[
		(\Delta_{\psi}A)(\Delta_{\psi}B) \geq \frac{1}{2} |(\psi, [\hat A, \hat B]\psi)|
	.\]
\end{theorem}

\begin{proofbox}
	Note $(\Delta_{\psi}A)^2 = ((\hat A - \langle \hat A \rangle_{\psi} \hat I) \psi, (\hat A - \langle \hat A \rangle_{\psi} \hat I)\psi)$, and similarly for $B$.

	Define $\hat A' = \hat A - \langle \hat A \rangle_{\psi} \hat I$, $\hat B' = \hat B - \langle \hat B \rangle_{\psi} \hat I$. Then, using the Schwarz inequality,
	\[
		(\Delta_{\psi}A)^2(\Delta_{\psi}B)^2 \geq |(\hat A' \psi, \hat B' \psi)|^2 = |(\psi, \hat A' \hat B' \psi)|^2
	.\]
	Define $\{\hat A', \hat B'\} = \hat A' \hat B' + \hat B' \hat A'$, the anti-commutator\index{anti-commutator}. Then this is symmetric, so we can write
	\[
		\hat A' \hat B' = \frac{1}{2} ([\hat A', \hat B'] + \{\hat A', \hat B'\})
	.\]
	Note under the Hermitian conjugate, $[\hat A', \hat B']$ flips sign, whereas $\{\hat A', \hat B'\}$ stays the same. Hence $(\psi, [\hat A', \hat B']\psi)$ is purely imaginary and $(\psi, \{\hat A', \hat B'\}\psi)$ is real. So,
	\[
		(\Delta_{\psi} A)^2 (\Delta_{\psi}B)^2 = \frac{1}{4} (|(\psi, [\hat A', \hat B']\psi)|^2 + |(\psi, \{\hat A', \hat B'\}\psi)|^2)
	.\]
	Dropping the anti-commutator term and taking the square roots gives our result.
\end{proofbox}

\subsubsection{Consequences of the Uncertainty Theorem}%
\label{subsub:consequences_of_the_uncertainty_theorem}

We have shown that $[ \hat A, \hat B] = 0 \iff$ we can simultaneously diagonalize $A$ and $B$. But this is if and only if $A$ and $B$ can be measured simultaneously to arbitrary precision on a given state.

Moreover, if we take $\hat A = \hat x$, $\hat B = \hat p$, then what we recover is \textit{Heisenberg's Uncertainty principle}\index{Heisengberg's uncertainty principle}
\[
	(\Delta_{\psi} x)(\Delta_{\psi} p) \geq \frac{\hbar}{2}
.\]
Note if $\psi = \psi_{GP}$, then at $t = 0$, we have equality in the above. This is due to the following two lemmas:

\begin{lemma}
	$\psi$ is a state of minimal uncertainty if and only if $\hat x \psi = ia \hat p \psi$, for $a \in \mathbb{R}$.
\end{lemma}

From this, we can deduce the following:

\begin{lemma}
	Minimal uncertainty occurs if and only if
	\[
		\psi(x) = C e^{-kx^2}
	,\]
	for $c \in \mathbb{C}$, and $k \in \mathbb{R}^{+}$.
\end{lemma}

\subsection{Ehrenfest Theorem}%
\label{sub:ehrenfest_theorem}

We look at the Ehrenfest theorem, which tells us the evolution of the operators over time.

\begin{theorem}[Ehrenfest theorem]\index{Ehrenfest theorem}
	The expectation value of a Hermitian operator $\hat A$ evolves according to
	\[
		\frac{\diff}{\diff t} \langle \hat A \rangle_{\psi} = \frac{i}{\hbar} \langle [\hat H, \hat A ] \rangle_{\psi} + \langle \frac{\partial \hat A}{\partial t} \rangle_{\psi}
	.\]
\end{theorem}

\begin{proofbox}
	We will integrate over parts:
	\begin{align*}
		\frac{\diff}{\diff t} \langle \hat A \rangle_{\psi} &= \frac{\diff}{\diff t} \int_{-\infty}^{\infty} \psi^{\ast} \hat A \psi \diff x = \int_{-\infty}^{\infty} \frac{\partial}{\partial t} (\psi^{\ast} \hat A \psi) \diff x \\
							   &= \int_{-\infty}^{\infty} \biggl( \frac{\partial \psi^{\ast}}{\partial t} \hat A \psi + \psi^{\ast} \frac{\partial \hat A}{\partial t} \psi + \psi^{\ast} \hat A \frac{\partial \psi}{\partial t} \biggr) \diff x \\
							   &= \frac{i}{\hbar} \int_{-\infty}^{\infty} \psi^{\ast} (\hat H \hat A - \hat A \hat H)\psi \diff x + \langle \frac{\partial \hat A}{\partial t} \rangle_{\psi} \\
							   &= \frac{i}{\hbar} \langle [\hat H, \hat A]\rangle_{\psi} + \langle \frac{\partial \hat A}{\partial t} \rangle_{\psi}
	\end{align*}
\end{proofbox}

\begin{exbox}
	\begin{enumerate}[1.]
		\item If we take $\hat A = \hat H$, then $[\hat H, \hat H] = 0$ and since $\hat H$ doesn't depend on time,
			\[
				\frac{\diff}{\diff t} \langle \hat H \rangle_{\psi} = 0
			.\]
		\item Take $\hat A = \hat p$. Then,
			\begin{align*}
				[\hat H, \hat p]\psi &= \biggl[ \frac{\hat p^2}{2m} + U(\hat x), \hat p \biggr] \psi = [U(\hat x), \hat p] \psi \\
						     &= U(x) \biggl( - i \hbar \frac{\partial}{\partial x}\biggr) \psi(x, t) - \biggl( - i \hbar \frac{\partial}{\partial x}\biggr) (U(x)\psi(x, t)) \\
						     &= - i \hbar U(x) \frac{\partial \psi}{\partial x}(x, t) + i \hbar U(x) \frac{\partial \psi}{\partial x}(x, t) + i \hbar \frac{\partial U}{\partial x}\psi(x, t) \\
						     &= i \hbar \frac{\partial U}{\partial x} \psi(x, t).
			\end{align*}
			Hence we get
			\[
				\frac{\diff \langle \hat p \rangle_{\psi}}{\diff t} = \frac{i}{\hbar} \langle [\hat H, \hat p] \rangle_{\psi} = - \langle \frac{\partial U}{\partial X} \rangle_{\psi}
			.\]
			This is the quantum mechanics equivalent of Newton's second law.
		\item If we let $\hat A = \hat x$, then
			\begin{align*}
				[\hat H, \hat x] &= \biggl[ \frac{\hat p^2}{2m} + U(\hat x), \hat x\biggr] = \frac{1}{2m} [\hat p^2, \hat x] \\
						 &= \frac{1}{2m} (\hat p[\hat p, \hat x] + [\hat p, \hat x]\hat p) \\
						 &= - \frac{i \hbar}{m} \hat p.
			\end{align*}
			So this gives
			\[
				\frac{\diff \langle \hat x \rangle_{\psi}}{\diff t}= \frac{i}{t} \langle [\hat H, \hat x[ \rangle_{\psi} = \frac{\langle \hat p \rangle_{\psi}}{m}
			,\]
			which is the quantum mechanics reformulation of $p = mv$, i.e. momentum is velocity times mass.
	\end{enumerate}
\end{exbox}

\subsection{Harmonic Oscillator Revisited}%
\label{sub:harmonic_oscillator_revisited}

Remember the Hamiltonian of the harmonic oscillator is
\[
\hat H = \frac{\hat p^2}{2m} + \frac{1}{2} m \omega^2 \hat x^2
,\]
where $k = m \omega^2$ is the elastic function. To find the eigenvalues and eigenfunctions of $\hat H$, we rewrite
\begin{align*}
	\hat H &= \frac{1}{2m} (\hat p + i m \omega \hat x)(\hat p - i m \omega \hat x) + \frac{i \omega}{2} [\hat p, \hat x] \\
	&= \frac{1}{2m}(\hat p + i m \omega \hat x)(\hat p - i m \omega \hat x) + \frac{\hbar \omega}{2} \hat I
\end{align*}

\begin{definition}
	We define the linear operators
	\[
		\hat a = \frac{1}{\sqrt{2m}}(\hat p - i m \omega \hat x), \qquad \hat a^{\dagger} = \frac{1}{\sqrt{2m}} (\hat p + i m \omega \hat x)
	.\]
	Then we can rewrite the Hamiltonian as
	\[
	\hat H = \hat a^{\dagger}\hat a + \frac{\hbar \omega}{2} \hat I
	.\]
\end{definition}

We can compute the value
\begin{align*}
	[\hat a, \hat a^{\dagger}] &= \frac{1}{2m} [\hat p - i m \omega \hat x, \hat p + i m \omega \hat x] \\
				   &= - \frac{i m \omega}{2m} [\hat x, \hat p] + \frac{i m \omega}{2m}[\hat p, \hat x] \\
				   &= \hbar \omega \hat I.
\end{align*}

This lets us compute
\begin{align*}
	[\hat H, \hat a] &= [\hat a^{\dagger} \hat a, \hat a] = - \hbar \omega \hat a, \\
	[\hat H, \hat a^{\dagger}] &= \hbar \omega \hat a^{\dagger}.
\end{align*}

Suppose $\chi$ is an eigenfunction of $\hat H$ with eigenvalue $E$. We consider the energy of $(\hat a \chi)$:
\begin{align*}
	\hat H(\hat a \chi) &= [\hat H, \hat a]\chi + \hat a \hat H \chi = - \hbar \omega \hat a \chi + E \hat a \chi \\
			    &= (E - \hbar \omega)\hat a \chi,
\end{align*}
so $(\hat a \chi)$ is an eigenfunction of $\hat H$ with eigenvalue $(E - \hbar \omega)$, and similarly $(\hat a^{\dagger} \chi)$ is an eigenfunction of $\hat H$ with eigenvalue $(E + \hbar \omega)$.

By induction, we can prove that $(\hat a^{n} \chi)$ is an eigenfunction with eigenvalue $(E - n \hbar \omega)$, and $( \hat{a}^{\dagger n} \chi)$ is an eigenfunction with eigenvalue $(E + n \hbar \omega)$.

Using the fact that the energies are non-zero, there exist an eigenfunction $\chi_0$ such that $\hat a \chi_0 = 0$. Finding $\chi_0$,
\begin{align*}
	\frac{1}{\sqrt{2m}} (\hat p - i m \omega \hat x) \chi_0 = 0, \\
	- i \hbar \frac{\partial \chi_0}{\partial x} - i m \omega x \chi_0 = 0, \\
	\chi_0(x) = C e^{-m\omega x^2/2\hbar},
\end{align*}
which is a Gaussian. The excited states with $E > E_0$ are the given by
\begin{align*}
	\chi_n &= (a^{\dagger})^{n} \chi_0 = \frac{1}{(\sqrt{2m})^{n}} (\hat p + i m \omega \hat x)^{n} \chi_0 \\
	       &= \frac{C}{(\sqrt{2m})^{n}}\biggl(- i \hbar \frac{\partial}{\partial x} + i m \omega x \biggr)^{n} e^{-m \omega^2/2\hbar},
\end{align*}
with eigenvalues
\[
	E_n = \frac{\hbar \omega}{2} + n \hbar \omega = \biggl(n + \frac{1}{2} \biggr)\hbar \omega
.\]

\newpage

\section{3D Solutions of the Schr\"{o}dinger Equation}%
\label{sec:3d_solutions_of_the_schr"_o_dinger_equation}

\subsection{TISE for Spherically Symmetric Potentials}%
\label{sub:tise_for_spherically_symmetric_potentials}

In three dimensions, the TISE is
\[
	- \frac{\hbar^2}{2m} \nabla^2 \chi(\mathbf{x}) + U(\mathbf{x}) \chi(\mathbf{x}) = E \chi(\mathbf{x})
.\]
Here, we have replaced the double derivative with the Laplacian operator $\nabla^2$.
\begin{itemize}
	\item In Cartesian coordinates $(x, y, z)$, the Laplacian\index{Laplacian} is
		\[
		\nabla^2 = \frac{\partial^2}{\partial x^2} + \frac{\partial^2}{\partial y^2} + \frac{\partial^2}{\partial z^2}
		.\]
	\item In spherical coordinates $(r, \theta, \phi)$, we have
		\[
			\nabla^2 = \frac{1}{r} \frac{\partial^2}{\partial r^2} r + \frac{1}{r^2 \sin^2 \theta} \biggl[ \sin \theta \frac{\partial}{\partial \theta} \biggl( \sin \theta \frac{\partial}{\partial \theta} \biggr) + \frac{\partial^2}{\partial \phi^2} \biggr]
		.\]
\end{itemize}

\begin{definition}
	A spherically symmetric potential\index{spherically symmetric potential} satisfies
	\[
		U(\mathbf{x}) = U(r, \theta, \phi) = U(r)
	.\]
\end{definition}

We start by looking at even, spherically symmetric stationary states $\chi(r, \theta, \phi) = \chi(r)$. The TISE becomes
\[
	- \frac{\hbar^2}{2mr} \frac{\Diff2}{\diff r^2} (r\chi(r)) + U(r) \chi(r) = E \chi(r)
,\]
which we can write as
\[
	-\frac{\hbar^2}{2m} \biggl( \frac{\Diff2 \chi}{\diff r^2} + \frac{2}{r} \frac{\diff \chi}{\diff r} \biggr) + U(r) \chi = E \chi
.\]
The normalisation condition for $\chi$ says
\[
	\int_{\mathbb{R}^3}|\chi(r, \theta, \phi)|^2 \diff V < \infty \iff \int_{0}^{\infty} |\chi(r)|^2r^2 \diff r < \infty
.\]
Hence the eigenfunction $\chi(r)$ must go to $0$ sufficiently fast as $r \to \infty$, and behave well at $r \to 0$.

We can solve the TISE by defining $\sigma(r) = r \chi(r)$. Then, in terms of $\sigma$, the TISE becomes
\[
	- \frac{\hbar^2}{2m} \frac{\Diff2 \sigma}{\diff r^2} + U(r) \sigma = E \sigma
.\]
This is just the one-dimensional TISE, defined only on $\mathbb{R}^{+}$, and with the usual normalisation conditions
\[
	\int_{0}^{\infty}|\sigma(r)|^2\diff r < \infty
.\]
If $\chi$ is to be defined, we must have $\sigma(0) = 0$ and $\sigma'(0)$ finite. Indeed, if $\sigma(r) \sim a \neq 0$ as $r \to 0$, then $\hat H$ is not Hermitian.

\begin{proofbox}
	For $\hat H$ to be Hermitian, we need $(\phi, \hat H \chi) = (\hat H \phi, \chi)$. This is
	\begin{align*}
		(\phi, \hat H \chi) &= \int_{0}^{\infty} r^2 \phi(r) \hat H \chi(r) \diff r = - \frac{\hbar^2}{2m} \int_{0}^{\infty} \phi \frac{\diff}{\diff r} \biggl( r^2 \frac{\diff \chi}{\diff r} \biggr) \diff r \\
				    &= - \frac{\hbar^2}{2m} \biggl[ r^2 \phi \frac{\diff \chi}{\diff r} - r^2 \chi \frac{\diff \phi}{\diff r} \biggr] - \frac{\hbar^2}{2m} \int_{0}^{\infty} \frac{\diff}{\diff r} \biggl(r^2 \frac{\diff \phi}{\diff r} \biggr) \chi \diff r.
	\end{align*}
	Note the last term is simply $(\hat H \phi, \chi)$. Now if $\phi(r) \sim B \neq 0$ as $r \to 0$, then $\chi(r) \sim \frac{A}{r}$ with $A \neq 0$. So,
	\[
	r^2 \phi \frac{\diff \chi}{\diff r} - r^2 \chi \frac{\diff \phi}{\diff r} \not \to 0
	,\]
	as $r \to 0$.
\end{proofbox}

Then, we can solve over the entirety of $\mathbb{R}$ by letting $U(-r) = U(r)$, and look for odd solutions to the TISE.

\begin{exbox}
	Take the spherically symmetric potential well
	\[
		U(r) =
		\begin{cases}
			0 & r \leq a, \\
			U_0 & r > a,
		\end{cases}
	\]
	where $a, U_0$ are positive. We solve for odd $\sigma$:
	\[
		-\frac{\hbar^2}{2m} \frac{\Diff2 \sigma}{\diff r^2} + U(r) \sigma = E \sigma
	.\]
	We look for odd parity bound states with $0 \leq E \leq U_0$. Then, letting
	\[
		k = \sqrt{\frac{2mE}{\hbar^2}}, \qquad \bar k = \sqrt{\frac{2m(U_0 - E)}{\hbar^2}}
	,\]
	we find odd solutions
	\[
		\sigma(r) =
		\begin{cases}
			A \sin (kr) & |r| \leq a, \\
			B e^{-\bar k r} & r > a, \\
			-B e^{\bar k r} & r < -a.
		\end{cases}
	\]
	The boundary conditions imply continuity of $\sigma(r)$ and $\sigma'(r)$ at $r = a$, so
	\[
	\begin{cases}
		A \sin ka = B e^{-\bar k a}, \\
		k A \cos ka = -\bar k B e^{-\bar k a}.
	\end{cases}
	\]
	This gives two equations
	\[
		- k \cot (ka) = \bar k, \qquad k^2 + \bar k^2 = \frac{2mU_0}{\hbar^2}
	.\]
	Substituting $\zeta = ka$, $\eta = \bar k a$, we get $\eta = - \zeta \cot \zeta$ and $\eta^2 + \zeta^2 = r_0^2$. (insert picture)

	Now if $r_0 < \frac{\pi}{2}$, then there are no solutions.
\end{exbox}

Compared to the same equation in one-dimension, there are two differences:
\begin{enumerate}[1.]
	\item Below a given threshold for $U_0$, there are no bound states in three dimensions.
	\item Our solution looks like
		\[
			\chi(r) =
			\begin{cases}
				a \frac{\sin (kr)}{r} & r < a, \\
				B \frac{e^{-\bar k r}}{r} & r \geq a.
			\end{cases}
		\]
\end{enumerate}

\subsection{Angular Momentum}%
\label{sub:angular_momentum}

In classical mechanics, we have an important quantity $\mathbf{L} = \mathbf{x} \times \mathbf{p}$, the \textit{angular momentum}\index{angular momentum}. This is conserved for spherically symmetric potentials, as
\[
\frac{\diff \mathbf{L}}{\diff t} = \mathbf{\dot x} \times \mathbf{p} + \mathbf{x} \times \mathbf{\dot p} = 0
.\]
This has important corollaries: in dynamics and relativity, we saw the conservation of angular momentum could take a three-dimensional problem (the two-body problem), to a one-dimensional problem.

\begin{definition}
	The \textit{angular momentum operator}\index{angular momentum operator} is defined by
	\[
	\mathbf{\hat L} = \mathbf{\hat x} \times \mathbf{\hat p} = -i\hbar \mathbf{x} \times \nabla
	.\]
	In Cartesian coordinates, this can be expressed as
	\[
	\hat L_i = \eps_{ijk} \hat x_j \hat p_k = -i\hbar \eps_{ijk}x_j \frac{\partial}{\partial x_k}
	.\]
\end{definition}

Then $\hat L_i$ satisfies the following properties:
\begin{itemize}
	\item $\hat L_i$ is Hermitian.
	\item $[\hat L_i, \hat L_j] \neq 0$ for $i \neq j$. Hence, different components of $\mathbf{L}$ cannot be determined simultaneously. In fact, we can prove
		\[
			[\hat L_i, \hat L_j] = i \hbar \eps_{ijk} \hat L_k
		.\]
\end{itemize}

\begin{proofbox}
	\begin{align*}
		[\hat L_1, \hat L_2]\chi(\mathbf{x}) &= - \hbar^2 \biggl[ \biggl( x_2 \frac{\partial}{\partial x_3} - x_3 \frac{\partial}{\partial x_2} \biggr) \biggl( x_3 \frac{\partial}{\partial x_1} - x_1 \frac{\partial}{\partial x_3} \biggr) \\
						     & \quad - \biggl( x_3 \frac{\partial}{\partial x_1} - x_1 \frac{\partial}{\partial x_3} \biggr) \biggl( x_2 \frac{\partial}{\partial x_3} - x_3 \frac{\partial}{\partial x_2} \biggr) \biggr] \chi(\mathbf{x}) \\
						     &= - \hbar^2 \biggl(x_2 \frac{\partial}{\partial x_1} - x_1 \frac{\partial}{\partial x_2} \biggr) \chi(\mathbf{x}) \\
						     &= i \hbar \hat L_3 \chi(\mathbf{x}).
	\end{align*}
\end{proofbox}

\begin{definition}
	We define the \textit{total angular momentum operator}\index{total angular momentum operator} $\hat L^2$ as
	\[
	\hat L^2 = \hat L_1^2 + \hat L_2^2 + \hat L_3^2
	.\]
\end{definition}

The total angular momentum satisfies the following properties:
\begin{itemize}
	\item $[\hat L^2, \hat L_1] = 0$.
	\item For $U(r)$, $[\hat L^2, \hat H] = [\hat L_i, \hat H] = 0$.
\end{itemize}

\begin{proofbox}
	\begin{align*}
		[\hat L_i, \hat x_j] &= [\eps_{imn}\hat x_m \hat p_n, \hat x_j] = \eps_{imn} [\hat x_m \hat p_n, \hat x_j] \\
				     &= \eps_{imn}(\hat x_m \underbrace{[\hat p_n, \hat x_j]}_{-i\hbar\delta_{nj}} + \underbrace{[\hat x_m, \hat x_j]}_{0} \hat p_n) \\
				     &= - i \hbar \eps_{imj} \hat x_m = i \hbar \eps_{ijm}\hat x_m, \\
		[\hat L_i, \hat x_j^2] &= [\hat L_i, \hat x_j] \hat x_j + \hat x_j [\hat L_i, \hat x_j] \\
				       &= i \hbar \eps_{ijm}(\hat x_m \hat x_j + \hat x_j \hat x_m) = 0.
	\end{align*}
	This prove that $[\hat L_i, U(r)] = 0$, as $U(r)$ is simply $U(\sqrt{\hat x_1^2 + \hat x_2^2 + \hat x_3^2})$. Similarly, we can prove
	\begin{align*}
		[\hat L_i, \hat p_j] &= i \hbar \eps_{ijm} \hat p_m, \\
		[\hat L-i, \hat p^2] &= 0, \\
		\implies [\hat L_i, \hat H] &= 0, \\
		[\hat L^2, \hat H] &= 0.
	\end{align*}
\end{proofbox}

We have the following observations:
\begin{enumerate}[1.]
	\item We can find joint eigenstates of the three operators $\{\hat H, \hat L^2, \hat L_i\}$ which form a basis of $\mathcal{H}$.
	\item The eigenvalues of the three operators can be simultaneously measured at an arbitrary precision.
	\item This set of operators is \textit{maximal}\index{maximal}, meaning we cannot construct another operator (other than $\hat I$) that commutes with all three.
\end{enumerate}

To find joint eigenfunction of, say $\hat L^2$ and $\hat L_3$, we can write $\mathbf{\hat L}$ is spherical coordinates:
\begin{align*}
	\hat L_3 &= - i \hbar \frac{\partial}{\partial \phi}, \\
	\hat L^2 &= - \frac{\hbar^2}{\sin^2 \theta} \bigg[ \sin \theta \frac{\partial}{\partial \theta} \biggl( \sin \theta \frac{\partial}{\partial \theta} \biggr) + \frac{\partial^2}{\partial \phi^2} \biggr].
\end{align*}
Now, we look for joint eigenfunctions $Y(\theta, \phi) = y(\theta) X(\phi)$. Then, as it is an eigenfunction of $\hat L_3$,
\begin{align*}
	- i \hbar \biggl( \frac{\partial}{\partial \phi} X(\phi) \biggr) y(\theta) &= \hbar m X(\phi) y (\theta), \\
	\implies X(\phi) &= e^{i m \phi}.
\end{align*}
As the wave-function is single-valued in $\mathbb{R}^3$, $X(\phi)$ must be invariant under $\phi \to \phi + 2\pi$. Hence, $m \in \mathbb{Z}$. Now, as $y(\theta)e^{im\phi}$ is an eigenfunction of $\hat L^2$, we find
\begin{align*}
	\frac{1}{\sin \theta}\frac{\partial}{\partial \theta} \biggl( \sin \theta \frac{\partial y}{\partial \theta} \biggr) - \frac{m^2}{\sin^2 \theta}y = - \frac{\lambda}{\hbar^2} y.
\end{align*}
This is the associated Legendre equation, and it has solution
\[
	y(\theta) = P_{l, m}(\cos \theta) = (\sin \theta)^{|m|}\frac{\Diff{|m|}}{\diff (\cos \theta)^{|m|}} P_l(\cos \theta)
.\]
Because $P_l(\cos \theta)$ is a polynomial is $\cos \theta$ of degree $l$, we need $-l \leq m \leq l$. This gives eigenvalues $\lambda = \hbar^2 l(l+1)$.

Combining these, the joint eigenfunctions are
\[
	Y_{l,m}(\theta, \phi) = P_{l, m}(\cos \theta) e^{i m \phi}
.\]
Then the eigenvalue of $\hat L^2$ is $\hbar^2 l(l+1)$, and the eigenvalue of $\hat L_3$ is $m \hbar$.

Here, $l$ and $m$ are quantum numbers that characterise the total angular momentum (as $l$), and the azimuthal  number, which is the $z$-component of angular momentum (as $m$). We write out the first few spherical harmonics:
\begin{align*}
	Y_{0,0}(\theta, \phi) &= \frac{1}{\sqrt{4\pi}}, & l = 0,& m = 0, \\
	Y_{1,0}(\theta, \phi) &= \sqrt{\frac{3}{4\pi}} \cos \theta, & l = 1,& m = 0, \\
	Y_{1, \pm 1}(\theta, \phi) &= \mp \sqrt{\frac{3}{8\pi}} \sin \theta e^{\hbar i \phi}, & l = 1,&m = \pm 1.
\end{align*}
All the spherical harmonics are orthonormal.

\subsection{The Hydrogen Atom}%
\label{sub:the_hydrogen_atom}

Consider a nucleus of a hydrogen atom, consisting of a proton $p$ with charge $+e$, and electron $e^{-1}$ with charge $-e$, at a radius of $r$ from the proton.

We model the atom by letting the proton be stationary at the origin. The force provided by the proton (the Coulomb force) is given by
\begin{align*}
	F_{coulomb}(r) &= - \frac{e^2}{4 \pi \eps_0} \frac{1}{r^2} = - \frac{\partial U_{coulomb}}{\partial r}, \\
	U_{coulomb}(r) &= - \frac{e^2}{4 \pi \eps_0} \frac{1}{r}.
\end{align*}
This force, and hence the potential, is spherically symmetric. The Schr\"{o}dinger equation is
\[
	- \frac{\hbar^2}{2m_e} \nabla^2\chi(r, \theta, \phi) - \frac{e^2}{4 \pi \eps_0}\frac{1}{r} \chi(r, \theta, \phi) = E \chi(r, \theta, \phi)
.\]
In spherical coordinates, we can write
\begin{align*}
	- \hbar^2 \nabla^2 &= \frac{-\hbar^2}{r} \frac{\partial^2}{\partial r^2}r - \frac{\hbar^2}{r^2\sin^2 \theta} \biggl( \sin \theta \frac{\partial}{\partial \theta} \sin \theta \frac{\partial}{\partial \theta} + \frac{\partial^2}{\partial \phi^2} \biggr) \\
			   &= - \frac{\hbar^2}{r} \frac{\partial^2}{\partial r^2} r + \frac{\hat L^2}{r^2}.
\end{align*} 
If we put this in the TDSE,
\[
	- \frac{\hbar^2}{2m_e} \frac{1}{r} \biggl( \frac{\partial^2}{\partial r^2} r \chi \biggr) + \frac{\hat L^2}{2m_e r^2} \chi - \frac{e^2}{4 \pi \eps_0 r} \chi = E \chi
.\]
Now, notice that eigenfunctions of $\hat H$ are also eigenfunctions of $\hat L^2$ and $\hat L_3$, so $\chi$ must also be an eigenfunction of $\hat L^2$, $\hat L_3$. Hence, if we write $\chi(r, \theta, \phi) = R(r) Y_{l,m}(\theta, \phi)$, then we get
\[
	- \frac{\hbar^2}{2m_e} \biggl( \frac{\Diff2 R}{\diff r^2} + \frac{2}{r} \frac{\diff R}{\diff r} \biggr) Y + \frac{\hbar^2}{2m_e r^2} l(l+1) + RY - \frac{e^2}{4 \pi \eps_0}RY = ERY
.\]
We can divide by $Y$, and end up with a one-dimensional equation for the radial part:
\[
	- \frac{\hbar^2}{2m} \biggl( \frac{\Diff2 R}{\diff r^2} + \frac{2}{r} \frac{\diff R}{\diff r} \biggr) + \biggl( - \frac{e^2}{4 \pi \eps_0}\frac{1}{r} + \frac{\hbar^2 l(l+1)}{2m_e r^2} \biggr)R = ER
,\]
which is in the form of a TISE.

\subsubsection{Spherically Symmetric Solutions}%
\label{subsub:spherically_symmetric_solutions}

We first look at the solutions where $l = 0$. Then $Y_{0,0}(\theta, \phi) = (4\pi)^{-1/2}$. We solve in terms of the rescaled variables
\[
\nu^2 = -\frac{2mE}{\hbar^2} > 0, \qquad \beta = \frac{e^2m_e}{2\pi\eps_0\hbar^2}
.\]
Then $R$ satisfies
\[
	\frac{\Diff2 R}{\diff r^2} + \frac{2}{r} \frac{\diff R}{\diff r} + \biggl( \frac{\beta}{r} - \nu^2 \biggr)R = 0
.\]
\begin{remark}
	\begin{enumerate}[(i)]
		\item[]
		\item The asymptotic behaviour (as $r\to \infty$) is determined by
			\[
				\frac{\Diff2R}{\diff r^2} - \nu^2 R = 0 \implies R(r) \sim e^{\pm r \nu}
			.\]
			We must have $R(r) \sim e^{-r \nu}$, by normalisability.
		\item At $r = 0$, the eigenfunction has to be finite.
	\end{enumerate}
\end{remark}
By our observation in (i), we write $R(r) = f(r) e^{- \nu r}$. Then, we get that $f$ satisfies
\[
	f''(r) + \frac{2}{r}(1 - \nu r)f'(r) + \frac{1}{r}(\beta - 2 \nu)f(r) = 0
.\]
This is a homogeneous linear ODE with a regular point at $r = 0$, so we may write it as a power series. Let
\[
	f(r) = r^{c} \sum_{n = 0}^{\infty} a_n r^{n}
.\]
Then,
\[
	\sum_{n = 0}^{\infty} \biggl[ a_n (c+n)(c+n-1)r^{c+n-2} + \frac{2}{r} (1 - \nu r) a_n (c+n) r^{c+n-1} + (\beta - 2 \nu) r^{c+n-1}\biggr] = 0
.\]
The lowest power of $r$ has coefficient
\[
	a_0 c(c-1) + 2 a_0 c = 0 \implies c = 0, c = -1
.\]
If $c = -1$, then $\chi \sim Ar^{-1}$ near $r = 0$, hence is not normalisable. So we must have $c = 0$. Then, for $n \geq 1$, we find
\[
	\sum_{n = 1}^{\infty} [a_n n(n+1) + a_{n-1}(\beta - 2 \nu n)]r^{n-2} = 0
,\]
which recursively gives
\[
	a_n = \frac{2 \nu n - \beta}{n(n+1)}a_{n-1}
.\]
Now we would like to say that these terms eventually die off, as in the case for the Hermite polynomials.

\begin{proposition}
	If $f(r) = \sum a_n r^{n}$ is infinite, then $R(r)$ is not normalisable.
\end{proposition}

\begin{proofbox}
	The asymptotic behaviour of $f(r)$ is determined by the ratio of the coefficients, which is
	\[
	\frac{a_n}{a_{n-1}} \to \frac{2 \nu}{n}
	.\]
	This satisfies the same asymptotic behaviour as
	\[
		g(r) = e^{2 \nu r} = \sum_{n = 0}^{\infty} \frac{(2 \nu)^{n}}{n!}r^{n}
	.\]
	Now, if asymptotically $f(r) \sim e^{2 \nu r}$, then $R(r) = f(r) e^{- \nu r} \sim e^{\nu r}$, which is not normalisable. Hence, the series must terminate.
\end{proofbox}

Therefore, there exists $N > 0$ such that $a_N = 0$, and $a_{N-1} \neq 0$. Thus,
\[
2 \nu N - \beta = - \implies \nu = \frac{\beta}{2 N}
.\]
Substituting in $\nu$ and $\beta$, we get that the energies are
\[
E_N = - \frac{e^{4}m_e}{32 \pi^2 \eps_0^2 \hbar^2} \frac{1}{N^2}
,\]
which is identical to the Bohr atom.

To find the eigenfunctions $R_N(r)$, we can substitute $2N \nu = \beta$, to find that
\[
	\frac{a_n}{a_{n-1}} = - 2 \nu \frac{N - n}{n(n+1)}
.\]
This can be used inductively to find $R_N$.

\begin{center}
\begin{tabular}{c|c}
	$N$ & $R_N(r)$ \\
	\hline
	$1$ & $A_1 e^{-\nu r}$ \\
	$2$ & $A_2(1 - \nu r)e^{-\nu r}$ \\
	$3$ & $A_3(1 - 2 \nu r + \frac{2}{3} \nu^2 r^2)e^{-\nu r}$ \\
\end{tabular}
\end{center}

In general, $R_N(r) = L_N(\nu r)e^{-\nu r}$, where $L_N$ are the \textit{Laguerre polynomials}\index{Laguerre polynomials}.

\subsubsection{Asymmetric Solutions}%
\label{subsub:asymmetric_solutions}

We look at the general case, when $l \geq 0$. Then again, we have TISE
\[
	\frac{\Diff2 R}{\diff r^2} + \frac{2}{r}\frac{\diff R}{\diff r} + \biggl( \frac{\beta}{r} - 2 \nu - \frac{l(l+1)}{r^2}\biggr) R = 0
.\]
Looking at the asymptotic behaviour, we again write $R(r) = f(r) e^{- \nu r}$. Then
\[
	f'' + \frac{2}{r}(1 - \nu r)f' + \biggl( \frac{\beta}{r} - 2 \nu - \frac{l(l+1)}{r^2}\biggr)f = 0
.\]
If we write $f$ as
\[
	f(r) = r^{\sigma}\sum_{n = 0}^{\infty}a_nr^{n}
.\]
The lowest power coefficient is $r^{\sigma - 2}$, which has coefficient
\[
	\sigma(\sigma + 1) - l(l+1) = 0 \implies \sigma = -l-1, \sigma = l
.\]
However $\sigma = -l-1$ gives $R \sim r^{-l-1}$ around $r = 0$, which is not integrable, so we must have
\[
	f(r) = r^{l} \sum_{r = 0}^{\infty} a_n r^{n}
.\]
This gives recursive formula
\[
	a_n = \frac{2\nu(n + l) - \beta}{n(n+2l - 1)}a_{n-1}
.\]
As before, this is not normalisable unless there exists $n_{max} > 0$ such that $a_{n_{max}} = 0$, and $a_{n_{max}-1} \neq 0$. Hence
\[
	2\nu\underbrace{(n_{max} + l)}_{N} = \beta = 0 \implies 2 \nu N - \beta = 0 \implies \nu = \frac{\beta}{2N}
.\]

Again the energy values are
\[
E_N = - \frac{e^{4}m_e}{32 \pi^2 \eps_0^2\hbar^2} \frac{1}{N^2}
.\]
However, the \index{degeneracy}degeneracy, which is the number of eigenstates with same eigenvalue, is larger. If $N = n_{max} + l$, then we can take any $0 \leq l \leq N-1$ and any $-l \leq m \leq l$. This gives $N^2$ possible eigenstates.

The eigenfunctions can then be written as
\[
	\chi_{N,l,n}(r, \theta, \phi) = R_{N,l}(r)Y_{l, m}(\theta, \phi) = r^{l}f_{N,l}(r)e^{-\nu r} Y_{l,m}(\theta, \phi)
.\]
Now $f_{N,l}(r)$ is a polynomial of degree $N - l - 1$, with coefficients
\[
	a_k = \frac{2\nu}{k} \frac{k + l - N}{k + 2l + 1}
.\]
These are known as the \textit{generalized Laguerre polynomials}\index{generalized Laguerre polynomials}. The eigenstates are defined by the quantum numbers
\begin{align*}
	N &= 0, 1, 2, \ldots, & &\text{principal quantum numbers}, \\
	l &= 0, \ldots, N-1, & &\text{total angular momentum}, \\
	m &= -l, \ldots, l, & &\text{azimuthal quantum numbers}.
\end{align*}

Hence, while the Bohr model accurately predicted the energies of the eigenstates, the quantization of the angular momentum was not correct, as $L^2 = l(l+1)\hbar^2$, for any $l < N$.
\newpage

\printindex

\end{document}
