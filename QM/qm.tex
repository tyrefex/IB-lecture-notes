\documentclass[12pt]{article}
\usepackage{amsmath}
\usepackage[a4paper]{geometry}
\usepackage{fancyhdr}
\usepackage{tikz}
\usepackage{amssymb}
\usepackage{graphicx}
\usepackage{amsthm}
\usepackage{import}
\usepackage{xifthen}
\usepackage{pdfpages}
\usepackage{transparent}
\usepackage{adjustbox}
\usepackage[shortlabels]{enumitem}
\usepackage{parskip}
\makeatletter
\newcommand{\@minipagerestore}{\setlength{\parskip}{\medskipamount}}
\makeatother
\usepackage{imakeidx}

\DeclareMathOperator{\Ker}{Ker}
\DeclareMathOperator{\Img}{Im}
\DeclareMathOperator{\rank}{rank}
\DeclareMathOperator{\nullity}{null}
\DeclareMathOperator{\spn}{span}
\DeclareMathOperator{\tr}{tr}
\DeclareMathOperator{\adj}{adj}
\DeclareMathOperator{\id}{id}
\DeclareMathOperator{\Sym}{Sym}
\DeclareMathOperator{\Orb}{Orb}
\DeclareMathOperator{\Stab}{Stab}
\DeclareMathOperator{\ccl}{ccl}
\DeclareMathOperator{\Aut}{Aut}
\DeclareMathOperator{\Syl}{Syl}
\DeclareMathOperator{\sgn}{sgn}
\DeclareMathOperator{\Fit}{Fit}
\DeclareMathOperator{\Ann}{Ann}


\newcommand{\incfig}[1]{%
	\def\svgwidth{\columnwidth}
	\import{./figures/}{#1.pdf_tex}
}

\setlength\parindent{0pt}

\newcommand{\course}{QM }
\newcommand{\lecnum}{}

\newtheorem{theorem}{Theorem}[section]
\newtheorem{corollary}{Corollary}[section]
\newtheorem{lemma}{Lemma}[section]
\newtheorem{proposition}{Proposition}[section]

\theoremstyle{definition}
\newtheorem{definition}{Definition}[section]

\theoremstyle{remark}
\newtheorem*{remark}{Remark}

\pagestyle{fancy}
\fancyhf{}
\rhead{\leftmark}
\lhead{Page \thepage}
\setlength{\headheight}{15pt}

\newcommand{\mapsfrom}{\mathrel{\reflectbox{\ensuremath{\mapsto}}}}

\makeindex[intoc]

\usepackage{hyperref}
\hypersetup{
    colorlinks,
    citecolor=black,
    filecolor=black,
    linkcolor=black,
    urlcolor=black
    pdfauthor={Ishan Nath}
}

\begin{document}

\hypersetup{pageanchor=false}
\begin{titlepage}
	\begin{center}
		\vspace*{1em}
		\Huge
		\textbf{IB Quantum Mechanics}

		\vspace{1em}
		\large
		Ishan Nath, Michaelmas 2022

		\vspace{1.5em}

		\Large

		Based on Lectures by Prof. Maria Ubiali

		\vspace{1em}

		\large
		\today
	\end{center}
	
\end{titlepage}
\hypersetup{pageanchor=true}

\tableofcontents

\newpage

\section{Historical Introduction}%
\label{sec:historical_introduction}

\subsection{Particles and Waves in Classical Mechanics}%
\label{sub:particles_and_waves_in_classical_mechanics}

These are the basic concepts of particle mechanics. We begin by looking at particles.

\begin{definition}
	A point particle\index{point particle} is an object carrying energy $E$ and momentum $p$ in an infinitesimally small point of space.
\end{definition}

A particle is defined by its position $\mathbf{x}$ and velocity $\mathbf{v} = \mathbf{\dot v} = \frac{d}{dt} \mathbf{x}$. From Newton's second law, we have
\[
	\mathbf{F}(\mathbf{x}(t), \mathbf{\dot x}(t)) = m \mathbf{\ddot x}(t)
.\]
Solving this determines $\mathbf{x}(t), \mathbf{\dot x}(t)$ for all $t$ once the initial conditions $\mathbf{x}(t_0), \mathbf{\dot x}(t_0)$ are known.

\begin{definition}
	A wave\index{wave} is any real or complex-valued function with periodicity in time or space.
\end{definition}

If we take a function of time $t$, such that $f(t + T) = f(t)$, where $T$ is the period, then $\nu = 1/T$ is the frequency, and the angular frequency is $\omega = 2 \pi \nu = 2 \pi/T$. Examples of such functions are $f(t) = \sin \omega t, \cos \omega t, e^{i \omega t}$.

If we take a function of space $x$, such that $f(x + \lambda) = f(x)$, where $\lambda$ is the wave length, then $k = 2 \pi/\lambda$ is the wave number. Some examples are $f(x) = \cos \omega x, \sin \omega x, e^{i \omega x}$.

In one dimension, an EM wave obeys the equation\index{wave equation}
\[
	\frac{\partial^2 f(x, t)}{\partial t^2} - c^2 \frac{\partial^2 f(x, t)}{\partial x^2} = 0
,\]
where $c \in \mathbb{R}$. This has solutions
\[
	f_{\pm}(x, t) = A_{\pm} \exp(\pm ikx - i\omega t)
,\]
provided that the wavelength and frequency are related by $\omega = ck$ or $\lambda \nu = c$. Here $A_{\pm}$ is the amplitude of the wave, and $\omega = ck$ is the dispersion relation\index{dispersion relation}.

In three dimensions, an EM wave obeys the equation
\[
	\frac{\partial^2 f(\mathbf{x}, t)}{\partial t^2} - c^2 \nabla^2 f(\mathbf{x}, t) = 0
.\]
Here we need $f(x, t_0)$ and $\frac{df}{dt} (x, t_0)$ to determine a unique solution. The periodic solutions are
\[
	f(\mathbf{x}, t) = A \exp(i \mathbf{k} \cdot \mathbf{x} - i \omega t)
,\]
where $\omega = c |\mathbf{k}|$.

\begin{remark}
	\begin{itemize}[(i)]
		\item[]
		\item Other kind of waves arise as solution of other governing equations provided a different dispersion relation.
		\item If the governing equation is linear, the superposition principle holds, stating if $f_1, f_2$ are solutions, then $f = f_1 + f_2$ is a solution.
	\end{itemize}
	
\end{remark}

\subsection{Particle-like behaviour of waves}%
\label{sub:particle_like_behaviour_of_waves}

\subsubsection{Black-body radiation}%
\label{subsub:black_body_radiation}

When a body is heated at temperature $T$, it radiates light at different frequencies. The classical prediction is that $E = k_B T$, where $E$ is the energy of the wave and $k_B$ is the Boltzmann constant. This gives
\[
	I(\omega) \propto k_B T \frac{\omega^2}{\pi^2 c^3}
.\]
This diverges as $\omega \to \infty$. Planck's model stated
\[
	I(\omega) \propto \frac{\omega^2}{\pi^2 c^3} \frac{\hbar \omega}{\exp(\hbar \omega / k_B T) - 1}
.\]
Here $\hbar - h/2\pi$ is the reduced Planck constant, with $h \approx 6.6 \cdot 10^{-34} \text{Joule} \times \text{sec}$. This only makes sense if $E = \hbar \omega$.


\newpage

\printindex

\end{document}
