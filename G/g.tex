%%% TO DO %%%

\documentclass[12pt]{article}

\usepackage{ishn}
\usetikzlibrary{decorations.markings}
\usetikzlibrary{patterns}
\usetikzlibrary{patterns.meta}
\DeclareMathOperator{\hash}{\sharp}% connected sum
\usepackage{caption}
\usepackage{subcaption}

\makeindex[intoc]

\begin{document}

\tikzset{middlearrow/.style={
        decoration={markings,
            mark= at position 0.5 with {\arrow{#1}} ,
        },
        postaction={decorate}
    }
}

\hypersetup{pageanchor=false}
\begin{titlepage}
	\begin{center}
		\vspace*{1em}
		\Huge
		\textbf{IB Geometry}

		\vspace{1em}
		\large
		Ishan Nath, Lent 2023

		\vspace{1.5em}

		\Large

		Based on Lectures by Prof. Gabriel Paternain

		\vspace{1em}

		\large
		\today
	\end{center}
	
\end{titlepage}
\hypersetup{pageanchor=true}

\tableofcontents

\newpage

\section{Surfaces}
\label{sec:surfaces}

\begin{definition}
	A \emph{topological surface}\index{topological surface} is a topological space $\Sigma$ such that
	\begin{enumerate}[(a)]
		\item for all $p \in \Sigma$, there is an open neighbourhood $p \in U \subset \Sigma$ such that $U$ is homeomorphic to $\mathbb{R}^2$, or a disc $D^2 \subset \mathbb{R}^2$, with its usual Euclidean topology.
		\item $\Sigma$ is Hausdorff and second countable.
	\end{enumerate}
\end{definition}

\begin{remark}
	\begin{enumerate}
		$\mathbb{R}^2 \simeq D(0,1) = \{x \in \mathbb{R}^2 \mid \|x\| \leq 1\}$.
		\item A space $X$ is \emph{Hausdorff}\index{Hausdorff} if for $p \neq q$ in $X$, there exist disjoint open sets $U, V$ with $p \in U, q \in V$.

			A space is \emph{second countable}\index{second countable} if it has a countable base, i.e there exist open sets $\{U_i\}_{i \in \mathbb{N}}$, such that every open set is a union of some of the $U_i$.

			The key point of defining surfaces is point (a), point (b) is for ruling out surfaces that are too weird.
		\item If $X$ is Hausdorff or second countable, then so are subspaces of $X$. Moreover Euclidean space has these properties (to show it is second countable, consider open balls $B(c, r)$ with $c \in \mathbb{Q}^{n} \subset \mathbb{R}^{n}$, and $r \in \mathbb{Q}_{+} \subset \mathbb{R}_{+}$).
	\end{enumerate}
\end{remark}

\begin{exbox}
	\begin{enumerate}[(i)]
		\item The plane $\mathbb{R}^2$.
		\item Any open set in $\mathbb{R}^2$ is a surface, i.e. $R^2 \setminus Z$ where $Z$ is closed is a surface.
		\item Graphs of functions. Suppose $f : \mathbb{R}^2 \to \mathbb{R}$ is continuous. Then the graph of $f$ is
			\[
				\Gamma_f = \{(x, y, f(x, y)) \mid (x, y) \in \mathbb{R}^2\}
			.\]
			This is a subspace of $\mathbb{R}^3$, so we can endow it with the subspace topology. We claim it is a subspace homeomorphic to $\mathbb{R}^2$.
	\end{enumerate}
	Recall that if $X, Y$ are topological spaces, then the product topology $X \times Y$ has a basis of open sets $U \times V$, where $U \subset X$, $V \subset Y$ are open

	A feature is that if $g : Z \to X \times Y$ is continuous if and only if $\Pi_x \circ g : Z \to X$ and $\Pi_y \circ g : Z \to Y$ are continuous, where $\Pi_x$, $\Pi_y$ are the canonical projectors.
	\begin{enumerate}[resume*]
		\item[]We can now show that if $f : X \to Y$ is continuous, then $\Gamma_f \subset X \times Y$ is homeomorphic to $X$, as $s(x) = (x,f(x))$ is a continuous function from $X$ to $\Gamma_f$, $\Pi_x|_{\Gamma_f}$ and $s$ are inverse homeomorphisms.

		In particular, for our example $\Gamma_f \simeq \mathbb{R}^2$. So any $f : \mathbb{R}^2 \to \mathbb{R}$ continuous produces a surface $\Gamma_f$.
	\item The sphere: $S^2 = \{(x, y, z) \in \mathbb{R}^3 \mid x^2 + y^2 + z^2 = 1\}$ (with the subspace topology). To show this is a surface, we can consider the stereographic projection $\Pi_{+}: S^2 \setminus\{(0,0,1)\} \to \mathbb{R}^2$:
			\[
				(x, y, z) \mapsto \biggl( \frac{x}{1-z}, \frac{y}{1-z} \biggr)
			.\]
	Then $\Pi_{+}$ is continuous and has an inverse
	\[
		(u, v) \mapsto \biggl( \frac{2u}{u^2 + v^2 + 1}, \frac{2v}{u^2 + v^2 + 1}, \frac{u^2 + v^2 - 1}{u^2 + v^2 + 1} \biggr)
	.\]
	So $\Pi_{+}$ is a continuous bijection with continuous inverse, and hence a homeomorphism.

	Similarly, taking a stereographic projection from the south pole $\Pi_{-} : S^2 \setminus \{(0, 0, -1)\} \to \mathbb{R}^2$, by
	\[
		(x, y, z) \mapsto \biggl( \frac{x}{1+z}, \frac{y}{1+z} \biggr)
	\]
	is another homeomorphism. Hence $S^2$ is a topological surface, as the open sets $S^2 \setminus \{(0, 0, 1)\}$ and $S^2 \setminus \{(0, 0, -1)\}$ cover $S^2$, and it is Hausdorff and second countable as it is a subspace of $\mathbb{R}^3$.
\item The \emph{real projective plane}\index{real projective plane}. The group $\mathbb{Z}_2$ acts on $S^{2}$ by homeomorphisms, via the antipodal map
	\begin{align*}
		a : S^2 &\to S^2 \\
		a(x, y, z) &\mapsto (-x, -y, -z)
	\end{align*}
	\end{enumerate}
\end{exbox}
\begin{definition}
	The real projective plane is the quotient of $S^2$ by identifying every point with its antipodal image:
	\[
	\mathbb{RP}^2 = S^2/\mathbb{Z}_2 = S^2/\sim
	.\]
\end{definition}

\begin{lemma}
	As a set, $\mathbb{RP}^2$ is naturally in bijection with the set of straight lines through $0$.
\end{lemma}

This is because any straight line through $0 \in \mathbb{R}^3$ intersects $S^2$ in exactly a pair of antipodal points, and each such pair determines a straight line.

\begin{lemma}
	$\mathbb{RP}^2$ is a topological surface with the quotient topology.
\end{lemma}

Recall the quotient topology: given the quotient map $q : X \to Y$, we say $V \subset Y$ is open if and only if $q^{-1}(V) \subset X$ is open in $X$.

\begin{proofbox}
	First we show that $\mathbb{RP}^2$ is Hausdorff. If $[p] \neq [q] \in \mathbb{RP}^2$, then $\pm p$, $\pm q$ are distinct, antipodal pairs.

	We take open discs centred on $p$ and $q$ and their antipodal images, such that no two discs intersect. The images of these discs give open images of $[p]$ and $[q]$ in $\mathbb{RP}^2$. Indeed, $q(B_{\delta}(p))$ is open since $q^{-1}(q(B_{\delta}(p))) = B_{\delta}(p) \cup (-B_{\delta}(p))$.

	Now we show $\mathbb{RP}^2$ is second countable. Let $U$ be a countable base of $S^2$, and let $\overline{U} = \{q(u) \mid u \in U\}$. Then $q(u)$ is open, as $q(u) = u \cup (-u)$, and $\overline{U}$ is clearly countable as $U$ is.

	Take $V \subset \mathbb{RP}^2$ open. By definition, $q^{-1}(V)$ is open, so let $q^{-1}(V) = \bigcup U_{\alpha}$, for $U_{\alpha} \in U$. Then
	\[
	V = q(q^{-1}(V)) = q\biggl( \bigcup_{\alpha} U_{\alpha}\biggr) = \bigcup_{\alpha}q(U_{\alpha})
	.\]

	Finally, let $p \in S^2$ and $[p] \in \mathbb{RP}^2$ be its image. Let $\overline{D}$ be a small closed disc neighbourhood of $p \in S^2$, so that $q|_{\overline{D}}$ is injective and continuous, and has image a Hausdorff space.

	Now recall that a countinuous bijection from a compact space to a Hausdorff space is a homeomorphism.

	So $q|_{\overline{D}} : \overline{D} \to q(\overline{D)}$ is a homeomorphism. This induces a homeomorphism
	\[
	q|_D : D \to q(D) \subset \mathbb{RP}^2
	,\]
	where $D$ is an open disc contained in $\overline{D}$. So $[p] \in q(D)$ has an open neighbourhood in $\mathbb{RP}^2$ homeomorphic to an open disc.
\end{proofbox}

\begin{exbox}
	We continue looking at examples of surfaces.
	\begin{enumerate}[(i)]
		\setcounter{enumi}{5}
		\item Let $S^{1}=  \{z \in \mathbb{C} \mid |z| = 1\}$. Then the \emph{torus}\index{torus} is $S^{1} \times S^{1}$ with the subspace topology of $\mathbb{C}^2$ (this is the same as taking the product topology).
	\end{enumerate}
\end{exbox}

\begin{lemma}
	The torus is a topological surface.
\end{lemma}

\begin{proofbox}
	We consider the map
	\begin{align*}
		\mathbb{R}^2 &\overset{e}{\to} S^{1} \times S^{1} \subset \mathbb{C} \times \mathbb{C} \\
		(s, t) & \mapsto (e^{2\pi i s}, e^{2 \pi i t}).
	\end{align*}
	We can view this map using the following diagram:
	\begin{center}
		\begin{tikzcd}[column sep = small]
		\mathbb{R}^2 \arrow[r, "e"] \arrow[d,"q"] & S^{1} \times S^{1} \\
		\mathbb{R}^2/\mathbb{Z}^2 \arrow[ru, "{\hat e}", dashed] &
	\end{tikzcd}
	\end{center}
	There is an equivalence relation on $\mathbb{R}^2$ given by translating by $\mathbb{Z}^2$. Now consider the map
	\[
		[0,1]^2 \injto \mathbb{R}^2 \overset{q}{\to} \mathbb{R}^2/\mathbb{Z}^2
	\]
	is onto, so $\mathbb{R}^2/\mathbb{Z}^2$ is compact. Now note that $\hat e$ is a continuous bijection, so since it is onto a Hausdorff space, it is a homeomorphism.

	Similar to $\mathbb{RP}^2$, for $[p] \in q(p)$, take a small closed disc $\overline{D} \subset \mathbb{R}^2$ such that, for all $(m, n) \in \mathbb{Z}^2$, $\overline{D} \cap (\overline{D} + (m, n)) = \emptyset$.

	Then $e|_{\overline{D}}$ and $q|_{\overline{D}}$ are injective. Now restricting to an open disc as before, we get an open disc as a neighbourhood of $[p]$, so $S^{1} \times S^{1}$ is a topological surface.
\end{proofbox}

Another viewpoint for a torus is by imposing on $[0,1]^2$ the equivalence relations
\begin{align*}
	(x, 0) &\sim (x, 1), & (0, y) &\sim (1, y).
\end{align*}

\begin{figure}[h]
	\centering
	\caption{Identification of a Torus}
	\label{fig:torus}
	\begin{tikzpicture}
		\draw[middlearrow={<}] (45:3) -- (135:3); 
		\draw[middlearrow={<<}] (135:3) -- (225:3);
		\draw[middlearrow={>}] (225:3) -- (315:3);
		\draw[middlearrow={>>}] (315:3) -- (45:3);
	\end{tikzpicture}
\end{figure}

\begin{exbox}
	We look at yet another example of a surface.
	\begin{enumerate}[(i)]
		\setcounter{enumi}{6}
		\item Let $P$ be a planar Euclidean polyon. Assume that the edges are oriented and paired, and for simplicity assume the Euclidean lengths of $e$ and $\hat e$ are equal if $\{e, \hat e\}$ are paired.

			Label by letters, and describe the orientation by a sign of $\pm$ relative to the clockwise orientation in $\mathbb{R}^2$.

			More precisely, if $\{e, \hat e\}$ are paired edges, there is a unique isometry from $e$ to $\hat e$ respecting their orientations, say
			\[
			f_{e \hat e} : e \to \hat e
			.\]
			These maps generate an equivalence relation on $P$, where we identify $x \in \partial P$ with $f_{e \hat e}(x)$ whenever $x \in e$.
	\end{enumerate}
\end{exbox}

\begin{lemma}
	$P/\sim$ (with the quotient topology) is a topological surface.
\end{lemma}

\begin{figure}[h]
	\centering
	\caption{Orientation of Edges of a Hexagon}
	\label{fig:hexagon}
	\begin{tikzpicture}
		\draw[middlearrow={<}] (0:3) -- (60:3); 
		\draw[middlearrow={<<}] (60:3) -- (120:3);
		\draw[middlearrow={>>>}] (120:3) -- (180:3);
		\draw[middlearrow={>>}] (180:3) -- (240:3);
		\draw[middlearrow={>>>}] (240:3) -- (300:3);
		\draw[middlearrow={>}] (300:3) -- (0:3);
	\end{tikzpicture}
\end{figure}


\begin{proofbox}
	We begin by looking at a special case of the torus $T^2$ as $[0,1]^2/\sim$. Then if $p$ is an interior point, we pick $\delta > 0$ small such that $\overline{B_{\delta}(p)}$ lies in the interior of the polygon $P$. Now we argue as before: the quotient map is injective on $\overline{B_{\delta}(p)}$ and is a homeomorphism on its interior.

	Now suppose $p$ is on an edge of $P$, but not a vertex. The idea is to take the two points in $q^{-1}(p)$, take half discs around them, and join them up to form a disc.

	Say $p = (0, y_0) \sim (1, y_0) = p'$. Take $\delta$ small enough so the half discs of radius $\delta$ do not meet the vertices and don't intersect. Let $U$ be the half disc around $p$ and $V$ the half disc around $p'$.

	Define a map as follows:
	\begin{align*}
		U : (x, y) &\overset{f_u}{\to} (x, y - y_0), \\
		V : (x, y) & \overset{f_v}{\to} (x - 1, y - y_0).
	\end{align*}
	We want to show these maps glue well together. To do this, we use the following fact:

	If $X = A \cup B$, $A$ and $B$ are closed, and $f : A \to Y$ and $g : B \to Y$ are continuous and $f|_{A \cap B} = g|_{A \cap B}$, then they define a continuous map on $X$.

	Now $f_u$ and $f_v$ are continuous on $U, V \subset [0,1]^2$, so they induce continuous maps on $q(U)$ and $q(V)$.

	In $T^2$, the intersection of the discs overlap on the paired edges, but our maps agree, so they are compatible with the equivalence relation. Hence $f_u$ and $f_v$ give a continuous map on an open image of $[p] \in T^2$ to $\mathbb{R}^2$. By the usual argument, we can show if $[p] \in T^2$ lies on an edge of $P$ it has a neighbourhood homeomorphic to a disc.

	Finally, we look at a vertex of $[0,1]^2$. In the image, there is really only one vertex. To find a homeomorphism to the open disc, we can take four quarter circles at each corner, and glue them appropriately.

	For a general polygon, it is a similar idea. Interior and edge points are done analogously to $T^2$. For vertices, it is a bit different. We have different equivalence classes of vertices caused by orienting the edges in different ways.

	If $v$ is a vertex of $P$ with $k$ vertices in its equivalence class, then we have $k$ sectors in $P$. Any sector can be identified with out favourite sector in $\mathbb{R}^2$, i.e. $(r, \theta) \in \mathbb{R}^2$ with $0 \leq r < \delta$ and $\theta \in [0, 2\pi/k]$. Gluing these together, we get an open disc as a neighbourhood of $v$.

	This works unless $k = 1$, in which case we have two paired edges coming into or out of a vertex in $P$. But this is homeomorphic to a cone, which is homeomorphic to a disc.

	These neighbourhoods of points in $P/\sim$ show that $P$ is locally homeomorphic to a disc, and we can easily check that $P/\sim$ is Hausdorff and second countable.
\end{proofbox}

\begin{exbox}
	One more example now.
	\begin{enumerate}[(i)]
		\setcounter{enumi}{7}
		\item We now consider connecting surfaces. Given topological surfaces $\Sigma_1$ and $\Sigma_2$, we can remove an open disc from each, and glue the resulting boundary circles.

			Explicitly, we take $\Sigma_1 \setminus D_1 \cup \Sigma_2 \setminus D_2$ as a disjoint union, and impose the quotient relation
			\[
			\theta \in \partial D_1 \sim \theta \in \partial D_2
			,\]
			where $\theta$ parametrizes $S^{1} = \partial D_i$.

			The result $\Sigma_1 \hash \Sigma_2$ is called the \emph{connected sum}\index{connected sum} of $\Sigma_1$ and $\Sigma_2$.

			In principle, this depends on the choices of discs, and it takes some effort to prove that it is well-defined.
	\end{enumerate}
\end{exbox}

\begin{lemma}
	The connected sum $\Sigma_1 \hash \Sigma_2$ is a topological surface.
\end{lemma}

We will not prove this lemma in this course.

\begin{figure}[h]
	\centering
	\caption{Octagon}
	\label{fig:octagon}
	\begin{tikzpicture}
		\draw[middlearrow={<<}] (22.5:3) -- (67.5:3); 
		\draw[middlearrow={<}] (67.5:3) -- (112.5:3);
		\draw[middlearrow={>>>>}] (112.5:3) -- (157.5:3);
		\draw[middlearrow={>>>}] (157.5:3) -- (202.5:3);
		\draw[middlearrow={<<<<}] (202.5:3) -- (247.5:3);
		\draw[middlearrow={<<<}] (247.5:3) -- (292.5:3);
		\draw[middlearrow={>>}] (292.5:3) -- (337.5:3);
		\draw[middlearrow={>}] (337.5:3) -- (22.5:3);
		\draw[blue,dashed] (112.5:3) -- (292.5:3);
	\end{tikzpicture}
\end{figure}

As another example the octagon is homeomorphic to a double torus: cutting along the blue line reveals two copies of a torus, which are joined together.

\begin{figure}[h]
	\centering
	\caption{Identification of $\mathbb{RP}^2$}
	\label{fig:square_rp2}
	\begin{subfigure}{.5\textwidth}
		\centering
	\begin{tikzpicture}
		\draw[middlearrow={<<}] (45:3) -- (135:3);
		\node[draw] at (45:3) {};
		\draw[middlearrow={<}] (135:3) -- (225:3);
		\draw[fill,red] (135:3) circle (0.1);
		\draw[middlearrow={<<}] (225:3) -- (315:3);
		\node[draw] at (225:3) {};
		\draw[middlearrow={<}] (315:3) -- (45:3);
		\draw[fill,red] (315:3) circle (0.1);
	\end{tikzpicture}
	\end{subfigure}%
	\begin{subfigure}{.5\textwidth}
		\centering
	\begin{tikzpicture}
		\draw[middlearrow={<}] (0:2.5) to[out=90,in=0] (90:2.5); 
		\draw[fill] (0:2.5) circle (0.075);
		\draw[middlearrow={<<}] (90:2.5) to[out=180,in=90] (180:2.5);
		\draw[fill] (90:2.5) circle (0.075);
		\draw[middlearrow={<}] (180:2.5) to[out=-90,in=180] (270:2.5);
		\draw[fill] (180:2.5) circle (0.075);
		\draw[middlearrow={<<}] (270:2.5) to[out=0,in=-90] (0:2.5);
		\draw[fill] (270:2.5) circle (0.075);
		\draw[red, dashed] (60:2.5) -- (240:2.5);
		\node[above right,red] at (60:2.5) {$\theta$};
		\node[below left,red] at (240:2.5) {$-\theta$};
	\end{tikzpicture}
	\end{subfigure}
\end{figure}

Similarly, we can find $\mathbb{RP}^2$ as the quotient of a square: this can be seem by morphing it into a circle with antipodes identified, which is then homeomorphic to $\mathbb{RP}^2$, seen by `squishing down' $\mathbb{RP}^2$ or projecting it onto a plane.

\begin{figure}[h]
	\centering
	\caption{Squishing down $\mathbb{RP}^2$}
	\label{fig:squish_rp2}
	\begin{subfigure}{.5\textwidth}
		\centering
	\begin{tikzpicture}
		\draw (0,0) circle (2.5);
		\draw (-2.5,0) arc(180:360:2.5 and 0.5);
		\draw[dashed] (2.5,0) arc(0:180:2.5 and 0.5);
		\draw[red, dashed] (60:2.5) -- (240:2.5);
		\node[above right,red] at (60:2.5) {$x$};
		\node[below left,red] at (240:2.5) {$-x$};
	\end{tikzpicture}
	\end{subfigure}%
	\begin{subfigure}{.5\textwidth}
		\centering
	\begin{tikzpicture}
		\draw (0:2.5) to[out=90,in=0] (90:2.5) to[out=180,in=90] (180:2.5);
		\draw (-2.5,0) arc(180:360:2.5 and 0.5);
		\draw[dashed] (2.5,0) arc(0:180:2.5 and 0.5);
		\draw[blue, dashed] (40:0.75) -- (220:0.75);
		\node[above right,blue] at (40:0.75) {$\theta$};
		\node[below left,blue] at (220:0.75) {$-\theta$};
	\end{tikzpicture}
	\end{subfigure}
\end{figure}

\subsection{Triangulation and Euler Characteristic}
\label{sub:triangulation_and_euler_characteristic}

\begin{definition}
	A \emph{subdivision}\index{subdivision} of a compact topological surface $\Sigma$ comprises of:
	\begin{enumerate}[(i)]
		\item a finite set $V$ of \emph{vertices}\index{vertex},
		\item a finite collection of edges $E + \{e_i : [0,1] \to \Sigma\}$ such that
			\begin{itemize}
				\item for all $i$, $e_i$ is a continuous injection on its interior and $e_i^{-1}(V) = \{0, 1\}$,
				\item $e_i$ and $e_j$ have disjoint images except perhaps at their endpoints in $V$.
			\end{itemize}
		\item We require that each connected component of
			\[
				\Sigma \setminus \Biggl( \bigcup_{i} e_i ([0,1]) \cup V \Biggr)
			\]
			is homeomorphic to an open disc, called a \emph{face}\index{face}.
	\end{enumerate}
	Hence the closure of a face $\overline{F} \setminus F$ has boundary lying in
	\[
		\bigcup_{i} e_i ([0,1]) \cup V
	.\]
	A subdivision is a \emph{triangulation}\index{triangulation} if every closed face (closure of a face) contains exactly three edges, and two closed faces are disjoint, meet in exactly one edge or just one vertex.
\end{definition}

\begin{exbox}
	A cube displays a subdivision of $S^2$, and a tetrahedron displays a triangulation of $S^2$.

	Moreover figure~\ref{fig:torus} displays a subdivision of $T^2$, with one vertex, two edges and one face.

	In figure~\ref{fig:torus_triangulation}, only the right triangulation is a valid triangulation: in the left figure, the two triangles share more than one edge.

	As well, figure~\ref{fig:s2_subdivision} is a degenerate subdivision of the sphere, with one vertex, no edges and one face.
\end{exbox}

\begin{figure}[h]
	\centering
	\caption{Triangulations of the Torus}
	\label{fig:torus_triangulation}
	\begin{subfigure}{0.5\textwidth}
		\centering
	\begin{tikzpicture}
		\draw[middlearrow={<}] (45:3) -- (135:3); 
		\draw[middlearrow={<<}] (135:3) -- (225:3);
		\draw[middlearrow={>}] (225:3) -- (315:3);
		\draw[middlearrow={>>}] (315:3) -- (45:3);
		\draw (45:3) -- (225:3);
	\end{tikzpicture}
	\caption{Invalid}
	\end{subfigure}%
	\begin{subfigure}{0.5\textwidth}
		\centering
	\begin{tikzpicture}
		\draw[middlearrow={<}] (45:3) -- (135:3); 
		\draw[middlearrow={<<}] (135:3) -- (225:3);
		\draw[middlearrow={>}] (225:3) -- (315:3);
		\draw[middlearrow={>>}] (315:3) -- (45:3);
		\draw (45:3) -- (225:3);
		\draw (2.12,0.7) -- (-2.12,0.7);
		\draw (2.12,-0.7) -- (-2.12,-0.7);
		\draw (0.7,2.12) -- (0.7,-2.12);
		\draw (-0.7,2.12) -- (-0.7,-2.12);
		\draw (-2.12,0.7) -- (-0.7,2.12);
		\draw (-2.12,-0.7) -- (0.7,2.12);
		\draw (-0.7,-2.12) -- (2.12, 0.7);
		\draw (0.7,-2.12) -- (2.12, -0.7);
	\end{tikzpicture}
	\caption{Valid}
	\end{subfigure}%
\end{figure}

\begin{figure}
	\centering
	\caption{Subdivision of $S^2$ }
	\label{fig:s2_subdivision}
	\begin{tikzpicture}
		\draw (0,0) circle (2.5);
		\draw (-2.5,0) arc(180:360:2.5 and 0.5);
		\draw[dashed] (2.5,0) arc(0:180:2.5 and 0.5);
		\draw[fill] (0,2.5) circle (0.1);
	\end{tikzpicture}
\end{figure}

\begin{definition}
	The \emph{Euler characteristic}\index{Euler characteristic} of a subdivision is
	\[
	|V| - |E| + |F|
	.\]
\end{definition}

\begin{theorem}
	\begin{enumerate}[\normalfont(i)]
		\item[]
		\item Every compact topological surface admits subdivisions and triangulations.
		\item The Euler characteristic, denoted $\chi(\Sigma)$, does not depend on the subdivision and defined a topological invariant of the surface.
	\end{enumerate}
\end{theorem}

\begin{remark}
	This is hard to prove, particularly (ii). There are cleaner approaches to this (seen in algebraic topology).
\end{remark}

\begin{exbox}
	\begin{enumerate}[1.]
		\item $\chi(S^2) = 2$.
		\item $\chi(T^2) = 0$.
		\item Let $\Sigma_1, \Sigma_2$ be compact topological spaces, and we form $\Sigma_1 \hash \Sigma_2$. We remove open discs $D_i \subset \Sigma_i$ which is a face of a triangulation in each surface. Hence,
			\[
			\chi(\Sigma_1 \hash \Sigma_2) = \chi(\Sigma_1) + \chi(\Sigma_2) - 2
			.\]
			In particular if $\Sigma_g$ is a surface with $g$ holes, i.e.
			\[
			\Sigma_g = \hash_{i = 1}^{g} T^2
			,\]
			then $\chi(\Sigma_g) = 2 - 2g$. $g$ is called the \emph{genus}\index{genus}.
	\end{enumerate}
\end{exbox}

\newpage

\section{Abstract Smooth Surfaces}
\label{sec:abstract_smooth_surfaces}

\begin{definition}
	A pair $(U, \varphi)$ where $U \subset \Sigma$ is open and $\varphi : U \to V \subset \mathbb{R}^2$ is called a \emph{chart}\index{chart}.

	The inverse $\sigma = \varphi^{-1} : V \to U \subset \Sigma$ is called a \emph{local parametrization}\index{local parametrization} of $\Sigma$.
\end{definition}

\begin{definition}
	A collection of charts
	\[
		\{ (U_i, \varphi_i)_{i \in I}\}
	\]
	such that
	\[
	\bigcup_{i \in I} U_i = \Sigma
	\]
	is called an \emph{atlas}\index{atlas} of $\Sigma$.
\end{definition}

\begin{exbox}
	\begin{enumerate}[1.]
	\item If $Z \subset \mathbb{R}^2$ is closed, then $\mathbb{R}^2 \setminus Z$ is a topological surface with an atlas with one chart: $(\mathbb{R}^2 \setminus Z, \varphi = \id)$.
	\item For $S^2$ we have an atlas with 2 charts: the two stereographic projections.
	\end{enumerate}
\end{exbox}

\begin{definition}
	Let $(U_i, \varphi_i)$ for $i = 1, 2$ be two charts containing $p \in \Sigma$. The map
	\[
	\varphi_2 \circ \varphi_1^{-1}|_{\varphi_1(U_1 \cap U_2)}
	\]
	is called the \emph{transition map}\index{transition map} between charts.
\end{definition}

Note that
\[
	\varphi_1(U_1 \cap U_2) \overset{\varphi_2 \circ \varphi_1^{-1}}{\to} \phi_2(U_1 \cap U_2)
\]
is a \emph{homeomorphism}.

\begin{figure}[h]
	\centering
	\caption{Transition Map on $S^2$}
	\label{fig:transition_map}
	\begin{tikzpicture}
		\draw (0,0) circle (2.5);
		\draw (-2.5,0) arc(180:360:2.5 and 0.5);
		\draw[dashed] (2.5,0) arc(0:180:2.5 and 0.5);
		\draw (-0.5,1.5) circle (0.5 and 0.3);
		\draw (-0.1,1.6) circle (0.2 and 0.5);
		\begin{scope}
			\clip (-0.5,1.5) circle (0.5 and 0.3);
			\draw[pattern = {north west lines}] (-0.1,1.6) circle (0.2 and 0.5);
		\end{scope}
		\draw [->] (-0.7, 1.1) -- (-2, -3);
		\draw [->] (0.15, 1.2) -- (1.8, -3);
		\draw (-2,-4.5) circle (1);
		\begin{scope}
			\clip (-2,-4.5) circle (1);
			\draw[pattern = {north west lines}] (-2,-5.5) rectangle +(3,3);
		\end{scope}
		\draw (2,-4.5) circle (1);
		\begin{scope}
			\clip (2,-4.5) circle (1);
			\draw[pattern = {north east lines}] (2,-5.5) rectangle +(-3,3);
		\end{scope}
		\draw[->] (-0.5, -4.5) -- (0.5, -4.5);
		\node[left] at (-0.85,1.8) {$U_1$};
		\node[right] at (0.1,2.1) {$U_2$};
		\node[right] at (-1.5,-1.5) {$\varphi_1$};
		\node[left] at (1.25,-1.5) {$\varphi_2$};
		\node[below] at (0,-4.5) {$\varphi_2 \circ \varphi_1^{-1}$};
	\end{tikzpicture}
\end{figure}

Recall if $V \subset \mathbb{R}^{n}$ and $V' \subset \mathbb{R}^{m}$ are open, a map $f : V \to V'$ is called \emph{smooth}\index{smooth} if it is infinitely differentiable, so it has continuous partial derivatives of all orders.

A homeomorphism $f : V \to V'$ is called a \emph{diffeomorphism}\index{diffeomorphism} if it is smooth and it has a smooth inverse.

\begin{definition}
	An \emph{abstract smooth surface}\index{abstract smooth surface} $\Sigma$ is a topological surface with an atlas of charts $\{U_i, \varphi_i)\}$ such that all transition maps are diffeomorphisms.
\end{definition}

\begin{exbox}
	\begin{enumerate}[1.]
		\item The atlas of two charts with stereographic projections gives $S^2$ the structure of an abstract smooth surface.
		\item The torus $T^2 = \mathbb{R}^2 / \mathbb{Z}^2$ is an abstract smooth surface. Recall that we obtained charts from (the inverses of) the projection restricted to small discs in $\mathbb{R}^2$. In particular, consider the atlas
			\[
				\{(e(D_{\eps}(x,y)), e^{-1} \text{ on its image})\}
			,\]
			where $\eps < 1/3$. Here the transition maps are translations, so $T^2$ inherits the structure of a smooth surface.
	\end{enumerate}
\end{exbox}

\begin{definition}
	Let $\Sigma$ be an abstract smooth surface and $f : \Sigma \to \mathbb{R}^{n}$ a map. We say that $f$ is \emph{smooth}\index{smooth map} at $p \in \Sigma$ if whenever $(U, \varphi)$ is a chart at $p$ belonging to the smooth atlas of $\Sigma$, then the map
	\[
	f \circ \varphi^{-1} : \varphi(U) \to \mathbb{R}^{n}
	\]
	is smooth at $\phi(p) \in \mathbb{R}^2$.
\end{definition}

Note if this holds for one chart at $p$, then it holds for all charts at $p$, as
\[
f \circ \varphi^{-1} = f \circ \varphi_2^{\-1} \circ (\varphi_2 \circ \varphi_1^{-1})
,\]
and $(\varphi_2 \circ \varphi_1^{-1})$ is a diffeomorphism.

Related, if $\Sigma_1, \Sigma_2$ are abstract smooth surfaces, then a map $f : \Sigma_1 \to \Sigma_2$ is \emph{smooth} if it is smooth at the local charts: there are charts $(U, \varphi)$ at $p$ and $(V, \psi)$ at $f(p)$ with $f(U) \subset V$ such that $\psi \circ f \circ \varphi^{-1}$ is smooth at $\varphi(p)$.

Again, if $f$ is smooth at $p$, then the smoothness of the local representation of $f$ at $p$ will hold for all charts at $p$ and $f(p)$ in the given atlases.

\begin{definition}
	$\Sigma_1$ and $\Sigma_2$ are \emph{diffeomorphic}\index{diffeomorphic} if there exists $f : \Sigma_1 \to \Sigma_2$ that is smooth with smooth inverse.
\end{definition}

\begin{definition}
	If $Z \subset \mathbb{R}^{n}$ is an arbitrary subset, we say that $f : Z \to \mathbb{R}^{m}$ is smooth near $p \in Z$ if there exists open $B$ with $p\ in B \subset \mathbb{R}^{n}$ and smooth $F : B \to \mathbb{R}^{m}$ such that
	\[
	F|_{B \cap Z} = f|_{B \cap Z}
	.\]
	So $f$ is locally the restriction of a smooth map defined on an open set.
\end{definition}

\begin{definition}
	If $X \subset \mathbb{R}^{n}$ and $Y \subset \mathbb{R}^{m}$ are subsets, we say that $X$ and $Y$ are \emph{diffeomorphic} if there exists $f : X \to Y$, smooth with smooth inverse.
\end{definition}

\begin{definition}
	A \emph{smooth surface}\index{smooth surface} in $\mathbb{R}^3$ is a subset $\Sigma \subset \mathbb{R}^3$ such that for all $p \in \Sigma$, there exists an open set $p \in U \subset \Sigma$ such that $U$ is diffeomorphic to an open set in $\mathbb{R}^2$.

	In other words, for all $p \in \Sigma$, there exists an open ball $B$ such that $p\in B \subset \mathbb{R}^3$ and $F : B \to V \subset \mathbb{R}^2$ smooth, with
	\[
	F|_{B \cap \Sigma} : B \cap \Sigma \to V
	\]
	a homeomorphism with inverse $V \to B \cap \Sigma$ smooth.
\end{definition}

Hence we have two notions of a smooth surface: one abstract, and one taking advantage of the ambient space $\mathbb{R}^3$.

\begin{theorem}\label{thm:smoothness}
	For a subset $\Sigma \subset \mathbb{R}^3$, the following are equivalent:
	\begin{enumerate}[\normalfont(a)]
		\item $\Sigma$ is a smooth surface in $\mathbb{R}^3$.
		\item $\Sigma$ is locally the graph of a smooth function over one of the coordinate planes, so for all $p \in \Sigma$, there exists open $p \in B \subset \mathbb{R}^3$ and open $V \subset \mathbb{R}^2$ such that
			\[
				\Sigma \cap B = \{(x, y, g(x, y) \mid g : V \to \mathbb{R}\}
			,\]
			with $g$ smooth.
		\item $\Sigma$ is locally cut out by a smooth function with non-zero derivative, so for all $p \in \Sigma$, there open exists $p \in B \subset \mathbb{R}^{n}$ and $f : B \to \mathbb{R}$ such that
			\[
			\Sigma \cap B = f^{-1}(0), \qquad Df|_x \neq 0,
			\]
			for all $x \in B$.
		\item $\Sigma$ is locally the image of an \emph{allowable parametrization}, so if $p \in \Sigma$, there exists open $p \in U \subset \Sigma$ and smooth $\sigma : V \to U$, such that $\sigma$ is a homeomorphism and $D\sigma|_x$ has rank $2$ for all $x \in V$.
	\end{enumerate}
\end{theorem}

\begin{remark}
	(b) says that if $\Sigma$ is a smooth surface in $\mathbb{R}^3$, then each $p \in \Sigma$ belongs to a chart $(U, \varphi)$ where $\varphi$ is the restriction of $\pi_{xy}, \pi_{yz}, \pi_{xz}$ from $\mathbb{R}^3$ to $\mathbb{R}^2$.
\end{remark}

\subsection{Inverse Function Theorem}
\label{sub:inverse_function_theorem}

\begin{theorem}\index{inverse function theorem}
	Let $U \subset \mathbb{R}^{n}$ be open and let $f : U \to \mathbb{R}^{n}$ be a continuously differentiable map. Let $p \in U$, $f(p) = q$, and suppose $Df|_p$ is invertible. Then there is an open neighbourhood $V$ of $q$ and a differentiable map $g : V \to \mathbb{R}^{n}$ with $g(q) = p$, with image an open neighbourhood $U' \subset U$ of $p$, such that
	\[
	f \circ g = \id_V, \qquad g \circ f = \id_{u'}
	.\]
	If $f$ is smooth, then so is $g$.
\end{theorem}

\begin{remark}
	$(Dg|_q) = (Df|_p)^{-1}$ by the chain rule.
\end{remark}

If we have a map $f : \mathbb{R}^{n} \to \mathbb{R}^{m}$ where $n > m$, then
\[
Df|_p = \biggl( \frac{\partial f_i}{\partial x_j} \biggr)_{m \times n}
\]
having full rank means that, permuting coordinates if necessary, we can assume that the first $m$ columns are linearly independent.

\begin{theorem}[Implicit Function theorem]\index{implicit function theorem}
	Let $p = (x_0, y_0) \in U$, where $U \subset \mathbb{R}^{k} \times \mathbb{R}^{\ell}$ is open, and $f : U \to \mathbb{R}^{\ell}$ be a continuously differentiable map with $f(p) = 0$, and
	\[
		\biggl( \frac{\partial f_i}{\partial y_j} \biggr)_{\ell \times \ell} \text{ is an isomorphism at } p
	.\]
	Then there exists an open neighbourhood $x_0 \in V \subset \mathbb{R}^{k}$ and a continuously differentiable map $g : V \to \mathbb{R}^{\ell}$ taking $x_0$ to $y_0$, such that if $(x, y) \in U \cap (V \times \mathbb{R}^{\ell})$, then
	\[
	f(x, y) = 0 \iff y = g(x)
	.\]
\end{theorem}

\begin{proofbox}
	Introduce $F: U \to \mathbb{R}^{k} \times \mathbb{R}^{\ell}$, where $(x, y) \mapsto (x, f(x, y))$. Then
	\[
	DF =
	\begin{pmatrix}
		I & \ast \\
		0 & \frac{\partial f_i}{\partial y_j}
	\end{pmatrix}
	.\]
	So $DF|_{(x_0, y_0)}$ is an isomorphism. The inverse function says that $F$ is locally invertible near $F(x_0, y_0) = (x_0, f(x_0, y_0)) = (x_0, 0)$.

	Take a product of open neighbourhoods $(x_0,0) \in V \times V'$, where $V \subset \mathbb{R}^{k}$, $V' \subset \mathbb{R}^{\ell}$ are open. Then there is some continuously differentiable inverse $G : V \times V' \to U' \subset U$ such that $F \circ G = \id_{V \times V'}$ 

	Write $G(x, y) = (\varphi(x,y), \psi(x, y))$. Then,
	\[
	F \circ G(x, y) = (\varphi(x, y) f(\varphi(x, y), \psi(x,y))) = (x, y)
	.\]
	Hence $\varphi(x, y) = x$, $f(x, \psi(x, y)) = y$. Thus, $f(x, y) = 0 \iff y = \psi(x, 0)$.

	Define $g : V \to \mathbb{R}^{\ell}$ by $g(x) = \psi(x,0)$. Then $g(x_0) = y_0$, and this is the required function $g$.
\end{proofbox}

\begin{exbox}
	\begin{enumerate}[1.]
		\item Take $f : \mathbb{R}^2 \to \mathbb{R}$ smooth, and $f(x_0, y_0) = 0$. Assume
			\[
			\frac{\partial f}{\partial y} \biggr|_{(x_0, y_0)} \neq 0
			.\]
			Then there exists smooth $g : (x_0 - \eps, x_0 + \eps) \to \mathbb{R}$. Such that $g(x_0) = y_0$ and $f(x, y) = 0 \iff y = g(x)$.

			Since $f(x, g(x)) = 0$ by chain rule
			\[
			\frac{\partial f}{\partial x} + \frac{\partial f}{\partial y} g'(x) = 0 \implies g'(x) = - \frac{\partial f/\partial x}{\partial f/ \partial y}
			.\]
		\item Let $f : \mathbb{R}^3 \to \mathbb{R}$ smooth and $f(x_0, y_0, z_0) = 0$, and assume
			\[
			Df|_{(x_0, y_0, z_0)} \neq 0
			.\]
			Permuting coordinates if necessary, we may assume that
			\[
			\frac{\partial f}{\partial z}|_{(x_0, y_0, z_0)} \neq 0
			.\]
			Then there exists an open neighbourhood $(x_0, y_0) \in V \subset \mathbb{R}^2$ and a smooth $g : V \to \mathbb{R}$, with $g(x_0, y_0) = z_0$, such that for an open set $(x_0, y_0, z_0) \in U$,
			\[
				f^{-1}(0) \cap U = \{(x, y, g(x, y)) \mid (x, y) \in V\}
			.\]
	\end{enumerate}	
\end{exbox}

We return to theorem~\ref{thm:smoothness}, which we can now prove.

\begin{proofbox}
	Note (b) implies all other statements. If $\Sigma$ is locally $\{(x, y, g(x, y)) \mid (x, y) \in V\}$, then we get a chart from the projection $\Pi_{xy}$, which is smooth and defined on an open neighbourhood of $\Sigma$, hence (b) implies (a).

	Also, it is cut out by
	\[
	f(x, y, z) = z - g(x, y)
	.\]
	Clearly $\frac{\partial f}{\partial z} = 1 \neq 0$, so (b) implies (c).

	Also, $\sigma(x, y) = (x, y, g(x, y))$ is allowable and smooth, with
	\[
	\sigma_x = (1, 0, g_x), \qquad \sigma_y = (0, 1, g_y)
	\]
	linearly independent. So (b) implies (d).

	Now (a) implies (d), as this is part of the definition of being a smooth surface in $\mathbb{R}^3$.

	Moreover, (c) implies (b) from the above example of the implicit function theorem.

	We finally show that (d) implies (b). Let $p \in \Sigma$, and $\sigma : V \to U \subset \Sigma$ with $\sigma(0) = p \in U$, and $\sigma = (\sigma_1, \sigma_2, \sigma_3)$. Then
	\[
	D\sigma =
	\begin{pmatrix}
		\partial \sigma_1/\partial u & \partial \sigma_1/\partial v \\
		\partial \sigma_2/\partial u & \partial \sigma_2/\partial v \\
	\partial \sigma_3/\partial u & \partial \sigma_3/\partial v
	\end{pmatrix}
	.\]
	So there exists two rows defining an invertible matrix, as $D\sigma$ has rank two. Suppose the first two rows are. Then $\Pi_{xy} \circ \sigma : V \to \mathbb{R}^2$ satisfies $D(\Pi_{xy} \circ \sigma)|_0$ is an isomorphism.

	By the inverse function theorem, this is locally invertible, so if we let $\phi = \Pi_{xy} \circ \sigma$, then $\Sigma$ is the graph of $(x, y, \sigma_3(\phi^{-1}(x, y)))$.
\end{proofbox}

Using this, we can find many examples of smooth surfaces in $\mathbb{R}^3$.

\begin{exbox}
	\begin{enumerate}[1.]
		\item The \emph{ellipsoid}\index{ellipsoid} $E \subset \mathbb{R}^3$ is $f^{-1}(0)$ for $f : \mathbb{R}^3 \to \mathbb{R}$ satisfying
			\[
			f(x, y, z) = \frac{x^2}{a^2} + \frac{y^2}{b^2} + \frac{z^2}{c^2} - 1
			.\]
			For all $p \in E = f^{-1}(0)$, $Df|_p \neq 0$, so $E$ is a smooth surface in $\mathbb{R}^3$.
		\item Let $\gamma : [a, b] \to \mathbb{R}^3$ be a smooth map with image in the $xz$-plane, so
			\[
			\gamma(t) = (f(t), 0, g(t))
			.\]
			Assume $\gamma$ is injective, with $\gamma'(t) \neq 0$. Rotating this around the $z$-axis, we get a surface of revolution with allowable parametrization %to do: insert image
			\[
			\sigma(u, v) = (f(u) \cos v, f(u) \sin v, g(u))
			.\]
			For $(u, v) \in [a, b] \times [\theta, \theta + 2 \pi)$ for $\theta \in [0, 2\pi]$ fixed, $\sigma$ is a homeomorphism onto its image. Indeed,
			\begin{align*}
				\sigma_u &= (f' \cos v, f' \sin v, g'), \\
				\sigma_v &= (-f \sin v, f \cos v, 0).
			\end{align*}
			Moreover,
			\[
			\| \sigma_u \times \sigma_v \|^2 = f^2 (f'^2 + g'^2) \neq 0
			,\]
			proving $\sigma$ is allowable.
	\end{enumerate}
\end{exbox}

\subsection{Orientability}
\label{sub:orientability}

Consider $V, V' \subset \mathbb{R}^2$ open, with $f : V \to V'$ a diffeomorphism. Then at any $x \in V$,
\[
	Df|_x \in \GL(2,\mathbb{R})
.\]

Let $\GL^{+}(2, \mathbb{R}) \subset \GL(2, \mathbb{R})$ be the subgroup of matrices of positive determinant.

\begin{definition}
	We say that $f$ is \emph{orientation preserving}\index{orientation preserving} if $Df|_x \in \GL^{+}(2, \mathbb{R})$ for all $x \in V$.
\end{definition}

\begin{definition}
	An abstract smooth surface $\Sigma$ is \emph{orientable}\index{orientability} if it admits an atlas such that the transition maps are orientation preserving diffeomorphisms of open sets of $\mathbb{R}^2$.

	A choice of such atlas is an \emph{orientation}of $\Sigma$, and we say that $\Sigma$ is \emph{oriented}.
\end{definition}

\begin{lemma}
	If $\Sigma_1, \Sigma_2$ are abstract smooth surfaces, and they are diffeomorphic, then $\Sigma_1$ is orientable if and only if $\Sigma_2$ is orientable.
\end{lemma}

\begin{proofbox}
	Suppose $f : \Sigma_1 \to \Sigma_2$ is a diffeomorphism and $\Sigma_2$ is orientable and equipped with an oriented atlas.

	Consider the atlas on $\Sigma_1$ given by
	\[
		(f^{-1}U, \psi \circ f|_{f^{-1}U})
	,\]
	where $(U, \psi)$ is a chart of $\Sigma_2$. The transition function between two such charts is exactly the transition function in the $\Sigma_2$ atlas.

	The transition function between two such charts is exactly the transition function in the $\Sigma_2$-atlas.
\end{proofbox}

\begin{remark}
	\begin{enumerate}[1.]
		\item[]
		\item There is no sensible classification of all smooth or topological surfaces, for example $\mathbb{R}^2 \setminus Z$ for $Z$ closed.

			By contrast, compact smooth surfaces up to diffeomorphism are classified by Euler characteristic and orientability.
		\item There is a definition of orientation preserving homeomorphism that needs some algebraic topology.

			The M\"{o}bius band is the surface in figure~\ref{fig:mobius_band}. It turns out that abstract smooth surfaces are orientable if and only if it contains no subsurface homeomorphic to the M\"{o}bius band.

			So we can say that a topological surface is orientable if and only if it contains no subsurface homeomorphic to a M\"{o}bius band.
		\item We get other structures by demanding the transition maps to be such that
			\[
			D(\varphi_1 \varphi_2^{-1})|_x \in G \subset \GL(2,\mathbb{R})
			.\]
			For example, we can take $G = \GL(1, \mathbb{C}) \subset \GL(2, \mathbb{R})$, which give \emph{Riemann surfaces}\index{Riemann surfaces}.
	\end{enumerate}
\end{remark}

\begin{figure}[h]
	\centering
	\caption{M\"{o}bius band}
	\label{fig:mobius_band}
	\begin{tikzpicture}
		\draw[dashed] (45:3) -- (135:3); 
		\draw[middlearrow={<}] (135:3) -- (225:3);
		\draw[dashed] (225:3) -- (315:3);
		\draw[middlearrow={<}] (315:3) -- (45:3);
	\end{tikzpicture}
\end{figure}

\begin{exbox}
	\begin{enumerate}[1.]
		\item If we take $S^2$ with the atlas of the two stereographic projection, we can compute the transition map as
			\[
				(u, v) \mapsto \biggl( \frac{u}{u^2 + v^2}, \frac{v}{u^2 + v^2} \biggr)
			,\]
			on $\mathbb{R}^2 \setminus \{0\}$. This is orientation preserving.
		\item In $T^2$, the transition maps are translations of $\mathbb{R}^2$, so $T^2$ is oriented.
	\end{enumerate}
\end{exbox}

For surfaces in $\mathbb{R}^3$, we would like ot have orientability dictated by some ambient feature.

\begin{definition}
	Let $\Sigma \subset \mathbb{R}^3$ be a smooth surface, and $p \in \Sigma$. Fix an allowable parametrization $\sigma : V \to U \subset \Sigma$, $\sigma(0) = p$.

	Then, the \emph{tangent plane}\index{tangent plane} $T_p \Sigma$ of $\Sigma$ at $p$ is the image of $D\sigma|_{0}$, as a subset of $\mathbb{R}^3$. This is a 2d vector subspace of $\mathbb{R}^3$.

	The \emph{affine tangent plane}\index{affine tangent plane} of $\Sigma$ at $p$ is $p + T_p \Sigma \subset \mathbb{R}^3$.
\end{definition}

\begin{lemma}
	$T_p \Sigma$ is well-defined, so it is independent of the choice of allowable parametrization near $p$.
\end{lemma}



\newpage

\printindex

\end{document}
