%%% TO DO %%%

\documentclass[12pt]{article}

\usepackage{ishn}
\usepackage{relsize}
\DeclareMathOperator{\hash}{\sharp}% connected sum

\makeindex[intoc]

\begin{document}

\hypersetup{pageanchor=false}
\begin{titlepage}
	\begin{center}
		\vspace*{1em}
		\Huge
		\textbf{IB Geometry}

		\vspace{1em}
		\large
		Ishan Nath, Lent 2023

		\vspace{1.5em}

		\Large

		Based on Lectures by Prof. Gabriel Paternain

		\vspace{1em}

		\large
		\today
	\end{center}
	
\end{titlepage}
\hypersetup{pageanchor=true}

\tableofcontents

\newpage

\section{Surfaces}
\label{sec:surfaces}

\begin{definition}
	A \emph{topological surface}\index{topological surface} is a topological space $\Sigma$ such that
	\begin{enumerate}[(a)]
		\item for all $p \in \Sigma$, there is an open neighbourhood $p \in U \subset \Sigma$ such that $U$ is homeomorphic to $\mathbb{R}^2$, or a disc $D^2 \subset \mathbb{R}^2$, with its usual Euclidean topology.
		\item $\Sigma$ is Hausdorff and second countable.
	\end{enumerate}
\end{definition}

\begin{remark}
	\begin{enumerate}
		$\mathbb{R}^2 \simeq D(0,1) = \{x \in \mathbb{R}^2 \mid \|x\| \leq 1\}$.
		\item A space $X$ is \emph{Hausdorff}\index{Hausdorff} if for $p \neq q$ in $X$, there exist disjoint open sets $U, V$ with $p \in U, q \in V$.

			A space is \emph{second countable}\index{second countable} if it has a countable base, i.e there exist open sets $\{U_i\}_{i \in \mathbb{N}}$, such that every open set is a union of some of the $U_i$.

			The key point of defining surfaces is point (a), point (b) is for ruling out surfaces that are too weird.
		\item If $X$ is Hausdorff or second countable, then so are subspaces of $X$. Moreover Euclidean space has these properties (to show it is second countable, consider open balls $B(c, r)$ with $c \in \mathbb{Q}^{n} \subset \mathbb{R}^{n}$, and $r \in \mathbb{Q}_{+} \subset \mathbb{R}_{+}$).
	\end{enumerate}
\end{remark}

\begin{exbox}
	\begin{enumerate}[(i)]
		\item The plane $\mathbb{R}^2$.
		\item Any open set in $\mathbb{R}^2$ is a surface, i.e. $R^2 \setminus Z$ where $Z$ is closed is a surface.
		\item Graphs of functions. Suppose $f : \mathbb{R}^2 \to \mathbb{R}$ is continuous. Then the graph of $f$ is
			\[
				\Gamma_f = \{(x, y, f(x, y)) \mid (x, y) \in \mathbb{R}^2\}
			.\]
			This is a subspace of $\mathbb{R}^3$, so we can endow it with the subspace topology. We claim it is a subspace homeomorphic to $\mathbb{R}^2$.
	\end{enumerate}
	Recall that if $X, Y$ are topological spaces, then the product topology $X \times Y$ has a basis of open sets $U \times V$, where $U \subset X$, $V \subset Y$ are open

	A feature is that if $g : Z \to X \times Y$ is continuous if and only if $\Pi_x \circ g : Z \to X$ and $\Pi_y \circ g : Z \to Y$ are continuous, where $\Pi_x$, $\Pi_y$ are the canonical projectors.
	\begin{enumerate}[resume*]
		\item[]We can now show that if $f : X \to Y$ is continuous, then $\Gamma_f \subset X \times Y$ is homeomorphic to $X$, as $s(x) = (x,f(x))$ is a continuous function from $X$ to $\Gamma_f$, $\Pi_x|_{\Gamma_f}$ and $s$ are inverse homeomorphisms.

		In particular, for our example $\Gamma_f \simeq \mathbb{R}^2$. So any $f : \mathbb{R}^2 \to \mathbb{R}$ continuous produces a surface $\Gamma_f$.
	\item The sphere: $S^2 = \{(x, y, z) \in \mathbb{R}^3 \mid x^2 + y^2 + z^2 = 1\}$ (with the subspace topology). To show this is a surface, we can consider the stereographic projection $\Pi_{+}: S^2 \setminus\{(0,0,1)\} \to \mathbb{R}^2$:
			\[
				(x, y, z) \mapsto \biggl( \frac{x}{1-z}, \frac{y}{1-z} \biggr)
			.\]
	Then $\Pi_{+}$ is continuous and has an inverse
	\[
		(u, v) \mapsto \biggl( \frac{2u}{u^2 + v^2 + 1}, \frac{2v}{u^2 + v^2 + 1}, \frac{u^2 + v^2 - 1}{u^2 + v^2 + 1} \biggr)
	.\]
	So $\Pi_{+}$ is a continuous bijection with continuous inverse, and hence a homeomorphism.

	Similarly, taking a stereographic projection from the south pole $\Pi_{-} : S^2 \setminus \{(0, 0, -1)\} \to \mathbb{R}^2$, by
	\[
		(x, y, z) \mapsto \biggl( \frac{x}{1+z}, \frac{y}{1+z} \biggr)
	\]
	is another homeomorphism. Hence $S^2$ is a topological surface, as the open sets $S^2 \setminus \{(0, 0, 1)\}$ and $S^2 \setminus \{(0, 0, -1)\}$ cover $S^2$, and it is Hausdorff and second countable as it is a subspace of $\mathbb{R}^3$.
\item The \emph{real projective plane}\index{real projective plane}. The group $\mathbb{Z}_2$ acts on $S^{2}$ by homeomorphisms, via the antipodal map
	\begin{align*}
		a : S^2 &\to S^2 \\
		a(x, y, z) &\mapsto (-x, -y, -z)
	\end{align*}
	\end{enumerate}
\end{exbox}
\begin{definition}
	The real projective plane is the quotient of $S^2$ by identifying every point with its antipodal image:
	\[
	\mathbb{RP}^2 = S^2/\mathbb{Z}_2 = S^2/\sim
	.\]
\end{definition}

\begin{lemma}
	As a set, $\mathbb{RP}^2$ is naturally in bijection with the set of straight lines through $0$.
\end{lemma}

This is because any straight line through $0 \in \mathbb{R}^3$ intersects $S^2$ in exactly a pair of antipodal points, and each such pair determines a straight line.

\begin{lemma}
	$\mathbb{RP}^2$ is a topological surface with the quotient topology.
\end{lemma}

Recall the quotient topology: given the quotient map $q : X \to Y$, we say $V \subset Y$ is open if and only if $q^{-1}(V) \subset X$ is open in $X$.

\begin{proofbox}
	First we show that $\mathbb{RP}^2$ is Hausdorff. If $[p] \neq [q] \in \mathbb{RP}^2$, then $\pm p$, $\pm q$ are distinct, antipodal pairs.

	We take open discs centred on $p$ and $q$ and their antipodal images, such that no two discs intersect. The images of these discs give open images of $[p]$ and $[q]$ in $\mathbb{RP}^2$. Indeed, $q(B_{\delta}(p))$ is open since $q^{-1}(q(B_{\delta}(p))) = B_{\delta}(p) \cup (-B_{\delta}(p))$.

	Now we show $\mathbb{RP}^2$ is second countable. Let $U$ be a countable base of $S^2$, and let $\overline{U} = \{q(u) \mid u \in U\}$. Then $q(u)$ is open, as $q(u) = u \cup (-u)$, and $\overline{U}$ is clearly countable as $U$ is.

	Take $V \subset \mathbb{RP}^2$ open. By definition, $q^{-1}(V)$ is open, so let $q^{-1}(V) = \bigcup U_{\alpha}$, for $U_{\alpha} \in U$. Then
	\[
	V = q(q^{-1}(V)) = q\biggl( \bigcup_{\alpha} U_{\alpha}\biggr) = \bigcup_{\alpha}q(U_{\alpha})
	.\]

	Finally, let $p \in S^2$ and $[p] \in \mathbb{RP}^2$ be its image. Let $\overline{D}$ be a small closed disc neighbourhood of $p \in S^2$, so that $q|_{\overline{D}}$ is injective and continuous, and has image a Hausdorff space.

	Now recall that a countinuous bijection from a compact space to a Hausdorff space is a homeomorphism.

	So $q|_{\overline{D}} : \overline{D} \to q(\overline{D)}$ is a homeomorphism. This induces a homeomorphism
	\[
	q|_D : D \to q(D) \subset \mathbb{RP}^2
	,\]
	where $D$ is an open disc contained in $\overline{D}$. So $[p] \in q(D)$ has an open neighbourhood in $\mathbb{RP}^2$ homeomorphic to an open disc.
\end{proofbox}

\begin{exbox}
	We continue looking at examples of surfaces.
	\begin{enumerate}[(i)]
		\setcounter{enumi}{5}
		\item Let $S^{1}=  \{z \in \mathbb{C} \mid |z| = 1\}$. Then the \emph{torus}\index{torus} is $S^{1} \times S^{1}$ with the subspace topology of $\mathbb{C}^2$ (this is the same as taking the product topology).
	\end{enumerate}
\end{exbox}

\begin{lemma}
	The torus is a topological surface.
\end{lemma}

\begin{proofbox}
	We consider the map
	\begin{align*}
		\mathbb{R}^2 &\overset{e}{\to} S^{1} \times S^{1} \subset \mathbb{C} \times \mathbb{C} \\
		(s, t) & \mapsto (e^{2\pi i s}, e^{2 \pi i t}).
	\end{align*}
	We can view this map using the following diagram:
	\begin{center}
		\begin{tikzcd}[column sep = small]
		\mathbb{R}^2 \arrow[r, "e"] \arrow[d,"q"] & S^{1} \times S^{1} \\
		\mathbb{R}^2/\mathbb{Z}^2 \arrow[ru, "{\hat e}", dashed] &
	\end{tikzcd}
	\end{center}
	There is an equivalence relation on $\mathbb{R}^2$ given by translating by $\mathbb{Z}^2$. Now consider the map
	\[
		[0,1]^2 \injto \mathbb{R}^2 \overset{q}{\to} \mathbb{R}^2/\mathbb{Z}^2
	\]
	is onto, so $\mathbb{R}^2/\mathbb{Z}^2$ is compact. Now note that $\hat e$ is a continuous bijection, so since it is onto a Hausdorff space, it is a homeomorphism.

	Similar to $\mathbb{RP}^2$, for $[p] \in q(p)$, take a small closed disc $\overline{D} \subset \mathbb{R}^2$ such that, for all $(m, n) \in \mathbb{Z}^2$, $\overline{D} \cap (\overline{D} + (m, n)) = \emptyset$.

	Then $e|_{\overline{D}}$ and $q|_{\overline{D}}$ are injective. Now restricting to an open disc as before, we get an open disc as a neighbourhood of $[p]$, so $S^{1} \times S^{1}$ is a topological surface.
\end{proofbox}

Another viewpoint for a torus is by imposing on $[0,1]^2$ the equivalence relations
\begin{align*}
	(x, 0) &\sim (x, 1), & (0, y) &\sim (1, y).
\end{align*}

\begin{exbox}
	We look at yet another example of a surface.
	\begin{enumerate}[(i)]
		\setcounter{enumi}{6}
		\item Let $P$ be a planar Euclidean polyon. Assume that the edges are oriented and paired, and for simplicity assume the Euclidean lengths of $e$ and $\hat e$ are equal if $\{e, \hat e\}$ are paired.

			Label by letters, and describe the orientation by a sign of $\pm$ relative to the clockwise orientation in $\mathbb{R}^2$.

			More precisely, if $\{e, \hat e\}$ are paired edges, there is a unique isometry from $e$ to $\hat e$ respecting their orientations, say
			\[
			f_{e \hat e} : e \to \hat e
			.\]
			These maps generate an equivalence relation on $P$, where we identify $x \in \partial P$ with $f_{e \hat e}(x)$ whenever $x \in e$.
	\end{enumerate}
\end{exbox}

\begin{lemma}
	$P/\sim$ (with the quotient topology) is a topological surface.
\end{lemma}

\begin{proofbox}
	We begin by looking at a special case of the torus $T^2$ as $[0,1]^2/\sim$. Then if $p$ is an interior point, we pick $\delta > 0$ small such that $\overline{B_{\delta}(p)}$ lies in the interior of the polygon $P$. Now we argue as before: the quotient map is injective on $\overline{B_{\delta}(p)}$ and is a homeomorphism on its interior.

	Now suppose $p$ is on an edge of $P$, but not a vertex. The idea is to take the two points in $q^{-1}(p)$, take half discs around them, and join them up to form a disc.

	Say $p = (0, y_0) \sim (1, y_0) = p'$. Take $\delta$ small enough so the half discs of radius $\delta$ do not meet the vertices and don't intersect. Let $U$ be the half disc around $p$ and $V$ the half disc around $p'$.

	Define a map as follows:
	\begin{align*}
		U : (x, y) &\overset{f_u}{\to} (x, y - y_0), \\
		V : (x, y) & \overset{f_v}{\to} (x - 1, y - y_0).
	\end{align*}
	We want to show these maps glue well together. To do this, we use the following fact:

	If $X = A \cup B$, $A$ and $B$ are closed, and $f : A \to Y$ and $g : B \to Y$ are continuous and $f|_{A \cap B} = g|_{A \cap B}$, then they define a continuous map on $X$.

	Now $f_u$ and $f_v$ are continuous on $U, V \subset [0,1]^2$, so they induce continuous maps on $q(U)$ and $q(V)$.

	In $T^2$, the intersection of the discs overlap on the paired edges, but our maps agree, so they are compatible with the equivalence relation. Hence $f_u$ and $f_v$ give a continuous map on an open image of $[p] \in T^2$ to $\mathbb{R}^2$. By the usual argument, we can show if $[p] \in T^2$ lies on an edge of $P$ it has a neighbourhood homeomorphic to a disc.

	Finally, we look at a vertex of $[0,1]^2$. In the image, there is really only one vertex. To find a homeomorphism to the open disc, we can take four quarter circles at each corner, and glue them appropriately.

	For a general polygon, it is a similar idea. Interior and edge points are done analogously to $T^2$. For vertices, it is a bit different. We have different equivalence classes of vertices caused by orienting the edges in different ways.

	If $v$ is a vertex of $P$ with $k$ vertices in its equivalence class, then we have $k$ sectors in $P$. Any sector can be identified with out favourite sector in $\mathbb{R}^2$, i.e. $(r, \theta) \in \mathbb{R}^2$ with $0 \leq r < \delta$ and $\theta \in [0, 2\pi/k]$. Gluing these together, we get an open disc as a neighbourhood of $v$.

	This works unless $k = 1$, in which case we have two paired edges coming into or out of a vertex in $P$. But this is homeomorphic to a cone, which is homeomorphic to a disc.

	These neighbourhoods of points in $P/\sim$ show that $P$ is locally homeomorphic to a disc, and we can easily check that $P/\sim$ is Hausdorff and second countable.
\end{proofbox}

\begin{exbox}
	One more example now.
	\begin{enumerate}[(i)]
		\setcounter{enumi}{7}
		\item We now consider connecting surfaces. Given topological surfaces $\Sigma_1$ and $\Sigma_2$, we can remove an open disc from each, and glue the resulting boundary circles.

			Explicitly, we take $\Sigma \setminus D_1 \cup \Sigma_2 \setminus D_2$ as a disjoint union, and impose the quotient relation
			\[
			\theta \in \partial D_1 \sim \theta \in \partial D_2
			,\]
			where $\theta$ parametrizes $S^{1} = \partial D_i$.

			The result $\Sigma_1 \hash \Sigma_2$ is called the \emph{connected sum}\index{connected sum} of $\Sigma_1$ and $\Sigma_2$.

			In principle, this depends on the choices of discs, and it takes some effort to prove that it is well-defined.
	\end{enumerate}
\end{exbox}

\begin{lemma}
	The connected sum $\Sigma_1 \hash \Sigma_2$ is a topological surface.
\end{lemma}

We will not prove this lemma in this course.


\newpage

\printindex

\end{document}
