\documentclass[12pt]{article}
\usepackage{amsmath}
\usepackage{mathtools}
\usepackage[a4paper]{geometry}
\usepackage{fancyhdr}
\usepackage{tikz}
\usepackage{amssymb}
\usepackage{graphicx}
\usepackage{amsthm}
\usepackage{import}
\usepackage{xifthen}
\usepackage{pdfpages}
\usepackage{transparent}
\usepackage{adjustbox}
\usepackage[shortlabels]{enumitem}
\usepackage{parskip}
\makeatletter
\newcommand{\@minipagerestore}{\setlength{\parskip}{\medskipamount}}
\makeatother
\usepackage{imakeidx}

\DeclareMathOperator{\Ker}{Ker}
\DeclareMathOperator{\Img}{Im}
\DeclareMathOperator{\rank}{rank}
\DeclareMathOperator{\nullity}{null}
\DeclareMathOperator{\spn}{span}
\DeclareMathOperator{\tr}{tr}
\DeclareMathOperator{\adj}{adj}
\DeclareMathOperator{\id}{id}
\DeclareMathOperator{\Sym}{Sym}
\DeclareMathOperator{\Orb}{Orb}
\DeclareMathOperator{\Stab}{Stab}
\DeclareMathOperator{\ccl}{ccl}
\DeclareMathOperator{\Aut}{Aut}
\DeclareMathOperator{\Syl}{Syl}
\DeclareMathOperator{\sgn}{sgn}
\DeclareMathOperator{\Fit}{Fit}
\DeclareMathOperator{\Ann}{Ann}
\DeclareMathOperator{\epi}{epi}


\newcommand{\incfig}[1]{%
	\def\svgwidth{\columnwidth}
	\import{./figures/}{#1.pdf_tex}
}

\setlength\parindent{0pt}

\newtheorem{theorem}{Theorem}[section]
\newtheorem{corollary}{Corollary}[section]
\newtheorem{lemma}{Lemma}[section]
\newtheorem{proposition}{Proposition}[section]

\theoremstyle{definition}
\newtheorem{definition}{Definition}[section]
\newtheorem{example}{Example}[section]

\theoremstyle{remark}
\newtheorem*{remark}{Remark}

\pagestyle{fancy}
\fancyhf{}
\rhead{\leftmark}
\lhead{Page \thepage}
\setlength{\headheight}{15pt}

\makeindex[intoc]

\usepackage{hyperref}
\hypersetup{
    colorlinks,
    citecolor=black,
    filecolor=black,
    linkcolor=black,
    urlcolor=black
}

\newcommand{\mapsfrom}{\mathrel{\reflectbox{\ensuremath{\mapsto}}}}

\begin{document}

\hypersetup{pageanchor=false}
\begin{titlepage}
	\begin{center}
		\vspace*{1em}
		\Huge
		\textbf{IB Linear Algebra}

		\vspace{1em}
		\large
		Ishan Nath, Michaelmas 2022

		\vspace{1.5em}

		\Large

		Based on Lectures by Prof. Pierre Raphael

		\vspace{1em}

		\large
		\today
	\end{center}
	
\end{titlepage}
\hypersetup{pageanchor=true}

\tableofcontents

\newpage

\section{Vector Spaces and Subspaces}%
\label{sec:vector_spaces_and_subspaces}

Let $F$ be an arbitrary field.

\begin{definition}[$F$ vector space]\index{vector space}
	A $F$ vector space is an abelian group $(V, +)$ equipped with a function
	\begin{align*}
		F \times V &\to V \\
		(\lambda, v) &\mapsto \lambda v
	\end{align*}
	such that
	\begin{itemize}
		\item $\lambda(v_1 + v_2) = \lambda v_1 + \lambda v_2$, 
		\item $(\lambda_1 + \lambda_2)v = \lambda_1 v + \lambda_2 v$,
		\item $\lambda(\mu v) = (\lambda \mu) v$,
		\item $1 \cdot v = v$.
	\end{itemize}
\end{definition}
We know how to
\begin{itemize}
	\item Sum two vectors
	\item Multiply a vector $v \in V$ by a scalar $\lambda \in F$.
\end{itemize}

\begin{adjustbox}{minipage = \columnwidth - 25.5pt, margin=1em, frame=1pt, margin=0em}
\begin{example}
	\begin{enumerate}[(i)]
		\item[]
		\item Take $n \in \mathbb{N}$, then $F^{n}$ is the set of column vectors of length $n$ with elements in $F$. We have
			\[
			v \in F^{n}, v =
			\begin{pmatrix}
				x_1 \\
				\vdots \\
				x_n
			\end{pmatrix}
			, x_i \in F
			,\]
			\[
			v + w =
			\begin{pmatrix}
				v_1 \\
				\vdots \\
				v_n
			\end{pmatrix}
			+
			\begin{pmatrix}
				w_1 \\
				\vdots \\
				w_n
			\end{pmatrix}
			=
			\begin{pmatrix}
				v_1 + w_1 \\
				\vdots \\
				v_n + w_n
			\end{pmatrix}
			,\]
			\[
			\lambda v =
			\begin{pmatrix}
				\lambda v_1 \\
				\vdots \\
				\lambda v_n
			\end{pmatrix}
			.\]
			Then $F^{n}$ is a $F$ vector space.
	\end{enumerate}
	
\end{example}

\end{adjustbox}

\begin{adjustbox}{minipage = \columnwidth - 25.5pt, margin=1em, frame=1pt, margin=0em}
	\begin{enumerate}[(i)]
		\item For any set $X$, take
			\[
				\mathbb{R}^{X} = \{f : X \to \mathbb{R}\}
			.\]
			Then $\mathbb{R}^{X}$ is an $\mathbb{R}$ vector space.
		\item Take $M_{n, m}(F)$, the set of $n \times m$ $F$ valued matrices. Then $M_{n, m}(F)$ is a $F$ vector space.
\end{enumerate}

\end{adjustbox}

\begin{remark}
	The axiom of scalar multiplication implies that for all $v \in V$, $0 \cdot v = \mathbf{0}$.
\end{remark}

\begin{definition}[Subspace]\index{subspace}
	Let $V$ be a vector space over $F$. A subset $U$ of $V$ is a vector subspace of $V$ (denoted $U \leq V$) if
	\begin{itemize}
		\item $0 \in U$,
		\item $(u_1, u_2) \in U \times U$ implies $u_1 + u_2 \in U$,
		\item $(\lambda, u) \in F \times U$ implies $\lambda u \in U$.
	\end{itemize}
\end{definition}
Note if $V$ is an $F$ vector space, and $U \leq V$, then $U$ is an $F$ vector space.

\begin{adjustbox}{minipage = \columnwidth - 25.5pt, margin=1em, frame=1pt, margin=0em}
\begin{example}
	\begin{enumerate}[(i)]
		\item[]
		\item Take $V = \mathbb{R}^{\mathbb{R}}$, the space of functions $f : \mathbb{R} \to \mathbb{R}$. Let $\mathcal{C}(\mathbb{R})$ be the space of continuous function $f : \mathbb{R} \to \mathbb{R}$. Then $\mathcal{C}(\mathbb{R}) \leq \mathbb{R}^{\mathbb{R}}$.
		\item Take the elements of $\mathbb{R}^3$ which sum up to $t$. This is a subspace if and only if $t = 0$.
	\end{enumerate}	
\end{example}

\end{adjustbox}

Note that the union of two subspaces is generally not a subspace, as it is usually not closed under addition.

\begin{proposition}
	Let $V$ be an $F$ vector space, and $U, W \leq V$. Then $U \cap W \leq V$.
\end{proposition}

\begin{adjustbox}{minipage = \columnwidth - 25.5pt, margin=1em, frame=1pt, margin=0em}
	\textbf{Proof:} Since $0 \in U, 0 \in W$, $0 \in U \cap W$. Now consider $(\lambda, \mu) \in F^2$, and $(v_1, v_2) \in (U \cap W)^2$. Take $\lambda_1 v_1 + \lambda_2 v_2$. Since $u_1,v_1 \in U$, this is in $U$. Similarly, it is in $W$. So it is in $U \cap W$, and $U \cap W \leq V$.
\end{adjustbox}

\begin{definition}[Sum of subspaces]\index{subspace sum}
	Let $V$ be an $F$ vector space. Let $U, W \leq V$. Then the \textbf{sum} of $U$ and $W$ is the set
	\[
		U + W = \left\{ u + w \mid (u, w) \in U \times W \right\}
	.\]
\end{definition}

\begin{adjustbox}{minipage = \columnwidth - 25.5pt, margin=1em, frame=1pt, margin=0em}
\textbf{Proof:} Note $0 = 0 + 0 \in U + W$. Take $\lambda_1 f + \lambda_2 g$, where $f, g \in U + W$. Then we can write $f = f_1 + f_2, g = g_1 + g_2$, where $f_1, g_1 \in U$, $f_2, g_2 \in W$. Then 
\[
	\lambda_1 f + \lambda_2 g = \lambda_1 (f_1 + f_2) + \lambda_2(g_1 + g_2) = (\lambda_1 f_1 + \lambda_2 g_1) + (\lambda_1 f_2 + \lambda_2 g_2) \in U + W
.\]
\end{adjustbox}

\begin{remark}
	$U + W$ is the smallest subspace of $V$ which contains both $U$ and $W$.
\end{remark}

\subsection{Subspaces and Quotients}%
\label{sub:subspaces_and_quotients}

\begin{definition}[Quotient]\index{quotient}
	Let $V$ be an $F$ vector space. Let $U \leq V$. The quotient space $V / U$ is the abelian group $V/U$ equipped with the scalar product multiplication
	\begin{align*}
		F \times V/U &\to V/U \\
		(\lambda, v + U) &\mapsto \lambda v + U
	\end{align*}
\end{definition}

\begin{proposition}
	$V/U$ is an $F$ vector space.
\end{proposition}

\newpage

\printindex

\end{document}
