\documentclass[12pt]{article}
\usepackage{amsmath}
\usepackage[a4paper]{geometry}
\usepackage{fancyhdr}
\usepackage{tikz}
\usepackage{amssymb}
\usepackage{graphicx}
\usepackage{amsthm}
\usepackage{import}
\usepackage{xifthen}
\usepackage{pdfpages}
\usepackage{transparent}
\usepackage{adjustbox}
\usepackage[shortlabels]{enumitem}
\usepackage{parskip}
\makeatletter
\newcommand{\@minipagerestore}{\setlength{\parskip}{\medskipamount}}
\makeatother
\usepackage{imakeidx}

\DeclareMathOperator{\Ker}{Ker}
\DeclareMathOperator{\Img}{Im}
\DeclareMathOperator{\rank}{rank}
\DeclareMathOperator{\nullity}{null}
\DeclareMathOperator{\spn}{span}
\DeclareMathOperator{\tr}{tr}
\DeclareMathOperator{\adj}{adj}
\DeclareMathOperator{\id}{id}
\DeclareMathOperator{\Sym}{Sym}
\DeclareMathOperator{\Orb}{Orb}
\DeclareMathOperator{\Stab}{Stab}
\DeclareMathOperator{\ccl}{ccl}
\DeclareMathOperator{\Aut}{Aut}
\DeclareMathOperator{\Syl}{Syl}
\DeclareMathOperator{\sgn}{sgn}
\DeclareMathOperator{\Fit}{Fit}
\DeclareMathOperator{\Ann}{Ann}


\newcommand{\incfig}[1]{%
	\def\svgwidth{\columnwidth}
	\import{./figures/}{#1.pdf_tex}
}

\setlength\parindent{0pt}

\newcommand{\course}{AT }
\newcommand{\lecnum}{}

\newtheorem{theorem}{Theorem}[section]
\newtheorem{corollary}{Corollary}[section]
\newtheorem{lemma}{Lemma}[section]
\newtheorem{proposition}{Proposition}[section]

\theoremstyle{definition}
\newtheorem{definition}{Definition}[section]

\theoremstyle{remark}
\newtheorem*{remark}{Remark}

\pagestyle{fancy}
\fancyhf{}
\rhead{\leftmark}
\lhead{Page \thepage}
\setlength{\headheight}{15pt}

\newcommand{\mapsfrom}{\mathrel{\reflectbox{\ensuremath{\mapsto}}}}

\makeindex[intoc]

\usepackage{hyperref}
\hypersetup{
    colorlinks,
    citecolor=black,
    filecolor=black,
    linkcolor=black,
    urlcolor=black
    pdfauthor={Ishan Nath}
}

\begin{document}

\hypersetup{pageanchor=false}
\begin{titlepage}
	\begin{center}
		\vspace*{1em}
		\Huge
		\textbf{IB Analysis \& Topology}

		\vspace{1em}
		\large
		Ishan Nath, Michaelmas 2022

		\vspace{1.5em}

		\Large

		Based on Lectures by Dr. Paul Russell

		\vspace{1em}

		\large
		\today
	\end{center}
	
\end{titlepage}
\hypersetup{pageanchor=true}

\tableofcontents

\newpage

\part{Generalizing Continuity and Convergence}%
\label{prt:generalizing_continuity_and_convergence}

\section{Three Examples of Convergence}%
\label{sec:three_examples_of_convergence}

\subsection{Convergence in \texorpdfstring{$\mathbb{R}$}{R}}%
\label{sub:convergence_in_r_}

Let $(x_n)$ be a sequence in $\mathbb{R}$ and $x \in \mathbb{R}$. We say $(x_n)$ \textbf{converges}\index{convergence in $\mathbb{R}$} to $x$, and write $x_n \to x$, if for all $\varepsilon > 0$, there exists $N$ such that for all $n \geq N$, $|x_n - x| < \varepsilon$.

In $\mathbb{R}$, one useful fact is the \textbf{triangle inequality} -- $|a+b| \leq |a| + |b|$. We also have two key theorems:

\begin{theorem}[Bolzano-Weierstrass Theorem]\index{Bolzano-Weierstrass theorem}
\item
	A bounded sequence in $\mathbb{R}$ must have a convergent subsequence.
\end{theorem}

Recall that a sequence $(x_n)$ in $\mathbb{R}$ is \textbf{Cauchy}\index{Cauchy sequence} if for all $\varepsilon > 0$, there exists $N$, such that for all $m, n \geq N$, $|x_m - x_n| < \varepsilon$. It is easy to show every convergent sequence is Cauchy. We also have the following:

\begin{theorem}[General Principle of Convergence]\index{general principle of convergence}
\item
	Any Cauchy sequence in $\mathbb{R}$ converges.
\end{theorem}

This can be proven by Bolzano-Weierstrass theorem.

\subsection{Convergence in \texorpdfstring{$\mathbb{R}^2$}{R\^2}}%
\label{sub:convergence_in_r_2_}

Let $(z_n)$ be a sequence in $\mathbb{R}^2$, and $z \in \mathbb{R}^2$. We wish to define $(z_n) \to z$.

In $\mathbb{R}$, we used the norm $|x|$. In $\mathbb{R}^2$, if we have $z = (x, y)$, then we can say $\|z\| = \sqrt{x^2 + y^2}$. This also satisfies the triangle inequality\index{triangle inequality} -- $\|a + b\| \leq \|a\| + \|b\|$.

\begin{definition}
	Let $(z_n)$ be a sequence in $\mathbb{R}^2$, and $z \in \mathbb{R}^2$. We say that $(z_n)$ \textbf{converges} to $z$, and write $z_n \to z$, if for all $\varepsilon > 0$, there exists $N$ such that for all $n \geq N$, $\|z_n - z\| < \varepsilon$.

	Equivalently, $z_n \to z$ if and only if $\|z_n - z\| \to 0$.
\end{definition}

\begin{adjustbox}{minipage = \columnwidth - 25.5pt, margin=1em, frame=1pt, margin=0em}
	\begin{lemma}
	If $(z_n)$, $(w_n)$ are sequences in $\mathbb{R}^2$ with $z_n \to z$, $w_n \to w$. Then $z_n + w_n \to z + w$.
\end{lemma}

\textbf{Proof:}
\begin{align*}
	\|(z_n + w_n) - (z + w)\| \leq \|z_n - z\| + \|w_n - w\| \to 0 + 0 = 0.
\end{align*}

\end{adjustbox}

In fact, given convergence in $\mathbb{R}$, convergence in $\mathbb{R}^2$ is easy.

\begin{proposition}
	Let $(z_n)$ be a sequence in $\mathbb{R}^2$ and let $z \in \mathbb{R}^2$. Write $z_n = (x_n, y_n)$ and $z = (x, y)$. Then $z_n \to z$ if and only if $x_n \to x$ and $y_n \to y$.
\end{proposition}

\begin{adjustbox}{minipage = \columnwidth - 25.5pt, margin=1em, frame=1pt, margin=0em}
\textbf{Proof:}

First, note $|x_n - x|, |y_n - y| \leq \|z_n - z\|$, so $\|z_n - z\| \to 0$ implies $|x_n - x|, |y_n - y| \to 0$.

Now, if $|x_n - x|, |y_n - y| \to 0$, then $\|z_n - z\| = \sqrt{|x_n - x|^2 + |y_n - y|^2} \to 0$.
\end{adjustbox}

\begin{definition}
	A sequence $(z_n)$ in $\mathbb{R}^2$ is \textbf{bounded} if there exists $M \in \mathbb{R}$ such that for all $n$, $\|z_n\| \leq M$.
\end{definition}

\begin{theorem}[Bolzano-Weierstrass in $\mathbb{R}^2$]
\item
	A bounded sequence in $\mathbb{R}^2$ must have a convergent subsequence.
\end{theorem}

\begin{adjustbox}{minipage = \columnwidth - 25.5pt, margin=1em, frame=1pt, margin=0em}
	\textbf{Proof:} Let $(z_n)$ be a bounded subsequence in $\mathbb{R}^2$. Write $z_n = (x_n, y_n)$. Now $|x_n|, |y_n| \leq \|z_n\|$, so $x_n, y_n$ are bounded in $\mathbb{R}$.

	By Bolzano-Weierstrass, $x_n$ has a convergent subsequence, say $x_{n_j} \to x \in \mathbb{R}$. Similarly $(y_{n_j})$ is bounded, so it has a convergent subsequence $y_{n_{j_k}} \to y$. Since we know $x_{n_{j_k}} \to x$, $z_{n_{j_k}} \to z = (x, y)$.
\end{adjustbox}

\begin{definition}
	A sequence $(z_n) \in \mathbb{R}^2$ is \textbf{Cauchy} if for all $\varepsilon > 0$, there exists $N$ such that for all $m, n \geq N$, $\|z_m - z_n\| < \varepsilon$.
\end{definition}

It is easy to show a convergent sequence in $\mathbb{R}^2$ is Cauchy.

\begin{theorem}[General Principle of Convergence for $\mathbb{R}^2$]
\item
	Any Cauchy sequence in $\mathbb{R}^2$ converges.
\end{theorem}

\begin{adjustbox}{minipage = \columnwidth - 25.5pt, margin=1em, frame=1pt, margin=0em}
	\textbf{Proof:} Let $(z_n)$ be a Cauchy sequence in $\mathbb{R}^2$. Write $z_n = (x_n, y_n)$. For all $m, n$, $|x_m - x_n| \leq \|z_m - z_n\|$, so $(x_n)$ is a Cauchy sequence in $\mathbb{R}$, thus it converges in $\mathbb{R}$. Similarly, $(y_n)$ converges in $\mathbb{R}$, so $(z_n)$ converges.
\end{adjustbox}


\newpage

\printindex

\end{document}
